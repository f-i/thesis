\chapter{Methods for Producing a Reliable APWP}\label{chap:Reliab}  % chktex 36
\textit{This chapter mainly describes how to generate paleomagnetic APWPs using
168 different methods, and then the application of the new APW path similarity
measuring tool described in the last chapter in finding the best APWP generating
method(s). The final results tell us that the ``Age Position Picking (APP)''  % chktex 36
method is better than the ``Age Mean Picking (AMP)'' method for making a
reliable paleomagnetic APWP and weighting is actually unnecessary.}
\vfill
\minitoc\newpage

\section{Introduction}

A paleomagnetic pole, also known as paleopole, has many attributes, including
sampling site, number of samples, sample rock characteristics, age determination
and its uncertainty, pole location and its uncertainty etc. And APWPs are
generated by combining paleopoles into mean poles, for a particular rigid block
over the desired age range to produce a smoothed path (see detailed definition
of APWP in Section~\ref{sec:applPlateTec}). APWP reflects cumulative motions of
a continental fragment relative to the Earth's spin axis. So, in order to build
an APWP, an object continent or continental fragment should be picked first. And
then the general ``boundary'' of this continent or continental fragment should
be determined tectonically. This closed tectonic ``boundary'' is then used to
constrain the paleomagnetic datasets through picking only those with their
sampling sites lying inside the closed ``boundary''. See
Appendix~\ref{appen4chp3}
for examples about how the paleopole datasets are constrained for a particular
tectonic plate during a specific time interval. Then the paleopoles that derive
from those sampling sites are statistically combined together into mean poles
when their ages are close. The final products, those mean poles, are then
connected according to their estimated ages to make a time-series path which is
called paleomagnetic APWP\@. So it is important to ask a question like what can
be called a reliable paleomagnetic APWP or how to build a reliable APWP,
because, for example (and also what we want to focus our work on here is), there
are so many different perspectives to consider when we combine paleopoles into
mean poles. For instance, how close a group of paleopoles' ages should be to be
combined together into a mean pole, what statistical way should be used to do
this combining, should those paleopoles be treated equally in this process of
combining, if no what factor affects more than others, etc. These are the
factors that might affect the end result.

In this chapter, paleopoles' attributes are used to weigh their influence on
mean poles (see details in Section~\ref{sec:w}), or to determine if they should
be omitted for producing a `better' mean pole (see details in Section~\ref{sec:f}).
Different key attributes, that can be quantified, or their combination are
considered. In addition, when we are merging paleopoles to produce a smoothed
mean path, we make choices not only about which data are included or excluded,
but how data are combined. Moving averaging is used to combine data (see
details in Section~\ref{sec:p}). With application of two moving average methods,
where each moving window includes data with (1) only middle point of lower and
higher age limits considered as paleopole's age and (2) whole dating uncertainty
considered as paleopole's age range, different APWPs are produced to see which
attribute is more important and also which moving average method performs
better. To answer these questions, we need a reference path (see details in
Section~\ref{sec:refpath}) and a tool to measure similarity between the paleomagnetic APWPs
and the reference path. Then the measuring tool developed and introduced in
Chapter~\ref{chap:Metho} is used to measure the similarity. We have two reference paths derived
from other plate kinematics models. These reference models are thought to be
more accurate and more precise, so that if the paleomagnetic APWPs are similar
to the reference path, they are regarded as more `reliable'. The question is
which model predicted reference path is a better reference or is there any
difference between these two reference paths? In this chapter, we will also try
to answer this question.

Finally no single attribute generally works better than others for all the
continents or continental fragments. However, the moving average method, that
considers the whole range of paleopoles' temporal uncertainty when calculating
mean pole's age is needed, is proved that it is obviously a better way of
generating reliable APWPs than the one that only considers the mid-point of
temporal uncertainty as a paleopole's age. And weighting is proved to be
actually unnecessary. The reference paths derived from other models are not
significantly different.

\subsection{How An APWP Is Generated and Key Factors}

As mentioned above, the first question to ask is that which tectonic plate the
paleopole we are working on belongs to (see Section~\ref{sec:f1} and details
about how it is solved in Appendix~\ref{appen4chp3}). The fact is that
paleomagnetic data is not just about poles but multiple attributes integrated
and not all data are created equal. So once the host tectonic plate is
determined, we encounter another problem: the attributes, for example,
uncertainties of ages and also locations, can vary greatly for different
paleopoles (see details in Sections~\ref{sec:ageu} and~\ref{sec:posu}).
Paleopole's attributes like age and location obviously can be
quantified and then used to weigh paleopoles or to omit some of them through
setting up a range of acceptable values. In addition, the age uncertainty can
also be used to change the way how we do moving averaging when we calculate mean
poles through considering the whole uncertainty range instead of only
considering the mid-point of the uncertainty range in the traditional way (this
is also how we get two different moving average methods). Besides age and
position uncertainties of paleopoles, the data consistency is also needed to be
investigated carefully (see details in Section~\ref{sec:datcons}), because some
paleopoles' polarity given in the GPMDB4.6b~\cite[updated in 2016 by the Ivar
Giaever Geomagnetic Laboratory team, in collaboration with Pisarevsky]{M96,P05}
could be wrong although most are correct. Data density is also vital (see
details in Section~\ref{sec:datden}) because paleomagnetism is basically built
on statistics. In addition, when the paleopoles were published (publication
year) is also a key factor that potentially indicates the quality of the
paleopoles (see details in Section~\ref{sec:puby}), simply because in old times,
technology of magnetism measuring was not that advanced as today. In
consideration of all these factors, we organise them in the general processing
steps below in order to generate a paleomagnetic APWP\@.

\subsubsection{Picking/Binning Step}

The moving average method is used to combine paleopoles into mean poles. The
moving time window/bin picks a group of paleopoles at each time step (see
description in Section~\ref{sec:p} and Fig.~\ref{fig-nac-maplat}). So the key
factors are:

\paragraph{Width of time window}
(trade off between number of paleopoles and amount of smoothing): Smoothness of
paleomagnetic APWPs generated by moving averaging could change with different
time window lengths and steps. If window length is too wide and step is too
long, final path would be so smoothed that actual details would be missed. If
window length is too narrow and step is too short, final path would be jerky
piecewise so that too much noise would be introduce because there are too few
datasets in each window. So a balance needs to be achieved.

\paragraph{Age uncertainty in magnetisation,}
particularly when that uncertainty is larger than the selection window. The
mid-point is commonly used as a selection point (see an example in upper panel
of Fig.~\ref{fig-nac-maplat}), but when the minimum age is constrained by a
field test (common when the age range is large; see Section~\ref{sec:ageu}), the
assumption that this is the most likely age is questionable. As it is mentioned
in Section~\ref{sec:ageu}, if field test shows magnetisation acquired prior to,
for example, a folding event that is tens of millions of years after initial
rock formation, a passed field test is not actually very useful. So age
uncertainty is very large, the possibility of remagnetisation is certainly very
high, which means the age is probably closer to the minimum age instead of the
middle age.

\subsubsection{Filtering Step}

This step filters out bad data that is likely to be unreliable, based on known
information about the paleopoles. These can be subdivided into two groups:

\paragraph{Characteristics that indicate the paleopole is well-constrained
(precise)}
(e.g., has a small $\alpha$95 uncertainty ellipse, large $\kappa$)

\paragraph{Characteristics that indicate the paleopole is reliable (accurate)}:
\begin{itemize}
  \item Lithology, particularly the risk of inclination flattening in sediments
        (see Section~\ref{sec:datcons})
  \item Risk of unidentified local rotations in deformed areas (see
        Section~\ref{sec:datcons})
  \item Publication year \textemdash{} younger studies that use stepwise
        demagnetisation techniques more likely to remove overprints and isolate
        a primary remanence (see Section~\ref{sec:puby}).
  \item Sampling strategy \textemdash{} need sufficient number of contributing
        samples and sites (N and B in Table~\ref{tab-fld}) to have a good chance
        of averaging out secular variation (see Section~\ref{sec:datcons}).
  \item Field tests that constrain the magnetisation age (see
        Section~\ref{sec:ageu}).
\end{itemize}

\subsubsection{Calculation Step}

Normally all paleopoles are treated equally when calculating the Fisher
mean~\citep[see how to calculate a Fisher mean using the formulas 6.12, 6.13,
6.14 and 6.15 in][chap. 6; note that instead of direction declination and
inclination expected in those formulas, pole longitude and latitude should be
used]{B92}, but as an alternative or supplement to filtering paleopoles can
potentially be weighted according to the factors that potentially influence
their precision and/or accuracy such as the A95 uncertainty or age uncertainty,
and a weighted Fisher mean calculated, giving the `better' paleopoles more
influence on the result mean pole. So, for example, each paleopole has got a
weight before calculating for a mean. In this dissertation these weights $w_i$
are integrated into the $\sum\limits_{i=1}^N l_i$, $\sum\limits_{i=1}^N m_i$ and
$\sum\limits_{i=1}^N n_i$ of the formulas 6.13 and 6.14 in~\citet[chap.~6]{B92},
where $w$ means weight and $i$ is the same as in those two formulas. Then for a
weighted Fisher mean, the two formulas 6.13 and 6.14 in~\citet[chap.~6]{B92}
become
%
\begin{equation}
  l = \frac{\sum\limits_{i=1}^N l_{i}w_i}{R} \quad\quad
  m = \frac{\sum\limits_{i=1}^N m_{i}w_i}{R} \quad\quad
  n = \frac{\sum\limits_{i=1}^N n_{i}w_i}{R}
\end{equation}
%
and
%
\begin{equation}
  R = {\left( \sum\limits_{i=1}^N l_{i}w_i \right)}^2
    + {\left( \sum\limits_{i=1}^N m_{i}w_i \right)}^2
    + {\left( \sum\limits_{i=1}^N n_{i}w_i \right)}^2
\end{equation}
%
Note that the trade-offs and difficulties are similar for filtering, e.g.\
giving poles with low A95 more weight presumes that they are closer to the
actual mean, which is not necessarily the case. Therefore, weighing on different
paleomagnetic factors is hoped to help us have some insight into these complex
issues.

\subsection{Existing Solutions and General Issues}\label{sec:si}

As mentioned in the concluding paragraph of Section~\ref{sec:f2}, if not all
paleopoles are created equal, the question becomes: how should paleopoles of
varying quality be combined? For a certain set of paleopoles, how can we produce
an APWP that is both
\begin{enumerate}
  \item well constrained: the mean pole for each time step has a low spatial
  uncertainty (small 95\% confidence ellipse), and
  \item accurate: the mean pole position is close to its actual position at each
  time step.
\end{enumerate}

Previous work on this question have focussed on the filtering out so-called
``bad'' data before calculation of the mean pole. Commonly used schemes include:
\begin{enumerate}
  \item~\citet{v90} (V90). V90 includes seven criteria (see the details in the
  concluding paragraph of Section~\ref{sec:f2}). The Q (quality) factor is the
  total number of criteria satisfied (0\textendash7)~\citep{v88}. The Q factor
  is a very straightforward way to get a quantitative reliability score, and is
  widely used for filtering paleopoles prior to calculating
  APWPs~\citep[e.g.][]{T12,Ma16,F19}. Further each paleopole can also be
  conveniently weighted in proportion to its Q factor in the calculations of
  APWPs~\citep{T92}. But at the same time this is a fairly basic filter that
  lumps together criteria that may not be equally important.
  \item~\citet{B02} (BC02). BC02 data quality criteria use only paleopoles with
  $\alpha$95 less than 10\degree\ in the Cenozoic and 15\degree\ in the
  Mesozoic, derived from at least 36 samples from at least 6 sampling sites (see
  also the concluding paragraph of Section~\ref{sec:f2}). While straightforward and
  convenient to apply, these stringent criteria mean some useful data may be
  filtered out and wasted, especially for a period where there are only limited
  number of paleopoles.
  \item~\citet{S05} (SS05). SS05 is similar to, but less stringent than, BC02.
  SS05 uses only paleopoles with $\alpha$95 of $\leq$15\degree\ and an age
  uncertainty of $\leq$40 Myr, derived from at least $\geq$4 sampling sites with
  $\geq$4 samples/site, and at least some sample demagnetisation (See also the
  last paragraph of Section~\ref{subp:ss05}).
\end{enumerate}

Although they are often used, and effect the path and uncertainties of the
resulting APWP, there has been limited study of how effective these filtering
methods are at reconstructing a `true' APWP compared to unfiltered data.
Furthermore, the focus on the filtering stage ignores the possible impact of
binning/windowing (i.e.\ combining paleopoles through moving-averaging) and
weighting. This study more rigorously investigates the effects of the choices
made at every stage on the resulting APWP\@. By focussing on paleomagnetic data
from the last $\sim$120 Myr, where plate motions are independently constrained
by reconstructions of seafloor spreading tied into a hotspot reference frame, we
can also verify which choices actually lead to a better-constrained and accurate
record of past plate motions.


\section{Methods}

\subsection{General Approach}

In this study, we use paleopoles extracted from the GPMDB4.6b~\citep{M96,P05} to
generate APWPs for the period 120\textendash0 Ma. A range of possible APW paths
for North America, India and Australia can be generated from the extracted sets
of paleopoles using various binning, filtering and weighting methods
(Table.~\ref{tab-pick} and~\ref{tab-weit}). These paths can then be compared to
synthetic APWPs independently generated from an absolute plate motion model. The
three plates chosen have different attributes, both in terms of the input data
set and the nature of the reference APWP\@.

\subsection{Paleomagnetic Data}

\subsubsection{GPMDB Field Codes}

Data analysis includes manipulation of data fields/columns in the GPMDB\@. In
the following content, the codes of the several specific fields used will be
referred to. They are listed in Table~\ref{tab-fld} for easy reference.

\begin{table}
\centering
\caption{List of the used fields and field codes of the GPMDB.}\label{tab-fld}
\begin{tabular}{p{0.16\linewidth} p{0.79\linewidth}}
\toprule
Field Code & Meaning \\ \midrule
LOMAGAGE & Lower best estimate of the magnetic age of the magnetisation component \\
HIMAGAGE & Upper best estimate of the magnetic age of the magnetisation component \\
B & Number of sites \\
N & Number of samples \\
ED95 & Radius of circle of 95\% confidence about mean direction, i.e. $\alpha$95 \\
EP95 & Radius of circle of 95\% confidence about paleopole position \\
KD & Fisher precision parameter for mean direction \\
DP & Half-angle of confidence on the pole in the direction of paleomeridian \\
DM & Half-angle of confidence on the pole perpendicular to paleomeridian \\
K\_NORM & Fisher precision parameter for Normal directions \\
ROCKNAME & Name of sample rock \\
ROCKTYPE & Type of sample rock \\
STATUS & Indicates if results have been superseded \\
COMMENTS & General information including details of the origin of LOMAGAGE and
           HIMAGAGE\@. If the result is a combined pole this field contains information on the data included in the combined result \\
\bottomrule
\end{tabular}
\end{table}

\subsubsection{Paleomagnetic Data of Three Representative Continents}

Collections of paleopoles with a minimum age (LOMAGAGE) $\leq$135 Ma for the
North American, the Indian and Australian
plates, were extracted from the GPMDB\@. In order to include valid paleopoles
from blocks that moved independently prior to 120 Ma, which therefore have
different assigned plate codes in the GPMDB, the spatial join
technique~\citep{J07} was used to find all poles within the geographic region
that defines the rigid plate within the period of interest (see
Appendix~\ref{appen4chp3} for details):

\paragraph{For North America,}
the search region was defined by the North American craton (NAC), Avalon/Acadia
and Piedmont blocks, as defined by the recently published plate model
of~\citet{Y18}. Following extraction, 58 palepoles from southwestern North America
that have been affected by regional rotations since 36 Ma~\citep{Mc06}, were
removed. The final dataset consists of 135 paleopoles (Fig.~\ref{fig-120NAhist}),
with 76 of them (${\sim}56$\%; average age uncertainty ${\sim}14$ Myr, average
EP95 ${\sim}9.3$\degree) sampled from dominantly igneous sequences, 56
(${\sim}42$\%; average age uncertainty ${\sim}27.5$ Myr, average EP95 ${\sim}10.5$\degree)
sampled from mostly sedimentary sequences, including 6 from redbeds, and 3
(${\sim}2$\%) from metamorphic sequences. The principal features of the age
distribution are a larger number of young ($<$5 Ma) poles, and relatively fewer
poles in the Late Cretaceous and Miocene.

\begin{figure*}
\centering
\includegraphics[width=.85\textwidth]{../../paper/tex/GeophysJInt/figures/120NAhist.pdf}
\caption[Distribution of 120\textendash0 Ma North American paleopoles]{Temporal
distribution of 120\textendash0 Ma $NAC$ paleopoles in 10 Myr window length and
5 Myr step length. For distribution a, each bin only counts in the midpoints
(circles) of pole uncertainty bars (not including those right at bin edges); For
distribution b, as long as the bar intersects with the bin (not including those
intersecting only at one of bin edges), it is counted in. Inside the
parentheses, i means igneous rocks derived (red bars; only two poles,
83\textendash77 Ma and 80\textendash65 Ma, from igneous and also sedimentary;
only one pole, 72\textendash40 Ma, from igneous and also metamorphic), r means
sedimentary rocks with redbeds involved derived (orange bars), and m means
metamorphic rocks derived (blue bars); the left are non-redbed sedimentary rocks
derived (black bars; only two poles, 146\textendash65 Ma [RESULTNO 6679] and
2\textendash0 Ma [RESULTNO 1227], are from sedimentary and also metamorphic).
The data published before 1984 are shown as circles with a
dot.}\label{fig-120NAhist}
\end{figure*}

\paragraph{For India,}
the Indian block as defined by~\citet{Y18} was used, but following extraction 31
paleopoles associated with parts of the northern margin that have undergone
regional rotations since the Jurassic~\citep{G15} were removed. The final
dataset consists of 75 paleopoles (Fig.~\ref{fig-120INhist}), with 39 of them
(52\%; average age uncertainty ${\sim}5$ Myr, average EP95 ${\sim}7.7$\degree)
sampled from dominantly igneous sequences and 36 (48\%; average age uncertainty
${\sim}14$ Myr, average EP95 ${\sim}7$\degree) sampled from mostly sedimentary
sequences, including 3 from redbeds. There is a high concentration of poles from
the latest Cretaceous\textendash{}Early Cenozoic (${\sim}70$\textendash60 Ma),
many of which are igneous; in younger and older intervals, there are fewer,
mostly sedimentary poles.

\begin{figure}[!ht]
\centering
\includegraphics[width=.91\textwidth]{../../paper/tex/GeophysJInt/figures/120INhist.pdf}
\caption[Distribution of 120\textendash0 Ma Indian paleopoles]{Temporal
distribution of 120\textendash0 Ma Indian paleopoles. For red bars, only one
pole, 67\textendash64 Ma (RESULTNO 8593), is from igneous and also sedimentary.
See Fig.~\ref{fig-120NAhist} for more information.}\label{fig-120INhist}
\end{figure}

\paragraph{For Australia,}
the Australia, Sumba, and Timor blocks as defined by~\citet{Y18} were used, in
combination with data from the Tasmania block younger than ${\sim}100$ Ma (with
a maximum age (HIMAGAGE) $\leq$100 Ma), prior to which it was not fixed with
respect to Australia~\citep{Y18}. The final dataset consists of 99 paleopoles
(Fig.~\ref{fig-120AUhist}), with 61 of them (${\sim}62$\%; average age
uncertainty ${\sim}23.5$ Myr, average EP95 ${\sim}10.9$\degree) sampled from
dominantly igneous sequences, and 38 (${\sim}38$\%; average age uncertainty
${\sim}23$ Myr, average EP95 ${\sim}9.4$\degree) sampled from mostly sedimentary
sequences, including 9 from redbeds. The temporal distribution of poles is
relatively uniform.

\begin{figure*}
\centering
\includegraphics[width=.91\textwidth]{../../paper/tex/GeophysJInt/figures/120AUhist.pdf}
\caption[Distribution of 120\textendash0 Ma Australian paleopoles]{Temporal
distribution of 120\textendash0 Ma Australian paleopoles. For black bars,
only four poles, 100\textendash80 Ma (RESULTNO 1106), 10\textendash2 Ma
(RESULTNO 1208), 4\textendash2 Ma (RESULTNO 140) and 1\textendash0 Ma (RESULTNO
1963), are from sedimentary and also igneous. For red bars, only one pole,
65\textendash25 Ma (RESULTNO 1872), is from igneous and also sedimentary rocks,
and only one pole, 1\textendash0 Ma (RESULTNO 1147), is from igneous and also
metamorphic rocks. See Fig.~\ref{fig-120NAhist} for more
information.}\label{fig-120AUhist}
\end{figure*}

\bigskip
Compared with North American (Fig.~\ref{fig-120NAhist}) and Australian
(Fig.~\ref{fig-120AUhist}) paleopoles, Indian paleopoles have a relatively lower
density and a higher prevalence of sedimentary paleopoles, except during the
period of ${\sim}70$\textendash60 Ma (Fig.~\ref{fig-120INhist}). For North
America and India, sedimentary paleopoles have significantly larger average age
uncertainty, but about same for Australia, there is no igneous and sedimentary
${\alpha}95$ difference.

\subsection{APWP Generation}\label{sec:apwpg}

Multiple APWPs were generated using the selected poles for each of the three
plates as follows:

\paragraph{Picking/binning.} A moving average or moving window technique was
used: paleopoles were selected for each APWP time step (initially 5 Myr step
length from 0 to 120 Ma) if their age fell within a window centered on the
current step age. In this study, the width of the moving window was always twice
that of time step (i.e.\ initially 10 Myr), such that each window half-overlaps
with its neighbours.

\paragraph{Filtering.} Poles with characteristics thought to correspond to
poor data quality, or lacking characteristics thought to correspond to good
data quality, were discarded (or in some cases, corrected).

\paragraph{Weighting.} Calculation of a weighted Fisher mean~\citep{F53} of
the remaining poles within each window, using weighting functions intended to
increase the influence of higher quality poles relative to lower quality ones.

\bigskip
Twenty-eight different picking and filtering algorithms were tested
(Table~\ref{tab-pick}, referred to hereafter as Pk), in combination with 6
different weighting algorithms (Table~\ref{tab-weit}, referred to hereafter as
Wt), for the three plates. The effects of changing the time step length and
width of the moving window, and the reference path, were also examined.

\subsubsection{Picking/Binning}\label{sec:p}

In studies where the moving window method is used to calculate an
APWP~\citep{T99,T08}, a paleopole is generally considered to fall in the current
window only if the mid-point of its age limits fall within that window. If
paleopole has a large age uncertainty compared to the size of the moving window,
it will not be included in the moving windows close to the beginning and end of
the age range, which are arguably more likely magnetisation ages than the
mid-point. To investigate this issue, we compare the performance of moving
windows populated using the mid-point picking criterion, referred to hereafter
as ``Age Mean Picking'' (AMP\@; even-numbered algorithms in
Table~\ref{tab-pick}, Fig.~\ref{fig-dif} and subsequent figures), to a less
restrictive picking criterion where a paleopole is included in the current
moving window if any part of its age limits falls within that window, referred
to hereafter as ``Age Position Picking'' (APP\@; odd-numbered algorithms in
Table~\ref{tab-pick}, Fig.~\ref{fig-dif} and subsequent figures). The APP
algorithm will pick more paleopoles in each moving window than the AMP algorithm
(Figs.~\ref{fig-120NAhist}-\ref{fig-120AUhist}; Fig.~\ref{fig-nac-maplat}).

If, for example, we have a paleopole with an acquisition age of 10\textendash20
Ma, and we have a 2 Myr moving window with a 1 Myr age step, then it is included
in just the 14\textendash16 Ma window (for the mid-point age of 15 Ma) for
AMP\@. For APP, this paleopole falls in the 9\textendash11, 10\textendash12,
11\textendash13, 12\textendash14 \ldots 17\textendash19, 18\textendash20 and
19\textendash21 Ma windows. Each original paleopole is therefore represented in
the mean poles calculated over its entire possible acquisition age.

\begin{figure}[!ht]
\centering
\includegraphics[width=.7\textwidth]{../../paper/tex/GeophysJInt/figures/binning.pdf}
\caption[How moving average (MA) methods work]{An example of 10 Myr moving
window and 5 Myr step in the two moving average methods, AMP and APP, based on
poles of the $NAC$. White circles are the midpoints of low and high magnetic
ages. The vertical axis has no specific meaning here. For example, for the
window of 15 Ma to 5 Ma (the dashed-line bin), the AMP method calculates the
Fisher mean pole (dark triangle in Fig.~\ref{fig-mhsPred}) of only 5 paleopoles,
while the APP method calculates the mean pole (dark circle in
Fig.~\ref{fig-mhsPred}) of 9 paleopoles.}\label{fig-nac-maplat}
\end{figure}

A shorter step and narrower window will potentially increase the clustering of
the selected paleopoles, but will reduce their number. Conversely, a longer step
and wider window will increase the number of poles, but potentially decrease
their clustering. To investigate these trade-offs, every picking/filtering and
weighting method was also used to generate APWPs with a time step and window
doubled to 10 Myr and 20 Myr, respectively. Paths generated using AMP and APP
with no filtering, and every weighting method, with time steps from 1 Myr to 15
Myr in 1 Myr increments and window widths from 2 Myr to 30 Myr in 2 Myr
increments, were also analysed. In all cases the oldest time step was 120 Ma.

\subsubsection{Filtering}\label{sec:f}

14 different filters or corrections (Table~\ref{tab-pick}) were applied to both
data picked using the AMP moving window method (even numbers) and data picked
using the APP moving window method (odd numbers), resulting in a total of 28
unique picking algorithms. The filters or corrections can be characterised as
follows:

\begin{table}[!ht]
\centering
\caption{List of all Picking/Binning algorithms developed here.}\label{tab-pick}
\begin{tabular}{@{}ll@{}}
\toprule
No. & Picking Algorithm \\ \midrule
0 & AMP\@: Age Mean Picking, see Section~\ref{sec:apwpg} \\
1 & APP\@: Age Position Picking \\
2 & AMP (``$\alpha$95/Age range'' no more than ``15/20'') \\
3 & APP (``$\alpha$95/Age range'' no more than ``15/20'') \\
4 & AMP (mainly or only igneous) \\
5 & APP (mainly or only igneous) \\
6 & AMP (contain igneous and not necessarily mainly) \\
7 & APP (contain igneous and not necessarily mainly) \\
8 & AMP (unflatten sedimentary) \\
9 & APP (unflatten sedimentary) \\
10 & AMP (nonredbeds) \\
11 & APP (nonredbeds) \\
12 & AMP (unflatten redbeds) \\
13 & APP (unflatten redbeds) \\
14 & AMP (published after 1983) \\
15 & APP (published after 1983) \\
16 & AMP (published before 1983) \\
17 & APP (published before 1983) \\
18 & AMP (exclude commented local rot or secondary print) \\
19 & APP (exclude commented local rot or secondary print) \\
20 & AMP (exclude local rot or correct it if suggested) \\
21 & APP (exclude local rot or correct it if suggested) \\
22 & AMP (filtered using SS05 palaeomagnetic reliability criteria) \\
23 & APP (filtered using SS05 palaeomagnetic reliability criteria) \\
24 & AMP (exclude superseded data already included in other results) \\
25 & APP (exclude superseded data already included in other results) \\
26 & AMP (comb of 22 and 24) \\
27 & APP (comb of 23 and 25) \\ \bottomrule
\end{tabular}
\raggedright{\\Notes: SS05,~\citet{S05}}
\end{table}

\paragraph{No modification (Pk 0,1).}

\paragraph{Removal of poles with large spatial and temporal uncertainties
(Pk 2,3).} Paleopoles with both large $\alpha$95 (ED95 $>15\degree$,
following the BC02 threshold for the Mesozoic) and wide acquisition age limits
(difference between HIMAGAGE and LOMAGAGE $>20$ Myr, following the V90 criteria
about age within a half of a geological period; the average of the geological
periods between 120 and 0 Ma [Quaternary, Neogene, Paleogene and Cretaceous] is
about 20 Myr) which are less likely to provide a good estimate of the actual
pole position within any specific age window, were excluded.

\paragraph{Prefer poles from igneous rocks (Pk 4,5, 6,7).} Pk 4,5 removes
paleopoles potentially affected by inclination flattening by selecting only
paleopoles coded as igneous or mostly igneous (ROCKTYPE starting with
``intrusive'' or ``extrusive''). In fact, most of the paleopoles picked by Pk
4,5 are derived from igneous-only rocks. Pk 6,7 select paleopoles coded as
containing igneous (ROCKTYPE containing ``intrusive'' or ``extrusive''); this is
a less strict filter, because the dominant rock type could potentially be
another lithology. Therefore Pk 6 also includes paleopoles from Pk 4, and Pk 7
also includes paleopoles from Pk 5.

\paragraph{Correct sedimentary poles for inclination shallowing (Pk 8,9).}
Rather than excluding paleopoles from sedimentary rocks, paleopoles coded as
sedimentary or redbeds were instead corrected for inclination flattening using
the flattening function $\tan I_o = f \tan I_f$~\citep{K55}, where $I_o$ is the
observed inclination, $I_f$ is the unflattened inclination, and $f$ is the
flattening factor (also known as shallowing coefficient; 1=no flattening,
0=completely flattened). Here $f=0.6$ is used in all cases,
following~\citet{T12}, unless the rock type (ROCKTYPE field in the database) is
not sedimentary dominated but contains sedimentary rocks. In these cases,
$f=0.8$ is used instead, following the minimum anisotropy-of-thermal-remanence
determined f-correction~\citep{D11,Do11}.

\paragraph{Remove redbeds (Pk 10,11) or correct them for inclination
shallowing (Pk 12,13).} Bias toward shallow inclinations is also observed
in paleomagnetic data derived from red-beds~\cite[e.g., in central Asia,
Mediterranean region, North America, etc.]{T04,K04,T07,B10}. This bias can be
addressed by removing the source (Pk 10,11; ROCKTYPE containing redbeds), or
correcting for inclination flattening, setting $f=0.6$ as previously described
(Pk 12,13). In the latter case, the assumption is being made that the
redbeds are carrying a detrital paleomagnetic signal.

\paragraph{Prefer poles from younger (Pk 14,15, 24,25) or older (Pk 16,17)
studies.} Advancements in equipment (e.g., cryogenic magnetometers) and
analytical techniques (e.g., stepwise demagnetisation) mean that more recently
published paleopoles are potentially more reliable than older ones. Pk 14,15
removes any paleopoles published prior to 1983 (YEAR $>$ 1983), the mean
publication date for paleopoles in the GPMDB\@. Pk 24,25 takes a similar but
less aggressive approach by excluding paleopoles that have been superseded (99
paleopoles) by later studies from the same sequence, which are presumed to
represent a more accurate determination of the paleopole position. The
``STATTUS'' field of the GPMDB4.6b indicates if a paleopole has been superseded.
Conversely, removing paleopoles published after 1983 (Pk 16,17; YEAR $\leq$
1983) should have a negative effect.

\paragraph{Exclude suspected local rotations and secondary overprints (Pk
18,19), or correct for them where possible (Pk 20,21).} Secondary remanence
components and local tectonic deformation can both displace the measured pole
position away from its ``true'' position. Such poles can be identified based on
demagnetisation data, or comparison to the pre-existing APWP\@. Pk 18,19
removes paleopoles that were identified as such in the COMMENTS field: 66
paleopoles affected by local rotations are identified by carefully going through
all the COMMENTs containing information about rotation; paleopoles affected by
secondary overprints are extracted with the keyword ``econd'' identified in the
COMMENTS\@. A subset (19) of the 66 paleopoles identified have a suggested
correction associated with them; Pk 20,21 retains these paleopoles after
applying the suggested correction.

\paragraph{SS05 quality criteria (Pk 22,23).}\label{subp:ss05} As with Pk 2,3,
SS05~\citep{S05} removes paleopoles with high spatial ($\alpha$95 $>15\degree$)
and temporal (age range $>$ 40 Myr) uncertainty, but additionally remove
paleopoles where samples had poor sampling coverage (sampling sites' quantity
[B] of $<4$, samples' quantity [N] of $<4$ times of the sites [B]) and were not
subjected to even a blanket demagnetisation treatment (laboratory cleaning
procedure code DEMAGCODE $<2$). Pk 26,27 also use these criteria, but
further excludes superseded data.

\bigskip
Some of the picking (Table~\ref{tab-pick}) and weighting (Table~\ref{tab-weit})
methods developed here are also connected with the V90 Q factor (see
Section~\ref{sec:si}). For example, Pk 2,3 and Wt 2, 4, 5 are related to the V90
criteria 1; Pk 2,3, 22,23, 26,27 and Wt 1, 3 are related to the V90 criteria 2;
Pk 22,23 are related to the V90 criteria 3; The data constraining described in
Appendix~\ref{appen4chp3} is related to the V90 criteria 5; and Pk 18,19 are
related to the V90 criteria 7.

\subsubsection{Weighting}\label{sec:w}

Following filtering, weights were assigned to each of the remaining paleopoles
using one of the following six algorithms (Table~\ref{tab-weit}), prior to
calculation of a weighted Fisher mean:

\paragraph{No weighting (Wt 0).} Weighting factor=1 for all
paleopoles.
%
\begin{longtable}[h]{@{}c|l@{}}
\caption{List of all weighting algorithms developed here.}\label{tab-weit}
\\\hline\multicolumn{1}{|p{.25in}|}{\textbf{No.}} & \multicolumn{1}{p{5.5in}|}{\textbf{Weighting Algorithm}} \\
\hline\endfirsthead%
\multicolumn{2}{r}{{\bfseries \tablename\ \thetable{} --- continued from previous page}} \\ \hline
\multicolumn{1}{|p{.25in}|}{\textbf{No.}} & \multicolumn{1}{p{5.5in}|}{\textbf{Weighting Algorithm}} \\ \hline
\endhead%
\hline\multicolumn{2}{|r|}{{\bfseries continued on next page}} \\ \hline
\endfoot\hline
\endlastfoot0 & None (No weighting) \\ \hline
1 & Larger numbers of sites (B) \& observations (N), greater $weight$ ($w$): \\
  & \begin{minipage}{5.5in}\begin{equation*}w=\left\{\begin{array}{ll}
    0.2 & \textrm{, if both B \& N are missing, or B$\leq1$ \& N$\leq1$} \\
    (1-\frac{1}{B})\times0.5 & \textrm{, if N is missing or N$\leq1$, \& B$>$1} \\
    (1-\frac{1}{N})\times0.5 & \textrm{, if B is missing or B$\leq1$, \& N$>$1} \\
    (1-\frac{1}{B})\times(1-\frac{1}{N}) & \textrm{, if B$>$1 \& N$>$1}
\end{array}\right.\end{equation*}\end{minipage} \\ \hline
2 & Lower age uncertainty, greater weight: \\
  & \begin{minipage}{5.5in}age\_range=HIMAGAGE-LOMAGAGE \\
    age\_midpoint = (HIMAGAGE+LOMAGAGE)$\times$0.5 \\
    if age\_midpoint$<$2.58 (Ma; start of the Quaternary, according to GSA Geologic Time Scale), \\
    \vbox{\begin{equation*}w=\left\{\begin{array}{ll}
    1 & \textrm{, if age\_range$\leq$1.29 (from $\frac{2.58\textendash0}{2}$)} \\
    \frac{1.29}{age\_range} & \textrm{, if age\_range$>$1.29}
    \end{array}\right.\end{equation*}} \\
    if 2.58$\leq$age\_midpoint$<$23.03 (Ma; Neogene), \\
    \vbox{\begin{equation*}w=\left\{\begin{array}{ll}
    1 & \textrm{, if age\_range$\leq$10.225 (from $\frac{23.03\textendash2.58}{2}$)} \\
    \frac{10.225}{age\_range} & \textrm{, if age\_range$>$10.225}
    \end{array}\right.\end{equation*}} \\
    if 23.03$\leq$age\_midpoint$<$201.3 (Ma; Paleogene, Cretaceous, Jurassic), \\
    \vbox{\begin{equation*}w=\left\{\begin{array}{ll}
    1 & \textrm{, if age\_range$\leq$15} \\
    \frac{15}{age\_range} & \textrm{, if age\_range$>$15}
    \end{array}\right.\end{equation*}}
    \end{minipage} \\ \hline
3 & Lower $\alpha$95, greater weight: \\
  & \begin{minipage}{5.5in}Positive half Normal distribution with a mean and standard deviation \\
    of 0 and 10, scaled with $10\sqrt{2\pi}$ (to make the peak reach 1) \\
    \vbox{\begin{equation*}w=e^{-\frac{\alpha_{95}^2}{200}},\end{equation*}} \\
    where \\
    \vbox{\begin{equation*}\alpha95=\left\{\begin{array}{ll}
    ED95 & \\
    DP & \textrm{, if ED95 is missing} \\
    \frac{140}{\sqrt{KD\times{}N}} & \textrm{, if ED95 \& DP are missing} \\
    \frac{140}{\sqrt{K\_NORM\times{}N}} & \textrm{, if ED95, DP \& KD are missing} \\
    \frac{140}{\sqrt{K\_NORM\times{}B}} & \textrm{, if ED95, DP, KD \& N are missing} \\
    \frac{140}{\sqrt{1.7\times{}B}} & \textrm{, if ED95, DP, KD \& K\_NORM are missing,} \\
    & \textrm{using the lowest KD value ${\sim}1.7$ in GPMDB4.6b,}
    \end{array}\right.\end{equation*}}
    finally $w$=0 if this $\alpha$95 completely overlaps with another smaller
    $\alpha$95 whose paleopole is exactly derived from the same place and same
    rock.
    \end{minipage} \\ \hline
4 & Age uncertainty Position to bin (more overlap, greater weight): \\
  & \begin{minipage}{5.5in}wha, window high age; wla, window low age \\
    \vbox{\begin{equation*}w=\left\{\begin{array}{ll}
    \frac{wha-LOMAGAGE}{age\_range} & \textrm{, if LOMAGAGE$>$wla \& HIMAGAGE$>$wha} \\
    \frac{HIMAGAGE-wla}{age\_range} & \textrm{, if LOMAGAGE$<$wla \& HIMAGAGE$<$wha} \\
    \frac{wha-wla}{age\_range} & \textrm{, if LOMAGAGE$<$wla \& HIMAGAGE$>$wha} \\
    1 & \textrm{, if LOMAGAGE$>$wla \& HIMAGAGE$<$wha}
    \end{array}\right.\end{equation*}}
    \end{minipage} \\ \hline
5 & Combining 3 and 4: average of the two weights from 3 and 4 \\
\end{longtable}
%
\paragraph{Weighting by sample and site number (Wt 1).} Paleopoles derived
from more individually oriented samples (observations; N) collected from more
sampling levels/sites (B) are more likely to average out secular variation and
accurately sample the GAD field~\citep{T19,B02,v90}, and are given a weighting
closer to 1. Unfortunately, in the GPMDB, a paleopole's B or N is not always
given. As shown in (Table~\ref{tab-weit} no. 1), in such cases the calculated
weights were modified to give lower weights overall whilst still accounting for
partial information, such as N being reported but not B.

For number of sampling sites B$>$1 and number of samples N$\leq$1, there are 8
such paleopoles for 120\textendash0 Ma North America, India 4, and Australia 1.
For B$\leq$1 and N$>$1, there are 20 such paleopoles for 120\textendash0 Ma
North America, India 26, and Australia 22. For B$\leq$1 and N$\leq$1, there are
only 23 such paleopoles for the whole GPMDB 4.6b, including 18 with neither B
value nor N value given; while for 120\textendash0 Ma there is no such paleopole
found in North America, India and Australia.

\paragraph{Weighting by age uncertainty (Wt 2)} Above a maximum age range
that represents a well-constrained age, defined as half of each geological
period in the Phanerozoic Eon (e.g., Quaternary, Neogene; here the geological
period that the middle point of the paleopole's age range falls within is
assigned)~\citep{v90,T19} or 15 Myr (the halves of the Paleogene, Cretaceous,
and Jurassic periods are all at least 20 Myr, which is large for these
relatively young geologic periods), whichever is smaller, paleopoles are given
an increasingly small weight as the age uncertainty (the high magnetic age $-$
the low magnetic age) increases (No. 2 in Table~\ref{tab-weit}).

\paragraph{Weighting by spatial uncertainty (Wt 3).} Paleopoles with a
smaller $\alpha$95 confidence ellipse are given a higher weighting than those
with a larger $\alpha$95, using a Gaussian/Normal distribution centered on 0
with standard deviation of 10. However, a paleopole's $\alpha$95 is not always
given in the database. If $\alpha$95 is not given, DP (the semi-axis of the
confidence ellipse along the great circle path from site to pole) is assigned as
$\alpha$95. If DP is not given either, $\alpha$95 was further approximated by
$\frac{140}{\sqrt{KD\times{}N}}$, where KD is Fisher precision parameter for
mean direction if this parameter is given, or Fisher precision parameter for
Normal directions (K\_NORM) if only K\_NORM is given when KD is missing. If N is
not given, B is used as N. If even K\_NORM is also missing, the lowest KD value
${\sim}1.7$ in GPMDB 4.6b is used as KD\@. It is also worthwhile to mention that
if specimens, where two paleopoles are derived, are exactly from the same place
and same rock (by checking if ROCKNAME, ROCKTYPE and sampling site are the
same), and one $\alpha$95 is completely inside the other $\alpha$95, a zero is
assigned as the weight of the data with the larger $\alpha$95. In fact, in the
above described procedure A95 (circle of 95\% confidence about mean pole) is a
better alternative instead of $\alpha$95, because A95 is directly reflecting the
spatial uncertainty of the paleopoles. However, most paleopoles' A95s are not
given in GPMDB 4.6b, so $\alpha$95 is used instead since $\alpha$95 is also
indirectly reflecting the quality of the paleopole.

\paragraph{Weighting by degree of overlap between moving window and pole age
uncertainty (Wt 4).} If a large fraction of the age range for an individual
paleopole falls within the current window, it is given a higher weighting than a
pole where the overlap is smaller, because it is more likely to be close to the
true pole position in the window interval. In other words, if window intersects
with part of age range, weight= (intersecting part) / (age range width).

\paragraph{Weighting by both spatial and temporal uncertainty (Wt 5).} This
weighting method is a combination of Wt 3 (but here the standard deviation of
the weighting function is 15 though) and Wt 4. It takes the average of sums of
the weights generated by Wt 3 and 4.

\subsubsection{Scaling of Weights}

The weights obtained from the six different weighting functions
(Table~\ref{tab-weit}) are then integrated into Fisher mean function~\citep{F53}
to calculate a weighted Fisher mean. First, weight values are used to scale
Cartesian x, y and z components of each individual paleopole's geographic
coordinate. Then these individual coordinates are combined through Fisher
resultant vector $R$ function~\citep[see][chap.~11]{T19}. Therefore the mean
pole location, its spatial uncertainty A95, and precision parameter are all
weighted along with $R$.

Traditionally, weights are directly integrated by being multiplied with the
variable we would like to do weighting to. For example, here weights can be
directly multiplied with the Cartesian x, y and z components of each paleopole.
However, this direct multiplying causes the decreasing of $R$, which further
sensitively and extremely increases $\alpha$95 (Fig.~\ref{fig-alpha95}),
especially because $R$ is always less than $N$ and $N$ is usually not that high
(more than ${\sim}50$ is rather rare, around 10 averagely). The consequence
would be that all the $\alpha$95s of mean poles are extremely large in size and
difficult to be spatially differentiated.

\begin{figure*}
	\centering
	\begin{subfigure}{.49\textwidth}
		\includegraphics[width=\textwidth]{../../paper/tex/GeophysJInt/figures/alpha95_000082.png}
		\caption{perspective view a}
	\end{subfigure}
	\begin{subfigure}{.49\textwidth}
		\includegraphics[width=\textwidth]{../../paper/tex/GeophysJInt/figures/alpha95_000094.png}
		\caption{perspective view b}
	\end{subfigure}
	\caption[Visualization of the equation that relates $\alpha$95, and R and
N]{Visualization of Equation 11.9 of Essentials of Paleomagnetism: Fifth Web
Edition, illustrating the relationship between the radius of the circle of 95\%
confidence ($p$=0.05) about the mean, $\alpha$95, resultant vector $R$ and number of directions (or
paleopoles) $N$.}\label{fig-alpha95}
\end{figure*}

Therefore, here weights are scaled before being multiplied with the Cartesian x,
y and z components by
%
\begin{equation*}
  ScaledWeight=\frac{Weight_i * N}{\sum_{i}^{N} Weight_i},
\end{equation*}
%
where $N$ is number of paleopoles for making a mean pole. So the scaled weight
could be greater than 1 because it is actually scaled through being divided by
the mean of the weights. This scaling does not only keep the effect of weighting
but also avoid dramatically changing $R$ and indirectly and extremely changing
$\alpha$95.

\subsection{Reference Paths}\label{sec:refpath}

A prediction of the expected APWP for any plate can be generated using a plate
kinematic model (e.g.\ the last ${\sim}180$\textendash200 Myr of plate motions
reconstructed from spreading ridges in ocean basins) that is tied in to an
absolute reference frame. Here, we use the rotations of~\citet{O05}, which
describe motion of Nubian Plate relative to the Indo-Atlantic hotspots back to
120 Ma. North America is linked to this absolute frame of reference across the
Mid-Atlantic ridge, using North America-Nubia rotations from chron C1n
(0\textendash0.78 Ma) from~\citet{D10}, to chron C2An (2.7 Ma)
from~\citet{Sh12}, to C5n.1ny (9.74 Ma) from~\citet{M99}, to C5n.2o (10.949 Ma)
from~\citet{G13}, to C6ny (19.05 Ma) from~\citet{M99}, to C6no (20.131 Ma)
from~\citet{G13}, to C34ny (83.5 Ma) from~\citet{M99}, from C34ny to
${\sim}118.1$ Ma from~\citet{S12}, and to closure at C34no (120.6 Ma)
from~\citet{G13}. India is linked via the East African Rift Valley
(Somalia-Nubia rotations from chron C1n (0\textendash0.78 Ma) from~\citet{D17},
to C2A.2no (3.22 Ma) from~\citet{H05}, and to closure at C7.2m (25.01 Ma) and
chron C34 (85 Ma) from~\citet{R16}, and finally extended to 120 Ma because there
was no known relative motion between Somalia and Nubia from 120 Ma to 85 Ma
according to the rotations from~\citet{M16}). Australia is linked via the East
African Rift Valley, then the SW Indian Ridge (E Antarctica-Somalia rotations
from chron C1n (0\textendash0.78 Ma) from~\citet{D17}, to C2A.2no (3.22 Ma)
from~\citet{H05}, to C5n.2no (10.95 Ma) from~\citet{L02}, to C13ny (33.06 Ma)
from~\citet{P08}, to C29no (64.75 Ma) from~\citet{C10}, to C34y (83 Ma)
from~\citet{R16}, to 96 Ma from~\citet{M01}, and to closure at chron M0 (120.6
Ma) from~\citet{M08}), and SE Indian Ridge (Australia-East Antarctica rotations
from chron C1n (0\textendash0.78 Ma) from~\citet{D17}, to C6no (20.13 Ma)
from~\citet{C04}, to C8o (26 Ma) from~\citet{G18}, to C17n.3no (38.11 Ma)
from~\citet{C04}, to C34ny (83.5 Ma) from~\citet{Wh13}, to the Quiet Zone
Boundary (96 Ma) from~\citet{W07}, to full closure at 136 Ma from~\citet{Wh13}).
The geomagnetic polarity timescales of~\citet{C95} for Late Cretaceous and
Cenozoic, and of~\citet{Gr94} for Early to Late Cretaceous time are used to
convert from chron boundaries to absolute ages. A long table listing these
rotation parameters with covariance uncertainties (Table~\ref{tab:rot}) is
included in Appendix~\ref{appen4chp3}, and the calculated rotations for the
North American, Indian and Australian reference APWPs in the hotspot reference
(Figs.~\ref{fig-mhsPred}-\ref{fig-mhsPred801}) are also listed in
Table~\ref{tab:refAPWP}.

\begin{figure}[!ht]
\centering
\includegraphics[width=1.01\textwidth]{../../paper/tex/GeophysJInt/figures/fhs_m.pdf}
\caption[120\textendash0 Ma MHM vs FHM predicted APWP of North America]{MHM
predicted 120\textendash0 Ma APWP (solid line) for $NAC$ through the North
America\textendash{}Nubia\textendash{}Mantle plate circuit. The FHM predicted
path (dashed line with shaded uncertainties) is also shown for comparison. The
age step is 5 Myr. Compared with the 10 Ma paleomagnetic mean pole calculated by
the AMP method (dark triangle), the coeval mean pole derived from the APP method
is closer to both FHM and MHM predicted 10 Ma poles, which indicates more data
diluting the effect of outliers. See also the paleopoles that the two mean poles
are composed of in Fig.~\ref{fig-nac-maplat}.}\label{fig-mhsPred}
\end{figure}

Where possible, poles which were published uncertainty estimates were used.
Where no uncertainty estimates were available, values of the covariance matrix
were set to an arbitrarily small value (1E–15). Where this occurs, the spatial
uncertainties for the reference APWP are likely underestimated. However, but
because the uncertainties for the Nubia-hotspot rotations are substantially
larger than for rotations derived from fitting magnetic isochrons, the effect is
small.

To reconstruct a reference APWP at the required time steps for comparison with
the paleomagnetic APWPs, rotations and their associated uncertainties were
interpolated between constraining finite rotation poles according to the method
of~\citet{D08}, assuming constant rates.

Neither the hotspot reference frame nor the paleomagnetic reference frame are
truly fixed with respect to the solid Earth. In the former case, hotspots are
not truly stationary in the mantle~\citep{S98}; in the latter, true polar wander
(TPW) may also lead to differential movements of the solid earth with respect to
the spin axis~\citep{E03}. In reality, it is difficult to untangle these
effects. Whilst there is little clear evidence for significant TPW in the past
${\sim}120$ Myr~\citep{C00,R04}, modeling suggests that the effects of hotspot
drift can start to become significant over 80\textendash100 Myr
timescales~\citep{O05}. Because paleomagnetic APWPs have large associated
spatial uncertainties, a synthetic APWP calculated using a fixed hotspot
reference frame is unlikely to deviate significantly from the `true' APWP, and
most comparison experiments use a fixed hotspot model (FHM) reference path for
North America (Fig.~\ref{fig-mhsPred}), India (Fig.~\ref{fig-mhsPred501}) and
Australia (Fig.~\ref{fig-mhsPred801}). However, the full set of comparisons for
the 28 picking methods and 6 weighting methods was also run for reference paths
generated using the moving hotspot model (MHM) rotations of~\citet{O05}, which
incorporate motions of the Indo-Atlantic hotspots relative to the mantle derived
from mantle convection modeling.

\begin{figure}[!ht]
\centering
\includegraphics[width=1.01\textwidth]{../../paper/tex/GeophysJInt/figures/fhs501_m.pdf}
\caption[120\textendash0 Ma MHM vs FHM predicted APWP of India]{MHM predicted
120\textendash0 Ma APWP (solid line) for India through the
India\textendash{}Somalia\textendash{}Nubia\textendash{}Mantle plate circuit.
Its age step is 5 Myr. The dashed line is the FHM predicted path shown for
comparison. The inset shows paths for fast moving India and also much slower
moving North America shown in Fig.~\ref{fig-mhsPred}.}\label{fig-mhsPred501}
\end{figure}

When comparing the synthetic APW paths for the three plates (inset,
Fig.~\ref{fig-mhsPred801}), there are clear differences. The predicted mean
north pole for North America at 120 Myr is still at ${\sim}75^{\circ}$N
(Fig.~\ref{fig-mhsPred}), indicating rather slow drift with respect to the spin
axis; this is due to a large component of the North American plate's absolute
motion in the past 120 Myr being to the east. In contrast, the rapid northward
motion of the Indian plate in the same period, particularly prior to its
collision with Asia at ${\sim}50$\textendash55 Ma~\citep{N10}, is reflected by
the 120 Ma predicted mean north pole being located at ${\sim}20^{\circ}$N
(Fig.~\ref{fig-mhsPred501}). Australia represents an intermediate case, with
north westerly plate motion from ${\sim}120$\textendash60 Myr changing to more
rapid northward motion from ${\sim}60$\textendash55 Ma to the present~\citep{W07}.
When comparing the FHM and MHM tracks, significant differences in the oldest
parts (before ${\sim}80$ Ma) are apparent for India and North America.

\begin{figure}[!ht]
\centering
\includegraphics[width=1.01\textwidth]{../../paper/tex/GeophysJInt/figures/mhs801.pdf}
\caption[120\textendash0 Ma MHM vs FHM predicted APWP of Australia]{MHM
predicted 120\textendash0 Ma APWP (solid line) for Australia through the
Australia\textendash{}East
Antarctica\textendash{}Somalia\textendash{}Nubia\textendash{}Mantle plate
circuit. Its age step is 5 Myr. The dashed line is the FHM predicted path shown
for comparison. The inset shows paths for fast moving India shown in
Fig.~\ref{fig-mhsPred501}, much slower moving North America shown in
Fig.~\ref{fig-mhsPred}, and also relatively intermediate moving
Australia.}\label{fig-mhsPred801}
\end{figure}

These differences in the reference path due to different plate kinematics is
another variable that may affect the performance of the different weighting
algorithm for different plates, in addition to the distribution and type of the
contributing mean poles used to generate the paleomagnetic APWPs.

\subsection{Comparison Algorithm}

Comparisons between APWPs generated using different picking and weighting
algorithms and the synthetic reference APWPs were performed using the composite
path difference ($\mathcal{CPD}$) algorithm described in
Chapter~\ref{chap:Metho}, with equal weighting given to the spatial, length and
angular differences (i.e., W$_s$ = W$_l$ = W$_a$ = $\frac{1}{3}$). A lower
$\mathcal{CPD}$ score generally indicates a `better' fit, although a lower score
can also potentially result from comparison to a more poorly constrained path
with large uncertainties, which are less likely to be significantly different.
An additional `Fit Quality' (FQ) metric can help to distinguish such cases, by
assigning the two paths being compared a letter score based on the average size
of their uncertainty ellipses (Chapter~\ref{sec:FQ}). Here, the first letter
refers to the generated APWP, and the second letter refers to the reference
path. In this study, the reference path FQ score is fixed for each plate;
because the spatial uncertainties for paths generated from plate motion models
are small compared to those typical for paleomagnetic data, this reference path
FQ is always rated `A'.

This does not only help find the most similar paleomagnetic APWP (from the best
algorithm) to the reference APWP, but also help further test and demonstrate the
validity of the similarity measuring tool in practise.

\section{Results}

\subsection{Baseline results: 10 Myr window, 5 Myr step, fixed hotspot
reference}\label{sec:base}

Fig.~\ref{fig-dif} shows the $\mathcal{CPD}$ scores for the APWPs generated with
all 28 picking methods (AMP and APP with one of 14 separate filters applied) and
one of 6 weighted mean calculations then applied, compared to the FHM reference
path (squares and dashed line in Figs.~\ref{fig-mhsPred}-\ref{fig-mhsPred801})
for North America (Fig.~\ref{fig-na-dif}), India (Fig.~\ref{fig-in-dif}) and
Australia (Fig.~\ref{fig-au-dif}). The 27 lowest and highest of the 168
scores for each plate (values greater than 1 standard deviation from the mean)
are marked in green and red, respectively. Different combinations of windowing
method, filtering and weighting clearly affect the difference score, with
$\mathcal{CPD}$ values ranging from 0.0023 to 0.5137. The fits for paths with
low difference scores are clearly much better than for those with high ones
(Fig.~\ref{fig-difbw}). From Fig.~\ref{fig-dif}, it is clear that:

\begin{figure*}
  \vspace*{-1.1cm}
	\centering
	\begin{subfigure}{.94\textwidth}
		\includegraphics[width=\textwidth]{../../paper/tex/GeophysJInt/figures/101_120_0.pdf}
		\caption{North America: minimum 0.00573 (25(0)), maximum
		0.08601 (16(2)), mean 0.032403, median 0.032395}\label{fig-na-dif} % subcaption
	\end{subfigure}
	\vspace{.1em} % here you can insert horizontal or vertical space
	\begin{subfigure}{.94\textwidth}
		\includegraphics[width=\textwidth]{../../paper/tex/GeophysJInt/figures/501_120_0.pdf}
		\caption{India: minimum 0.023 (19(0)), maximum 0.5137 (8(3)),
		mean 0.1182, median 0.0835}\label{fig-in-dif} % subcaption
	\end{subfigure}
	\vspace{.1em}
	\begin{subfigure}{.94\textwidth}
		\includegraphics[width=\textwidth]{../../paper/tex/GeophysJInt/figures/801_120_0.pdf}
		\caption{Australia: minimum 0.00227 (11(0)), maximum
		0.3934 (26(3)), mean 0.08373, median 0.05}\label{fig-au-dif} % subcaption
	\end{subfigure}
	\caption[$\mathcal{CPD}$ of each plate's paleomagnetic APWPs vs its FHM
predicted APWP]{Equal-weight composite path difference ($\mathcal{CPD}$) values
between each continent's paleomagnetic APWPs and its predicted APWP from FHM and
related plate circuits. The paths are in 10 Myr bin and 5 Myr step. The
difference values less than one-standard-deviation interval of the whole 168
values (lower 15.866 per cent) are colored in green, more than
one-standard-deviation interval (upper 15.866 per cent) colored in red. Exactly
the same columns are connected. The percentages of removed paleopoles are
derived relative to Pk 1, corrected relative to each corresponding picking
method (Pk 8,9, 12,13; 1 paleopole removed and 1 corrected by Pk 20,21 for
India). Fit quality (FQ) for each score is color coded.}\label{fig-dif} % caption for whole figure
\end{figure*}

\begin{figure*}
	\centering
	\begin{subfigure}{.42\textwidth} % width of left subfigure
		\includegraphics[width=\textwidth]{../../paper/tex/GeophysJInt/figures/ay18_101comb_10_5_25_0.pdf}
		\caption{North America: minimum 0.00573 (25(0)), FQ
		B-A}\label{fig-nac-105250}
	\end{subfigure}
	\begin{subfigure}{.42\textwidth} % width of right subfigure
		\includegraphics[width=\textwidth]{../../paper/tex/GeophysJInt/figures/ay18_101comb_10_5_16_2.pdf}
		\caption{North America: maximum 0.08601 (16(2)), FQ
		B-A}\label{fig-nac-105162}
	\end{subfigure}
	\vspace{.1em}
	\begin{subfigure}{.42\textwidth}
		\includegraphics[width=\textwidth]{../../paper/tex/GeophysJInt/figures/ay18_501comb_10_5_19_0.pdf}
		\caption{India: minimum 0.023 (19(0)), FQ C-A}\label{fig-ind-105190}
	\end{subfigure}
	\begin{subfigure}{.42\textwidth}
		\includegraphics[width=\textwidth]{../../paper/tex/GeophysJInt/figures/ay18_501comb_10_5_8_3.pdf}
		\caption{India: maximum 0.5137 (8(3)), FQ C-A}\label{fig-ind-10583}
	\end{subfigure}
	\vspace{.1em}
	\begin{subfigure}{.42\textwidth}
		\includegraphics[width=\textwidth]{../../paper/tex/GeophysJInt/figures/ay18_801comb_10_5_11_0.pdf}
		\caption{Australia: minimum 0.00227 (11(0)), FQ C-A}\label{fig-au-105110}
	\end{subfigure}
	\begin{subfigure}{.42\textwidth}
		\includegraphics[width=\textwidth]{../../paper/tex/GeophysJInt/figures/ay18_801comb_10_5_26_3.pdf}
		\caption{Australia: maximum 0.3934 (26(3)), FQ C-A}\label{fig-au-105263}
	\end{subfigure}
	\caption[Best and worst differences (10 Myr bin, 5 Myr step)]{Path
comparisons with best and worst difference values shown in Fig.~\ref{fig-dif}.
The parenthetical remarks are Pk No and Wt No.}\label{fig-difbw}
\end{figure*}

\begin{enumerate}
  \item There is much more variation in scores along the horizontal axes than
	the vertical axes (see self-explanatory topography of bands in
	Fig.~\ref{fig-dif}), suggesting that the choice of windowing and filtering
	method (Table~\ref{tab-pick}) has a much greater impact than weighting
	(Table~\ref{tab-weit}).
  \item Many of the highest scores (worst fits) are associated with
	even-numbered picking and filtering methods, i.e., those which use the AMP
	windowing algorithm. Even so, Pk 4 and 6 are among the best methods for
	India.
  \item The magnitude and range of $\mathcal{CPD}$ scores for each of the three
	plates is different, with the North American plate having the lowest
	magnitudes and range (Fig.~\ref{fig-na-dif}), and the Indian plate having
	the highest (Fig.~\ref{fig-in-dif}). The scores for the Australian plate are
	generally closer to the equivalent scores for the North American plate than
	to the scores for the Indian plate (Fig.~\ref{fig-d-dif}).
  \item Although there is some overlap (e.g., Pk 19, 21 [best], and
	2, 16 [worst] for all the three plates or for both India and Australia; 1,
	11, 13, 19, 21, 25 [best], and 2, 14, 16, 22, 26 [worst] for both North
	America and Australia; 5, 7, 19, 21 [best], 2, 16, 18 [worst] for both North
	America and India), the best- and worst-performing picking/filtering and
	weighting algorithms are not exactly the same for each plate.
  \item Relating to FQ ratings: North America's lower $\mathcal{CPD}$ scores are
	also associated with consistently good FQ B-A grades; India's higher
	$\mathcal{CPD}$ scores are also associated with lower FQ (mostly C-A,
	including the no-filter methods, so there are some improvements in FQ with
	some filters), and Australia is a mix of B-A (including the no-filter
	methods) and C-A (which represent a reduction in FQ with some filters). In
	general, lower $\mathcal{CPD}$ scores also appear to be associated with
	better FQ (e.g. B-A rather than C-A).
\end{enumerate}

\subsection{Effects of windowing method}

Dividing $\mathcal{CPD}$ scores according to whether the AMP or APP windowing
method was used (Fig.~\ref{fig-difAMPvsAPP}) confirms that whilst the lowest
$\mathcal{CPD}$ scores for paths generated by the AMP windowing algorithm are
close to the lowest scores generated using the APP method, the highest scores
are much higher (Fig.~\ref{fig-boxAMPvsAPP}). The mean of the $\mathcal{CPD}$
scores for APWPs generated using AMP is greater than the maximum APP-derived
score for the Indian and Australian plates
(Figs.~\ref{fig-in-difAMPvsAPP},~\ref{fig-au-difAMPvsAPP},~\ref{fig-in-boxAMPvsAPP},~\ref{fig-au-boxAMPvsAPP}),
and more than 1 standard deviation greater than the APP mean for North American
APWPs (Figs.~\ref{fig-na-difAMPvsAPP},~\ref{fig-na-boxAMPvsAPP}).

\begin{figure*}
	\centering
	\begin{subfigure}{1.01\textwidth}
		\includegraphics[width=\textwidth]{../../paper/tex/GeophysJInt/figures/101_120_0APPvsAMP.pdf}
		\caption{AMP\@: minimum 0.0265 (4(1)), maximum 0.086 (16(2)), mean
		0.0451, median 0.0457; APP\@: minimum 0.0057 (25(0)), maximum
		0.05835 (17(5)), mean 0.0197, median 0.01494}\label{fig-na-difAMPvsAPP}
	\end{subfigure}
	\vspace{.1em}
	\begin{subfigure}{1.01\textwidth}
		\includegraphics[width=\textwidth]{../../paper/tex/GeophysJInt/figures/501_120_0APPvsAMP.pdf}
		\caption{AMP\@: minimum 0.047 (4(3)), maximum 0.5137 (8(3)), mean
		0.1688, median 0.1892; APP\@: minimum 0.023 (19(0)), maximum 0.1359
		(3(4)), mean 0.0676, median 0.0602}\label{fig-in-difAMPvsAPP}
	\end{subfigure}
	\vspace{.1em}
	\begin{subfigure}{1.01\textwidth}
		\includegraphics[width=\textwidth]{../../paper/tex/GeophysJInt/figures/801_120_0APPvsAMP.pdf}
		\caption{AMP\@: minimum 0.0286 (3(0)), maximum 0.3934 (26(3)), mean
		0.12675, median 0.0938; APP\@: minimum 0.00227 (11(0)), maximum 0.11445
		(3(2)), mean 0.0407, median 0.0256}\label{fig-au-difAMPvsAPP}
	\end{subfigure}
	\caption[$\mathcal{CPD}$ of each plate's paleomagnetic APWPs vs its FHM
predicted APWP (AMP vs APP)]{Separated results from AMP and APP in
Fig.~\ref{fig-dif}. For each grid block (left: AMP, right: APP), the difference
values less than one-standard-deviation interval of the whole 84 values are
labeled in green, more than one-standard-deviation interval labeled in
red.}\label{fig-difAMPvsAPP}
\end{figure*}

For each of the 84 possible combinations of filter method and weighting, the
AMP-derived score is typically 3\textendash5 times higher than the equivalent
APP-derived score (Fig.~\ref{fig-boxAMPvsAPP}, insets). APP-generated paths
yield a lower $\mathcal{CPD}$ score than the equivalent AMP-generated path for
82 (97.6\%) of the North America scores, 72 (85.7\%) of the India scores, and 84
(100\%) of the Australia scores. For the North American and Indian plates,
filtering that prefers igneous poles (Pk 4,5 and 6,7) or removes
poles with large various uncertainties (Pk 22,23 and 26,27) is most likely
to produce AMP scores close to (less than 1.5 times) or less than the APP scores
(Figs.~\ref{fig-na-difAMPvsAPP},~\ref{fig-in-difAMPvsAPP}). In the former case,
the scores are comparable and relatively low; in the latter case, they are
comparable but relatively high. For the Australian plate, only correcting for
sedimentary inclination shallowing (Pk 8,9) produces comparable but moderate
scores (Fig.~\ref{fig-au-difAMPvsAPP}).

\begin{figure*}
	\centering
	\begin{subfigure}{.9\textwidth}
		\includegraphics[width=\textwidth]{../../paper/tex/GeophysJInt/figures/101_10_5FHM.pdf}
		\caption{}\label{fig-na-boxAMPvsAPP}
	\end{subfigure}
	\vspace{.1em}
	\begin{subfigure}{.9\textwidth}
		\includegraphics[width=\textwidth]{../../paper/tex/GeophysJInt/figures/501_10_5FHM.pdf}
		\caption{}\label{fig-in-boxAMPvsAPP}
	\end{subfigure}
	\vspace{.1em}
	\begin{subfigure}{.9\textwidth}
		\includegraphics[width=\textwidth]{../../paper/tex/GeophysJInt/figures/801_10_5FHM.pdf}
		\caption{}\label{fig-au-boxAMPvsAPP}
	\end{subfigure}
	\caption[]{Box-and-whisker and cross (inset) plots of
Fig.~\ref{fig-difAMPvsAPP}. The $\mathcal{CPD}$s from same filter and weighting
method (red and blue dots plotted with box-and-whisker) are connected; some
special cases where $\mathcal{CPD}$ from AMP lower than from APP are highlighted
using darker connecting lines. Dot symbols are semi-transparent so a darker
color indicates a greater number of data at a given
$\mathcal{CPD}$.}\label{fig-boxAMPvsAPP}
\end{figure*}

On the North American plate, the APP-derived $\mathcal{CPD}$ score is most
likely to be significantly better (defined as more than 5 times higher) when
Wt 0 (no weighting) and 1 (by sample and site number) have been
applied (Fig.~\ref{fig-na-difAMPvsAPP}). This pattern is also seen for the
Australian plate, but removal of poles published after 1983 (Pk 16,17) also
results in significantly better performance of the APP method
(Fig.~\ref{fig-au-difAMPvsAPP}). For the Indian plate, the largest difference
occurs when poles with sedimentary inclination shallowing (Pk 8,9) are
corrected, or suspected overprints or local rotations are removed (Pk
18,19, Fig.~\ref{fig-in-difAMPvsAPP}).

\subsection{Effects of filtering and weighting}

When $\mathcal{CPD}$ scores are separated by windowing method
(Fig.~\ref{fig-difAMPvsAPP}), the effects of particular filtering and weighting
methods become easier to discern. In general, different filters (rows) produce
larger variations in scores than different weighting methods (columns). With
the exception of AMP-derived paths for India, the $\mathcal{CPD}$ score for
paths with no filtering (Pk 0,1) and no weighting (Wt
0) is lower than the mean scores for that plate and windowing method. Therefore
filtering and weighting at best slightly improves, and at worst significantly
degrades, the APWP fit to the reference path.

\subsubsection{Filter aggression}

When considering the effects of filtering, it is important to consider how many
paleopoles within the data set have been removed (related to Pk 2\textendash7,
10,11, and 14\textendash27) or corrected (Pk 8,9 and 12,13): if there is very
little alteration of the data set, little change from no filtering (Pk 0,1)
would be expected. In terms of the numbers of paleopoles affected, the most
consequential filters are:
%
\begin{enumerate}
  \item removal of sedimentary paleopoles (Pk 4,5 and 6,7), which removes
		${\sim}40$\textendash50\% of the datasets on all 3 plates, with the
		highest proportion being removed on the Indian plate. Although Pk 4,5
		are more strict, it does not remove many more paleopoles than Pk 6,7.
		For North America and India, the numbers of filtered paleopoles for Pk
		4,5 and 6,7 are actually the same.
  \item correction of sedimentary paleopoles for inclination flattening (filter
		in Pk 8,9), which affects 38\textendash48\% of the dataset, with the
		highest affected proportion on the Indian plate.
  \item removal of paleopoles with large temporal and spatial uncertainty (Pk
		2,3, 22,23), particularly for the Australian plate, where the SS05
		filtering criteria removes ${\sim}70$\% of the paleopoles. Filter in Pk
		26,27 combines Pk 22,23 and 24,25, but no or very few (in the only case
		of Australia only 1 additional paleopole) are actually removed.
  \item filtering based on publication date (Pk 14,15 and 16,17), with the
		ratio of pre/post 1983 poles varying from about 50/50 on the North
		American plate to about 70/30 on the Australian plate.
\end{enumerate}

Conversely, filtering or correction for redbeds (Pk 10,11 and 12,13), local
rotations and overprints (Pk 18,19 and 20,21%[only 1 paleopole influenced by local rotation removed, and one corrected, for only India; ${\sim}2.7$\%, Fig.~\ref{fig-in-dif}]
), or superseded data (Pk 24,25) affected 4\% or less of
the paleopoles on any plate.

Note that there is an important trade-off that precision in a mean pole is
determined by the number of contributing paleopoles (A95 is proportional to
1/$\sqrt{n}$), so an overly aggressive filter might improve accuracy at the
expense of precision. This is particularly an issue where data density is low
(see Section~\ref{sec:datden}).

\subsubsection{Filter performance}

Focussing on the filtering and weighting methods with aggressive filtering,
some commonalities in the best- and worst-performing methods can be observed,
although there are usually exceptions for particular plates and/or windowing
methods:
%
\begin{enumerate}
  \item For all 3 plates, higher $\mathcal{CPD}$ scores are commonly associated
		with filtering based on $\alpha$95 and age range (Pk 2,3, 22,23, 26,27),
		with the exception of AMP-derived paths for India, where Pk 22 and 26
		produce some of the lowest scores, and also improve FQ\@.
  \item For North America and India, low scores are commonly associated with
		removal of non-igneous poles (Pk 4,5 and 6,7), particularly for
		AMP-derived paths. On the Australian plate, these filters are less
		effective, and also reduce FQ\@.
  \item For North America and Australia, correction for inclination flattening
		generates $\mathcal{CPD}$ scores very similar to scores with no
		filtering for AMP-derived paths (Pk 8 versus Pk 0; for Australia FQ
		reduced), and increases scores for APP-derived paths (Pk 9 versus Pk 1).
		In contrast, for India generally there is a small decrease in
		$\mathcal{CPD}$ scores compared to no filtering for both AMP- and
		APP-derived paths.
  \item Many of the highest difference scores for North America and India occur
		when paleopoles published after 1983 are removed (Pk 16,17), whilst
		removing paleopoles published before 1983 (Pk 14,15) generates
		$\mathcal{CPD}$ scores comparable to scores with no filtering
		(relatively much lower). In contrast, for Australia Pk 14,15
		produces some of the highest $\mathcal{CPD}$ scores, and Pk 16,17
		have little effect on the $\mathcal{CPD}$ score, although FQ is
		commonly reduced. It is noteworthy that the number of older studies
		(65/68; Fig.~\ref{fig-au-dif}) is almost 2.5 times of the number of
		newer studies (27/29) for the Australian plate.
\end{enumerate}

Whilst it is generally true that methods with low-aggression filters do not
generate scores that differ much from the no-filtering scores, there are some
exceptions:
%
\begin{enumerate}
  \item For North America and Australia, removing superseded paleopoles (Pk
		24,25) produces lower $\mathcal{CPD}$ scores.
  \item Removing paleopoles suspected to be affected by overprints or local
		rotations (Pk 18,19) consistently produces lower
		$\mathcal{CPD}$ scores for Indian APP-derived paths.
\end{enumerate}

\begin{figure*}
	\centering
	\begin{subfigure}{1.01\textwidth}
		\includegraphics[width=\textwidth]{../../paper/tex/GeophysJInt/figures/f501d101.pdf}
		\caption{Differences between Fig.~\ref{fig-in-dif} (India) and
		Fig.~\ref{fig-na-dif} (North America)}\label{fig-i-n-dif}
	\end{subfigure}
	\vspace{.1em}
	\begin{subfigure}{1.01\textwidth}
		\includegraphics[width=\textwidth]{../../paper/tex/GeophysJInt/figures/f801d101.pdf}
		\caption{Differences between Fig.~\ref{fig-au-dif} (Australia) and
		Fig.~\ref{fig-na-dif} (North America)}\label{fig-a-n-dif}
	\end{subfigure}
	\vspace{.1em}
	\begin{subfigure}{1.01\textwidth}
		\includegraphics[width=\textwidth]{../../paper/tex/GeophysJInt/figures/f501d801.pdf}
		\caption{Differences between Fig.~\ref{fig-in-dif} (India) and
		Fig.~\ref{fig-au-dif} (Australia)}\label{fig-i-a-dif}
	\end{subfigure}
	\caption[Differences of $\mathcal{CPD}$ of each plate's paleomagnetic
APWPs vs its FHM predicted APWP]{Differences between grids in
Fig.~\ref{fig-dif}. The absolute difference values less than
1.96-standard-deviation interval of the whole 168 values are labeled in green,
more than 1.96-standard-deviation interval labeled in red.}\label{fig-d-dif}
\end{figure*}

\subsubsection{Weighting performance}

Compared to the variations resulting from different windowing methods and
filters, Figs.~\ref{fig-difAMPvsAPP},~\ref{fig-nads},~\ref{fig-nadl}
and~\ref{fig-nada} indicate that the effect of weighting the data prior to
calculating a Fisher mean is generally small or diluted. Where an effect can be
seen, it is negative, generating larger $\mathcal{CPD}$ scores.

\begin{figure}[!ht]
	\centering
	\begin{subfigure}{.495\textwidth}
		\includegraphics[width=\textwidth]{../../paper/tex/GeophysJInt/figures/101npoles105w0.pdf}
		\caption{For Wt 0}\label{fig-na-dsw0}
	\end{subfigure}
	\vspace{.1em}
	\begin{subfigure}{.495\textwidth}
		\includegraphics[width=\textwidth]{../../paper/tex/GeophysJInt/figures/101npoles105w1.pdf}
		\caption{For Wt 1}\label{fig-na-dsw1}
	\end{subfigure}
	\vspace{.1em}
	\begin{subfigure}{.495\textwidth}
		\includegraphics[width=\textwidth]{../../paper/tex/GeophysJInt/figures/101npoles105w2.pdf}
		\caption{For Wt 2}\label{fig-na-dsw2}
	\end{subfigure}
	\vspace{.1em}
	\begin{subfigure}{.495\textwidth}
		\includegraphics[width=\textwidth]{../../paper/tex/GeophysJInt/figures/101npoles105w3.pdf}
		\caption{For Wt 3}\label{fig-na-dsw3}
	\end{subfigure}
	\vspace{.1em}
	\begin{subfigure}{.495\textwidth}
		\includegraphics[width=\textwidth]{../../paper/tex/GeophysJInt/figures/101npoles105w4.pdf}
		\caption{For Wt 4}\label{fig-na-dsw4}
	\end{subfigure}
	\vspace{.1em}
	\begin{subfigure}{.495\textwidth}
		\includegraphics[width=\textwidth]{../../paper/tex/GeophysJInt/figures/101npoles105w5.pdf}
		\caption{For Wt 5}\label{fig-na-dsw5}
	\end{subfigure}
	\caption[$d_s$ of each pair of poles for North American 10/5 Myr APWPs]{Tested
spatial difference ($d_s$) values (color shaded) between North American
paleomagnetic APWPs and its predicted APWP from the FHM and related plate
circuits. The paths are in 10 Myr bin and 5 Myr step. The number labels on the
grids (including grid heights) are the numbers of site mean poles that are
contributing to make each mean path pole.}\label{fig-nads}
\end{figure}

\begin{figure}[!ht]
	\centering
	\begin{subfigure}{.495\textwidth}
		\includegraphics[width=\textwidth]{../../paper/tex/GeophysJInt/figures/101nsegs105w0.pdf}
		\caption{For Wt 0}\label{fig-na-dlw0}
	\end{subfigure}
	\vspace{.1em}
	\begin{subfigure}{.495\textwidth}
		\includegraphics[width=\textwidth]{../../paper/tex/GeophysJInt/figures/101nsegs105w1.pdf}
		\caption{For Wt 1}\label{fig-na-dlw1}
	\end{subfigure}
	\vspace{.1em}
	\begin{subfigure}{.495\textwidth}
		\includegraphics[width=\textwidth]{../../paper/tex/GeophysJInt/figures/101nsegs105w2.pdf}
		\caption{For Wt 2}\label{fig-na-dlw2}
	\end{subfigure}
	\vspace{.1em}
	\begin{subfigure}{.495\textwidth}
		\includegraphics[width=\textwidth]{../../paper/tex/GeophysJInt/figures/101nsegs105w3.pdf}
		\caption{For Wt 3}\label{fig-na-dlw3}
	\end{subfigure}
	\vspace{.1em}
	\begin{subfigure}{.495\textwidth}
		\includegraphics[width=\textwidth]{../../paper/tex/GeophysJInt/figures/101nsegs105w4.pdf}
		\caption{For Wt 4}\label{fig-na-dlw4}
	\end{subfigure}
	\vspace{.1em}
	\begin{subfigure}{.495\textwidth}
		\includegraphics[width=\textwidth]{../../paper/tex/GeophysJInt/figures/101nsegs105w5.pdf}
		\caption{For Wt 5}\label{fig-na-dlw5}
	\end{subfigure}
	\caption[$d_l$ of each pair of segments for North American 10/5 Myr
APWPs]{Tested length difference ($d_l$) values (color shaded) between North
American paleomagnetic APWPs and its predicted APWP from FHM and related plate
circuits. The paths are in 10 Myr bin and 5 Myr step. The labeled numbers on the
grids are the averaged numbers of site mean poles that are contributing to each
segment's two mean path poles.}\label{fig-nadl}
\end{figure}

\begin{figure}[!ht]
	\centering
	\begin{subfigure}{.495\textwidth}
		\includegraphics[width=\textwidth]{../../paper/tex/GeophysJInt/figures/101nocs105w0.pdf}
		\caption{For Wt 0}\label{fig-na-daw0}
	\end{subfigure}
	\vspace{.1em}
	\begin{subfigure}{.495\textwidth}
		\includegraphics[width=\textwidth]{../../paper/tex/GeophysJInt/figures/101nocs105w1.pdf}
		\caption{For Wt 1}\label{fig-na-daw1}
	\end{subfigure}
	\vspace{.1em}
	\begin{subfigure}{.495\textwidth}
		\includegraphics[width=\textwidth]{../../paper/tex/GeophysJInt/figures/101nocs105w2.pdf}
		\caption{For Wt 2}\label{fig-na-daw2}
	\end{subfigure}
	\vspace{.1em}
	\begin{subfigure}{.495\textwidth}
		\includegraphics[width=\textwidth]{../../paper/tex/GeophysJInt/figures/101nocs105w3.pdf}
		\caption{For Wt 3}\label{fig-na-daw3}
	\end{subfigure}
	\vspace{.1em}
	\begin{subfigure}{.495\textwidth}
		\includegraphics[width=\textwidth]{../../paper/tex/GeophysJInt/figures/101nocs105w4.pdf}
		\caption{For Wt 4}\label{fig-na-daw4}
	\end{subfigure}
	\vspace{.1em}
	\begin{subfigure}{.495\textwidth}
		\includegraphics[width=\textwidth]{../../paper/tex/GeophysJInt/figures/101nocs105w5.pdf}
		\caption{For Wt 5}\label{fig-na-daw5}
	\end{subfigure}
	\caption[$d_a$ of each pair of segment-oreintation-changes for North American
10/5 Myr APWPs]{Tested angular difference ($d_a$) values (color shaded) between
North American paleomagnetic APWPs and its predicted APWP from FHM and related
plate circuits. The paths are in 10 Myr bin and 5 Myr step. The labeled numbers
on the grids are the averaged numbers of site mean poles that are contributing
to each segment-orientation-change's three mean path poles.}\label{fig-nada}
\end{figure}
%
\begin{enumerate}
  \item For APP-derived paths from the North American and Australian plates,
		Wt 0 (no weighting) and 1 (weighting by sample and site
		number) usually produce slightly better $\mathcal{CPD}$ scores than
		other weighting methods.
  \item Wt 3 (weighting by spatial uncertainty) seems most likely
		to generate much higher $\mathcal{CPD}$ scores, particularly for
		AMP-derived paths, and particular for the Indian plate.
  \item Wt 0, 1 and 5 generally produce lower similarity scores than Wt 2, 3 and
		4.
\end{enumerate}

\subsection{Effects of window size}

Fig.~\ref{fig-dif2010} shows the $\mathcal{CPD}$ scores for the APWPs generated
with the 28 picking and 6 weighting methods, compared to the FHM reference
paths, with the picking time window width increased from 10 to 20 Myr, and the
window step increased from 5 to 10 Myr.

\begin{figure*}
	\centering
	\begin{subfigure}{.96\textwidth}
		\includegraphics[width=\textwidth]{../../paper/tex/GeophysJInt/figures/101_20_10_120_0.pdf}
		\caption{North America: minimum 0.00991 (15(0)),
		maximum 0.0654 (14(2)), mean 0.0339, median 0.0296371}\label{fig-na-dif2010}
	\end{subfigure}
	\vspace{.1em}
	\begin{subfigure}{.96\textwidth}
		\includegraphics[width=\textwidth]{../../paper/tex/GeophysJInt/figures/501_20_10_120_0.pdf}
		\caption{India: minimum 0.0142822 (6(3)), maximum 0.154278 (16(3)),
		mean 0.07384, median 0.072088}\label{fig-in-dif2010}
	\end{subfigure}
	\vspace{.1em}
	\begin{subfigure}{.96\textwidth}
		\includegraphics[width=\textwidth]{../../paper/tex/GeophysJInt/figures/801_20_10_120_0.pdf}
		\caption{Australia: minimum 0 (11(0,1),19(3)), maximum
		0.324438 (22(3)), mean 0.0682, median 0.036432}\label{fig-au-dif2010}
	\end{subfigure}
	\caption[$\mathcal{CPD}$ of each plate's paleomagnetic APWPs vs its FHM
predicted APWP (20/10 Myr bin/step)]{As Fig.~\ref{fig-dif}, here the paths are
generated in 20 Myr bin and 10 Myr step. The difference values less than
one-standard-deviation interval of the whole 168 values are colored in green,
more than one-standard-deviation interval colored in red. Compare the numbers of
picked paleopoles with those in Fig.~\ref{fig-dif}.}\label{fig-dif2010}
\end{figure*}

\begin{figure*}
	\centering
	\begin{subfigure}{.43\textwidth}
		\includegraphics[width=\textwidth]{../../paper/tex/GeophysJInt/figures/ay18_101comb_20_10_15_0.pdf}
		\caption{North America: minimum 0.00991 (15(0)), FQ B-A}
	\end{subfigure}
	\begin{subfigure}{.43\textwidth}
		\includegraphics[width=\textwidth]{../../paper/tex/GeophysJInt/figures/ay18_101comb_20_10_14_2.pdf}
		\caption{North America: maximum 0.0654 (14(2)), FQ B-A}
	\end{subfigure}
	\vspace{.1em}
	\begin{subfigure}{.43\textwidth}
		\includegraphics[width=\textwidth]{../../paper/tex/GeophysJInt/figures/ay18_501comb_20_10_6_3.pdf}
		\caption{India: minimum 0.0143 (6(3)), FQ C-A}
	\end{subfigure}
	\begin{subfigure}{.43\textwidth}
		\includegraphics[width=\textwidth]{../../paper/tex/GeophysJInt/figures/ay18_501comb_20_10_16_3.pdf}
		\caption{India: maximum 0.1543 (16(3)), FQ C-A}
	\end{subfigure}
	\vspace{.1em}
	\begin{subfigure}{.43\textwidth}
		\includegraphics[width=\textwidth]{../../paper/tex/GeophysJInt/figures/ay18_801comb_20_10_11_0.pdf}
		\caption{Australia: minimum 0 (11(0)), FQ B-A}\label{fig-au-2010110}
	\end{subfigure}
	\begin{subfigure}{.43\textwidth}
		\includegraphics[width=\textwidth]{../../paper/tex/GeophysJInt/figures/ay18_801comb_20_10_22_3.pdf}
		\caption{Australia: maximum 0.3244 (22(3)), FQ C-A}\label{fig-au-2010223}
	\end{subfigure}
	\caption[Best and worst differences (20 Myr bin, 10 Myr step)]{Path
comparisons with best and worst difference values shown in
Fig.~\ref{fig-dif2010}. The parenthetical remarks are Pk No and Wt
No.}\label{fig-dif2010bw}
\end{figure*}

\subsubsection{Overall change}

The overall effect of increased window and step size varies between plates
(Fig.~\ref{fig-f105d2010}). 118/168 (${\sim}70$\%) of equivalent $\mathcal{CPD}$
scores for the North American plate increase, indicating reduced similarity with
respect to the reference path (Fig.~\ref{fig-101f105d2010}). The increased
occurrence of A-A rather than B-A FQ ratings (Fig.~\ref{fig-na-dif2010} versus
Fig.\ref{fig-na-dif}) indicate that in some cases this may be due to a better
constrained path becoming more distinguishable. Decreased scores are confined to
particular, mostly AMP-derived picking methods: Pk 2, 22 and 26 (filtering based
on ${\alpha}95$ and age uncertainties), 4 and 6 (removal of sedimentary
paleopoles), and 16,17 (removal of post–1983 results). It is clearly due to the
corresponding increase of N in each increased window for these AMP-derived
quality filters.

For the Indian and Australian plates, there is a trend towards decreased
$\mathcal{CPD}$ scores, with 153/168 (${\sim}91$\%) and 104/168 (${\sim}62$\%)
of equivalent $\mathcal{CPD}$ scores being lower than for the 10/5 Myr
window/step, respectively (Fig.~\ref{fig-501f105d2010},~\ref{fig-801f105d2010}).
Where the score increases, the actual difference is close to 0, indicating very
little change, with the exception of Pk 14 (AMP, removal of pre–1983 results) on
the Australian plate. The largest score decreases are associated with Wt 3 (AMP,
${\alpha}95$ and age filtering) on the Indian plate, and Pk 15 (APP, removal of
pre–1983 results) on the Australian plate. Most FQ ratings remain unchanged
(Fig.~\ref{fig-in-dif2010},~\ref{fig-au-dif2010}): most obviously, ratings for
Pk 5 and 7 (APP, removal of sedimentary poles) on the Indian plate and Pk 14 on
the Australian plate change from B-A to C-A.

\begin{figure*}
	\centering
	\begin{subfigure}{1\textwidth}
		\includegraphics[width=\textwidth]{../../paper/tex/GeophysJInt/figures/101f105d2010.pdf}
		\caption{North America: Generally Bin/Step 10/5 Myr is better
		($\frac{59}{84}\approx70.24\%$)}\label{fig-101f105d2010}
	\end{subfigure}
	\vspace{.1em}
	\begin{subfigure}{1\textwidth}
		\includegraphics[width=\textwidth]{../../paper/tex/GeophysJInt/figures/501f105d2010.pdf}
		\caption{India: Generally Bin/Step 20/10 Myr is better
		($\frac{153}{168}\approx91.07\%$)}\label{fig-501f105d2010}
	\end{subfigure}
	\vspace{.1em}
	\begin{subfigure}{1\textwidth}
		\includegraphics[width=\textwidth]{../../paper/tex/GeophysJInt/figures/801f105d2010.pdf}
		\caption{Australia: Generally Bin/Step 20/10 Myr is better
		($\frac{13}{21}\approx61.9\%$)}\label{fig-801f105d2010}
	\end{subfigure}
	\caption[]{Differences between grids in Fig.~\ref{fig-dif} (10 Myr bin, 5
Myr step) and Fig.~\ref{fig-dif2010} (20 Myr bin, 10 Myr step). The absolute
difference values less than 1.96-standard-deviation interval of the whole 168
values are labeled in green, more than 1.96-standard-deviation interval labeled
in red.}\label{fig-f105d2010}
\end{figure*}

\subsubsection{Relative performance of methods}

As for the baseline results (Section~\ref{sec:base}, Fig.~\ref{fig-dif}), the
effects of windowing and filtering, particularly the windowing method (APP
versus AMP), are much more apparent than the effects of weighting. APP-derived
scores still outperform AMP-derived ones, but the overall degree of difference
is reduced for the North American and Indian plates, with percentage of
APP-derived $\mathcal{CPD}$ scores more than 3 times the equivalent AMP-derived
score reducing to ${\sim}26$\% and ${\sim}4$\% (from ${\sim}44$\% and
${\sim}40$\%), respectively. For the Australian plate, this percentage increases
from ${\sim}58$\% to ${\sim}70$\%.

This is largely the result of larger changes in the AMP-derived $\mathcal{CPD}$
scores compared to changes in the APP-derived ones with the wider time window.
The scores for APP-derived methods Pk 1 (no filtering), other non-aggressive
filters (Pk 13, 19, 21, 25), and Pk 9 (correction for inclination flattening),
are particularly stable (Fig.~\ref{fig-f105d2010}) for all three plates.
Filtering according to the SS05 criteria (Pk 23 and 27) also yields comparable
scores for both time windows on the North American and Indian plates.

As for the 10/5 window, the lowest AMP/APP differences are for Pk 4,5 and 6,7
(igneous poles preferred \textendash{} low scores) and Pk 2,3 and 22,23
(filtering for spatial and temporal uncertainty \textendash{} high scores) for
the North American and Indian plates; and 8,9 (correction for inclination
shallowing \textendash{} moderate scores) for the Australian plates. Likewise,
the highest AMP/APP difference is still associated with Pk 0,1 (no filtering) on
the North American and Australian plates, and Pk 16,17 on the Australian plate.

Overall, the relative performance of the different filtering methods when using
the 20/10 window/step remains similar to that observed for the 10/5 window/step
(Figs.~\ref{fig-dif},~\ref{fig-dif2010}). Some of the lowest scores are still
produced by Pk 1 (APP, no filtering). Of the aggressive filters, removing
sedimentary paleopoles (Pk 4,5 and 6,7) still perform well, selecting only
spatially and temporally well-constrained paleopoles (Pk 2,3 and 22,23) still
perform relatively poorly, and correction for inclination flattening (Pk 8,9)
has little effect.

In addition, using only paleopoles published before 1983 (Pk 16,17) still
produces relatively high $\mathcal{CPD}$ scores for the North American and
Indian plates, although the scores for the AMP-derived Pk 16 are substantially
reduced with the wider time window (Figs.~\ref{fig-101f105d2010},~\ref{fig-501f105d2010}).
Likewise, for Australia, using only paleopoles published after 1983 (Pk 14,15)
still produces higher scores than Pk 16,17: scores for the wider time window are
even higher for the AMP-derived Pk 14, but substantially lower for the
APP-derived Pk 15 (Fig.~\ref{fig-801f105d2010}).

\begin{landscape}
\begin{table}[]
\centering
\caption{Consistency check on comparisons of picking methods' performance between 20/10 and 10/5 Myr window/step. Notes: E means expected; UE means unexpected.}\label{tab-2010vs105F}
\resizebox{.9\width}{!}{%
\begin{tabular}{@{}cccllcllcccccl@{}}
\toprule
\multicolumn{2}{c}{Comparisons} & \multicolumn{3}{c}{Consistency of best} & \multicolumn{3}{c}{Consistency of worst} & \multicolumn{6}{c}{If CPD values for 20/10 are lower (Y/N)} \\ \midrule
10/5 & 20/10 & Y/N & \multicolumn{1}{c}{Special case} & Same Pk & Y/N & \multicolumn{1}{c}{Special case} & \multicolumn{1}{c}{Same Pk} & Mean & Median & Max & Min & All & If N, UE case \\ \midrule
\multicolumn{14}{c}{FHM} \\ \midrule
\multirow{4}{*}{\begin{tabular}[c]{@{}c@{}}Fig.\\\ref{fig-na-dif}\end{tabular}} & \multirow{4}{*}{\begin{tabular}[c]{@{}c@{}}Fig.\\\ref{fig-na-dif2010}\end{tabular}} & \multirow{4}{*}{Y} & \multirow{4}{*}{\begin{tabular}[c]{@{}l@{}}3 more best:\\ Pk 4, 6, 9\\ only for\\ 20/10 (E)\end{tabular}} & \multirow{4}{*}{\begin{tabular}[c]{@{}l@{}}1, 5, 7,\\ 11, 13,\\ 15, 19,\\ 21, 25\end{tabular}} & \multirow{4}{*}{N} & \multirow{4}{*}{\begin{tabular}[c]{@{}l@{}}Pk 2, 5, 7, 17, 22,\\ 26 for 10/5 (E);\\ 0, 8, 10, 12, 20,\\ 24 for 20/10 (UE)\end{tabular}} & \multirow{4}{*}{\begin{tabular}[c]{@{}l@{}}14,\\ 16,\\ 18\end{tabular}} & \multirow{4}{*}{N} & \multirow{4}{*}{Y} & \multirow{4}{*}{Y} & \multirow{4}{*}{N} & \multirow{4}{*}{N} & \multirow{4}{*}{\begin{tabular}[c]{@{}l@{}}Positive\\ values\\ in\\ Fig.~\ref{fig-101f105d2010}\end{tabular}} \\
 &  &  &  &  &  &  &  &  &  &  &  &  &  \\
 &  &  &  &  &  &  &  &  &  &  &  &  &  \\
 &  &  &  &  &  &  &  &  &  &  &  &  &  \\ \midrule
\multirow{4}{*}{\begin{tabular}[c]{@{}c@{}}Fig.\\\ref{fig-in-dif}\end{tabular}} & \multirow{4}{*}{\begin{tabular}[c]{@{}c@{}}Fig.\\\ref{fig-in-dif2010}\end{tabular}} & \multirow{4}{*}{Y} & \multirow{4}{*}{\begin{tabular}[c]{@{}l@{}}4 more best:\\ Pk 9, 21, 22,\\ 26 only for\\ 10/5 (UE)\end{tabular}} & \multirow{4}{*}{\begin{tabular}[c]{@{}l@{}}4\textendash7,\\ 19\end{tabular}} & \multirow{4}{*}{N} & \multirow{4}{*}{\begin{tabular}[c]{@{}l@{}}Pk 8 for\\ 10/5 (E);\\ 23, 27 for\\ 20/10 (UE)\end{tabular}} & \multirow{4}{*}{\begin{tabular}[c]{@{}l@{}}0, 2,\\ 10, 12,\\ 16, 18,\\ 20, 24\end{tabular}} & \multirow{4}{*}{Y} & \multirow{4}{*}{Y} & \multirow{4}{*}{Y} & \multirow{4}{*}{Y} & \multirow{4}{*}{N} & \multirow{4}{*}{\begin{tabular}[c]{@{}l@{}}Positive\\ values\\ in\\ Fig.~\ref{fig-501f105d2010}\end{tabular}} \\
 &  &  &  &  &  &  &  &  &  &  &  &  &  \\
 &  &  &  &  &  &  &  &  &  &  &  &  &  \\
 &  &  &  &  &  &  &  &  &  &  &  &  &  \\ \midrule
\multirow{3}{*}{\begin{tabular}[c]{@{}c@{}}Fig.\\\ref{fig-au-dif}\end{tabular}} & \multirow{3}{*}{\begin{tabular}[c]{@{}c@{}}Fig.\\\ref{fig-au-dif2010}\end{tabular}} & \multirow{3}{*}{Y} & \multirow{3}{*}{\begin{tabular}[c]{@{}l@{}}2 more best:\\ Pk 23, 27 only\\ for 20/10 (E)\end{tabular}} & \multirow{3}{*}{\begin{tabular}[c]{@{}l@{}}1, 11, 13,\\ 17, 19,\\ 21, 25\end{tabular}} & \multirow{3}{*}{Y} & \multirow{3}{*}{\begin{tabular}[c]{@{}l@{}}1 more worst:\\ Pk 4 only for\\ 10/5 (E)\end{tabular}} & \multirow{3}{*}{\begin{tabular}[c]{@{}l@{}}2, 14,\\ 16, 22,\\ 26\end{tabular}} & \multirow{3}{*}{Y} & \multirow{3}{*}{Y} & \multirow{3}{*}{Y} & \multirow{3}{*}{Y} & \multirow{3}{*}{N} & \multirow{3}{*}{\begin{tabular}[c]{@{}l@{}}Positive\\ values in\\ Fig.~\ref{fig-801f105d2010}\end{tabular}} \\
 &  &  &  &  &  &  &  &  &  &  &  &  &  \\
 &  &  &  &  &  &  &  &  &  &  &  &  &  \\ \midrule
\multicolumn{14}{c}{MHM} \\ \midrule
\multirow{5}{*}{\begin{tabular}[c]{@{}c@{}}Fig.\\\ref{fig-na-difm}\end{tabular}} & \multirow{5}{*}{\begin{tabular}[c]{@{}c@{}}Fig.\\\ref{fig-na-dif2010m}\end{tabular}} & \multirow{5}{*}{N} & \multirow{5}{*}{\begin{tabular}[c]{@{}l@{}}Pk 1, 9, 11,\\ 13, 19, 21,\\ 25 for 10/5\\ (UE); 22, 26\\ for 20/10 (E)\end{tabular}} & \multirow{5}{*}{\begin{tabular}[c]{@{}l@{}}5,\\ 7,\\ 15\end{tabular}} & \multirow{5}{*}{N} & \multirow{5}{*}{\begin{tabular}[c]{@{}l@{}}Pk 16,17\\ only for 10/5\\ (E); 8 only\\ for 20/10\\ (UE)\end{tabular}} & \multirow{5}{*}{\begin{tabular}[c]{@{}l@{}}0, 10\\ 12, 14,\\ 18,\\ 20,\\ 24\end{tabular}} & \multirow{5}{*}{N} & \multirow{5}{*}{Y} & \multirow{5}{*}{N} & \multirow{5}{*}{N} & \multirow{5}{*}{N} & \multirow{5}{*}{\begin{tabular}[c]{@{}l@{}}(0,8,10,11,12,14,18,20,24,25)(0\textendash5)\\ (1,13,19,21)(0,1,5)  3(0,3,5)\\ (5,7)(0\textendash2,4,5)  15(0\textendash4)\\ (9,23,27)(0,1,3,5)\\ account for 60.71\%\end{tabular}} \\
 &  &  &  &  &  &  &  &  &  &  &  &  &  \\
 &  &  &  &  &  &  &  &  &  &  &  &  &  \\
 &  &  &  &  &  &  &  &  &  &  &  &  &  \\
 &  &  &  &  &  &  &  &  &  &  &  &  &  \\ \midrule
\multirow{5}{*}{\begin{tabular}[c]{@{}c@{}}Fig.\\\ref{fig-in-difm}\end{tabular}} & \multirow{5}{*}{\begin{tabular}[c]{@{}c@{}}Fig.\\\ref{fig-in-dif2010m}\end{tabular}} & \multirow{5}{*}{N} & \multirow{5}{*}{\begin{tabular}[c]{@{}l@{}}Pk 19, 21\\ only for 10/5\\ (UE); 4, 6\\ only for\\ 20/10 (E)\end{tabular}} & \multirow{5}{*}{\begin{tabular}[c]{@{}l@{}}5,\\ 7,\\ 22,\\ 26\end{tabular}} & \multirow{5}{*}{Y} & \multirow{5}{*}{\begin{tabular}[c]{@{}l@{}}3 more\\ worst: Pk\\ 8, 14, 20\\ for 10/5\\ (E)\end{tabular}} & \multirow{5}{*}{\begin{tabular}[c]{@{}l@{}}0, 2,\\ 10, 12,\\ 16,\\ 18,\\ 24\end{tabular}} & \multirow{5}{*}{Y} & \multirow{5}{*}{Y} & \multirow{5}{*}{Y} & \multirow{5}{*}{Y} & \multirow{5}{*}{N} & \multirow{5}{*}{\begin{tabular}[c]{@{}l@{}}(1,25)(0,3,5)  15(3)\\ (19,23,27)(0\textendash5)\\ 21(0,1,3\textendash5)\\ (22,26)(4);\\ account for 19.05\%\end{tabular}} \\
 &  &  &  &  &  &  &  &  &  &  &  &  &  \\
 &  &  &  &  &  &  &  &  &  &  &  &  &  \\
 &  &  &  &  &  &  &  &  &  &  &  &  &  \\
 &  &  &  &  &  &  &  &  &  &  &  &  &  \\ \midrule
\multirow{5}{*}{\begin{tabular}[c]{@{}c@{}}Fig.\\\ref{fig-au-difm}\end{tabular}} & \multirow{5}{*}{\begin{tabular}[c]{@{}c@{}}Fig.\\\ref{fig-au-dif2010m}\end{tabular}} & \multirow{5}{*}{Y} & \multirow{5}{*}{\begin{tabular}[c]{@{}l@{}}3 more best:\\ Pk 15, 23,\\ 27 only\\ for 20/10\\ (E)\end{tabular}} & \multirow{5}{*}{\begin{tabular}[c]{@{}l@{}}1, 11,\\ 13, 17,\\ 19,\\ 21,\\ 25\end{tabular}} & \multirow{5}{*}{Y} & \multirow{5}{*}{\textit{None}} & \multirow{5}{*}{\begin{tabular}[c]{@{}l@{}}2,\\ 14,\\ 16,\\ 22,\\ 26\end{tabular}} & \multirow{5}{*}{Y} & \multirow{5}{*}{Y} & \multirow{5}{*}{Y} & \multirow{5}{*}{N} & \multirow{5}{*}{N} & \multirow{5}{*}{\begin{tabular}[c]{@{}l@{}}(0,20)(5)  (5,7)(3)\\ (1,11,13,16,18,19,21,25)(0,1,3,5)\\ (8,17)(0\textendash3,5)  9(0,1)  10(1,5)\\ 12(2)  14(2,4,5)  (22,26)(0\textendash2,4,5)\\ 24(1,2,5); account for 39.88\%\end{tabular}} \\
 &  &  &  &  &  &  &  &  &  &  &  &  &  \\
 &  &  &  &  &  &  &  &  &  &  &  &  &  \\
 &  &  &  &  &  &  &  &  &  &  &  &  &  \\
 &  &  &  &  &  &  &  &  &  &  &  &  &  \\ \bottomrule
\end{tabular}%
}
\end{table}
\end{landscape}

\subsubsection{Other window sizes and steps}

\paragraph{What to expect is}
the difference values for larger window/step size should be generally lower than
those for smaller window/step size, which further could result in more best
methods and less worst methods.

\begin{landscape}
\begin{table*}
\centering
\caption{Equal-weight 120\textendash0 Ma $\mathcal{CPD}$s for the three
representative continents' paleomagnetic APWPs compared with their FHM predicted
APWPs. The best are in dark green and underlined, second best in green and third
in light green.}\label{tab-pk0vs1bs}
\resizebox{.73\width}{!}{%
\begin{tabular}{|l|l|l|l|l|l|l|l|l|l|l|l|l|}
\hline
\multicolumn{1}{|c|}{} & \multicolumn{6}{c|}{N America Pk 0} &
  \multicolumn{6}{c|}{N America Pk 1} \\ \cline{2-13} 
\multicolumn{1}{|c|}{\multirow{-2}{*}{Window/Step size (Myr)}} & Wt 0 & Wt 1 & Wt 2 & Wt 3 & Wt 4 & Wt 5 & Wt 0 & Wt 1 & Wt 2 & Wt 3 & Wt 4 & Wt 5 \\ \hline
2/1 & 0.2864 & 0.2925 & 0.28685 & 0.280944 & 0.287 & 0.29324 &
  {\color[HTML]{34FF34} \textbf{0.005759}} & 0.0091586 & {\color[HTML]{32CB00}
  \textbf{0.01528}} & {\color[HTML]{009901} {\ul\textbf{0.0090683}}} & 0.03601
  & {\color[HTML]{32CB00} \textbf{0.009034}} \\ \hline
4/2 & 0.10584 & 0.09228 & 0.10938 & 0.11301 & 0.1131 & 0.11263 &
  {\color[HTML]{009901} {\ul\textbf{0.00525}}} &
  0.007665 & 0.01701 & {\color[HTML]{32CB00} \textbf{0.009612}} & 0.03504 & {\color[HTML]{009901} {\ul\textbf{0.007863}}} \\ \hline
6/3 & 0.06486 & 0.064078 & 0.077426 & 0.073621 & 0.082867 & 0.075522 &
  {\color[HTML]{32CB00} \textbf{0.005418}} & {\color[HTML]{34FF34}
  \textbf{0.007522}} & 0.017936 & {\color[HTML]{34FF34} \textbf{0.0097317}} &
  {\color[HTML]{34FF34} \textbf{0.01983}} & 0.0095782 \\ \hline
8/4 & 0.06934 & 0.07168 & 0.082474 & 0.0707846 & 0.11473 & 0.071686 &
  0.006 & {\color[HTML]{32CB00} \textbf{0.006217}} & {\color[HTML]{34FF34}
  \textbf{0.01556}} & 0.01242 & 0.02163 & 0.011483 \\ \hline
10/5 & 0.04412 & 0.04486 & 0.04739 & 0.04212 & 0.04753 & 0.04445 &
  0.00591 & {\color[HTML]{009901} {\ul\textbf{0.00616}}} & {\color[HTML]{009901} {\ul\textbf{0.01262}}} & 0.01208 &
  {\color[HTML]{009901} {\ul\textbf{0.01529}}} & {\color[HTML]{34FF34} \textbf{0.00941}} \\ \hline
12/6 & {\color[HTML]{32CB00} \textbf{0.03271}} & {\color[HTML]{32CB00}
  \textbf{0.02987}} & {\color[HTML]{009901} {\ul\textbf{0.03024}}} &
  {\color[HTML]{009901} {\ul\textbf{0.0276143}}} & {\color[HTML]{009901}
  {\ul\textbf{0.03012}}} & {\color[HTML]{009901} {\ul\textbf{0.02988}}} & 0.008736
  & 0.00902 & 0.01715 & 0.015665 & {\color[HTML]{32CB00} \textbf{0.01642}} &
  0.01453 \\ \hline
14/7 (119\textendash0 Ma path) & 0.0367 & 0.03695 & 0.03498 & 0.0347 &
  {\color[HTML]{34FF34} \textbf{0.0319}} & 0.0352 & 0.0115 & 0.01198 & 0.023922
  & 0.01477 & 0.02205 & 0.012985 \\ \hline
16/8 & 0.052386 & 0.04923 & 0.04964 & 0.04653 & 0.047712 & 0.048523 & 0.01138
  & 0.0117 & 0.02177 & 0.015409 & 0.023068 &
  0.01616 \\ \hline
20/10 & 0.04972 & 0.05356 & 0.0562 & 0.05352 & 0.0542 & 0.04902 & 0.014 &
  0.01728 & 0.02896 & 0.0181 & 0.02702 & 0.0185 \\ \hline
24/12 & {\color[HTML]{009901} {\ul\textbf{0.031642}}} & {\color[HTML]{34FF34}
  \textbf{0.0342}} & {\color[HTML]{34FF34} \textbf{0.03273}} &
  {\color[HTML]{34FF34} \textbf{0.03469}} & {\color[HTML]{32CB00} \textbf{0.0314765}} & {\color[HTML]{32CB00} \textbf{0.031253}} & 0.0164 &
  0.016737 & 0.02521 & 0.01907 & 0.02397 & 0.01924 \\ \hline
30/15 & {\color[HTML]{34FF34} \textbf{0.0345}} & {\color[HTML]{009901} {\ul\textbf{0.0298}}} & {\color[HTML]{32CB00} \textbf{0.0317}} &
  {\color[HTML]{32CB00} \textbf{0.0307}} & 0.03402 & {\color[HTML]{34FF34}
  \textbf{0.0341}} & 0.0206 & 0.0211 & 0.0313 & 0.0272 & 0.0293 & 0.0277 \\ \hline
\end{tabular}%
}
\resizebox{.7\width}{!}{%
\begin{tabular}{|l|l|l|l|l|l|l|l|l|l|l|l|l|}
\hline
\multicolumn{1}{|c|}{} & \multicolumn{6}{c|}{India Pk 0} &
  \multicolumn{6}{c|}{India Pk 1} \\ \cline{2-13} 
\multicolumn{1}{|c|}{\multirow{-2}{*}{Window/Step size (Myr)}} & Wt 0 & Wt 1 & Wt 2 & Wt 3 & Wt 4 & Wt 5 & Wt 0 & Wt 1 & Wt 2 & Wt 3 & Wt 4 & Wt 5 \\ \hline
2/1 & 0.25838 & 0.258757 & 0.258701 & 0.275184 & 0.265442 & 0.258692 &
  0.05446 & 0.05639 & 0.06456 & 0.06075 & 0.0677596 & 0.0583844 \\ \hline
4/2 & 0.19345 & 0.20276 & 0.19344 & 0.24597 & 0.203264 & 0.19423 & 0.05588 &
  0.056691 & 0.066803 & 0.062283 & 0.067216 & 0.060121 \\ \hline
6/3 & 0.17396 & 0.17476 & 0.173975 & 0.229325 & 0.18514 & 0.175174 & 0.057887
  & 0.05882 & 0.0664754 & 0.0627534 & 0.06725 & 0.06083 \\ \hline
8/4 & 0.15309 & 0.14966 & 0.15321 & 0.17412 & 0.172154 & 0.15124 & 0.05761 &
  0.05848 & 0.06671 & 0.06139 & 0.0669 & 0.05941 \\ \hline
10/5 & 0.1839 & 0.1845 & 0.1909 & 0.2586 & 0.1972 & 0.1852 &
  0.0545 & 0.0554 & 0.0662 & 0.0589 & 0.066 & 0.06 \\ \hline
12/6 & 0.108924 & 0.1035 & {\color[HTML]{34FF34} \textbf{0.11205}} & 0.11831 &
  0.121497 & {\color[HTML]{34FF34} \textbf{0.10587}} & 0.059897 & 0.06068 &
  0.06993 & 0.064499 & 0.06951 & 0.062945 \\ \hline
14/7 (119\textendash0 Ma path) & 0.112537 & 0.112885 & 0.126554 & 0.139516 &
  0.132359 & 0.114195 & {\color[HTML]{32CB00} \textbf{0.04942}} & {\color[HTML]{34FF34} \textbf{0.0502588}} & 0.0579931 & 0.060018 & 0.0654112 & 0.0582519 \\ \hline
16/8 & {\color[HTML]{34FF34} \textbf{0.104461}} & 0.104463 & 0.121002 &
  0.110942 & 0.119599 & 0.118336 & {\color[HTML]{34FF34} \textbf{0.051735}} &
  0.052813 & {\color[HTML]{34FF34} \textbf{0.055188}} & {\color[HTML]{34FF34} \textbf{0.056389}} & 0.0574883 & 0.055042 \\ \hline
20/10 & 0.1052 & {\color[HTML]{34FF34} \textbf{0.1015}} & 0.1198 &
  {\color[HTML]{34FF34} \textbf{0.1096}} & {\color[HTML]{34FF34}
  \textbf{0.1174}} & 0.1166 & {\color[HTML]{009901} {\ul\textbf{0.0492}}} &
  {\color[HTML]{32CB00} \textbf{0.0501}} & 0.0585 & {\color[HTML]{009901} {\ul\textbf{0.053}}} & {\color[HTML]{009901} {\ul\textbf{0.0536}}} & {\color[HTML]{009901} {\ul\textbf{0.052}}} \\ \hline
24/12 & {\color[HTML]{009901} {\ul\textbf{0.053143}}} & {\color[HTML]{009901}
  {\ul\textbf{0.05356}}} & {\color[HTML]{009901} {\ul\textbf{0.056986}}} &
  {\color[HTML]{009901} {\ul\textbf{0.05747}}} & {\color[HTML]{009901} {\ul\textbf{0.05558}}} & {\color[HTML]{009901} {\ul\textbf{0.0553047}}} &
  0.051926 & {\color[HTML]{009901} {\ul\textbf{0.045995}}} & {\color[HTML]{009901} {\ul\textbf{0.04868}}} &
  {\color[HTML]{32CB00} \textbf{0.0557455}} & {\color[HTML]{34FF34} \textbf{0.05698}} & {\color[HTML]{34FF34} \textbf{0.05456}} \\ \hline
30/15 & {\color[HTML]{32CB00} \textbf{0.05617}} & {\color[HTML]{32CB00}
  \textbf{0.0754578}} & {\color[HTML]{32CB00} \textbf{0.0775947}} &
  {\color[HTML]{32CB00} \textbf{0.0575459}} & {\color[HTML]{32CB00}
  \textbf{0.0565421}} & {\color[HTML]{32CB00} \textbf{0.056635}} & 0.0523614 &
  0.0519862 & {\color[HTML]{32CB00} \textbf{0.054158}} & 0.0563985 &
  {\color[HTML]{32CB00} \textbf{0.0555998}} & {\color[HTML]{32CB00} \textbf{0.0543501}} \\ \hline
\end{tabular}%
}
\resizebox{.7\width}{!}{%
\begin{tabular}{|l|l|l|l|l|l|l|l|l|l|l|l|l|}
\hline
\multicolumn{1}{|c|}{} & \multicolumn{6}{c|}{Australia Pk 0} &
  \multicolumn{6}{c|}{Australia Pk 1} \\ \cline{2-13} 
\multicolumn{1}{|c|}{\multirow{-2}{*}{Window/Step size (Myr)}} & Wt 0 & Wt 1 & Wt 2 & Wt 3 & Wt 4 & Wt 5 & Wt 0 & Wt 1 & Wt 2 & Wt 3 & Wt 4 & Wt 5 \\ \hline
2/1 & 0.471554 & 0.529417 & 0.52834 & 0.563622 & 0.628921 & 0.55071 &
  0.00498465 & 0.0047458 & 0.0247455 & 0.0072776 & 0.035936 &
  {\color[HTML]{34FF34} \textbf{0.0039155}} \\ \hline
4/2 & 0.26822 & 0.29862 & 0.25881 & 0.276464 & 0.33741 & 0.268965 & 0.004543 &
  0.004578 & 0.023895 & 0.005417 & 0.032511 & {\color[HTML]{009901} {\ul\textbf{0.0037674}}} \\ \hline
6/3 & 0.090985 & 0.0947676 & 0.09448 & 0.08977 & 0.10083 & 0.091376 &
  {\color[HTML]{32CB00} \textbf{0.0040955}} & 0.004847 & 0.02199 &
  {\color[HTML]{34FF34} \textbf{0.0046074}} & 0.023489 & 0.0064495 \\ \hline
8/4 & 0.0870445 & 0.097201 & 0.057284 & 0.086078 & 0.06245 & 0.04463 &
  0.004265 & {\color[HTML]{34FF34} \textbf{0.004191}} & 0.026156 & 0.0067075 & 0.02533 & 0.00757145 \\ \hline
10/5 & 0.0315 & 0.039 & {\color[HTML]{34FF34} \textbf{0.0509}} & 0.0305 & 0.0601 & 0.031 &
  0.0045 & 0.0048 & 0.0199 & 0.0058 & 0.0288 & 0.0089 \\ \hline
12/6 & 0.027348 & {\color[HTML]{34FF34} \textbf{0.026942}} & 0.057592 & 0.026687 & 0.055426 &
  {\color[HTML]{009901} {\ul\textbf{0.0270495}}} & 0.004341 & 0.0049061 &
  0.020529 & 0.007813 & 0.020667 & 0.0054794 \\ \hline
14/7 (119\textendash0 Ma path) & 0.02725 & 0.033687 &
  {\color[HTML]{32CB00} \textbf{0.046242}} & {\color[HTML]{34FF34}
  \textbf{0.0252737}} & {\color[HTML]{34FF34} \textbf{0.04651}} &
  {\color[HTML]{32CB00} \textbf{0.027222}} & {\color[HTML]{009901} {\ul\textbf{0.0017226}}} & {\color[HTML]{009901} {\ul\textbf{0.00354029}}} & {\color[HTML]{34FF34} \textbf{0.0119085}} &
  {\color[HTML]{009901} {\ul\textbf{0.00157378}}} & 0.0121774 & 0.00765843 \\ \hline
16/8 & 0.028261 & 0.037993 & 0.0536856 & 0.0278993 & 0.05263 &
  {\color[HTML]{34FF34} \textbf{0.0282745}} & 0.0050338 & {\color[HTML]{32CB00}
  \textbf{0.0037223}} & 0.018136 & 0.0047548 & 0.014537 & 0.01061 \\ \hline
20/10 & {\color[HTML]{34FF34} \textbf{0.0257}} & 0.0378 & 0.0616 &
  0.0254 & 0.0526 & 0.0373 & 0.00544 & 0.012 &
  0.023 & 0.0162 & {\color[HTML]{34FF34} \textbf{0.0119}} & 0.0137 \\ \hline
24/12 & {\color[HTML]{32CB00} \textbf{0.0219624}} & {\color[HTML]{32CB00}
  \textbf{0.0214394}} & {\color[HTML]{009901} {\ul\textbf{0.043667}}} &
  {\color[HTML]{32CB00} \textbf{0.0216832}} & {\color[HTML]{32CB00}
  \textbf{0.0372685}} & 0.0345906 & 0.0058473 & 0.0063788 &
  {\color[HTML]{32CB00} \textbf{0.0059937}} & 0.0125858 &
  {\color[HTML]{009901} {\ul\textbf{0.002742}}} & 0.0051643 \\ \hline
30/15 & {\color[HTML]{009901} {\ul\textbf{0.014614}}} & {\color[HTML]{009901}
  {\ul\textbf{0.0183819}}} & 0.0786527 & {\color[HTML]{009901} {\ul\textbf{0.014254}}} & {\color[HTML]{009901} {\ul\textbf{0.031029}}} &
  0.0293416 & {\color[HTML]{34FF34} \textbf{0.00412276}} &
  0.00444694 & {\color[HTML]{009901} {\ul\textbf{0.00448627}}} & {\color[HTML]{32CB00} \textbf{0.00397289}} &
  {\color[HTML]{32CB00} \textbf{0.0054747}} & {\color[HTML]{32CB00}
  \textbf{0.00387337}} \\ \hline
\end{tabular}%
}
\end{table*}
\end{landscape}

\begin{figure}[!ht]
\centering
\includegraphics[width=1\textwidth]{../../paper/tex/GeophysJInt/figures/WinStpVsCPD.pdf}
\caption[Sliding window and step sizes vs $\mathcal{CPD}$]{Plot of the
equal-weight $\mathcal{CPD}$ scores collected in Table~\ref{tab-pk0vs1bs}. Note
that here the step size is always half of the sliding window size and the
reference path is the FHM derived.}\label{fig-WinStpVsCPD}
\end{figure}

\paragraph{The results}
are summarised in Table~\ref{tab-2010vs105F}, Table~\ref{tab-pk0vs1bs} and
Fig.~\ref{fig-WinStpVsCPD}. Based on these results, the effect of doubling the
size of the time window and step is negligible. $\mathcal{CPD}$ scores for APWPs
generated with no filtering (Pk 0,1) over a wider range of time windows and
steps, from 2/1 to 30/15 are shown in Fig.~\ref{fig-WinStpVsCPD} and
Table~\ref{tab-pk0vs1bs}. Scores for APP-derived paths (Pk 1) remain stable over
the whole tested range. AMP-derived paths (Pk 0) have higher scores for windows
narrower than 10 Myr, but are much more stable for windows wider than
10\textendash12 Myr, whilst remaining generally higher than the equivalent APP
scores.


\subsection{Moving versus fixed hotspot reference}

Fig.~\ref{fig-difm} and Fig.~\ref{fig-dif2010m} show the $\mathcal{CPD}$ scores
for the APWPs generated with all 28 picking and 6 weighting methods, compared to
the MHM reference paths (stars and solid line in Figs.~\ref{fig-mhsPred}-\ref{fig-mhsPred801})
with picking time windows and steps of 10/5 Myr and 20/10 Myr.

\subsubsection{Overall change}

Although the reference path is changed to MHM, mean, median and range values of
$\mathcal{CPD}$ scores are almost unchanged and comparable (Fig.~\ref{fig-dif}
versus Fig.~\ref{fig-difm}, and Fig.~\ref{fig-dif2010} versus
Fig.~\ref{fig-dif2010m}).

\begin{figure*}
	\centering
	\begin{subfigure}{1.01\textwidth}
		\includegraphics[width=\textwidth]{../../paper/tex/GeophysJInt/figures/101_120_0m.pdf}
		\caption{North America: minimum 0.00268588 (5(1)),
		maximum 0.0892467 (16(3)), mean 0.03674, median 0.03177075}\label{fig-na-difm}
	\end{subfigure}
	\vspace{.1em}
	\begin{subfigure}{1.01\textwidth}
		\includegraphics[width=\textwidth]{../../paper/tex/GeophysJInt/figures/501_120_0m.pdf}
		\caption{India: minimum 0.0232517 (19(1)), maximum 0.51876 (8(3)),
		mean 0.11364, median 0.076556}\label{fig-in-difm}
	\end{subfigure}
	\vspace{.1em}
	\begin{subfigure}{1.01\textwidth}
		\includegraphics[width=\textwidth]{../../paper/tex/GeophysJInt/figures/801_120_0m.pdf}
		\caption{Australia: minimum 0.00122074 (17(5)), maximum
		0.367952 (26(3)), mean 0.077, median 0.03893}\label{fig-au-difm}
	\end{subfigure}
	\caption[$\mathcal{CPD}$ of each plate's paleomagnetic APWPs vs
its MHM predicted APWP]{As Fig.~\ref{fig-dif}, here the reference path is
predicted from MHM\@. See the numbers of the picked paleopoles for methods in
Fig.~\ref{fig-dif}.}\label{fig-difm}
\end{figure*}

\begin{figure*}
	\centering
	\begin{subfigure}{.42\textwidth}
		\includegraphics[width=\textwidth]{../../paper/tex/GeophysJInt/figures/ay18_101comb_10_5_5_1m.pdf}
		\caption{North America: minimum 0.002686 (5(1)), FQ C-A}
	\end{subfigure}
	\begin{subfigure}{.42\textwidth}
		\includegraphics[width=\textwidth]{../../paper/tex/GeophysJInt/figures/ay18_101comb_10_5_16_3m.pdf}
		\caption{North America: maximum 0.08925 (16(3)), FQ B-A}
	\end{subfigure}
	\vspace{.1em}
	\begin{subfigure}{.42\textwidth}
		\includegraphics[width=\textwidth]{../../paper/tex/GeophysJInt/figures/ay18_501comb_10_5_19_1m.pdf}
		\caption{India: minimum 0.0233 (19(1)), FQ C-A}
	\end{subfigure}
	\begin{subfigure}{.42\textwidth}
		\includegraphics[width=\textwidth]{../../paper/tex/GeophysJInt/figures/ay18_501comb_10_5_8_3m.pdf}
		\caption{India: maximum 0.5188 (8(3)), FQ C-A}
	\end{subfigure}
	\vspace{.1em}
	\begin{subfigure}{.42\textwidth}
		\includegraphics[width=\textwidth]{../../paper/tex/GeophysJInt/figures/ay18_801comb_10_5_17_5m.pdf}
		\caption{Australia: minimum 0.00122 (17(5)), FQ B-A}
	\end{subfigure}
	\begin{subfigure}{.42\textwidth}
		\includegraphics[width=\textwidth]{../../paper/tex/GeophysJInt/figures/ay18_801comb_10_5_26_3m.pdf}
		\caption{Australia: maximum 0.368 (26(3)), FQ C-A}
	\end{subfigure}
	\caption[Best and worst differences (10 Myr bin, 5 Myr
step)]{Path comparisons with best and worst difference values shown in
Fig.~\ref{fig-difm}. The parenthetical remarks are Pk No with Wt
No.}\label{fig-difbwm}
\end{figure*}

\begin{figure*}
	\centering
	\begin{subfigure}{1.01\textwidth}
		\includegraphics[width=\textwidth]{../../paper/tex/GeophysJInt/figures/101_20_10_120_0m.pdf}
		\caption{North America: minimum 0.00565784 (5(1)),
		maximum 0.104618 (14(2)), mean 0.043777, median 0.02679955}\label{fig-na-dif2010m}
	\end{subfigure}
	\vspace{.1em}
	\begin{subfigure}{1.01\textwidth}
		\includegraphics[width=\textwidth]{../../paper/tex/GeophysJInt/figures/501_20_10_120_0m.pdf}
		\caption{India: minimum 0.0177 (6(3)), maximum 0.15937 (16(1)), mean
		0.0745, median 0.061346}\label{fig-in-dif2010m}
	\end{subfigure}
	\vspace{.1em}
	\begin{subfigure}{1.01\textwidth}
		\includegraphics[width=\textwidth]{../../paper/tex/GeophysJInt/figures/801_20_10_120_0m.pdf}
		\caption{Australia: minimum 0.00282 (23(4)), maximum
		0.31998 (22(3)), mean 0.062766, median 0.02803}\label{fig-au-dif2010m}
	\end{subfigure}
	\caption[$\mathcal{CPD}$ of each plate's paleomagnetic APWPs vs its MHM
predicted APWP (20/10 Myr bin/step)]{As Fig.~\ref{fig-dif2010}, here the
reference path is predicted from MHM\@. See the numbers of picked paleopoles in
Fig.~\ref{fig-dif}.}\label{fig-dif2010m}
\end{figure*}

\begin{figure*}
	\centering
	\begin{subfigure}{.43\textwidth}
		\includegraphics[width=\textwidth]{../../paper/tex/GeophysJInt/figures/ay18_101comb_20_10_5_1m.pdf}
		\caption{North America: minimum 0.00566 (5(1)), FQ B-A}
	\end{subfigure}
	\begin{subfigure}{.43\textwidth}
		\includegraphics[width=\textwidth]{../../paper/tex/GeophysJInt/figures/ay18_101comb_20_10_14_2m.pdf}
		\caption{North America: maximum 0.1046 (14(2)), FQ B-A}
	\end{subfigure}
	\vspace{.1em}
	\begin{subfigure}{.43\textwidth}
		\includegraphics[width=\textwidth]{../../paper/tex/GeophysJInt/figures/ay18_501comb_20_10_6_3m.pdf}
		\caption{India: minimum 0.0177 (6(3)), FQ C-A}
	\end{subfigure}
	\begin{subfigure}{.43\textwidth}
		\includegraphics[width=\textwidth]{../../paper/tex/GeophysJInt/figures/ay18_501comb_20_10_16_1m.pdf}
		\caption{India: maximum 0.1594 (16(1)), FQ C-A}
	\end{subfigure}
	\vspace{.1em}
	\begin{subfigure}{.43\textwidth}
		\includegraphics[width=\textwidth]{../../paper/tex/GeophysJInt/figures/ay18_801comb_20_10_23_4m.pdf}
		\caption{Australia: minimum 0.0028 (23(4)), FQ C-A}
	\end{subfigure}
	\begin{subfigure}{.43\textwidth}
		\includegraphics[width=\textwidth]{../../paper/tex/GeophysJInt/figures/ay18_801comb_20_10_22_3m.pdf}
		\caption{Australia: maximum 0.32 (22(3)), FQ C-A}
	\end{subfigure}
	\caption[Best and worst differences (20 Myr bin, 10 Myr
step)]{Path comparisons with best and worst difference values shown in
Fig.~\ref{fig-dif2010m}. The parenthetical remarks are Pk No and Wt No.}\label{fig-dif2010bwm}
\end{figure*}

For 10/5 Myr bin/step, the absolute differences (Fig.~\ref{fig-dmf}) are all
lower than 0.066, and actually most are less than 0.01. This indicates that for
comparing with paleomagnetic APWPs choosing fixed or moving hotspot model for
generating a reference path is not quite different. Therefore selecting fixed or
moving model for having a reference path is not a priority. However, based on
the signs of the differences between the scores from FHM and MHM
(Fig.~\ref{fig-dmf}), for North America (Fig.~\ref{fig-mf101}), FHM derived path
is a slighly better reference in general, while for both India
(Fig.~\ref{fig-mf501}) and Australia (Fig.~\ref{fig-mf801}), generally MHM
derived path is a slightly better choice.

\begin{figure*}
	\centering
	\begin{subfigure}{1.01\textwidth}
		\includegraphics[width=\textwidth]{../../paper/tex/GeophysJInt/figures/mf101.pdf}
		\caption{Fig.~\ref{fig-na-difm}-Fig.~\ref{fig-na-dif}; Percentage for
		positive values: ${\sim}66.67$\%}\label{fig-mf101}
	\end{subfigure}
	\vspace{.1em}
	\begin{subfigure}{1.01\textwidth}
		\includegraphics[width=\textwidth]{../../paper/tex/GeophysJInt/figures/mf501.pdf}
		\caption{Fig.~\ref{fig-in-difm}-Fig.~\ref{fig-in-dif}; Percentage for
		positive values: ${\sim}34.52$\%}\label{fig-mf501}
	\end{subfigure}
	\vspace{.1em}
	\begin{subfigure}{1.01\textwidth}
		\includegraphics[width=\textwidth]{../../paper/tex/GeophysJInt/figures/mf801.pdf}
		\caption{Fig.~\ref{fig-au-difm}-Fig.~\ref{fig-au-dif}; Percentage for
		positive values: ${\sim}19.62$\%}\label{fig-mf801}
	\end{subfigure}
	\caption[Differences between results from FHM and MHM]{Differences between
results from two different reference paths, FHM (Fig.~\ref{fig-dif}) and MHM
(Fig.~\ref{fig-difm}) derived. The absolute difference values less than
1.96-standard-deviation interval of the whole 168 values are labeled in green,
more than 1.96-standard-deviation interval labeled in red.}\label{fig-dmf}
\end{figure*}

For North America, large changes seem favored by Wt 3, and small changes seem
favored by Wt 5 or Pk 15 (Fig.~\ref{fig-dmf}). Pk 3 brings minor changes to both
North America and India. Pk 22 and 26 show large changes and Pk 17, 19, 21 and
25 show minor changes for both India and Australia.

\subsubsection{Relative performance of methods}

\begin{enumerate}
  \item When 10/5 Myr bin/step is applied, Pk 19 is still among the best,
	and Pk 16 still one of the worst, for all 3 plates. Even when 20/10 Myr
	bin/step is applied, 19 is still relatively a ``good'' method, and 16 a
	``bad'' one.
  \item APP still outperforms AMP\@. For 10/5 Myr bin/step, the percentage of
	factor of greater than 3 (about 19.05\%, 38.1\% and 48.81\% for North
	America, India and Australia respectively) is less than FHM (about 40.5\%,
	39.3\% and 50\%). For 20/10 Myr bin/step, the percentage of factor of
	greater than 3 (about 3.6\% and 46.43\% for India and Australia) is still
	less than FHM (about 3.8\% and 71.4\%) whilst for North America the
	percentage (${\sim}44.05$\%) is more than FHM (28.6\%).
  \item Both scores and AMP/APP difference are still low for Pk 4,5, 6,7.
  \item Both scores and AMP/APP difference are still high for Pk 16,17
	(Pk 17 gives low scores for Australia).
  \item Still scores are high but AMP/APP difference are low for Pk 2,3.
  \item Still scores are low but AMP/APP difference are high for Pk 0,1 on
	Australia.
  \item Comparable AMP/APP (20/10 versus 10/5, or FHM versus MHM) still appears
        for Pk 8,9 on Australian plate.
  \item Pk 1 is still a good performer.
  \item Pk 4,5, 6,7 are still `good' aggressive filters.
  \item Pk 2,3 (at least 2) are still `bad' aggressive filters.
  \item For 10/5 Myr bin/step, Pk 22 is still better than 23 for India. For
	20/10 Myr bin/step, Pk 22 is still better than 23 for both North
	America and India.
  \item Pk 16,17 are still `bad' for North America and India.
\end{enumerate}

\begin{landscape}
\setlength\tabcolsep{3pt}
\begin{table}[]
\centering
\caption{Performance statistics of all the picking and weighting methods.}\label{tab-bw}
\resizebox{.7\width}{!}{%
\begin{tabular}{@{}clllllclllllcl@{}}
\toprule
\multirow{2}{*}{\begin{tabular}[c]{@{}c@{}}Hotspot model\\ for ref path\end{tabular}} & \multicolumn{1}{c}{\multirow{2}{*}{\begin{tabular}[c]{@{}c@{}}CPD\\ grid\end{tabular}}} & \multicolumn{2}{c}{Best} & \multicolumn{2}{c}{Worst} & \multicolumn{1}{c}{\multirow{2}{*}{\begin{tabular}[c]{@{}c@{}}\% of APP better\\ than AMP\end{tabular}}} & \multicolumn{6}{c}{Count occurrences of each Wt being best for 28 Pks} & \multicolumn{1}{c}{\multirow{2}{*}{\begin{tabular}[c]{@{}c@{}}Pk14,15(Newer Studies)\\ better than Pk16,17(Older)\end{tabular}}} \\ \cmidrule(lr){3-6} \cmidrule(lr){8-13}
 & \multicolumn{1}{c}{} & \multicolumn{1}{c}{Pk} & \multicolumn{1}{c}{Wt} & \multicolumn{1}{c}{Pk} & \multicolumn{1}{c}{Wt} & \multicolumn{1}{c}{} & \multicolumn{1}{c}{0} & \multicolumn{1}{c}{1} & \multicolumn{1}{c}{2} & \multicolumn{1}{c}{3} & \multicolumn{1}{c}{4} & \multicolumn{1}{c}{5} & \multicolumn{1}{c}{} \\ \midrule
\multirow{6}{*}{FHM} & Fig.~\ref{fig-na-dif} & 1, 5, 7, 11, 13, 15, \textbf{19}, \textbf{21}, 25 & 0, 1, 3, 5 & 2, 5, 7, 14, \textbf{16}, 17, 18, 22, 26 & 0\textendash5 & 97.6 & \textbf{9} & 4 & 4 & 7 & 0 & 5 & Y \\
 & Fig.~\ref{fig-in-dif} & 4\textendash7, 9, \textbf{19}, \textbf{21}, 22, 26 & 0\textendash5 & 0, 2, 8, 10, 12, \textbf{16}, 18, 20, 24 & 0\textendash5 & 85.7 & \textbf{15} & 2 & 2 & 6 & 2 & 1 & Y \\
 & Fig.~\ref{fig-au-dif} & 1, 11, 13, 17, \textbf{19}, \textbf{21}, 25 & 0, 1, 3, 5 & 2, 4, 14, \textbf{16}, 22, 26 & 0\textendash5 & 100 & \textbf{10} & 5 & 1 & 7 & 2 & 3 & N \\
 & Fig.~\ref{fig-na-dif2010} & 1, 4\textendash7, 9, 11, 13, 15, \textbf{19}, \textbf{21}, 25 & 0, 1, 3, 5 & 0, 8, 10, 12, 14, \textbf{16}, 18, 20, 24 & 0\textendash5 & 72.6 & \textbf{14} & 2 & 0 & 3 & 3 & 6 & N,Y \\
 & Fig.~\ref{fig-in-dif2010} & 4\textendash7, \textbf{19} & 0\textendash5 & 0, 2, 10, 12, \textbf{16}, 18, 20, 23, 24, 27 & 0\textendash5 & 69 & \textbf{11} & 6 & 0 & 9 & 1 & 1 & Y,4y2n \\
 & Fig.~\ref{fig-au-dif2010} & 1, 11, 13, 17, \textbf{19}, \textbf{21}, 23, 25, 27 & 0, 1, 3\textendash5 & 2, 14, \textbf{16}, 22, 26 & 0\textendash5 & 100 & 9 & 6 & 0 & \textbf{10} & 1 & 2 & N \\ \midrule
\multirow{6}{*}{MHM} & Fig.~\ref{fig-na-difm} & 1, 5, 7, 9, 11, 13, 15, \textbf{19}, \textbf{21}, 25 & 0\textendash5 & 0, 10, 12, 14, \textbf{16}-18, 20, 24 & 0\textendash5 & 97.6 & \textbf{10} & 9 & 3 & 4 & 0 & 2 & Y \\
 & Fig.~\ref{fig-in-difm} & 5, 7, \textbf{19}, \textbf{21}, 22, 26 & 0\textendash5 & 0, 2, 8, 10, 12, 14, \textbf{16}, 18, 20, 24 & 0\textendash5 & 85.7 & \textbf{12} & 2 & 0 & 7 & 4 & 3 & Y \\
 & Fig.~\ref{fig-au-difm} & 1, 11, 13, 17, \textbf{19}, \textbf{21}, 25 & 0\textendash3, 5 & 2, 14, \textbf{16}, 22, 26 & 0\textendash5 & 98.8 & 6 & 4 & 2 & \textbf{10} & 4 & 2 & N \\
 & Fig.~\ref{fig-na-dif2010m} & 5, 7, 15, 22, 26 & 0\textendash5 & 0, 8, 10, 12, 14, 18, 20, 24 & 0\textendash5 & 76.2 & \textbf{10} & 8 & 1 & 4 & 1 & 4 & 1y5n,Y \\
 & Fig.~\ref{fig-in-dif2010m} & 4\textendash7, 22, 26 & 0\textendash5 & 0, 2, 10, 12, \textbf{16}, 18, 24 & 0\textendash5 & 70.2 & 6 & \textbf{7} & 0 & 6 & 3 & 6 & Y,3y3n \\
 & Fig.~\ref{fig-au-dif2010m} & 1, 11, 13, 15, 17, \textbf{19}, \textbf{21}, 23, 25, 27 & 0\textendash5 & 2, 14, \textbf{16}, 22, 26 & 0\textendash5 & 100 & \textbf{12} & 1 & 0 & 10 & 4 & 1 & N \\ \bottomrule
\end{tabular}%
}
\end{table}
\end{landscape}

\section{Discussion}

\subsection{Question: Why the APP methods generally produce better
similarities than AMP methods do?}

\subsubsection{Perspective of conceptual difference between AMP and APP}

Paleomagnetic (Mean) A95 represents precision (how well constrained calculated
poles are), and (mean) coeval poles' GCD represents accuracy (how close
calculated poles are to the reference path; Fig.~\ref{fig-A95GCD105F} and
Fig.~\ref{fig-A95SGCD105F}). Compared with AMP, APP usually improves both and
generates paths with higher accuracy and also higher precision (generally
increasing number of contributing paleopoles).

\begin{figure}[!ht]
\captionsetup[subfigure]{singlelinecheck=off,justification=raggedright,aboveskip=-6pt,belowskip=-6pt}
\centering
  \begin{subfigure}[htbp]{.49\textwidth}
	\caption{}\includegraphics[width=\textwidth]{../../paper/tex/GeophysJInt/figures/a95gcd_10_5F.pdf}\label{fig-A95GCD105F}
  \end{subfigure}
  \begin{subfigure}[htbp]{.49\textwidth}
	\caption{}\includegraphics[width=\textwidth]{../../paper/tex/GeophysJInt/figures/a95Sgcd_10_5F.pdf}\label{fig-A95SGCD105F}
  \end{subfigure}
  \begin{subfigure}[htbp]{.49\textwidth}
	\caption{}\includegraphics[width=\textwidth]{../../paper/tex/GeophysJInt/figures/a95mSad10_5F.pdf}\label{fig-A95mSad105F}
  \end{subfigure}
  \begin{subfigure}[htbp]{.49\textwidth}
	\caption{}\includegraphics[width=\textwidth]{../../paper/tex/GeophysJInt/figures/a95mSld10_5F.pdf}\label{fig-A95mSld105F}
  \end{subfigure}
  \caption[APP spatially better than AMP (arrow)]{Paleomagnetic APWP's mean A95 versus
(a) ``mean GCD'', (b) ``mean significant GCD'', (c) ``mean significant
orientation difference'', and (d) ``mean significant length difference'' between
paleomagnetic APWP and its corresponding FHM-and-plate-circuit predicted APWP\@.
Arrowtails are the results from AMP, while arrowheads are from APP\@. Black
color filled arrowheads are the small number of special cases of AMP derived
equal-weight $\mathcal{CPD}$s better than APP (see details in Fig.~\ref{fig-dif}
and Table~\ref{tab-bw}).}\label{fig-A95mG105F}
\end{figure}

\begin{figure}
\captionsetup[subfigure]{singlelinecheck=off,justification=raggedright,aboveskip=-6pt,belowskip=-6pt}
\centering
  \begin{subfigure}[htbp]{.49\textwidth}
	\caption{}\includegraphics[width=\textwidth]{../../paper/tex/GeophysJInt/figures/a95gcdd.pdf}\label{fig-A95GCD105Fd}
  \end{subfigure}
  \begin{subfigure}[htbp]{.49\textwidth}
	\caption{}\includegraphics[width=\textwidth]{../../paper/tex/GeophysJInt/figures/a95Sgcd_10_5Fd.pdf}\label{fig-A95SGCD105Fd}
  \end{subfigure}
  \begin{subfigure}[htbp]{.49\textwidth}
	\caption{}\includegraphics[width=\textwidth]{../../paper/tex/GeophysJInt/figures/a95mSad10_5Fd.pdf}\label{fig-A95mSad105Fd}
  \end{subfigure}
  \begin{subfigure}[htbp]{.49\textwidth}
	\caption{}\includegraphics[width=\textwidth]{../../paper/tex/GeophysJInt/figures/a95mSld10_5Fd.pdf}\label{fig-A95mSld105Fd}
  \end{subfigure}
  \caption[APP spatially better than AMP (dot)]{Differences of APP and AMP coordinates
shown in Fig.~\ref{fig-A95mG105F}. Crosses locates the small minority cases of
AMP derived equal-weight $\mathcal{CPD}$s better than APP (see details in
Fig.~\ref{fig-dif} and Table~\ref{tab-bw}).}\label{fig-A95mG105Fd}
\end{figure}

The fact that APP increases the number of paleopoles (N) in each sliding window
would potentially average out some ``bad'' (i.e.\ inaccurate) poles and improves
the fit between the paleomagnetic APWPs and the model-predicted APWPs. The
general effects that APP brings include the decreases in paleomagnetic A95s,
or/and distances between compared coeval poles of paleomagnetic APWP and
reference APWP (Fig.~\ref{fig-A95mG105F} and Fig.~\ref{fig-A95mG105Fd}).
However, if the added paleopoles were all or mostly ``bad'', the improvement of
fit would not occur. So the improvement of fit is not only because of the
increase in N, but also because the majority of the additional poles are
``good''. AMP only regards the time uncertainty of each pole as one mid-point.
Then this mid-point is treated as the most likely age of that mean pole. This is
actually incorrect. The age uncertainty of paleopole is not obtained from a
probability density function derived from an observed frequency distribution. As
defined, the time uncertainty's lower (older) limit is a stratigraphic age, and
its upper (younger) limit could be also a stratigraphic age or be constrained by
a tectonic event using the field tests (e.g.\ fold/tilt test and conglomerate
test). So the true age of the pole could be any one that is not older than the
lower limit and also not younger than the upper limit. In other words, the
mid-point could be the true age of the pole, but it is not known as the most
likely age of that pole. If the mid-point is the most likely age of a pole, AMP
should generate a path that is closer to the reference. However, mostly APP
generates better similarities (See the high proportions of APP better than AMP
in Table~\ref{tab-bw}). Most reasonably, the mid-point should be regarded as
one possibility of all uniformly (not necessarily normally bell shaped, or U
shaped, or left or right skewed) distributed ages between the two time limits.

So APP remains the effect of a paleopole borne on the mean poles during all the
period of its age uncertainty, and use the increased number of paleopoles (N)
to average out the negative effect of those ``bad'' poles, including the
paleopoles that should not be included at that age for mean pole.

\subsubsection{Perspective of stability comparison between AMP and APP}

Fitting curves by moving averaging change with different time window lengths and
time increment lengths (i.e.\ steps) (e.g., the similarity of the pair in
Fig.~\ref{fig-au-2010110} is improved a bit compared to
Fig.~\ref{fig-au-105110}). A balance needs to be made between having windows
that are too wide and steps that are too long which will smooth the data so much
we miss actual details in the APWP (e.g.\ those 20 Myr window 10 Myr step
paleomagnetic paths in Fig.~\ref{fig-dif2010bw} and even 30/15 Myr window and
step; Table~\ref{tab-pk0vs1bs} and Fig.~\ref{fig-WinStpVsCPD}) and windows that
are too narrow and steps that are too short which introduces noise by having too
few poles in each window (e.g.\ 2 Myr window 1 Myr step;
Table~\ref{tab-pk0vs1bs} and Fig.~\ref{fig-WinStpVsCPD}). There is a dependence
here on data density: higher density allows smaller windows/steps (this is one
of the things we want to test with selective data removal in the future). A
variety of ways of binning the data (here 30\textendash2 Myr window size and
half of the size as step) are being tested to see which one produces the better
and more appropriately smoothed fit.

Note that there are 135, 75 and 99 paleopoles that compose of 120\textendash0 Ma
APWPs of North America, India and Australia respectively. Does the reason of
10/5 Myr generally better than 20/10 could be the relatively larger number of
paleopoles for North America? Since theoretically for each sliding window, the
more ``bad'' paleopoles it contains, the worse similarity we should obtain. In
the contrary, the less paleopoles the window contains, the weaker the effect of
averaging out ``bad'' poles' influence would be. So is there a threshold number
of paleopoles for making an paleomagnetic APWP\@? For example, for making a
120\textendash0 Ma APWP, do the results indicate the best number of paleopoles
we need should be some value between 99 and 135? Here a test is implemented as
follows. With the results from the 10/5 and 20/10 bin/step together, 2/1, 4/2,
6/3, 8/4, 12/6, 14/7, 16/8, 20/10, 24/12 and 30/15 Myr bin/step are also used to
generate paleomagnetic APWPs for North America, India and Australia to see which
bin/step size would make paleomagnetic APWP closest to reference path. Will the
similarities they generate be generally worse than those the 10/5 Myr bin/step
generates? Or will they be better first and then worse than those the 10/5 Myr
bin/step generates when the bin/step sizes increase up to 20/10 Myr? For the
best results (Table~\ref{tab-pk0vs1bs}), as expected, AMP needs wider sliding
window and step to get closer to the reference path while APP does not
(Fig.~\ref{fig-WinStpVsCPD}). Even the best sizes of sliding window and step are
assigned for AMP, the results from APP are still much better than those from
AMP\@. Picking methods (directly related to N) are still the key influence
factor of choosing a better sliding window size and step size of moving
averaging, although weighting methods are also important.

\subsubsection{Summary}

\paragraph{If AMP has to be used,} better results can be obtained through
using large sizes of sliding window and step, commonly more than 24/12 Myr. In
addition, we should be cautious when Wt 3 is used with AMP\@.

\paragraph{APP is still recommended,} not only because the temporal uncertainty
is incorporated into the algorithm but also the results from APP are not as that
sensitive as AMP to the changes of sliding window and step sizes. In fact, for
APP the results from different window and step sizes are much more stable than
those from AMP (Fig.~\ref{fig-WinStpVsCPD}). This means we actually do not need
to worry about what sizes should be chosen for the sliding window and step when
we use APP method.

\subsection{Question: Why the AMP methods sometimes unexceptionally produce
better similarities than APP methods do?}

Because of small number of paleopoles (not necessarily ``bad'') involved in
each sliding window, the produced mean poles by AMP should be relatively far
from its contemporary model-predicted pole. In other words, AMP intends to give
fairly small change in accuracy. This also could potentially bring more
distinguishable $d_s$ for AMP\@ if the corresponding A95 is not large enough.
For example, for Fig.~\ref{fig-na-dif}, there are only two special (of 84 APP
versus AMP comparisons) cases Pk (Wt) 4(3), 6(3) better than 5(3), 7(3)
respectively. Compared with the Pk (Wt) 4(3) APWP, although most of the mean
paleopoles are closer to the FHM predicted APWP and also the number of the
significant pole pairs is two less for the APP derived path (i.e. 5(3)), the A95s
are smaller and most importantly there are one more significant $d_a$
orientation-change pair and one more significant $d_l$ segment pair
(Table~\ref{tab-w3p4vs5}). If we observe carefully, it is because of the much
smaller 15 Ma A95 for 5(3). The similar phenomenon occurs to the case of 6(3)
versus 7(3), a relatively much smaller paleomagnetic A95 causes more distinguishable
$d_a$ and $d_l$ for the APP results, and they offset the improvement of spatial
similarity $d_s$ APP brings.

For Pk (Wt) 2(0) versus 3(0) for 20/10 Myr window/step North America, all their $d_a$ and
$d_l$ are indistinguishable. Compared with the results from AMP, although the
coeval pole GCDs are generally unchanged or decreased or even increased (but not
too much) for APP, this spatial improvement is not able to offset the negative
effects of also generally unchanged or decreased or even increased (but not too
much) paleomagnetic A95s, which potentially brings more statistically
distinguishable coeval poles (e.g.\ the 20 Ma and 110 Ma poles for Pk 3 and
Wt 0; Table~\ref{tab-w5p4vs5}). This further causes greater
distinguishable mean $d_s$ from the APP methods. The similar phenomenon occurs
to Fig.~\ref{fig-na-dif2010} Pk 2 versus 3 with Wt 2, 3 and 5, Pk
4 versus 5 with Wt 1, 3 and 5, and Pk 6 versus 7 with Wt 1, 3 and 5, and
so on.

In addition, compared with AMP, APP potentially could generate more mean poles,
because sometimes for some sliding window there is no paleopole involved at all
for AMP\@ while there are paleopoles involved for APP\@. For APP, the
mean poles at all ages should be composed of more paleopoles than it is for
AMP, which should generally decrease both coeval pole distance and paleomagnetic
A95. However, sometimes a rare case (e.g.\ the 0 Ma comparison shown in
Table~\ref{tab-501w0p22vs23}) happens. It is sometimes that an additional
very ``bad'' paleopole gets included by APP and this increases both coeval pole
distance and paleomagnetic A95 even though N increases. Such cases include
Fig.~\ref{fig-in-dif} Pk 22 versus 23 (actually exactly the same as Pk 26
versus 27) with all the six types of weightings.

So generally as we discussed in the last section APP decreases the distances
between paleomagnetic APWPs and the hotspot and ocean-floor spreading model
predicted APWP, and also the uncertainties of paleomagnetic APWPs. However, as
we described in this section, special cases like decreased A95 potentially
intends to make coeval poles differentiated if the coeval poles' distance is
not decreased effectively or even increased, or very ``bad'' paleopoles got
involved in some sliding windows, occurs. In summary, when the negative effect
from these types of rare cases is beyond the positive effect the generally
improved mean poles contribute, the composite difference score would increase.
However, this phenomenon seldom occur (Table~\ref{tab-bw}).

\begin{table}[!ht]
\centering
\caption{One example of the Type 1 rare cases where AMP gives better similarity
result than APP does from North America (Window size: 10 Myr, step size:
5 Myr). Only statistically significant values are listed here.}\label{tab-w3p4vs5}
\resizebox{\textwidth}{!}{%
\begin{tabular}{@{}llllllllll@{}}
\toprule
\multicolumn{2}{c}{\multirow{2}{*}{\begin{tabular}[c]{@{}c@{}}FHM\\ predicted\end{tabular}}} & \multicolumn{4}{c}{Pk4(Wt3)} & \multicolumn{4}{c}{Pk5(Wt3)} \\ \cmidrule(l){3-10} 
\multicolumn{2}{c}{} & \multicolumn{2}{c}{ds} & \multicolumn{2}{c}{dl} & \multicolumn{2}{c}{ds} & \multicolumn{2}{c}{da} \\ \midrule
Age (Ma) & DM/DP () & Pmag A95 () & Dist () & Age (Ma) & Diff () & Pmag A95 () & Dist () & Age (Ma) & Diff () \\ \midrule
10 & 1.44607/0.793714 & 14.876819 & \textbf{9.937} & 105\textendash110 & 5.91855 & \multicolumn{2}{l}{} & \textit{\textbf{10\textendash15\textendash20}} & \textbf{126.59} \\ \cmidrule(lr){5-6}
15 & 1.2875/0.816514 & \multicolumn{2}{l}{\multirow{2}{*}{}} & \multicolumn{2}{l}{\multirow{11}{*}{}} & 2.0857 & \textbf{11.805} & \multicolumn{2}{l}{} \\ \cmidrule(l){9-10} 
25 & 2.48031/1.10915 & \multicolumn{2}{l}{} & \multicolumn{2}{l}{} & 6.3358 & \textbf{6.873} & \multicolumn{2}{c}{dl} \\ \cmidrule(l){9-10} 
55 & 3.58782/2.14032 & 4.6347 & \textbf{5.372} & \multicolumn{2}{l}{} & \multicolumn{2}{l}{} & Age (Ma) & Diff () \\ \cmidrule(l){9-10} 
60 & 4.85938/3.17602 & \multicolumn{2}{l}{\multirow{2}{*}{}} & \multicolumn{2}{l}{} & 6.5922 & 6.215 & \textit{\textbf{10\textendash15}} & \textbf{13.52} \\
65 & 3.68984/2.30014 & \multicolumn{2}{l}{} & \multicolumn{2}{l}{} & 8.6632 & \textbf{7.6} & \textit{\textbf{15\textendash20}} & \textbf{14.68} \\ \cmidrule(l){9-10} 
75 & 2.6435/1.54052 & 9.0812 & \textbf{8.836} & \multicolumn{2}{l}{} & \multicolumn{2}{l}{\multirow{4}{*}{}} & \multicolumn{2}{l}{\multirow{6}{*}{}} \\
100 & 2.8983/2.68346 & 8.892 & \textbf{8.455} & \multicolumn{2}{l}{} & \multicolumn{2}{l}{} & \multicolumn{2}{l}{} \\
105 & 2.32328/1.74639 & 5.3 & \textbf{5.03} & \multicolumn{2}{l}{} & \multicolumn{2}{l}{} & \multicolumn{2}{l}{} \\
110 & 4.13015/2.25964 & 3.8 & \textbf{9.8064} & \multicolumn{2}{l}{} & \multicolumn{2}{l}{} & \multicolumn{2}{l}{} \\
115 & 4.63512/2.58006 & 19.6676 & \textbf{9.3345} & \multicolumn{2}{l}{} & 8.5 & \textbf{11.704} & \multicolumn{2}{l}{} \\
120 & 7.34408/4.06043 & 3.515 & \textbf{17.35} & \multicolumn{2}{l}{} & 7.728 & \textbf{15.258} & \multicolumn{2}{l}{} \\ \cmidrule(r){1-4} \cmidrule(lr){7-8}
\end{tabular}%
}
\end{table}

\begin{table}[!ht]
\centering
\caption{One example of the Type 2 rare cases where AMP gives better similarity
result than APP does from North America (Window size: 20 Myr, step size:
10 Myr). Only statistically significant values are listed here.}\label{tab-w5p4vs5}
\resizebox{\textwidth}{!}{%
\begin{tabular}{@{}llllll@{}}
\toprule
\multicolumn{1}{c}{\multirow{3}{*}{\begin{tabular}[c]{@{}c@{}}Age\\ (Ma)\end{tabular}}} & \multicolumn{1}{c}{\multirow{2}{*}{\begin{tabular}[c]{@{}c@{}}FHM\\ predicted\end{tabular}}} & \multicolumn{4}{c}{ds} \\ \cmidrule(l){3-6} 
\multicolumn{1}{c}{} & \multicolumn{1}{c}{} & \multicolumn{2}{c}{Pk2(Wt0)} & \multicolumn{2}{c}{Pk3(Wt0)} \\ \cmidrule(l){2-6} 
\multicolumn{1}{c}{} & DM/DP (\degree) & Pmag A95 (\degree) & Dist (\degree) & Pmag A95 (\degree) & Dist (\degree) \\ \midrule
0 & 0 & 3.97 & \textbf{5.714} & 3.97 & \textbf{5.714} \\
10 & 1.44607/0.793714 & 3.879 & \textbf{6.034} & 3.879 & \textbf{6.034} \\
20 & 1.58039/1.10047 & \multicolumn{2}{l}{} & 6.771 & \textbf{6.934} \\
50 & 3.57782/1.61328 & 3.8644 & \textbf{7.304} & 4.03 & \textbf{8.6} \\
60 & 4.85938/3.17602 & 5.716 & \textbf{8.457} & 5.55 & \textbf{7.367} \\
100 & 2.8983/2.68346 & 10.769 & \textbf{7.308} & 10.769 & \textbf{7.308} \\
110 & 4.13015/2.25964 & \multicolumn{2}{l}{} & 3.29 & \textbf{8.311} \\
120 & 7.34408/4.06043 & 3.38 & \textbf{16.41} & 3.083 & \textbf{16.728} \\ \bottomrule
\end{tabular}%
}
\end{table}

\begin{table}[!ht]
\centering
\caption{One example of the Type 3 rare cases where AMP gives better similarity
  result than APP does from India (Window size: 10 Myr, step size: 5 Myr).
  Only statistically significant values are listed here. Note that for the
  bold-number ages, there is no mean poles at all for the ``Pk 22 (AMP) +
  Wt 2'' case.}\label{tab-501w0p22vs23}
\resizebox{\textwidth}{!}{%
\begin{tabular}{@{}llllllllll@{}}
\toprule
\multicolumn{2}{c}{\multirow{2}{*}{\begin{tabular}[c]{@{}c@{}}FHM\\ predicted\end{tabular}}} & \multicolumn{3}{c}{Pk22(Wt 2)} & \multicolumn{5}{c}{Pk23(Wt2)} \\ \cmidrule(l){3-10} 
\multicolumn{2}{c}{} & \multicolumn{3}{c}{ds} & \multicolumn{3}{c}{ds} & \multicolumn{2}{c}{dl} \\ \midrule
\multicolumn{1}{c}{Age (Ma)} & DM/DP (\degree) & Pmag DM/DP (\degree) & Dist (\degree) & N & Pmag DM/DP (\degree) & Dist (\degree) & N & Age (Ma) & Diff (\degree) \\ \midrule
0 & 0 & \textbf{6.28} & \textbf{12.72} & \textit{\textbf{2}} & \textbf{23.54} & \textbf{18.14} & \textit{\textbf{3}} & \textbf{80\textendash85} & 6.286 \\
10 & 1.12124/0.673225 & 5.4/3.1 & 29.9 & 1 & 5.4/3.1 & 29.9 & 1 & \textbf{110\textendash115} & 16.684 \\ \cmidrule(l){9-10} 
\textbf{15} & 1.1347/0.8127 & \multicolumn{3}{l}{\multirow{5}{*}{}} & 5.4/3.1 & 28.28 & 1 & \multicolumn{2}{l}{\multirow{13}{*}{}} \\
\textbf{60} & 4.79687/3.07133 & \multicolumn{3}{l}{} & 8.817 & 8.28 & 20 & \multicolumn{2}{l}{} \\
\textbf{70} & 4.26508/2.48783 & \multicolumn{3}{l}{} & 3.26 & 4.464 & 20 & \multicolumn{2}{l}{} \\
\textbf{75} & 2.6777/1.57975 & \multicolumn{3}{l}{} & 5 & 4.477 & 1 & \multicolumn{2}{l}{} \\
\textbf{80} & 4.20828/2.50294 & \multicolumn{3}{l}{} & 5 & 3.358 & 1 & \multicolumn{2}{l}{} \\
85 & 2.50744/1.24746 & 5 & 7.632 & 1 & 5 & 7.632 & 1 & \multicolumn{2}{l}{} \\
\textbf{90} & 3.88998/1.43423 & \multicolumn{3}{l}{\multirow{5}{*}{}} & 5 & 10.884 & 1 & \multicolumn{2}{l}{} \\
\textbf{95} & 2.23389/1.6247 & \multicolumn{3}{l}{} & 5 & 11.099 & 1 & \multicolumn{2}{l}{} \\
\textbf{100} & 2.8062/2.59819 & \multicolumn{3}{l}{} & 5 & 11.4155 & 1 & \multicolumn{2}{l}{} \\
\textbf{105} & 2.32328/1.74639 & \multicolumn{3}{l}{} & 5 & 14.908 & 1 & \multicolumn{2}{l}{} \\
\textbf{110} & 4.55519/2.49218 & \multicolumn{3}{l}{} & 6.8/4.9 & 13.962 & 1 & \multicolumn{2}{l}{} \\
115 & 4.63512/2.58006 & 10.73 & 10.508 & 5 & 10.73 & 10.508 & 5 & \multicolumn{2}{l}{} \\
120 & 6.02639/3.3319 & 10.73 & 10.508 & 5 & 10.73 & 10.508 & 5 & \multicolumn{2}{l}{} \\ \cmidrule(r){1-8}
\end{tabular}%
}
\end{table}

Other Type 1 (e.g. Table~\ref{tab-w3p4vs5}) cases: Fig.~\ref{fig-na-dif2010}
Pk (Wt) 2(1) versus 3(1). Fig.~\ref{fig-na-difm} 4(3) versus 5(3), 6(3) versus 7(3).

Other Type 2 (e.g. Table~\ref{tab-w5p4vs5}) cases: Fig.~\ref{fig-na-dif2010}
Pk (Wt) 2(0, 2, 3, 5) versus 3(0, 2, 3, 5), 4(1, 3, 5) versus 5(1, 3, 5), 6(1,
3, 5) versus 7(1, 3, 5). Fig.~\ref{fig-in-dif2010} 4(0\textendash5) versus 5(0\textendash5),
6(0\textendash5) versus 7(0\textendash5), 14(2, 3) versus 15(2, 3), 22(0, 2, 3) versus 23(0, 2, 3),
26(0, 2, 3) versus 27(0, 2, 3). Fig.~\ref{fig-au-dif2010} 4(2) versus 5(2).
Fig.~\ref{fig-au-difm} 8(5) versus 9(5). Fig.~\ref{fig-na-dif2010m} 2(0\textendash5)
versus 3(0\textendash5), 4(2) versus 5(2), 6(2) versus 7(2), 22(0\textendash5) versus
23(0\textendash5), 26(0\textendash5) versus 27(0\textendash5).
Fig.~\ref{fig-in-dif2010m} 4(0\textendash5) versus 5(0\textendash5), 6(0\textendash5)
versus 7(0\textendash5), 14(3) versus 15(3).

Combined Type 1 and 2 cases: Fig.~\ref{fig-na-dif2010} Pk (Wt)
22(0\textendash5) versus 23(0\textendash5), 26(0\textendash5) versus 27(0\textendash5).
Fig.~\ref{fig-in-dif2010} Pk (Wt) 22(1, 4, 5) versus 23(1, 4, 5), 26(1, 4, 5) versus
27(1, 4, 5). Fig.~\ref{fig-in-dif2010m} 22(0\textendash5) versus 23(0\textendash5),
26(0\textendash5) versus 27(0\textendash5).

Other Type 3 (e.g. Table~\ref{tab-501w0p22vs23}) cases: Fig.~\ref{fig-na-difm}
4(2\textendash5) versus 5(2\textendash5). Fig.~\ref{fig-in-difm} 22(0\textendash5)
versus 23(0\textendash5), 26(0\textendash5) versus 27(0\textendash5).


\subsection{Question: Why weighting is not affecting?}

Generally, weighting does not affect the similarities dramatically, because the
six results from the six weighting methods are mostly very close to each other
(Fig.~\ref{fig-dif}, Fig.~\ref{fig-dif2010}, Fig.~\ref{fig-difm},
Fig.~\ref{fig-dif2010m}). This closeness is also generally observed in the form
of clusters in Fig.~\ref{fig-w} and Fig.~\ref{fig-wu}. In addition, from the
general statistics of performance of the six weighting methods shown in
Table~\ref{tab-bw}, Wt 0 mostly performs the best or at least the
second best, which means no weighting works better in general.

When the above-mentioned question, about why APP generally produces better fits
than AMP, is tackled, we already find that both accuracy (how closely do the
pairs/segments/angles match) and precision (how large are the uncertainties on
the pairs/segments/angles, i.e.\ how difficult do they distinguish) can be the
factors that finally determine the difference score. Although another factor,
resolution (how many pairs/segments/angles are actually being compared) can
also influence the difference score (e.g. Table~\ref{tab-pk0vs1bs}), here this
factor is not relevant to comparisons between different weightings for a
certain picking method, because the numbers of picked paleopoles are the same
for the six weighting methods.

Therefore, at the very basic level, for example, a lower score is the result of
one of, or combination of:

1. A reduction in the difference scores of significantly different
pairs/segments, straightforwardly interpreted as a better fit (improved
accuracy).

2. Previously significantly different pairs/segments becoming insignificant.
This can occur either because the fit is better (improved accuracy), or because
of an increase in the uncertainty of the mean poles (they become less
distinguishable \textemdash{} decreased precision).

Fig.~\ref{fig-wp} and Fig.~\ref{fig-wpu} show that the proportions of results
from Wt 1\textendash5 to result from 0 are generally in the second
quadrant of the coordinate plane, where proportional change in difference score
is positive whereas proportional change in paleomagnetic FQ score is negative.
This means the five weightings (Wt 1\textendash5) do make effects, not obviously
in accuracy but mainly in improving precision, which could potentially expose
more pairs of distinguishable poles/segments/angle-changes. Or even accuracy is
improved in a small amount, improved precision intends to cancel out the effects
from improved accuracy.

There are a few special cases that one or two of the six weighting methods gives
a result with a dramatically worsened difference score, e.g.\ for weighting
method Wt 3 (Fig.~\ref{fig-dif}). From the labeled dots of Wt 3 in
Fig.~\ref{fig-w}, Fig.~\ref{fig-wu}, Fig.~\ref{fig-wp} and Fig.~\ref{fig-wpu},
we can get a general impression that Wt 3 is indeed improving precision
but not accuracy (at least not enough to offset the effects from improved
precision) so that this precision improvement is also the culprit that worsens
the final score. Highly improved precision without corresponding improved
accuracy would potentially bring more significant differences in shape metrics.
Wt 3 is exactly performing in this form.

\begin{figure}
\centering
\includegraphics[width=1\textwidth]{../../paper/tex/GeophysJInt/figures/w.pdf}
\caption[Paleomagnetic APWP's FQ score vs significant difference score]{10/5 Myr
bin/step paleomagnetic APWP's FQ score (different from FQ, see
the definitions of FQ and FQ score in Chapter~\ref{chap:Metho}) versus significant
$\mathcal{CPD}$ score (reference path: FHM predicted) for the 28 different
picking methods. See the significant $\mathcal{CPD}$ scores and Ppath-Rpath FQ
in Fig.~\ref{fig-dif}. Only those results dramatically worsened by Wt 3
are labeled.}\label{fig-w}
\end{figure}

\begin{figure}
\centering
\includegraphics[width=1\textwidth]{../../paper/tex/GeophysJInt/figures/wu.pdf}
\caption[Paleomagnetic APWP's FQ score vs raw difference score]{10/5 Myr
bin/step paleomagnetic APWP's FQ score (different from FQ, see the definitions
of FQ and FQ score in Chapter~\ref{chap:Metho}) versus raw difference score (reference path:
FHM predicted) for the 28 different picking methods. See FQ versus significant
difference score in Fig.~\ref{fig-w}. Only those results dramatically worsened
by Wt 3 are labeled.}\label{fig-wu}
\end{figure}

\begin{figure}
\centering
\includegraphics[width=1\textwidth]{../../paper/tex/GeophysJInt/figures/wp.pdf}
\caption[Proportional changes of Wt 1\textendash5 to 0: Paleomagnetic
APWP's FQ score vs significant difference score]{Proportion of Wt
1\textendash5 to 0: Proportional change in 10/5 Myr bin/step paleomagnetic
APWP's FQ score (different from FQ, see the definitions of FQ and FQ score in
Chapter~\ref{chap:Metho}) versus proportional change in significant $\mathcal{CPD}$ score
(reference path: FHM predicted) for the 28 different picking methods. See the
significant $\mathcal{CPD}$ scores and Ppath-Rpath FQ in Fig.~\ref{fig-dif}.
Only those results dramatically worsened by Wt 3 are
labeled.}\label{fig-wp}
\end{figure}

\begin{figure}
\centering
\includegraphics[width=1\textwidth]{../../paper/tex/GeophysJInt/figures/wpu.pdf}
\caption[Proportional changes of Wt 1\textendash5 to 0: Paleomagnetic
APWP's FQ score vs raw difference score]{Proportion of Wt
1\textendash5 to 0: Proportional change in 10/5 Myr bin/step paleomagnetic
APWP's FQ score (different from FQ, see the definitions of FQ and FQ score in
Chapter~\ref{chap:Metho}) versus proportional change in raw difference score (reference path:
FHM predicted) for the 28 different picking methods. See FQ versus significant
difference score in Fig.~\ref{fig-wp}. Only those results dramatically worsened
by Wt 3 are labeled.}\label{fig-wpu}
\end{figure}

In addition, for Wt 3, a small size of $\alpha$95 (high precision) could be
caused by those sampled directions not covering enough long period (thought to
be at least about $10^4$ years) to ``average out'' secular variation for giving
a paleopole. That is to say, the smallest $\alpha$95s could get the greatest
weights that they should not deserve.

Generally, weighting is affecting because different weighting functions give
obviously different results. However, interestingly weighting does not improve
fit and generally no weighting (Wt 0) is giving the best fit, although in most
cases weighting does improve precision. Wt 2 or 4 is not recommended, because
they never have generated the best similarities (Table~\ref{tab-bw}), compared
with other weighting methods. There is no general pattern about which weighting
(of Wt 1\textendash5) is better or worse. So weighting, for making a
paleomagnetic APWP, is not absolutely necessary. However, there are some
patterns about which weighting is better or worse for some specific continent or
some specific picking methods. For example, Wt 3 works generally fine with
Australian data (Table~\ref{tab-bw}). However, Wt 3 is not recommended for North
America and India.

\subsection{Question: Why best and worst methods are not consistent?}

For all the three continents, North America, India and Australia, Pk
19 (APP with local rotation or secondary print excluded) and 21 (APP
with local rotation and secondary print corrected) are consistently the best or
at least the relatively better (for example, in Fig.~\ref{fig-in-dif2010},
Pk 21 results are not colored in green, but ranging from
0.0483\textendash0.0561 that is still much less than the mean, 0.074 and also
the median, 0.072); whereas Pk 16 (AMP with publications before
1983) and 18 (AMP with local rotation or secondary print excluded) are
consistently the worst or at least the relatively worse.

Nevertheless, for each single continent, they have their own consistently best
and worst picking methods that are not the best or worst for other continents.
For example, Pk 15 (APP with publications after 1983) works well with North
America. We know that 70 North American paleopoles (about 53\%) have contributed
to Pk 15 for 120\textendash0 Ma. However, only 28 Indian paleopoles (about 38\%)
and 29 Australian paleopoles (about 30\%) have contributed to Pk 15. In
contrast, the Pk 17 (APP with publications before 1983) works well with only
Australia. This also tells that number of paleopoles is a key factor that
affects the fit score. The fact that Pk 22 (AMP with SS05 criteria) doesn't work
well with only Australia also reflects the importance of this factor.

In summary, the reason why some best and worst methods are not consistent is
generally that different continents have their unique data sets and get their
own paleomagnetic studies in varying degrees.

\subsection{Question: Are there particular parts of the path that are more
variable? Do different methods affect different parts of the path differently?}

The results may highlight the trade-off between more data diluting the effect of
outliers, and fewer but `better' data being more easily affected by a bad point
that gets through the filters (Fig.~\ref{fig-nads}, Fig.~\ref{fig-nadl} and
Fig.~\ref{fig-nada}).



\section{Final Conclusions}

From the perspective of the general similarities between those paleomagnetic
APWPs and the hot spot model and ocean-floor spreading model predicted APWPs,
GAD hypothesis is proved valid for at least the last 120 Myr.

\subsection{Universal rules of ways of processing paleomagnetic data}
%
\begin{description}
  \item Although effects of filters (all the picking methods where number of
	paleopoles shrinks compared to Pk 0 and Pk 1; see Fig.~\ref{fig-dif}) have a
	marginal change in reducing N (precision potentially going down), some
	filters do improve the similarity score, for example, Pk 25 (APP without
	superseded data) is always giving better scores than Pk 1 for all the
	three continents. However, Pk 0 and Pk 1 (no filtering and corrections
	applied) still generally perform well.
  \item APP (adding data to a time window with overlapping age selection
	criterion) is better than AMP for making paleomagnetic APWPs, for both kinds
	of situations when there are lots of data (APP even better, e.g.\ for North
	America and Australia) and not much data (APP still a better option, e.g.\
	for India; Table~\ref{tab-bw}). APP with most filters/corrections (Pk 3, 5,
	7, 9, 11, \ldots, 27) are generally giving worse scores than APP without any
	filter/correction (Pk 1).
  \item In most cases the APP methods produce better similarities than the AMP
	methods (Table~\ref{tab-bw}).
  \item Actually weighting is not improving the fit but improving precision
	generally. For quite many of the methods, no weighting is the best performer
	(Table~\ref{tab-bw}). For example, score is likely worse for the combined
	methods of weight Pk 3 and AMP\@.
  \item APP itself helps incorporate temporal uncertainty into the algorithm.
	With the bootstraps test helping incorporate spatial uncertainty into the
	algorithm together, both spatial and temporal uncertainties are successfully
	considered in APP methods.
\end{description}

\subsection{Conditional rules of ways of processing paleomagnetic data}
%
\begin{description}
  \item Pk 16 (AMP with data from old studies) and 18 (AMP
	without data affected by rotations and secondary print) are not recommended
	for generating a paleomagnetic APWP\@.
\end{description}

\subsection{Summary}

According to the results we have from the three continents, North America, India
and Australia, using the similarity measuring tool developed in Chapter~\ref{chap:Metho}, it is
recommended that APP should be used to select the input paleopoles. According to
all the paleomagnetic data we have from the three continents, the results from
any size for sliding window and step are interestingly and extremely close to
each other when the APP method is used, compared with the results from the AMP
method (Table~\ref{tab-pk0vs1bs} and Fig.~\ref{fig-WinStpVsCPD}). So any size
for binning and stepping is ok when APP is used. Then filtering is actually not
necessary. However, some filtering methods (e.g.\ Pk 5, 7 (igneous-derived), 11,
13 (nonredbeds or corrected redbeds derived), 19, 21
(non-local-rotation/reprinted or corrected-rotation derived) and 25
(non-superseded data derived)) are fine too and will not give worst or worse
results than the other filtering methods (i.e.\ the left Picking methods). With
APP used, weighting is actually not necessary either. If a weighting has to be
used, Wt 1 (related to the number of paleomagnetic sampling sites and
observations) is generally better than the other four given weighting methods
(Fig.~\ref{fig-flow}).

If AMP has to be used, relatively wide sliding window and step are needed.
According to our tests, more than 20/10 Myr is recommended. In addition, AMP
works relatively better with igneous-derived data (i.e.\ Pk 4 and 6),
which indicates that if we have fewer data, these data need to be better in
quality (Fig.~\ref{fig-flow}).

\begin{figure}
  \centering
  \includegraphics[width=1.01\textwidth]{../../paper/tex/GeophysJInt/figures/flowchart.pdf}
  \caption[Flowchart of making a reliable paleomagnetic APWP]{Flowchart for
recommended procedure of processing paleomagnetic data.}\label{fig-flow}
\end{figure}

\section*{Acknowledgments}
Thanks to Ohio Supercomputer Center for their remote HPC resources.
