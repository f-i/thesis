\chapter{Finding the Way(s) to Make a Reliable APWP}\label{chap:Reliab}
\textit{This chapter mainly describes how to generate paleomagnetic APWPs using
168 different methods, and then the application of the new APW path similarity
measuring tool used in finding the best APWP generating method(s). The final
results tell us that the ``Age Position Picking (APP)'' method is better than
the ``Age Mean Picking (AMP)'' method for making a reliable paleomagnetic APWP
and weighting is actually unnecessary.}
\vfill
\minitoc\newpage

(Chapter 3 is also openly accessible from
\emph{https://github.com/f-i/making\_of\_reliable\_APWPs}.)

\section{Introduction}

APWPs are generated by combining paleomagnetic poles, also known as paleopoles,
for a particular rigid block over the desired age range to produce a smoothed
path. See the Appendix A for some examples how the paleopole datasets are
constrained for a particular tectonic plate during a specific time interval.

\subsection{Not All Data Are Created Equal}

However, uncertainties in the age and location of paleopoles can vary greatly
for different poles.

\subsubsection{Age Error}

Although remanent magnetizations are generally assumed to be primary, many
events can cause remagnetisation (in which case the derived pole is `younger'
than the rock). If an event that has occurred since the rock's formation that
should affect the magnetisation (e.g., folding, thermal overprinting due to
intrusion) can be shown to have affected it, then it constrains the
magnetisation to have been acquired before that event. Recognising or ruling
out remagnetisations depends on these field tests, which are not always
performed or possible. Even a passed field test may not be useful if field test
shows magnetisation acquired prior to a folding event tens of millions of years
after initial rock formation.

The most obvious characteristic we can observe from paleomagnetic data is that
some poles have very large age ranges, e.g., more than 100 Myr. The
magnetization age should be some time between the information of the rock and
folding events. There are also others where we have similar position but the age
constraint is much narrower, e.g. 10 Myr window or less. Obviously the latter
kind of data is more valuable than the one with large age range.

\subsubsection{Position Error}

The errors of pole latitudes and longitudes are 95\% confidence ellipses, which
also vary greatly in magnitude. All paleopoles have some associated
uncertainties due to measurement error and the nature of the geomagnetic field.
More uncertainties can be added by too few samples, sampling spanning too short
a time range to approximate a GAD field, failure to remove overprints during
demagnetisation, etc.

\subsubsection{Data Consistency}

Paleopoles of a rigid plate or block should be continuous time series. For a
rigid plate, two poles with similar ages shouldn't be dramatically different in
location. Sometimes, this is the case. Sometimes we have further separated poles
with close ages.

There are a number of possible causes for these outliers, including:

\verb"Lithology"

For poor consistency of data, it is potentially because of different
inclinations or declinations. The first thing we should consider about is their
lithology. We want to check if the sample rock are igneous or sedimentary,
because sediment compaction can result in anomalously shallow
inclinations~\cite{T19}. In addition, we also can check if the rock are redbeds
or non-redbeds. Although whether redbeds record a detrital signal or a later
Chemical Remanent Magnetization (CRM) is still somewhat controversial, both
sedimentary rocks and redbeds could lead to inconsistency in direction compared
to igneous rocks.

\verb"Local Rotations"

Local deformation between two paleomagnetic localities invalidates the rigid
plate assumption and could lead to inconsistent VGP directions. So if
discordance is due to local deformation, and we would ideally want to exclude
or correct (if possible) such poles from our APWP calculation.

\verb"Other Factors"

In most cases, mean pole age (centre of age error) has just been binned. If
any of the poles have large age errors, they could be different ages from each
other and sample entirely different parts of the APWP\@. Conversely, if any of
the poles have too few samples, or were not sampled over enough time to average
to a GAD field, a discordant pole may be due to unreduced secular variation.

\subsubsection{Data Density}

As we go back in time, we have lower quality and lower density (or quantity) of
data, for example, the Precambrian or Early Paleozoic paleomagnetic data are
relatively fewer than Middle-Late Phanerozoic ones, and most of them are not
high-quality, e.g., larger errors in both age and location. The combination of
lower data quality with lower data density means that a single `bad' pole (with
large errors in age and/or location) can much more easily distort the
reconstructed APWP, because there are few or no `good' poles to counteract its
influence.

Data density also varies between different plates. E.g., we have a relatively
high density of paleomagnetic data for North American Craton (NAC), but few
poles exist for Greenland and Arabia. Based on mean age (mean of lower and upper
magnetic ages), for 120\textendash0 Ma, the \textbf{Global Paleomagnetic
Database} (GPMDB) version 4.6b~\cite[updated in 2016 by the Ivar Giaever
Geomagnetic Laboratory team, in collaboration with Pisarevsky]{M96,P05} has more
than 130 poles for NAC, but only 17 for Greenland and 24 for Arabia.

\subsubsection{Publication Year}

The time when the data was published should also be considered, because
magnetism measuring methodology, technology and equipments have been improved
since the early 20th century. For example, stepwise demagnetisation, which is
the most reliable method of detecting and removing secondary overprints, has
only been in common use since the mid 1980s.

In summary, not all paleopoles are created equal, which leads to an important
question: how to best combine poles of varying quality into a coherent and
accurate APWP\@?

\subsection{Existing Solutions and General Issues}
