\chapter{Conclusions}\label{chap:Concl}
\textit{2\textendash3 pages of summary of results, significance and future
directions/work.}
\vfill
\minitoc\newpage

\section{Summary of Results and Significance}

\subsection{Summary of Results}

\begin{enumerate}
  \item The APP method (considering the whole age ranges of paleopoles) is
  recommended to replace the old AMP one (considering only middle points of age
  ranges) for making a paleomagnetic APWP, when moving average is the core
  technique of the methodology.
  \item Weighting paleopoles, a traditional processing step for making a
  paleomagnetic APWP, is actually unnecessary.
\end{enumerate}

\subsection{Significance}

\begin{enumerate}
  \item Similarity between paleomagnetic APWPs of different tectonic plates
  could indicate the plates were once part of a supercontinent. Therefore, this
  thesis proposed a similarity measuring algorithm, and meanwhile a
  corresponding Python package is provided online as an open-source software to
  allow automatically calculating similarity scores.
  \item The APP method without weighting paleopoles performs significantly
  better with modern ($\sim$120\textendash0 Ma) paleomagnetic data than other
  methods, and is expected to be applied onto deeper time datasets.
\end{enumerate}

\section{Recommendations for Future Research}

Although this thesis proposed methodologies that starts right from the
beginning of a data-analysis research, compiling data, describing data,
processing data \ldots step by step to analyzing results, and finally found and
validated an approach to make a reliable paleomagnetic APWP, there are still
many further studies that need to be done.
%
\begin{enumerate}
  \item As mentioned in the objectives of Chapter~\ref{chap:Intro}, we have
  investigated the limits of paleomagnetic data on reconstructing individual
  plate motions like of North America, India and Australia in
  Chapter~\ref{chap:Reliab}. We still need to further investigate the limits of
  paleomagnetic data on reconstructing supercontinents, and even global tectonic
  parameters like average rate of plate motion, number of plates and so on.
  \item This thesis studies methodologies that apply only on
  $\sim$120\textendash0 Ma paleomagnetic data, so there are no studies that
  indicate how these methodologies perform on deep-time data.
  \item For $\sim$120\textendash0 Ma, there are relatively more paleomagnetic
  data published. However, when geologic time goes deeper, both quantity and
  quality of paleomagnetic data decline. A potential question is how much
  paleomagnetic data do we need actually to accurately reconstruct known modern
  ($\sim$120\textendash0 Ma) known plate motions? What insights does this give
  us into the reliability of reconstructions from earlier in geologic history?
  \item Based on the analysis we have done in this thesis and answers from the
  above questions, can we develop algorithms that look for matching segments of
  APWPs from different cratons, that might indicate they were part of the same
  continent or supercontinent?
  \item Can we develop algorithms that use APWPs from multiple continents to
  estimate global average plate motion rates? Can we get a good sense of how
  much information is lost due to lack of data on longitudinal motions? Can we
  use this to draw any conclusions about long term trends (or lack thereof) in
  the style and vigour of global plate tectonics? (Possible further question:
  can data on relative continental motion acquired from matching APWP curves be
  incorporated to improve these estimates?)
\end{enumerate}
