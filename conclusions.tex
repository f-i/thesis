\chapter{Conclusions}\label{chap:Concl}
\textit{Here are 2\textendash3 pages of summary of results, significance and
future directions/work.}
\vfill
\minitoc\newpage

\section{Summary of Results and Significance}

\subsection{Summary of Results}

\begin{enumerate}
  \item The APP method (considering the whole age ranges of paleopoles) is
  recommended to supercede the old AMP one (considering only middle points of
  age ranges) for making a paleomagnetic APWP, when moving average is the core
  technique of the methodology.
  \item Weighting paleopoles, a traditional processing step for making a
  paleomagnetic APWP, is actually unnecessary.
\end{enumerate}

\subsection{Significance}

\begin{enumerate}
  \item Similarity between coeval paleomagnetic APWPs for different tectonic
    plates could indicate the plates were once part of a supercontinent. This
    thesis developed a similarity measuring algorithm that can more rigorously
    identify potential matches to guide paleo-geographic reconstruction. A
    corresponding Python package is provided online, as open-source software, to
    allow automatic calculation of similarity scores.
  \item The APP method without weighting paleopoles performs significantly
    better with modern ($\sim$120\textendash0 Ma) paleomagnetic data than other
    methods, and is expected to produce similar improvements when applied onto
    datasets from deeper time.
\end{enumerate}

\section{Recommendations for Future Research}

Although this thesis proposed methodologies that starts right from the
beginning of a data-analysis research, compiling data, describing data,
processing data \ldots step by step to analyzing results, and finally found and
validated an approach to make a reliable paleomagnetic APWP, there are still
many further studies that need to be done.
%
\begin{enumerate}
  \item A question that follows from the results of Chapter~\ref{chap:Reliab}
    is: Are there particular parts of the path that are more variable? And do
    different methods affect different parts of the path differently? The
    results may highlight the trade-off between more data diluting the effect of
    outliers, and fewer but `better' data being more easily affected by a bad
    pole that gets through the filters (Fig.~\ref{fig-nads}, Fig.~\ref{fig-nadl}
    and Fig.~\ref{fig-nada}).
  \item In Chapter~\ref{chap:Reliab}, we have investigated the limits of
    paleomagnetic data for reconstructing individual plate motions of North
    America, India and Australia. We still need to further investigate the
    limits of paleomagnetic data on reconstructing supercontinents, and even
    global tectonic parameters like average rate of plate motion, number of
    plates and so on.
  \item This thesis studies only $\sim$120\textendash0 Ma paleomagnetic data, so
    there are no studies that indicate how these methodologies perform on
    deep-time data.
  \item For $\sim$120\textendash0 Ma, there are relatively more paleomagnetic
    data published. However, when geologic time goes deeper, both quantity and
    quality of paleomagnetic data decline. A potential question is how much
    paleomagnetic data do we need actually to accurately reconstruct known
    modern ($\sim$120\textendash0 Ma) known plate motions? This is one of the
    things we want to test with selective data removal in future work. What
    insights does this give us into the reliability of reconstructions from
    earlier in geologic history?
  \item Based on the analysis that have been done in this thesis and answers
    from the above questions, can we develop algorithms that look for matching
    segments of APWPs from different cratons, that might indicate they were part
    of the same continent or supercontinent?
  \item Can we develop algorithms that use APWPs from multiple continents to
    estimate global average plate motion rates? Can we get a good sense of how
    much information is lost due to lack of data on longitudinal motions? Can we
    use this to draw any conclusions about long term trends (or lack thereof) in
    the style and vigour of global plate tectonics? (Possible further question:
    can data on relative continental motion acquired from matching APWP curves
    be incorporated to improve these estimates?)
\end{enumerate}
