\chapter{Methodologies}\label{chap:Metho}
\textit{This chapter mainly describes the development of a new APW path
similarity measuring tool used throughout the dissertation.
Apparent polar wander paths (APWPs) based on paleomagnetic data are the
principal means of describing plate motions through most of Earth history.
Comparing the spatio-temporal patterns and trends of APWPs between different
tectonic plates is important for testing proposed paleogeographic
reconstructions of past supercontinents. However, thus far there is no clearly
defined quantitative approach to determine the degree of similarity between
APWPs. This paper proposes a new method of determining the degree of similarity
between two APWPs that combines three separate difference metrics that assess
both spatial separation of coeval poles, and similarities in the bearing and
length of coeval segments using a weighted linear summation. Bootstrap tests are
used to determine whether the differences between coeval poles and segments are
significant for the given spatial uncertainties in pole positions. The Fit
Quality is used to discriminate between low significance scores caused by
comparing poorly constrained paths with large spatial uncertainties from those
caused by a close fit between well-constrained paths. The individual and
combined metrics are demonstrated using tests on synthetic pairs of APWPs with
varying degrees of spatial and geometric difference. In a test on real
paleomagnetic data, we show that these metrics can quantify the effects of
correction for inclination shallowing in sedimentary rocks on Gondwana and
Laurussia's 320\textendash0 Ma APWPs. For an APWP pair, when one APWP's three
individual metrics are all greater than or equal to, or less than or equal to
the other one's, weighting is dispensable because the similarity ranking order
becomes straightforward; otherwise assigning equal weights is recommended,
although then decision makers are allowed to arbitrarily change weights
according to their preferences.}
\vfill
\minitoc\newpage

(This chapter is also openly accessible from
\emph{https://github.com/f-i/APWP\_similarity}. Text:
\emph{https://github.com/f-i/APWP\_similarity/blob/master/2.pdf}; Figures:
\emph{https://github.com/f-i/APWP\_similarity/blob/master/2\_figures.pdf};
Supplementary:
\emph{https://github.com/f-i/APWP\_similarity/blob/master/2Supp.pdf})


\section{Introduction}

Paleomagnetism is an important source of information on the past motions of the
Earth's tectonic plates. The orientation of remanent magnetisations acquired by
rocks during their formation record the past position of the Earth's magnetic
poles. In older rocks, these virtual geomagnetic poles often appear to be
increasingly offset from the modern day geographic poles. Because the Earth's
geomagnetic field appears to have remained largely dipolar and centered on the
spin axis for at least the last 2 billion years~\cite{E06}, this divergence is
interpreted as recording the translation and rotation of a continent by the
motion of tectonic plates in the time since the rock formed. An Apparent Polar
Wander Path (APWP) is a time sequence of paleomagnetic poles (or, more commonly,
mean poles that average all regional paleopoles of similar age) that traces
the cumulative motion of a continental fragment relative to the Earth's spin
axis.

Investigations of the Earth's past tectonic evolution and paleogeography often
involve comparing APWPs. For example, if two now separated continental fragments
were once part of the same supercontinent, their APWPs should share the same
geometry during the interval that this supercontinent existed. If the
supercontinent has been correctly reconstructed, the APWPs should also overlap
during this interval (Fig.~\ref{fig:nambal_same_geometry}). APWP comparisons can
be used to assess plate motion models generated using different datasets and/or
fitting techniques~\cite[for example]{B02,B07,S07,T08,D11}; significant deviations
from the known APWP for a continent can also be used to identify local tectonic
rotations~\cite[for example]{G10,Ch13}. Despite the clear importance of measuring
APWP similarity, these comparisons remain largely qualitative in nature,
involving visual comparisons of specific APWP segments and checking if they have
overlapping 95\% confidence limits~\cite[for example]{B02,B07,G10,D11}. Where
quantitative measures are used, the mean great circle distance (GCD) between
coeval poles on the APWP pair has been commonly used as a generalised difference
metric for a pair of APWPs, with a lower score indicating that they are more
similar~\cite[for example]{S07,T08}. However, because GCD is simply a measure of
spatial separation and does not incorporate geometric information about the two
paths being compared, it is possible for pairs with clearly different
similarities to have similar mean GCD scores
(Fig.~\ref{fig:QualitativelyDifferentSameGCD}). Due to the inherent time
variability of the geomagnetic field, uncertainties arising from the sampling
and measurement of remanent magnetisations, and uncertainties in the
magnetization age, the mean paleopoles that make up an APWP also have associated
spatial uncertainties. The significance of a GCD score is therefore not
immediately obvious. A score that indicates a relatively large difference
between two paths may not be significant if the spatial uncertainties are large;
a small difference could be significant if the spatial uncertainties are small
(Fig.~\ref{fig:GCDlargeButIndistinguishable}).


\begin{figure*}[tbp]
\centering
\includegraphics[width=1.01\textwidth]{../../paper/tex/ComputGeosci/figures/nambal_same_geometry.pdf}
\caption[Parts of APWPs of supercontinent fragments share the same
geometry]{(a) The APWPs for North America (black) and Baltica (grey) are
spatially distinct, but their Late Paleozoic\textendash{}Early Mesozoic sections
are geometrically similar due to them both being part of the supercontinent
Pangaea. (b) Reversing the opening of the Atlantic Ocean by rotation around a
reconstruction pole (blue star) results in the overlap of these two APWPs
between 390 million years ago (Ma) and 220 Ma, validating the proposed
paleogeography. The effects of this rotation on Baltica and its APWP (BAL) are
illustrated by the motion of the circle marker (before: blank center; after:
dark center), respectively. General Perspective projection. APWPs and rotation
parameters from~\cite{To16}.}\label{fig:nambal_same_geometry}
\end{figure*}

\begin{figure*}[tbp]
  \captionsetup[subfigure]{labelformat=empty,aboveskip=-6pt,belowskip=-6pt}
  \centering
  \begin{subfigure}[htbp]{1.01\textwidth}
    \centering
    \includegraphics[width=1.01\linewidth]{../../paper/tex/ComputGeosci/figures/QualitativelyDifferentSameGCD.pdf}
    \caption{}\label{fig:QualitativelyDifferentSameGCD}
  \end{subfigure}
  \begin{subfigure}[htbp]{1.01\textwidth}
    \centering
    \includegraphics[width=1.01\linewidth]{../../paper/tex/ComputGeosci/figures/GCDlargeButIndistinguishable.pdf}
    \caption{}\label{fig:GCDlargeButIndistinguishable}
  \end{subfigure}
  \caption[Examples showing GCD is a bad index of similarity]{(a) How the
  average GCD similarity metric ignores path geometry: \emph{Pair}\textbf{1}
  (circles and squares, left) is clearly more similar than \emph{Pair}\textbf{2}
  (circles and triangles, right), but for both pairs each GCD remains constant.
  (b) How GCD also ignores spatial uncertainties. The average GCD separation
  between coeval points is smaller for \emph{Pair}\textbf{1} (circles and
  squares, left) than \emph{Pair}\textbf{2} (circles and triangles, right). But
  if spatial uncertainties (plotted as 95\% confidence ellipses) are considered,
  this ranking is not trustworthy: it is \emph{Pair}\textbf{2} that is
  statistically indistinguishable from the reference path. Azimuthal
  Orthographic projection.}\phantomsection\label{fig:GCDbadIndex}
\end{figure*}
