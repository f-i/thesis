\begin{abstract}

Paleomagnetism is the only accepted quantitative method for reconstructing plate
motions and past paleogeographies for most of Earth history, concretely for
times before ${\sim}170$ Ma, the age of the oldest identification of marine
magnetic anomalies. However, there are limitations to paleomagnetic data. The
effects of paleomagnetic data quality and density, which generally degrades
further back in geologic history, and data processing methodology on producing
reliable APWPs are not well known. Luckily, for the past
${\sim}130$\textendash200 Myr we have independent plate motion data from
reconstructions of ocean spreading combined with hotspot reference frames. These
independent data sources can help constrain plate motions in more accurate ways.
For this period of geological time, paleomagnetic data also reaches the highest
density. Then such questions can be asked: when we look back on the past
${\sim}130$\textendash200 Myr, how much good paleomagnetic data do we have, and
how well does it reconstruct `true' plate motions derived from other independent
plate motion data?

In this thesis, 168 methods based on different qualities and densities of
paleomagnetic data are developed to generate 168 different apparent polar wander
paths (APWPs), which are the principal means of describing plate motions. Then
these APWPs are compared to the reference paths derived from two plate
kinematics models. A new quantitative method of determining the degree of
similarity between paleomagnetic APWPs and the reference paths is then created
and publicly accessible as a Python package on GitHub.

The final results indicate that the new developed data processing methodology is
suppressing the necessity of considering data quality and density when
paleomagnetic data needs to be processed. The new proposed Age Position Picking
method (considering the whole age ranges of paleopoles) is recommended to
supercede the old Age Mean Picking one (considering only middle points of age
ranges) for making a paleomagnetic APWP, when moving average is the core
technique of the methodology. Weighting paleopoles, a traditional processing
step for making a paleomagnetic APWP, is actually unnecessary.

\end{abstract}
