\begin{abstract}

Paleomagnetic Apparent Polar Wander Paths (APWPs) are the principal means of
describing plate motions through most of Earth history. However, there are
limitations to paleomagnetic data such as the not-well-constrained longitudes of
paleo-plates and the degrading quality and density of paleomagnetic data with
increasing age. Yet comparing the spatio-temporal patterns and trends of APWPs
between different tectonic plates is important for testing proposed
paleogeographic reconstructions of past supercontinents. Similarity between
paleomagnetic APWPs of different tectonic plates could indicate the plates were
once part of a supercontinent. However, there is no clearly defined quantitative
approach to determine the degree of similarity between APWPs.

A new similarity measuring algorithm between two APWPs, that combines three
separate difference metrics that assess both spatial separation of coeval poles,
and similarities in the bearing and length of coeval segments using a weighted
linear summation, is proposed. Bootstrap tests are used to determine whether the
differences between coeval poles and segments are significant for the given
spatial uncertainties in pole positions. The Fit Quality is used to discriminate
between low significance scores caused by comparing poorly constrained paths
with large spatial uncertainties from those caused by a close fit between
well-constrained paths. The individual and combined metrics are demonstrated
using tests on synthetic pairs of APWPs with varying degrees of spatial and
geometric difference. In a test on real paleomagnetic data, these metrics can
quantify the effects of correction for inclination shallowing in sedimentary
rocks on Gondwana and Laurussia's 320–0 Ma APWPs. A Python package on GitHub is
provided online as an open-source software to allow automatically calculating
similarity scores.

APWPs using 168 different methods are generated, and then the new APW path
similarity measuring tool is applied to find the best APWP generating methods.
Paleopole attributes are used to weigh their influence on mean poles, or to
determine if they should be omitted for producing a `better' mean pole.
Different key attributes, that can be quantified, or their combination, are
considered. In addition, when merging paleopoles to produce a smoothed mean
path, choices are made not only about which data are included or excluded, but
how data are combined. Moving averaging is used to combine data. The results
indicate that our new Age Position Picking (APP) method (considering the whole
age ranges of paleopoles) generates more reliable APWPs than the traditional Age
Mean Picking (AMP) method (considering only middle points of age ranges) for
making an APWP, when moving average is the core technique of the methodology.
Additionally, weighting paleopoles, a traditional processing step for making a
paleomagnetic APWP, is actually unnecessary. The APP method (without weighting
paleopoles), which performs significantly better with modern ($\sim$120–0 Ma)
paleomagnetic data than other methods, should be applied to older `deep-time'
datasets.

\end{abstract}
