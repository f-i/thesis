\documentclass[12pt]{ociamthesis}  % default square logo; [12pt,beltcrest] old belt crest logo; [12pt,shieldcrest] older shield crest logo

%load any additional packages
\usepackage[margin=1.0in,includefoot,heightrounded]{geometry}
\usepackage{amssymb,booktabs,gensymb,url,textcomp,amsmath,array}
\usepackage[nohints]{minitoc}
\setcounter{minitocdepth}{2}
\usepackage{float}
\usepackage[font=small,labelfont=bf,belowskip=-1pt,aboveskip=8pt]{caption} %setup globally
\usepackage{subcaption}  %setup locally works
\usepackage{graphicx}
\usepackage{multirow}
\usepackage{rotating}
\usepackage{tablefootnote}
\usepackage[utf8x]{inputenc}
\usepackage[table,xcdraw]{xcolor}  %colorful words in tables
\usepackage[round]{natbib}  %note that when you use this package, \bibliographystyle{} needs the corresponding styles, see more at https://www.overleaf.com/learn/latex/Natbib_bibliography_styles; learn more about natbib at https://journals.aas.org/natbib/
\usepackage{lscape,longtable}
\usepackage[page,toc,titletoc,title]{appendix}

\usepackage[normalem]{ulem}
\useunder{\uline}{\ul}{}

%input macros (i.e. write your own macros file called mymacros.tex
%and uncomment the next line)
%\include{mymacros}

\title{GLOBAL PALEOMAGNETIC DATA ANALYSIS:\\[1ex]     %your thesis title,
IMPROVED METHODS OF RECONSTRUCTING PLATE MOTIONS USING PALEOMAGNETIC DATA}   %note \\[1ex] is a line break in the title

\college{Kent State University}  %your college
\degreen{Master of Science}     %the degree
\renewcommand{\submittedtext}{A thesis submitted\\ To Kent State University in partial\\ Fulfillment of the requirements for the\\ Degree of Master of Science}  %comment this line to use the default submittedtext defined in cls file
\author{Chenjian Fu}             %your name

\degreedate{DECEMBER 2021\\}         %the degree date

\providecommand\phantomsection{}

\renewcommand{\contentsname}{Table of Contents}

%end the preamble and start the document
\begin{document}

%%%%% CHOOSE YOUR LINE SPACING HERE
%this baselineskip gives double line spacing for an examiner to easily
%markup the thesis with comments; about 1.5-spaced: 18pt plus2pt minus1pt
\baselineskip=22pt plus2pt

%set the number of sectioning levels that get number and appear in the contents
\setcounter{secnumdepth}{3}
\setcounter{tocdepth}{3}


%\include{dedication}        % include a dedication.tex file

\begin{romanpages}          % start roman page numbering
\begin{abstract}

Paleomagnetism is the only accepted quantitative method for reconstructing plate
motions and past paleogeographies for most of Earth history, concretely for
times before ${\sim}170$ Ma, the age of the oldest identification of marine
magnetic anomalies. However, there are limitations to paleomagnetic data. The
effects of paleomagnetic data quality and density, which generally degrades
further back in geologic history, and data processing methodology on producing
reliable APWPs are not well known. Luckily, for the past
${\sim}130$\textendash200 Myr we have independent plate motion data from
reconstructions of ocean spreading combined with hotspot reference frames. These
independent data sources can help constrain plate motions in more accurate ways.
For this period of geological time, paleomagnetic data also reaches the highest
density. Then such questions can be asked: when we look back on the past
${\sim}130$\textendash200 Myr, how much good paleomagnetic data do we have, and
how well does it reconstruct `true' plate motions derived from other independent
plate motion data?

In this thesis, 168 methods based on different qualities and densities of
paleomagnetic data are developed to generate 168 different apparent polar wander
paths (APWPs), which are the principal means of describing plate motions. Then
these APWPs are compared to the reference paths derived from two plate
kinematics models. A new quantitative method of determining the degree of
similarity between paleomagnetic APWPs and the reference paths is then created
and publicly accessible as a Python package on GitHub.

The final results indicate that the newly developed data processing methodology
is suppressing the necessity of considering data quality and density when
paleomagnetic data needs to be processed. The new proposed Age Position Picking
method (considering the whole age ranges of paleopoles) is recommended to
supercede the old Age Mean Picking one (considering only middle points of age
ranges) for making a paleomagnetic APWP, when moving average is the core
technique of the methodology. Weighting paleopoles, a traditional processing
step for making a paleomagnetic APWP, is actually unnecessary.

\end{abstract}
          % include the abstract
\maketitle                  % create a title page from the preamble info
\include{signature}
  \addcontentsline{toc}{chapter}{\contentsname}
\tableofcontents            % generate and include a table of contents
  \cleardoublepage\addcontentsline{toc}{chapter}{\listfigurename}
\listoffigures              % generate and include a list of figures
  \cleardoublepage\addcontentsline{toc}{chapter}{\listtablename}
\listoftables
  \cleardoublepage\addcontentsline{toc}{chapter}{Acknowlegements}
\begin{acknowledgements}

I thank Dr Daniel Holm for reaching out every time when I need help. I also
  thank him for helping me finish the final polishing of this thesis. I
  especially thank him and Ms Kelly Thomasson for helping me get a parking
  permission when I was temperally disabled in the snowy winter of 2018.

I thank Dr Chris Rowan for advising me on this topic and his detailed feedback
  for my writing.

I thank Dr Joseph Ortiz for encouraging me to compelete a small fun project
  ``What if The Himalayas Doesn't Exist'' in his Paleoceanography class. When I
  was not very confident about my speaking and writing at the beginning of this
  graduate program, he said ``I couldn't imagine how terrible I do if I study in
  China.'' He is a gentleman.

I also want to thank my parents and sister for their unconditional support. My
  dad got a bad cancer early last year, immediately had a major surgery and is
  still doing chemotherapy. My mom and sister take care of him all the time and
  I didn't go home yet since then.

The similarity measuring tool developed in this thesis depends on the
  open-source softwares PmagPy~\citep{T16} and GMT~\citep{W13}. All images are
  produced using GMT\@.

Thanks to Ohio Supercomputer Center for their remote HPC resources.

\end{acknowledgements}
   % include an acknowledgements.tex file
\end{romanpages}            % end roman page numbering

%now include the files of latex for each of the chapters etc
\chapter{Introduction}\label{chap:Intro}
\textit{The first chapter introduces paleomagnetism-based paleogeographic
reconstruction technique and highlights the motivation of the research conducted
in the thesis.}
\vfill
\minitoc\newpage

\section{Background and Motivation}

Reconstructing past paleogeographies, especially the motion of plates and their
interactions through time, is a key component of understanding the Earth's
geological history, including deciphering tectonics (e.g.\ supercontinent
reconstruction), paleo-climate history, and the evolution of life. Since the
advent of plate tectonics, it has been the background for nearly all geologic
events. In addition, plate reconstructions form the basis of global or regional
geodynamic models.

\subsection{Techniques Used in Relative and Absolute Plate Motion Studies}

The earliest quantitative effort to model plate kinematics was fitting conjugate
passive margins of the Atlantic~\citep{B65,W07}. They showed that the Atlantic
could be closed using a single Euler pole (using Euler's theorem on rotation).
Then it became fitting conjugate isochrons based on best-fitting marine magnetic
anomaly and fracture zone data~\citep{M71}, which minimizes the misfit area
between two isochrons. The \emph{Hellinger} method~\citep{H81} is a more
advanced and generalised method which also fits conjugate isochrons based on
best-fitting marine magnetic anomaly and fracture zone data, which minimizes the
sum of the misfits of conjugate data points that belong to common isochron
segments~\citep{W07}. These conjugate-line-fitting techniques are relatively
accurate for quantitative analysis. However they give relative, not absolute,
motions between plates, because plate motions can't be tied into absolute
location on Earth's surface, since both plates are likely moving. In addition,
they are limited to survey data from the seafloor, with a maximum age of no more
than ${\sim}200$ Ma~\citep{M08}.

Reference frames are a means of describing the motion of geologic features
(e.g.\ tectonic plates) on the surface of the Earth, relative to a common point
or ``frame''~\citep{Sh12}. An absolute reference frame is a frame that can be
treated as fixed relative to the Earth's geographic reference frame. In reality,
it's impossible to find a truly absolute reference frame, so we are actually
looking for a frame that has limited (and hopefully known) motion, which
approximates as ``fixed'' over geologically useful timescales and provides the
most complete descriptions of plate motions. A commonly used absolute reference
frame is the ``Fixed-hotspot model''~\citep{M93,M99}, covering ages from
${\sim}132$ Ma to present-day, which assumes that the linear volcanic chains
found on most oceanic plates are artifacts of absolute plate motions over a
upwelling plume from the deep mantle, which is assumed to be relatively fixed.
The advantage of this ``Fixed-hotspot model'' is that it is fairly
straightforward if the assumption of fixed hotspots is correct. However, this
model is limited to plates with well-dated volcanic hotspot chains~\citep[e.g.\
the Ninetyeast Ridge on the Indian Ocean floor and the Walvis Ridge in the
southern Atlantic Ocean, see][]{O05} and dating can be difficult~\citep[e.g.\
diffuse volcanic centers possibly related to large diameter plume conduits could
cause the existence of time reversals, see][]{O05}. As for not well-dated
hotspot tracks, for example, only about 5\textperthousand\ of the seamounts
(thought to be volcanic) in the Pacific are thought be be related to hotspot
volcanism and radiometrically dated~\citep[39 per cent of these ages are less
than 10 Ma, see][]{H07}. In addition, the fixed-hotspot model is mostly confined
to existing oceanic or thin continental crust because older oceanic lithosphere
has been largely destroyed by subduction and old, thick continental crust mostly
removed by erosion~\citep{C13}. Last, but not least, hotspots can be susceptible
to drift that may be caused by changes in sub-lithospheric mantle
flow~\citep{T09}. Generally, however, the drift rate is considered to be an
order of magnitude less than the rate of plate motions, so only becomes
significant over timescales of ${\sim}50$ Myr or more~\citep{O05,T07}. To
overcome this source of error, the ``Moving-hotspot model''~\citep{O05} uses
mantle convection modeling to predict hotspot drift. This approach has achieved
some apparent success, e.g.\ by getting motions in the Indo-Atlantic and Pacific
hotspot clusters to agree with each other, but it's very dependent on the mantle
convection model used. Hybrid models attempt to overcome the shortcomings of
each reference frame by combining them, e.g.\ combining a fixed-hotspot frame
from 100 Ma to 0 Ma~\citep{M93} with a moving-hotspot frame from
${\sim}132$\textendash100 Ma~\citep{O05}~\citep[Hybrid hotspot model,
see][]{Sh12}, combining a moving-hotspot frame from 100\textendash0
Ma~\citep{O05} with a paleomagnetic model (which reflects plate motion relative
to the magnetic dipole axis but cannot provide paleolongitudes because of the
axial symmetry of the Earth's magnetic dipole field)~\citep{T08} from
140\textendash100 Ma~\citep[Hybrid paleomagnetic model, see][]{Sh12}, and
combining a moving-hotspot frame from 120\textendash100 Ma~\citep{O05} with a
true polar wander (TPW) corrected paleomagnetic model~\citep{S08} from
100\textendash0 Ma~\citep[Hybrid TPW-corrected model, see][]{Sh12}.

Recently another absolute reference frame ``Subduction reference
model''~\citep{v10} tries to connect orogenies/sutures/subduction complexes on
the Earth surface with their corresponding subducted slabs in the mantle.
Assuming that these remnants sank vertically through the mantle, the absolute
location at which they were subducted can be reconstructed. In this way, this
model mainly imposes a longitude correction on the above mentioned ``Hybrid
TPW-corrected model'', and can theoretically give past absolute locations of
plates back to ${\sim}260$ Ma based on the estimated age of the oldest slab
remnants that can be reliably located in the mantle. While the ``Subduction
reference model'' allows for reconstructions between ${\sim}260$ Ma and 140 Ma,
older than the other absolute models can predict, the model is strongly
dependent on the vertical subduction assumption and resolution of seismic
tomography models, so its uncertainty is high. Above all, importantly, if we can
describe the absolute motion of one or a few key plates, the techniques for
establishing the relative plate motions described in the second paragraph above
can be used to construct plate circuits that allow a full kinematic description
of plate tectonics to be developed.

As we can see, all of these above reconstruction methods are limited to recent
geological history. For most of Earth history, concretely for times before
${\sim}170$ Ma, the age of the oldest magnetic anomaly identification,
paleomagnetism is the only accepted quantitative method for reconstructing plate
motions and past paleogeographies.

\subsection{Application of Paleomagnetism to Plate
Tectonics}\label{sec:applPlateTec}

\begin{figure}[!ht]
  \centering
    \includegraphics[width=0.88\textwidth]{../sphinx/source/I/fig/GAD.pdf}
  \captionsetup{width=.95\textwidth}
  \caption[Geocentric axial dipole model]{GAD model: Inclination (angle I =
    $\tan^{-1}(2\tan\lambda)$) of the Earth's magnetic field and how it varies
    with latitude, redrawn from~\citet{B92,T08,T20}. Magnetic dipole M is placed
    at the center of the Earth and aligned with the rotation axis; $\lambda$ is
    the geographic latitude, and $\theta$ is the
    colatitude.}\label{Fig:chap_intro_gad}
\end{figure}

The geomagnetic field is generated by the convective flow of a liquid
iron-nickel alloy in the outer core of the Earth. It is largely dipolar and can
be represented by a dipole that points from the north magnetic pole to the south
pole. However, the geomagnetic field varies in strength and direction over
decadal\textendash{}millennial timescales due to quadropole and octopole
components of the field. The most spectacular variations in direction are
occasional polarity reversals (normal polarity: the same as the present
direction of the field; or the opposite, i.e.\ reverse polarity). Over a period
of a few thousand years, the magnetic axis slowly migrates around the Earth's
rotational (geographic) axis (secular variation), but when averaged over 10,000
year timescales, higher order components of the field are thought to largely
cancel out and the position of the magnetic poles aligns with the geographic
poles. This is the geocentric axial dipole (GAD) hypothesis. In a GAD field, at
the north magnetic pole the inclination (angle with respect to the local
horizontal plane, see for example angle I in Fig.~\ref{Fig:chap_intro_gad}) of
the field is +90\degree\ (straight down), at the Equator the field inclination
is 0\degree\ (horizontal) pointing north and at the south magnetic pole the
inclination is -90\degree\ (straight up) (Fig.~\ref{Fig:chap_intro_gad}).
Another direction parameter of the Earth's magnetic field is declination. It is
the angle with respect to the geographic meridian, which is 0\degree\ everywhere
in a time-averaged GAD field.

Magnetic remanence is the magnetization left behind in a ferromagnetic substance
in the absence of an external magnetic field~\citep{T20}. The remanent
magnetisation of rocks can preserve the direction and intensity of the
geomagnetic field when the rock was formed, e.g.\ in the process of cooling,
ferromagnetic materials in the lava flow are magnetized in the direction of the
Earth's magnetic field, so the local direction of the field vector is locked in
solidified lava. We are often interested in whether the geomagnetic pole has
changed, or whether a particular plate/terrane has rotated with respect to the
geomagnetic pole~\citep{T20}. By measuring the direction of the remanent
magnetisation, we can calculate a virtual geomagnetic pole (VGP) to represent
the geomagnetic pole of an imaginary geocentric dipole which would give rise to
the observed remanent declination and inclination. Collection of VGPs (or
site-mean directions) allow calculating a ``paleomagnetic pole'', also known as
paleopole, at the formation level. Commonly a paleopole is a \emph{Fisherian}
mean~\citep{F53} of VGPs with a spatial uncertainty. A paleopole that plots away
from the present geographic poles is assumed to be due to plate motions since
the lava was solidified, which causes the paleopole to move with the
plate~\citep{T08}. Based on measurements of the remanent inclination, the
ancient latitude $\lambda$ for a plate can be calculated when the rock formed
from the dipole formula $\tan(I) = 2 \times\tan(\lambda)$
(Fig.~\ref{Fig:chap_intro_gad}). In addition, the remanent declination provides
information about the rotation of a plate. Ideally, as a time average, a
paleopole (which can be calculated from declination, inclination and the current
geographic location of the sampling site) for a newly formed rock will
correspond with the geographic north or south pole. To perform a reconstruction
with paleopoles we therefore have to calculate the rotation (Euler) pole and
angle which will bring the paleopole back to the geographic north or south pole,
and then rotate the plate by the same amount of angle using the same Euler pole.
This is how paleomagnetism can be used to reconstruct past positions of a plate.
In our example (Fig.~\ref{Fig:chap_intro_reconstructpole}), a ${\sim}155$ Ma
paleopole (latitude=52.59\degree{}N, longitude=91.45\degree{}W) will be restored
to the geographic pole by an Euler rotation of pole (0\degree, 178.55\degree{}E)
with angle 37.41\degree, which rotates the sampling site from its present
position of (0\degree, 25\degree{}E) to the Africa paleo-continent at
(15.7\degree{}S, 20.11\degree{}E). So Africa must have drifted northwards since
the Late Jurassic.

\begin{figure}[!ht]
  \centering
    \includegraphics[width=0.88\textwidth]{../sphinx/source/I/fig/af.pdf}
  \captionsetup{width=.95\textwidth}
  \caption[The hemispheric ambiguity and absolute paleolongitude indeterminacy
  with a single paleomagnetic pole (paleopole)]{Reconstruction of Africa with
  its ${\sim}155$ Ma paleopole. The red polygon is today's position of Africa,
  while the blue and green ones shows its reconstructed position at ${\sim}155$
  Ma, if the pole was North and South pole, respectively. Dashed green polygon
  illustrates the ambiguity of paleolongitude from paleomagnetic data alone
  (sites at same latitude but different longitudes record the same Declination
  and Inclination in a GAD field).}\label{Fig:chap_intro_reconstructpole}
\end{figure}

However, there are 2 problems with using paleopoles for constraining finite
rotations~\citep{T20}. First, if only one paleopole is given alone without any
geologic context, its polarity can be ambiguous, i.e.\ an upward inclination may
be due to being located in the southern hemisphere during a normal polarity
chron, or in the northern hemisphere during a reversed polarity chron (cf.\ the
solid blue and solid green Africa in Fig.~\ref{Fig:chap_intro_reconstructpole}).
In other words, we can't know if it's North pole or South pole, especially for
paleomagnetic data with Precambrian and early Paleozoic ages. Returning to the
example above, if the ${\sim}155$ Ma paleopole (52.59\degree{}N,
91.45\degree{}W) was formed during a period of reversed polarity, then it needs
to be rotated to the South pole rather than the North pole. The necessary Euler
rotation of pole (0\degree, 1.45\degree{}W) and angle 142.59\degree\ rotates the
sampling site (0\degree, 25\degree{}E) on Africa to (15.7\degree{}N,
23.01\degree{}W) indicating southward motion since the Late Jurassic. Second,
because in a GAD field the declination equals zero everywhere
(Fig.~\ref{Fig:chap_intro_reconstructpole}), paleomagnetic data doesn't register
longitudinal motions of plates (the Euler pole for a plate moving purely to the
east or west is at the geographic poles, so preserved paleopoles will experience
zero rotation), which means we can position a plate at any longitude we wish
subject to other geological constraints (cf.\ the solid and dashed green Africa
in Fig.~\ref{Fig:chap_intro_reconstructpole}).

The data source used in this thesis is \emph{Global Paleomagnetic
Database} (GPMDB) Version 4.6b~\cite[updated in 2016 by the Ivar Giaever
Geomagnetic Laboratory team, in collaboration with Pisarevsky]{M96,P05}, which
includes 9514 paleopoles for ages of 3,500 Ma to the present published from 1925
to 2016. GPMDB has been published in two ways: (1) IAGA GPMDB 4.6 online query:
\url{http://www.ngu.no/geodynamics/gpmdb/}, which is now closed; (2) Microsoft
Access system in \emph{.mdb} format at NOAA's National Geophysical Data Center
\url{https://www.ngdc.noaa.gov/geomag/paleo.shtml}~\citep{P03}
and CESRE's Paleomagnetism and Rock Magnetism project
\url{https://wiki.csiro.au/display/cmfr/Palaeomagnetism+and+Rock+Magnetism},
which is later updated to 4.6b by Ivar Giaever Geomagnetic Laboratory
\url{http://www.iggl.no/resources.html\#data}.

\begin{figure}[!ht]
  \centering
  \includegraphics[width=1\textwidth]{../../pps/poster/fig/na.pdf}
  \captionsetup{width=1\textwidth}
  \caption[All published paleomagnetic data from North America]{Much
paleomagnetic data has been collected from the North American Craton. For
younger geologic times, do we really need so much data to reconstruct accurately
just like modern-day plate motions? The image shows distribution of all
published paleopoles of the NAC over time, which are compiled from GPMDB
4.6b~\citep{P05} and PALEOMAGIA~\citep{V14}.}\label{Fig:chap_intro_nacpole}
\end{figure}

An apparent polar wander path (APWP) is composed of poles of different ages
from different sampling sites on the same stable (non-deforming) continent,
chained together to form a record of motion relative to the fixed magnetic pole
over geological time. It represents a convenient way of summarizing
paleomagnetic data for a plate instead of producing paleogeographic maps at
each geological period~\citep{T08}. As a preliminary study, the \emph{North
American Craton} (NAC) is chosen as a research object to develop techniques we
want to think about. The NAC is one of best studied cratons in paleomagnetism
with the GPMDB 4.6b containing 2160 paleopoles published since 1948
(Fig.~\ref{Fig:chap_intro_nacpole}). If we observe the latitudes, longitudes and
age distribution of the NAC paleopoles (Fig.~\ref{Fig:chap_intro_nacpole}), we
actually can identify the general trend of its APWP\@. However, converting this
data into a reliable, well-defined APWP can be challenging, due to the following
issues:

\subsection{Fact 1: Not All Regions on the Earth Surface Are
Solid}\label{sec:f1}

If we consider the modern North American continent, the region west of the
Rockies is actively deforming. Paleomagnetic data from such areas are likely to
reflect local tectonic processes such as block rotation rather than rigid plate
motions, and should be excluded. For example, the Rockies Mountain area was not
included in my data selecting polygon (the transparent yellow area in
Fig.~\ref{Fig:chap_intro_nacpole}). In order to investigate a specific craton or
terrane or block's past paleogeographic motion, choosing an appropriate
subregion without active tectonics, e.g.\ rotation, uplift or rifting, to select
data is often required. Such tectonics-free regions are usually called rigid.
However, the difficulty of defining such tectonic boundaries makes appropriate
spatial and temporal choices very difficult, particularly further in the
geological past when cratonic configurations and active plate boundaries were
very different to today. This leads to a question: What is the best way to
constrain the data for a specific plate or block? The solution proposed in this
thesis is described in Appendix~\ref{appen4chp3}.

\subsection{Fact 2: Not All Data Are Created Equal}\label{sec:f2}

APWPs are generated by combining paleopoles for a particular rigid block over
the desired age range to produce a smoothed path. However, the NAC dataset
illustrates that uncertainties in the age and location of different paleopoles
in the GPMDB can vary greatly (Fig.~\ref{Fig:chap_intro_nacpole}).

\subsubsection{Age Uncertainty}\label{sec:ageu}

Although remanent magnetizations are generally assumed to be primary, many
events can cause remagnetisation (in which case the derived pole is `younger'
than the rock). If an event that has occurred since the rock's formation that
should affect the magnetisation (e.g.\ folding, thermal overprinting due to
igneous intrusion, etc.) can be shown to have affected it, then it constrains
the magnetisation to have been acquired before that event. Recognising or ruling
out remagnetisations depends on these field tests, which are not always
performed or possible. Even a passed field test may not be useful if field test
shows magnetisation acquired prior to a folding event tens of millions of years
after initial rock formation.

The most obvious characteristic we can observe from the NAC paleomagnetic data
(Fig.~\ref{Fig:chap_intro_nacpole}) is that some paleopoles have very large age
ranges, e.g.\ more than 100 Myr. The magnetization age should be some time
between the information of the rock and folding events. There are also others
where we have similar position but the age constraint is much narrower, e.g.\ 10
Myr window or less. Obviously the latter kind of data is more valuable than the
one with large age range.

\subsubsection{Position Uncertainty}\label{sec:posu}

The uncertainties of paleopole latitudes and longitudes are plotted as 95\%
confidence ellipses (cf.\ the transparent red ovals in
Fig.~\ref{Fig:chap_intro_nacpole}), which also vary greatly in magnitude. All
paleopoles have some associated uncertainties due to measurement error and the
nature of the geomagnetic field. More uncertainties can be added by too few
samples, sampling spanning too short a time range to approximate a GAD field,
failure to remove overprints during demagnetisation, etc.

\subsubsection{Data Consistency}\label{sec:datcons}

Paleopoles of a rigid plate or block should be continuous time series. For a
rigid plate, two paleopoles with similar ages shouldn't be dramatically
different in location. We want to look at the consistency of the NAC and India's
data over smaller time periods, so the data is binned over a small time interval
(e.g.\ 2 Myr) to see whether the paleopoles in each time interval overlap within
their uncertainty ellipses, as they should. Sometimes, this is the case
(Fig.~\ref{Fig:chap_intro_na6462agemean}). Sometimes we have further separated
poles with close ages (Fig.~\ref{Fig:chap_intro_in97agemean}).

\begin{figure*}[tbp]
  \captionsetup[subfigure]{labelformat=empty,aboveskip=-6pt,belowskip=-6pt}
  \centering
  \begin{subfigure}[htbp]{.49\textwidth}
    \captionsetup{skip=0pt}  % local setting for this subfigure
    \centering
    \includegraphics[width=1.01\linewidth]{../sphinx/source/I/fig/na6462.pdf}
    \caption{North America: 64\textendash62 Ma (mean age)}\label{Fig:chap_intro_na6462agemean}
  \end{subfigure}
  \begin{subfigure}[htbp]{.49\textwidth}
    \captionsetup{skip=0pt}
    \centering
    \includegraphics[width=1.01\linewidth]{../sphinx/source/I/fig/in97.pdf}
    \caption{India: 9\textendash7 Ma (mean age)}\label{Fig:chap_intro_in97agemean}
  \end{subfigure}
  \caption[Example of AMP moving averaging effects]{Overlapping and further
    separated paleopoles of the NAC\@ and India. The oval ellipses are their
    95\% confidence uncertainties. The labels are their result number given in
    GPMDB 4.6b.}\phantomsection\label{Fig:chap_intro_ma-amp}
\end{figure*}

There are a number of possible causes for these outliers, including:

\paragraph{Lithology}

For this poor consistency of data (Fig.~\ref{Fig:chap_intro_in97agemean}), it is
potentially because of different inclinations or declinations. The first thing
we should consider about is their lithology. We want to check if the sample rock
are igneous or sedimentary, because sediment compaction can result in
anomalously shallow inclinations~\citep{T20}. In addition, we also can check if
the rock are redbeds or non-redbeds. Although whether redbeds record a detrital
signal or a later \emph{chemical remanent magnetization} is still somewhat
controversial, both sedimentary rocks and redbeds could lead to inconsistency in
direction compared to igneous rocks. For this case, all the three paleopoles
(Fig.~\ref{Fig:chap_intro_in97agemean}) are from sedimentary rocks. In addition,
pole 1136 and 1137 (Result Number in GPMDB 4.6b)'s source rocks also contain
redbeds~\citep{O82}, although the authors did not mention about the potential
inclination shallowing. For pole 7095, although the source rocks do not contain
redbeds, the authors did mention about possible inclination shallowing due to
haematite grains~\citep{G94}.

\paragraph{Local Rotations}

As discussed previously, local deformation between two paleomagnetic localities
invalidates the rigid plate assumption and could lead to inconsistent paleopole
directions. All the three paleopoles (Fig.~\ref{Fig:chap_intro_in97agemean})
contain signals of local rotations~\citep{O82,G94}, e.g.\ pole 7095 has a signal
which suggests the presence of a counter-clockwise local rotation of the Tinau
Khola section~\citep{G94}, and therefore do not reflect motions of the whole
rigid India plate in this case. So the discordance is likely due to local
deformation (Fig.~\ref{Fig:chap_intro_in97agemean}), and we would ideally
want to exclude or correct such poles from our APWP calculation.

\paragraph{Other Factors}

In Fig.~\ref{Fig:chap_intro_ma-amp}, mean pole age (centre of age uncertainty)
has just been binned. If any of the paleopoles have large age uncertainties,
they could be different ages from each other and sample entirely different parts
of the APWP\@. Conversely, if any of the paleopoles have too few samples, or
were not sampled over enough time to average to a GAD field, a discordant pole
may be due to unreduced secular variation, because in order to average errors in
orientation of the samples and scatter caused by secular variation, a
``sufficient'' number of individually oriented samples from ``enough'' sites
must be satisfied~\citep{v90,B02,T20}. For example, pole 1136
(Fig.~\ref{Fig:chap_intro_in97agemean}) is from only 4 sampling sites, pole 1137
is from only 3 sites and number of pole 7095's sampling sites is not even given
in the GPMDB 4.6b.

\subsubsection{Data Density}\label{sec:datden}

As we go back in time, we have lower quality and lower density (or quantity) of
data, for example, Precambrian or Early Paleozoic paleopoles are relatively
fewer than Middle-Late Phanerozoic ones, and most of them are not high-quality,
e.g.\ larger uncertainties in both age and location
(Fig.~\ref{Fig:chap_intro_nacpole}). The combination of lower data quality with
lower data density means that a single `bad' paleopole (with large uncertainties
in age and/or location) can much more easily distort the reconstructed APWP,
because there are few or no `good' paleopoles to counteract its influence.

Data density also varies between different plates. For example, we have a
relatively high density of paleomagnetic data for the NAC, but few paleopoles
exist for Greenland and Arabia. Based on mean age (mean of lower and upper
magnetic ages), for 120\textendash0 Ma, GPMDB 4.6b has more than 130 paleopoles
for the NAC, but only 17 for Greenland and 24 for Arabia.

\subsubsection{Publication Year}\label{sec:puby}

The time when the data was published should also be considered, because
paleomagnetic measuring methodology, technology and equipment have been improved
since the early $20^{th}$ century. For example, stepwise demagnetisation, which
is the most reliable method of detecting and removing secondary overprints, has
only been in common use since the mid 1980s.

In summary, not all paleopoles are created equal, which leads to an important
question: how to best combine poles of varying quality into a coherent and
accurate APWP\@? Paleomagnetists have proposed a variety of methods to filter
so-called ``bad'' data, or give lower weights to those ``bad'' data before
generating an APWP, e.g.\ two widely used methods: the V90 reliability
criteria~\citep{v90} and the BC02 selection criteria~\citep{B02}. Briefly, the
V90 criteria for paleomagnetic results includes seven criteria: (1) Well
determined rock age and that magnetization age is the same is presumed; (2) At
least 25 samples reported with \emph{Fisher} precision $\kappa$~\citep{F53} greater
than 10 and $\alpha$95 less than 16\degree; (3) Detailed demagnetisation results
reported; (4) Passed field tests; (5) Tectonic coherence with continent and good
structural control; (6) Identified antipodal reversals; (7) Lack of similarity
with younger poles~\citep{T92}. Compared with V90, the BC02 criteria suggests
stricter filtering, e.g.\ using only poles with at least 6 sampling sites and 36
samples, each site having $\alpha$95 less than 10\degree\ in the Cenozoic and
15\degree\ in the Mesozoic. There are many potential ways to weight the data set
which could obviously greatly influence the final result, and we want to test
this. But there has been limited study of how effective these
filtering/weighting methods are at reconstructing a `true' APWP, and for most
studies after a basic filtering of `low quality' poles, the remaining poles are,
in fact, treated equally.

\section{Objectives}

Our overarching aims are to develop rigorous, consistent and well-documented
methods of reconstructing plate motions using paleomagnetic data, and to
investigate the limits of paleomagnetic data on reconstructing individual plate
motions.

\subsection{Motivation and General Approach}

How has plate tectonics evolved over geologic history, in terms of average plate
velocities, numbers of plates and so on? The only quantitative data we have
prior to ${\sim}170$ Ma are paleomagnetic data. We know there are limitations to
paleomagnetic data, because we can't constrain the longitudes of paleo-plates
very well. When we look back through geologic history, how much good
paleomagnetic data do we have, and how well does it reconstruct `true' plate
motions? We don't know well the effects of data quality and density, which
generally degrades further back in geologic history, on producing reliable
APWPs. For the past ${\sim}130$\textendash200 Myr we have the highest density of
paleomagnetic data and also independent plate motion data from reconstructions
of ocean spreading combined with hotspot reference frames. These independent
data sources can help constrain plate motions in more accurate ways. This allows
us to ask the question: What path-generating method and how much paleomagnetic
data do we need actually to reconstruct accurately known modern-day plate
motions? If we can handle that, we can go back in time. For a certain density of
paleomagnetic data that we have, how reliably can we talk about what's going on
in the past given the much lower data distribution? It might turn out we don't
need very much data to say something reasonably and reliably. We can test this
by looking at the last 0\textendash120 Ma where we can compare paleomagnetically
derived plate motions with other more accurate methods of paleogeographic
reconstruction. This does not only include the work of developing tools and
algorithms to generate those paleomagnetically derived plate motions (to use
paleomagnetic data to reconstruct APWP parameters that are known from other
sources like ocean basins and hotspots), but also need us to know how good these
tools are or which one is the best algorithm (to compare paleomagnetic APWPs
with the known data sources predicted APWP). In short, this can give insights
into how well we can `know' plate motions back in the past, and what
path-generating method, data quality and density are necessary to reliably
reconstruct a `true' APWP\@.

As a preliminary analysis, some algorithms were made to separate/calculate out
so-called good paleomagnetic data (at any particular time period for a
particular craton, like here from ${\sim}120$ Ma to the present day for the
NAC, India and Australia). We are interested in what makes `good' data, how we
can identify them and filter them out from the database, and how sometimes `bad'
data are only bad in the sense that it is poorly constrained in age or position
or any other parameter, in which cases it might be possible to include it by
e.g., weighting. A weighted mean pole can be calculated for a time interval with
`better' (more likely to be reliable) paleopoles counting more than `worse'. For
example, a paleopole with small $\alpha$95 and very well constrained age is more
likely to reflect APWP position at the selected age point than a paleopole with
large $\alpha$95 and very broad age range. Then an algorithm that compares
similarity between paleomagnetic APWP and known-data-predicted APWP and also
gives scores should be developed. So that best path-generating method and data
quality are used to make a reliable paleomagnetic APWP\@. The validity of this
algorithm should also be tested.

\subsection{Research Questions or Hypotheses}

Questions 1\textendash2 focus on method development, whereas 3 starts
using them for plate tectonic research in modern geologic era, potentially
further back in geological time.

\subsubsection{Question 1}

What is the best way to turn a collection of individual paleopoles, with
different age constraints and uncertainties, into a smoothed APW path? This
question, in fact, is about how to (1) choose a data-constraining polygon that
represents a solid continent during a certain period; (2) pick (or bin) data
within a certain window for \emph{Fisher} statistical~\citep{F53} calculation; (3) do
weighting for picked data according to different uncertainties or other kinds of
standards of qualifications; (4) if the derived APWP is still not smoothed
enough when compared with a reference path, is further smoothing necessary? Our
goal here actually is to get a reliable result, i.e.\ a path generated to
approximate the `real' APWP with appropriate uncertainties.

\subsubsection{Question 2}

Based on the consequences from the algorithms we developed, we can do research
on why some algorithms are good, others bad for all plates? Why some algorithm
performs well for a plate or two but not others?

\subsubsection{Question 3}

What kind of dataset (in terms of data density and quality) is needed to
accurately reconstruct a known APWP, or a shared APWP between two cratons? If
Some criteria could be established for this. Does it provide any insights into
past reconstructions of plate motions (e.g., Rodinia)?

\bigskip
In summary, this thesis starts the research from studying modern paleomagnetic
datasets to attempt to find a general or even universal methodology that also
could be applied onto deep-time paleopoles.

\chapter{Methodologies}\label{chap:Metho}
\textit{This chapter mainly describes the development of a new APW path
similarity measuring tool used throughout the dissertation.
Apparent polar wander paths (APWPs) based on paleomagnetic data are the
principal means of describing plate motions through most of Earth history.
Comparing the spatio-temporal patterns and trends of APWPs between different
tectonic plates is important for testing proposed paleogeographic
reconstructions of past supercontinents. However, thus far there is no clearly
defined quantitative approach to determine the degree of similarity between
APWPs. This paper proposes a new method of determining the degree of similarity
between two APWPs that combines three separate difference metrics that assess
both spatial separation of coeval poles, and similarities in the bearing and
length of coeval segments using a weighted linear summation. Bootstrap tests are
used to determine whether the differences between coeval poles and segments are
significant for the given spatial uncertainties in pole positions. The Fit
Quality is used to discriminate between low significance scores caused by
comparing poorly constrained paths with large spatial uncertainties from those
caused by a close fit between well-constrained paths. The individual and
combined metrics are demonstrated using tests on synthetic pairs of APWPs with
varying degrees of spatial and geometric difference. In a test on real
paleomagnetic data, we show that these metrics can quantify the effects of
correction for inclination shallowing in sedimentary rocks on Gondwana and
Laurussia's 320\textendash0 Ma APWPs. For an APWP pair, when one APWP's three
individual metrics are all greater than or equal to, or less than or equal to
the other one's, weighting is dispensable because the similarity ranking order
becomes straightforward; otherwise assigning equal weights is recommended,
although then decision makers are allowed to arbitrarily change weights
according to their preferences.}
\vfill
\minitoc\newpage

(This chapter is also openly accessible from
\emph{https://github.com/f-i/APWP\_similarity}. Text:
\emph{https://github.com/f-i/APWP\_similarity/blob/master/2.pdf}; Figures:
\emph{https://github.com/f-i/APWP\_similarity/blob/master/2\_figures.pdf};
Supplementary:
\emph{https://github.com/f-i/APWP\_similarity/blob/master/2Supp.pdf})


\section{Introduction}

Paleomagnetism is an important source of information on the past motions of the
Earth's tectonic plates. The orientation of remanent magnetisations acquired by
rocks during their formation record the past position of the Earth's magnetic
poles. In older rocks, these virtual geomagnetic poles often appear to be
increasingly offset from the modern day geographic poles. Because the Earth's
geomagnetic field appears to have remained largely dipolar and centered on the
spin axis for at least the last 2 billion years~\cite{E06}, this divergence is
interpreted as recording the translation and rotation of a continent by the
motion of tectonic plates in the time since the rock formed. An Apparent Polar
Wander Path (APWP) is a time sequence of paleomagnetic poles (or, more commonly,
mean poles that average all regional paleopoles of similar age) that traces
the cumulative motion of a continental fragment relative to the Earth's spin
axis.

Investigations of the Earth's past tectonic evolution and paleogeography often
involve comparing APWPs. For example, if two now separated continental fragments
were once part of the same supercontinent, their APWPs should share the same
geometry during the interval that this supercontinent existed. If the
supercontinent has been correctly reconstructed, the APWPs should also overlap
during this interval (Fig.~\ref{fig:nambal_same_geometry}). APWP comparisons can
be used to assess plate motion models generated using different datasets and/or
fitting techniques~\cite[for example]{B02,B07,S07,T08,D11}; significant deviations
from the known APWP for a continent can also be used to identify local tectonic
rotations~\cite[for example]{G10,Ch13}. Despite the clear importance of measuring
APWP similarity, these comparisons remain largely qualitative in nature,
involving visual comparisons of specific APWP segments and checking if they have
overlapping 95\% confidence limits~\cite[for example]{B02,B07,G10,D11}. Where
quantitative measures are used, the mean great circle distance (GCD) between
coeval poles on the APWP pair has been commonly used as a generalised difference
metric for a pair of APWPs, with a lower score indicating that they are more
similar~\cite[for example]{S07,T08}. However, because GCD is simply a measure of
spatial separation and does not incorporate geometric information about the two
paths being compared, it is possible for pairs with clearly different
similarities to have similar mean GCD scores
(Fig.~\ref{fig:QualitativelyDifferentSameGCD}). Due to the inherent time
variability of the geomagnetic field, uncertainties arising from the sampling
and measurement of remanent magnetisations, and uncertainties in the
magnetization age, the mean paleopoles that make up an APWP also have associated
spatial uncertainties. The significance of a GCD score is therefore not
immediately obvious. A score that indicates a relatively large difference
between two paths may not be significant if the spatial uncertainties are large;
a small difference could be significant if the spatial uncertainties are small
(Fig.~\ref{fig:GCDlargeButIndistinguishable}).


\begin{figure*}[tbp]
\centering
\includegraphics[width=1.01\textwidth]{../../paper/tex/ComputGeosci/figures/nambal_same_geometry.pdf}
\caption[Parts of APWPs of supercontinent fragments share the same
geometry]{(a) The APWPs for North America (black) and Baltica (grey) are
spatially distinct, but their Late Paleozoic\textendash{}Early Mesozoic sections
are geometrically similar due to them both being part of the supercontinent
Pangaea. (b) Reversing the opening of the Atlantic Ocean by rotation around a
reconstruction pole (blue star) results in the overlap of these two APWPs
between 390 million years ago (Ma) and 220 Ma, validating the proposed
paleogeography. The effects of this rotation on Baltica and its APWP (BAL) are
illustrated by the motion of the circle marker (before: blank center; after:
dark center), respectively. General Perspective projection. APWPs and rotation
parameters from~\cite{To16}.}\label{fig:nambal_same_geometry}
\end{figure*}

\begin{figure*}[tbp]
  \captionsetup[subfigure]{labelformat=empty,aboveskip=-6pt,belowskip=-6pt}
  \centering
  \begin{subfigure}[htbp]{1.01\textwidth}
    \centering
    \includegraphics[width=1.01\linewidth]{../../paper/tex/ComputGeosci/figures/QualitativelyDifferentSameGCD.pdf}
    \caption{}\label{fig:QualitativelyDifferentSameGCD}
  \end{subfigure}
  \begin{subfigure}[htbp]{1.01\textwidth}
    \centering
    \includegraphics[width=1.01\linewidth]{../../paper/tex/ComputGeosci/figures/GCDlargeButIndistinguishable.pdf}
    \caption{}\label{fig:GCDlargeButIndistinguishable}
  \end{subfigure}
  \caption[Examples showing GCD is a bad index of similarity]{(a) How the
  average GCD similarity metric ignores path geometry: \emph{Pair}\textbf{1}
  (circles and squares, left) is clearly more similar than \emph{Pair}\textbf{2}
  (circles and triangles, right), but for both pairs each GCD remains constant.
  (b) How GCD also ignores spatial uncertainties. The average GCD separation
  between coeval points is smaller for \emph{Pair}\textbf{1} (circles and
  squares, left) than \emph{Pair}\textbf{2} (circles and triangles, right). But
  if spatial uncertainties (plotted as 95\% confidence ellipses) are considered,
  this ranking is not trustworthy: it is \emph{Pair}\textbf{2} that is
  statistically indistinguishable from the reference path. Azimuthal
  Orthographic projection.}\phantomsection\label{fig:GCDbadIndex}
\end{figure*}

\chapter{Methods for Producing a Reliable APWP}\label{chap:Reliab}  % chktex 36
\textit{This chapter mainly describes how to generate paleomagnetic APWPs using
168 different methods, and then the application of the new APW path similarity
measuring tool described in the last chapter in finding the best APWP generating
method(s). The final results tell us that the ``Age Position Picking (APP)''  % chktex 36
method is better than the ``Age Mean Picking (AMP)'' method for making a
reliable paleomagnetic APWP and weighting is actually unnecessary.}
\vfill
\minitoc\newpage

\section{Introduction}

A paleomagnetic pole, also known as paleopole, has many attributes, including
sampling site, number of samples, sample rock characteristics, age determination
and its uncertainty, pole location and its uncertainty etc. And APWPs are
generated by combining paleopoles into mean poles, for a particular rigid block
over the desired age range to produce a smoothed path (see detailed definition
of APWP in Section~\ref{sec:applPlateTec}). APWP reflects cumulative motions of
a continental fragment relative to the Earth's spin axis. So, in order to build
an APWP, an object continent or continental fragment should be picked first. And
then the general ``boundary'' of this continent or continental fragment should
be determined tectonically. This closed tectonic ``boundary'' is then used to
constrain the paleomagnetic datasets through picking only those with their
sampling sites lying inside the closed ``boundary''. See
Appendix~\ref{appen4chp3}
for examples about how the paleopole datasets are constrained for a particular
tectonic plate during a specific time interval. Then the paleopoles that derive
from those sampling sites are statistically combined together into mean poles
when their ages are close. The final products, those mean poles, are then
connected according to their estimated ages to make a time-series path which is
called paleomagnetic APWP\@. So it is important to ask a question like what can
be called a reliable paleomagnetic APWP or how to build a reliable APWP,
because, for example (and also what we want to focus our work on here is), there
are so many different perspectives to consider when we combine paleopoles into
mean poles. For instance, how close a group of paleopoles' ages should be to be
combined together into a mean pole, what statistical way should be used to do
this combining, should those paleopoles be treated equally in this process of
combining, if no what factor affects more than others, etc. These are the
factors that might affect the end result.

In this chapter, paleopoles' attributes are used to weigh their influence on
mean poles (see details in Section~\ref{sec:w}), or to determine if they should
be omitted for producing a `better' mean pole (see details in
Section~\ref{sec:f}). Different key attributes, that can be quantified, or their
combination are considered. In addition, when we are merging paleopoles to
produce a smoothed mean path, we make choices not only about which data are
included or excluded, but how data are combined. Moving averaging is used to
combine data (see details in Section~\ref{sec:p}). With application of two
moving average methods, where each moving window includes data with (1) only
middle point of lower and higher age limits considered as paleopole's age and
(2) whole dating uncertainty considered as paleopole's age range, different
APWPs are produced to see which attribute is more important and also which
moving average method performs better. To answer these questions, we need a
reference path (see details in Section~\ref{sec:refpath}) and a tool to measure
similarity between the paleomagnetic APWPs and the reference path. Then the
measuring tool developed and introduced in Chapter~\ref{chap:Metho} is used to
measure the similarity. We have two reference paths derived from other plate
kinematics models. These reference models are thought to be more accurate and
more precise, so that if the paleomagnetic APWPs are similar to the reference
path, they are regarded as more `reliable'. The question is which model
predicted reference path is a better reference or is there any difference
between these two reference paths? In this chapter, we will also try to answer
this question.

Finally no single attribute generally works better than others for all the
continents or continental fragments. However, the moving average method, that
considers the whole range of paleopoles' temporal uncertainty when calculating
mean pole's age is needed, is proved that it is obviously a better way of
generating reliable APWPs than the one that only considers the mid-point of
temporal uncertainty as a paleopole's age. And weighting is proved to be
actually unnecessary. The reference paths derived from other models are not
significantly different.

\subsection{How An APWP Is Generated and Key Factors}

As mentioned above, the first question to ask is that which tectonic plate the
paleopole we are working on belongs to (see Section~\ref{sec:f1} and details
about how it is solved in Appendix~\ref{appen4chp3}). The fact is that
paleomagnetic data is not just about poles but multiple attributes integrated
and not all data are created equal. So once the host tectonic plate is
determined, we encounter another problem: the attributes, for example,
uncertainties of ages and also locations, can vary greatly for different
paleopoles (see details in Sections~\ref{sec:ageu} and~\ref{sec:posu}).
Paleopole's attributes like age and location obviously can be
quantified and then used to weigh paleopoles or to omit some of them through
setting up a range of acceptable values. In addition, the age uncertainty can
also be used to change the way how we do moving averaging when we calculate mean
poles through considering the whole uncertainty range instead of only
considering the mid-point of the uncertainty range in the traditional way (this
is also how we get two different moving average methods). Besides age and
position uncertainties of paleopoles, the data consistency is also needed to be
investigated carefully (see details in Section~\ref{sec:datcons}), because some
paleopoles' polarity given in the GPMDB4.6b~\cite[updated in 2016 by the Ivar
Giaever Geomagnetic Laboratory team, in collaboration with Pisarevsky]{M96,P05}
could be wrong although most are correct. Data density is also vital (see
details in Section~\ref{sec:datden}) because paleomagnetism is basically built
on statistics. In addition, when the paleopoles were published (publication
year) is also a key factor that potentially indicates the quality of the
paleopoles (see details in Section~\ref{sec:puby}), simply because in old times,
technology of magnetism measuring was not that advanced as today. In
consideration of all these factors, we organise them in the general processing
steps below in order to generate a paleomagnetic APWP\@.

\subsubsection{Picking/Binning Step}

The moving average method is used to combine paleopoles into mean poles. The
moving time window/bin picks a group of paleopoles at each time step (see
description in Section~\ref{sec:p} and Fig.~\ref{fig-nac-maplat}). So the key
factors are:

\paragraph{Width of time window}
(trade off between number of paleopoles and amount of smoothing): Smoothness of
paleomagnetic APWPs generated by moving averaging could change with different
time window lengths and steps. If window length is too wide and step is too
long, final path would be so smoothed that actual details would be missed. If
window length is too narrow and step is too short, final path would be jerky
piecewise so that too much noise would be introduce because there are too few
datasets in each window. So a balance needs to be achieved.

\paragraph{Age uncertainty in magnetisation,}
particularly when that uncertainty is larger than the selection window. The
mid-point is commonly used as a selection point (see an example in upper panel
of Fig.~\ref{fig-nac-maplat}), but when the minimum age is constrained by a
field test (common when the age range is large; see Section~\ref{sec:ageu}), the
assumption that this is the most likely age is questionable. As it is mentioned
in Section~\ref{sec:ageu}, if field test shows magnetisation acquired prior to,
for example, a folding event that is tens of millions of years after initial
rock formation, a passed field test is not actually very useful. So age
uncertainty is very large, the possibility of remagnetisation is certainly very
high, which means the age is probably closer to the minimum age instead of the
middle age.

\subsubsection{Filtering Step}

This step filters out bad data that is likely to be unreliable, based on known
information about the paleopoles. These can be subdivided into two groups:

\paragraph{Characteristics that indicate the paleopole is well-constrained
(precise)}
(e.g., has a small $\alpha$95 uncertainty ellipse, large $\kappa$)

\paragraph{Characteristics that indicate the paleopole is reliable (accurate)}:
\begin{itemize}
  \item Lithology, particularly the risk of inclination flattening in sediments
        (see Section~\ref{sec:datcons})
  \item Risk of unidentified local rotations in deformed areas (see
        Section~\ref{sec:datcons})
  \item Publication year \textemdash{} younger studies that use stepwise
        demagnetisation techniques more likely to remove overprints and isolate
        a primary remanence (see Section~\ref{sec:puby}).
  \item Sampling strategy \textemdash{} need sufficient number of contributing
        samples and sites (N and B in Table~\ref{tab-fld}) to have a good chance
        of averaging out secular variation (see Section~\ref{sec:datcons}).
  \item Field tests that constrain the magnetisation age (see
        Section~\ref{sec:ageu}).
\end{itemize}

\subsubsection{Calculation Step}

Normally all paleopoles are treated equally when calculating the Fisher
mean~\citep[see how to calculate a Fisher mean using the formulas 6.12, 6.13,
6.14 and 6.15 in][chap. 6; note that instead of direction declination and
inclination expected in those formulas, pole longitude and latitude should be
used]{B92}, but as an alternative or supplement to filtering paleopoles can
potentially be weighted according to the factors that potentially influence
their precision and/or accuracy such as the A95 uncertainty or age uncertainty,
and a weighted Fisher mean calculated, giving the `better' paleopoles more
influence on the result mean pole. So, for example, each paleopole has got a
weight before calculating for a mean. In this thesis these weights $w_i$
are integrated into the $\sum\limits_{i=1}^N l_i$, $\sum\limits_{i=1}^N m_i$ and
$\sum\limits_{i=1}^N n_i$ of the formulas 6.13 and 6.14 in~\citet[chap.~6]{B92},
where $w$ means weight and $i$ is the same as in those two formulas. Then for a
weighted Fisher mean, the two formulas 6.13 and 6.14 in~\citet[chap.~6]{B92}
become
%
\begin{equation}
  l = \frac{\sum\limits_{i=1}^N l_{i}w_i}{R} \quad\quad
  m = \frac{\sum\limits_{i=1}^N m_{i}w_i}{R} \quad\quad
  n = \frac{\sum\limits_{i=1}^N n_{i}w_i}{R}
\end{equation}
%
and
%
\begin{equation}
  R = {\left( \sum\limits_{i=1}^N l_{i}w_i \right)}^2
    + {\left( \sum\limits_{i=1}^N m_{i}w_i \right)}^2
    + {\left( \sum\limits_{i=1}^N n_{i}w_i \right)}^2
\end{equation}
%
Note that the trade-offs and difficulties are similar for filtering, e.g.\
giving poles with low A95 more weight presumes that they are closer to the
actual mean, which is not necessarily the case. Therefore, weighing on different
paleomagnetic factors is hoped to help us have some insight into these complex
issues.

\subsection{Existing Solutions and General Issues}\label{sec:si}

As mentioned in the concluding paragraph of Section~\ref{sec:f2}, if not all
paleopoles are created equal, the question becomes: how should paleopoles of
varying quality be combined? For a certain set of paleopoles, how can we produce
an APWP that is both
\begin{enumerate}
  \item well constrained: the mean pole for each time step has a low spatial
  uncertainty (small 95\% confidence ellipse), and
  \item accurate: the mean pole position is close to its actual position at each
  time step.
\end{enumerate}

Previous work on this question have focussed on the filtering out so-called
``bad'' data before calculation of the mean pole. Commonly used schemes include:
\begin{enumerate}
  \item~\citet{v90} (V90). V90 includes seven criteria (see the details in the
  concluding paragraph of Section~\ref{sec:f2}). The Q (quality) factor is the
  total number of criteria satisfied (0\textendash7)~\citep{v88}. The Q factor
  is a very straightforward way to get a quantitative reliability score, and is
  widely used for filtering paleopoles prior to calculating
  APWPs~\citep[e.g.][]{T12,Ma16,F19}. Further each paleopole can also be
  conveniently weighted in proportion to its Q factor in the calculations of
  APWPs~\citep{T92}. But at the same time this is a fairly basic filter that
  lumps together criteria that may not be equally important.
  \item~\citet{B02} (BC02). BC02 data quality criteria use only paleopoles with
  $\alpha$95 less than 10\degree\ in the Cenozoic and 15\degree\ in the
  Mesozoic, derived from at least 36 samples from at least 6 sampling sites (see
  also the concluding paragraph of Section~\ref{sec:f2}). While straightforward and
  convenient to apply, these stringent criteria mean some useful data may be
  filtered out and wasted, especially for a period where there are only limited
  number of paleopoles.
  \item~\citet{S05} (SS05). SS05 is similar to, but less stringent than, BC02.
  SS05 uses only paleopoles with $\alpha$95 of $\leq$15\degree\ and an age
  uncertainty of $\leq$40 Myr, derived from at least $\geq$4 sampling sites with
  $\geq$4 samples/site, and at least some sample demagnetisation (See also the
  last paragraph of Section~\ref{subp:ss05}).
\end{enumerate}

Although they are often used, and effect the path and uncertainties of the
resulting APWP, there has been limited study of how effective these filtering
methods are at reconstructing a `true' APWP compared to unfiltered data.
Furthermore, the focus on the filtering stage ignores the possible impact of
binning/windowing (i.e.\ combining paleopoles through moving-averaging) and
weighting. This study more rigorously investigates the effects of the choices
made at every stage on the resulting APWP\@. By focussing on paleomagnetic data
from the last $\sim$120 Myr, where plate motions are independently constrained
by reconstructions of seafloor spreading tied into a hotspot reference frame, we
can also verify which choices actually lead to a better-constrained and accurate
record of past plate motions.


\section{Methods}

\subsection{General Approach}

In this study, we use paleopoles extracted from the GPMDB4.6b~\citep{M96,P05} to
generate APWPs for the period 120\textendash0 Ma. A range of possible APW paths
for North America, India and Australia can be generated from the extracted sets
of paleopoles using various binning, filtering and weighting methods
(Table.~\ref{tab-pick} and~\ref{tab-weit}). These paths can then be compared to
synthetic APWPs independently generated from an absolute plate motion model. The
three plates chosen have different attributes, both in terms of the input data
set and the nature of the reference APWP\@.

\subsection{Paleomagnetic Data}

\subsubsection{GPMDB Field Codes}

Data analysis includes manipulation of data fields/columns in the GPMDB\@. In
the following content, the codes of the several specific fields used will be
referred to. They are listed in Table~\ref{tab-fld} for easy reference.

\begin{table}
\centering
\caption{List of the used fields and field codes of the GPMDB.}\label{tab-fld}
\begin{tabular}{p{0.16\linewidth} p{0.79\linewidth}}
\toprule
Field Code & Meaning \\ \midrule
LOMAGAGE & Lower best estimate of the magnetic age of the magnetisation component \\
HIMAGAGE & Upper best estimate of the magnetic age of the magnetisation component \\
B & Number of sites \\
N & Number of samples \\
ED95 & Radius of circle of 95\% confidence about mean direction, i.e. $\alpha$95 \\
EP95 & Radius of circle of 95\% confidence about paleopole position \\
KD & Fisher precision parameter for mean direction \\
DP & Half-angle of confidence on the pole in the direction of paleomeridian \\
DM & Half-angle of confidence on the pole perpendicular to paleomeridian \\
K\_NORM & Fisher precision parameter for Normal directions \\
ROCKNAME & Name of sample rock \\
ROCKTYPE & Type of sample rock \\
STATUS & Indicates if results have been superseded \\
COMMENTS & General information including details of the origin of LOMAGAGE and
           HIMAGAGE\@. If the result is a combined pole this field contains information on the data included in the combined result \\
\bottomrule
\end{tabular}
\end{table}

\subsubsection{Paleomagnetic Data of Three Representative Continents}

Collections of paleopoles with a minimum age (LOMAGAGE) $\leq$135 Ma for the
North American, the Indian and Australian
plates, were extracted from the GPMDB\@. In order to include valid paleopoles
from blocks that moved independently prior to 120 Ma, which therefore have
different assigned plate codes in the GPMDB, the spatial join
technique~\citep{J07} was used to find all poles within the geographic region
that defines the rigid plate within the period of interest (see
Appendix~\ref{appen4chp3} for details):

\paragraph{For North America,}
the search region was defined by the North American craton (NAC), Avalon/Acadia
and Piedmont blocks, as defined by the recently published plate model
of~\citet{Y18}. Following extraction, 58 palepoles from southwestern North America
that have been affected by regional rotations since 36 Ma~\citep{Mc06}, were
removed. The final dataset consists of 135 paleopoles (Fig.~\ref{fig-120NAhist}),
with 76 of them (${\sim}56$\%; average age uncertainty ${\sim}14$ Myr, average
EP95 ${\sim}9.3$\degree) sampled from dominantly igneous sequences, 56
(${\sim}42$\%; average age uncertainty ${\sim}27.5$ Myr, average EP95 ${\sim}10.5$\degree)
sampled from mostly sedimentary sequences, including 6 from redbeds, and 3
(${\sim}2$\%) from metamorphic sequences. The principal features of the age
distribution are a larger number of young ($<$5 Ma) poles, and relatively fewer
poles in the Late Cretaceous and Miocene.

\begin{figure}
  \centering
  \includegraphics[width=.87\textwidth]{../../paper/tex/GeophysJInt/figures/120NAhist.pdf}
  \caption[Distribution of 120\textendash0 Ma North American
paleopoles]{Temporal distribution of 120\textendash0 Ma North American
paleopoles in 10/5 Myr window/step length. For distribution a, each bin only
counts in the midpoints (circles) of pole uncertainty bars (not including those
right at bin edges); For distribution b, as long as the bar intersects with the
bin (not including those intersecting only at one of bin edges), it is counted
in. Inside the parentheses, i means igneous rocks derived (red bars; only two
poles, 83\textendash77 Ma and 80\textendash65 Ma, from igneous and also
sedimentary; only one pole, 72\textendash40 Ma, from igneous and also
metamorphic), r means sedimentary rocks with redbeds involved derived (orange
bars), and m means metamorphic rocks derived (blue bars); the left are
non-redbed sedimentary rocks derived (black bars; only two poles,
146\textendash65 Ma [RESULTNO 6679] and 2\textendash0 Ma [RESULTNO 1227], are
from sedimentary and also metamorphic). The data published before 1984 are shown
as circles with a dot.}\label{fig-120NAhist}
\end{figure}

\paragraph{For India,}
the Indian block as defined by~\citet{Y18} was used, but following extraction 31
paleopoles associated with parts of the northern margin that have undergone
regional rotations since the Jurassic~\citep{G15} were removed. The final
dataset consists of 75 paleopoles (Fig.~\ref{fig-120INhist}), with 39 of them
(52\%; average age uncertainty ${\sim}5$ Myr, average EP95 ${\sim}7.7$\degree)
sampled from dominantly igneous sequences and 36 (48\%; average age uncertainty
${\sim}14$ Myr, average EP95 ${\sim}7$\degree) sampled from mostly sedimentary
sequences, including 3 from redbeds. There is a high concentration of poles from
the latest Cretaceous\textendash{}Early Cenozoic (${\sim}70$\textendash60 Ma),
many of which are igneous; in younger and older intervals, there are fewer,
mostly sedimentary poles.

\begin{figure}
  \centering
  \includegraphics[width=.95\textwidth]{../../paper/tex/GeophysJInt/figures/120INhist.pdf}
  \caption[Distribution of 120\textendash0 Ma Indian paleopoles]{Temporal
distribution of 120\textendash0 Ma Indian paleopoles. For red bars, only one
pole, 67\textendash64 Ma (RESULTNO 8593), is from igneous and also sedimentary.
See Fig.~\ref{fig-120NAhist} for more information.}\label{fig-120INhist}
\end{figure}

\paragraph{For Australia,}
the Australia, Sumba, and Timor blocks as defined by~\citet{Y18} were used, in
combination with data from the Tasmania block younger than ${\sim}100$ Ma (with
a maximum age (HIMAGAGE) $\leq$100 Ma), prior to which it was not fixed with
respect to Australia~\citep{Y18}. The final dataset consists of 99 paleopoles
(Fig.~\ref{fig-120AUhist}), with 61 of them (${\sim}62$\%; average age
uncertainty ${\sim}23.5$ Myr, average EP95 ${\sim}10.9$\degree) sampled from
dominantly igneous sequences, and 38 (${\sim}38$\%; average age uncertainty
${\sim}23$ Myr, average EP95 ${\sim}9.4$\degree) sampled from mostly sedimentary
sequences, including 9 from redbeds. The temporal distribution of poles is
relatively uniform.

\begin{figure}
  \centering
  \includegraphics[width=.93\textwidth]{../../paper/tex/GeophysJInt/figures/120AUhist.pdf}
  \caption[Distribution of 120\textendash0 Ma Australian paleopoles]{Temporal
distribution of 120\textendash0 Ma Australian paleopoles. For black bars,
only four poles, 100\textendash80 Ma (RESULTNO 1106), 10\textendash2 Ma
(RESULTNO 1208), 4\textendash2 Ma (RESULTNO 140) and 1\textendash0 Ma (RESULTNO
1963), are from sedimentary and also igneous. For red bars, only one pole,
65\textendash25 Ma (RESULTNO 1872), is from igneous and also sedimentary rocks,
and only one pole, 1\textendash0 Ma (RESULTNO 1147), is from igneous and also
metamorphic rocks. See Fig.~\ref{fig-120NAhist} for more
information.}\label{fig-120AUhist}
\end{figure}

\bigskip
Compared with North American (Fig.~\ref{fig-120NAhist}) and Australian
(Fig.~\ref{fig-120AUhist}) paleopoles, Indian paleopoles have a relatively lower
density and a higher prevalence of sedimentary paleopoles, except during the
period of ${\sim}70$\textendash60 Ma (Fig.~\ref{fig-120INhist}). For North
America and India, sedimentary paleopoles have significantly larger average age
uncertainty, but about same for Australia, there is no igneous and sedimentary
${\alpha}95$ difference.

\subsection{APWP Generation}\label{sec:apwpg}

Multiple APWPs were generated using the selected poles for each of the three
plates as follows:

\paragraph{Picking/binning.} A moving average or moving window technique was
used: paleopoles were selected for each APWP time step (initially 5 Myr step
length from 0 to 120 Ma) if their age fell within a window centered on the
current step age. In this study, the width of the moving window was always twice
that of time step (i.e.\ initially 10 Myr), such that each window half-overlaps
with its neighbours.

\paragraph{Filtering.} Poles with characteristics thought to correspond to
poor data quality, or lacking characteristics thought to correspond to good
data quality, were discarded (or in some cases, corrected).

\paragraph{Weighting.} Calculation of a weighted Fisher mean~\citep{F53} of
the remaining poles within each window, using weighting functions intended to
increase the influence of higher quality poles relative to lower quality ones.

\bigskip
Twenty-eight different picking and filtering algorithms were tested
(Table~\ref{tab-pick}, referred to hereafter as Pk), in combination with 6
different weighting algorithms (Table~\ref{tab-weit}, referred to hereafter as
Wt), for the three plates. The effects of changing the time step length and
width of the moving window, and the reference path, were also examined.

\subsubsection{Picking/Binning}\label{sec:p}

In studies where the moving window method is used to calculate an
APWP~\citep{T99,T08}, a paleopole is generally considered to fall in the current
window only if the mid-point of its age limits fall within that window. If
paleopole has a large age uncertainty compared to the size of the moving window,
it will not be included in the moving windows close to the beginning and end of
the age range, which are arguably more likely magnetisation ages than the
mid-point. To investigate this issue, we compare the performance of moving
windows populated using the mid-point picking criterion, referred to hereafter
as ``Age Mean Picking'' (AMP\@; even-numbered algorithms in
Table~\ref{tab-pick}, Fig.~\ref{fig-dif} and subsequent figures), to a less
restrictive picking criterion where a paleopole is included in the current
moving window if any part of its age limits falls within that window, referred
to hereafter as ``Age Position Picking'' (APP\@; odd-numbered algorithms in
Table~\ref{tab-pick}, Fig.~\ref{fig-dif} and subsequent figures). The APP
algorithm will pick more paleopoles in each moving window than the AMP algorithm
(Figs.~\ref{fig-120NAhist}-\ref{fig-120AUhist}; Fig.~\ref{fig-nac-maplat}).

If, for example, we have a paleopole with an acquisition age of 10\textendash20
Ma, and we have a 2 Myr moving window with a 1 Myr age step, then it is included
in just the 14\textendash16 Ma window (for the mid-point age of 15 Ma) for
AMP\@. For APP, this paleopole falls in the 9\textendash11, 10\textendash12,
11\textendash13, 12\textendash14 \ldots 17\textendash19, 18\textendash20 and
19\textendash21 Ma windows. Each original paleopole is therefore represented in
the mean poles calculated over its entire possible acquisition age.

\begin{figure}[!ht]
  \centering
  \includegraphics[width=.7\textwidth]{../../paper/tex/GeophysJInt/figures/binning.pdf}
  \caption[How moving average methods work]{An example of 10 Myr moving window
and 5 Myr step in the two moving average methods, AMP and APP, based on
paleopoles of the $NAC$. White circles are the midpoints of low and high
magnetic ages. The vertical axis has no specific meaning here. For example, for
the window of 15 Ma to 5 Ma (the dashed-line bin), the AMP method calculates the
Fisher mean pole (dark triangle in Fig.~\ref{fig-mhsPred}) of only 5 paleopoles,
while the APP method calculates the mean pole (dark circle in
Fig.~\ref{fig-mhsPred}) of 9 paleopoles.}\label{fig-nac-maplat}
\end{figure}

A shorter step and narrower window will potentially increase the clustering of
the selected paleopoles, but will reduce their number. Conversely, a longer step
and wider window will increase the number of poles, but potentially decrease
their clustering. To investigate these trade-offs, every picking/filtering and
weighting method was also used to generate APWPs with a time step and window
doubled to 10 Myr and 20 Myr, respectively. Paths generated using AMP and APP
with no filtering, and every weighting method, with time steps from 1 Myr to 15
Myr in 1 Myr increments and window widths from 2 Myr to 30 Myr in 2 Myr
increments, were also analysed. In all cases the oldest time step was 120 Ma.

\subsubsection{Filtering}\label{sec:f}

14 different filters or corrections (Table~\ref{tab-pick}) were applied to both
data picked using the AMP moving window method (even numbers) and data picked
using the APP moving window method (odd numbers), resulting in a total of 28
unique picking algorithms. The filters or corrections can be characterised as
follows:

\begin{table}[!ht]
\centering
\caption{List of all picking/binning algorithms developed here.}\label{tab-pick}
\begin{tabular}{@{}ll@{}}
\toprule
No. & Picking Algorithm \\ \midrule
0 & AMP\@: Age Mean Picking, see Section~\ref{sec:apwpg} \\
1 & APP\@: Age Position Picking \\
2 & AMP (``$\alpha$95/Age range'' no more than ``15/20'') \\
3 & APP (``$\alpha$95/Age range'' no more than ``15/20'') \\
4 & AMP (mainly or only igneous) \\
5 & APP (mainly or only igneous) \\
6 & AMP (contain igneous and not necessarily mainly) \\
7 & APP (contain igneous and not necessarily mainly) \\
8 & AMP (unflatten sedimentary) \\
9 & APP (unflatten sedimentary) \\
10 & AMP (nonredbeds) \\
11 & APP (nonredbeds) \\
12 & AMP (unflatten redbeds) \\
13 & APP (unflatten redbeds) \\
14 & AMP (published after 1983) \\
15 & APP (published after 1983) \\
16 & AMP (published before 1983) \\
17 & APP (published before 1983) \\
18 & AMP (exclude commented local rot or secondary print) \\
19 & APP (exclude commented local rot or secondary print) \\
20 & AMP (exclude local rot or correct it if suggested) \\
21 & APP (exclude local rot or correct it if suggested) \\
22 & AMP (filtered using SS05 palaeomagnetic reliability criteria) \\
23 & APP (filtered using SS05 palaeomagnetic reliability criteria) \\
24 & AMP (exclude superseded data already included in other results) \\
25 & APP (exclude superseded data already included in other results) \\
26 & AMP (comb of 22 and 24) \\
27 & APP (comb of 23 and 25) \\ \bottomrule
\end{tabular}
\raggedright{\\Notes: SS05,~\citet{S05}}
\end{table}

\paragraph{No modification (Pk 0,1).}

\paragraph{Removal of poles with large spatial and temporal uncertainties
(Pk 2,3).} Paleopoles with both large $\alpha$95 (ED95 $>15\degree$,
following the BC02 threshold for the Mesozoic) and wide acquisition age limits
(difference between HIMAGAGE and LOMAGAGE $>20$ Myr, following the V90 criteria
about age within a half of a geological period; the average of the geological
periods between 120 and 0 Ma [Quaternary, Neogene, Paleogene and Cretaceous] is
about 20 Myr) which are less likely to provide a good estimate of the actual
pole position within any specific age window, were excluded.

\paragraph{Prefer poles from igneous rocks (Pk 4,5, 6,7).} Pk 4,5 removes
paleopoles potentially affected by inclination flattening by selecting only
paleopoles coded as igneous or mostly igneous (ROCKTYPE starting with
``intrusive'' or ``extrusive''). In fact, most of the paleopoles picked by Pk
4,5 are derived from igneous-only rocks. Pk 6,7 select paleopoles coded as
containing igneous (ROCKTYPE containing ``intrusive'' or ``extrusive''); this is
a less strict filter, because the dominant rock type could potentially be
another lithology. Therefore Pk 6 also includes paleopoles from Pk 4, and Pk 7
also includes paleopoles from Pk 5.

\paragraph{Correct sedimentary poles for inclination shallowing (Pk 8,9).}
Rather than excluding paleopoles from sedimentary rocks, paleopoles coded as
sedimentary or redbeds were instead corrected for inclination flattening using
the flattening function $\tan I_o = f \tan I_f$~\citep{K55}, where $I_o$ is the
observed inclination, $I_f$ is the unflattened inclination, and $f$ is the
flattening factor (also known as shallowing coefficient; 1=no flattening,
0=completely flattened). Here $f=0.6$ is used in all cases,
following~\citet{T12}, unless the rock type (ROCKTYPE field in the database) is
not sedimentary dominated but contains sedimentary rocks. In these cases,
$f=0.8$ is used instead, following the minimum anisotropy-of-thermal-remanence
determined f-correction~\citep{D11,Do11}.

\paragraph{Remove redbeds (Pk 10,11) or correct them for inclination
shallowing (Pk 12,13).} Bias toward shallow inclinations is also observed
in paleomagnetic data derived from red-beds~\cite[e.g., in central Asia,
Mediterranean region, North America, etc.]{T04,K04,T07,B10}. This bias can be
addressed by removing the source (Pk 10,11; ROCKTYPE containing redbeds), or
correcting for inclination flattening, setting $f=0.6$ as previously described
(Pk 12,13). In the latter case, the assumption is being made that the
redbeds are carrying a detrital paleomagnetic signal.

\paragraph{Prefer poles from younger (Pk 14,15, 24,25) or older (Pk 16,17)
studies.} Advancements in equipment (e.g., cryogenic magnetometers) and
analytical techniques (e.g., stepwise demagnetisation) mean that more recently
published paleopoles are potentially more reliable than older ones. Pk 14,15
removes any paleopoles published prior to 1983 (YEAR $>$ 1983), the mean
publication date for paleopoles in the GPMDB\@. Pk 24,25 takes a similar but
less aggressive approach by excluding paleopoles that have been superseded (99
paleopoles) by later studies from the same sequence, which are presumed to
represent a more accurate determination of the paleopole position. The
``STATTUS'' field of the GPMDB4.6b indicates if a paleopole has been superseded.
Conversely, removing paleopoles published after 1983 (Pk 16,17; YEAR $\leq$
1983) should have a negative effect.

\paragraph{Exclude suspected local rotations and secondary overprints (Pk
18,19), or correct for them where possible (Pk 20,21).} Secondary remanence
components and local tectonic deformation can both displace the measured pole
position away from its ``true'' position. Such poles can be identified based on
demagnetisation data, or comparison to the pre-existing APWP\@. Pk 18,19
removes paleopoles that were identified as such in the COMMENTS field: 66
paleopoles affected by local rotations are identified by carefully going through
all the COMMENTs containing information about rotation; paleopoles affected by
secondary overprints are extracted with the keyword ``econd'' identified in the
COMMENTS\@. A subset (19) of the 66 paleopoles identified have a suggested
correction associated with them; Pk 20,21 retains these paleopoles after
applying the suggested correction.

\paragraph{SS05 quality criteria (Pk 22,23).}\label{subp:ss05} As with Pk 2,3,
SS05~\citep{S05} removes paleopoles with high spatial ($\alpha$95 $>15\degree$)
and temporal (age range $>$ 40 Myr) uncertainty, but additionally remove
paleopoles where samples had poor sampling coverage (sampling sites' quantity
[B] of $<4$, samples' quantity [N] of $<4$ times of the sites [B]) and were not
subjected to even a blanket demagnetisation treatment (laboratory cleaning
procedure code DEMAGCODE $<2$). Pk 26,27 also use these criteria, but
further excludes superseded data.

\bigskip
Some of the picking (Table~\ref{tab-pick}) and weighting (Table~\ref{tab-weit})
methods developed here are also connected with the V90 Q factor (see
Section~\ref{sec:si}). For example, Pk 2,3 and Wt 2, 4, 5 are related to the V90
criteria 1; Pk 2,3, 22,23, 26,27 and Wt 1, 3 are related to the V90 criteria 2;
Pk 22,23 are related to the V90 criteria 3; The data constraining described in
Appendix~\ref{appen4chp3} is related to the V90 criteria 5; and Pk 18,19 are
related to the V90 criteria 7.

\subsubsection{Weighting}\label{sec:w}

Following filtering, weights were assigned to each of the remaining paleopoles
using one of the following six algorithms (Table~\ref{tab-weit}), prior to
calculation of a weighted Fisher mean:

\paragraph{No weighting (Wt 0).} Weighting factor=1 for all
paleopoles.
%
\begin{longtable}[h]{@{}c|l@{}}
\caption{List of all weighting algorithms developed here.}\label{tab-weit}
\\\hline\multicolumn{1}{|p{.25in}|}{\textbf{No.}} & \multicolumn{1}{p{5.5in}|}{\textbf{Weighting Algorithm}} \\
\hline\endfirsthead%
\multicolumn{2}{r}{{\bfseries \tablename\ \thetable{} --- continued from previous page}} \\ \hline
\multicolumn{1}{|p{.25in}|}{\textbf{No.}} & \multicolumn{1}{p{5.5in}|}{\textbf{Weighting Algorithm}} \\ \hline
\endhead%
\hline\multicolumn{2}{|r|}{{\bfseries continued on next page}} \\ \hline
\endfoot\hline
\endlastfoot0 & None (No weighting) \\ \hline
1 & Larger numbers of sites (B) \& observations (N), greater $weight$ ($w$): \\
  & \begin{minipage}{5.5in}\begin{equation*}w=\left\{\begin{array}{ll}
    0.2 & \textrm{, if both B \& N are missing, or B$\leq1$ \& N$\leq1$} \\
    (1-\frac{1}{B})\times0.5 & \textrm{, if N is missing or N$\leq1$, \& B$>$1} \\
    (1-\frac{1}{N})\times0.5 & \textrm{, if B is missing or B$\leq1$, \& N$>$1} \\
    (1-\frac{1}{B})\times(1-\frac{1}{N}) & \textrm{, if B$>$1 \& N$>$1}
\end{array}\right.\end{equation*}\end{minipage} \\ \hline
2 & Lower age uncertainty, greater weight: \\
  & \begin{minipage}{5.5in}age\_range=HIMAGAGE-LOMAGAGE \\
    age\_midpoint = (HIMAGAGE+LOMAGAGE)$\times$0.5 \\
    if age\_midpoint$<$2.58 (Ma; start of the Quaternary, according to GSA Geologic Time Scale), \\
    \vbox{\begin{equation*}w=\left\{\begin{array}{ll}
    1 & \textrm{, if age\_range$\leq$1.29 (from $\frac{2.58\textendash0}{2}$)} \\
    \frac{1.29}{age\_range} & \textrm{, if age\_range$>$1.29}
    \end{array}\right.\end{equation*}} \\
    if 2.58$\leq$age\_midpoint$<$23.03 (Ma; Neogene), \\
    \vbox{\begin{equation*}w=\left\{\begin{array}{ll}
    1 & \textrm{, if age\_range$\leq$10.225 (from $\frac{23.03\textendash2.58}{2}$)} \\
    \frac{10.225}{age\_range} & \textrm{, if age\_range$>$10.225}
    \end{array}\right.\end{equation*}} \\
    if 23.03$\leq$age\_midpoint$<$201.3 (Ma; Paleogene, Cretaceous, Jurassic), \\
    \vbox{\begin{equation*}w=\left\{\begin{array}{ll}
    1 & \textrm{, if age\_range$\leq$15} \\
    \frac{15}{age\_range} & \textrm{, if age\_range$>$15}
    \end{array}\right.\end{equation*}}
    \end{minipage} \\ \hline
3 & Lower $\alpha$95, greater weight: \\
  & \begin{minipage}{5.5in}Positive half Normal distribution with a mean and standard deviation \\
    of 0 and 10, scaled with $10\sqrt{2\pi}$ (to make the peak reach 1) \\
    \vbox{\begin{equation*}w=e^{-\frac{\alpha_{95}^2}{200}},\end{equation*}} \\
    where \\
    \vbox{\begin{equation*}\alpha95=\left\{\begin{array}{ll}
    ED95 & \\
    DP & \textrm{, if ED95 is missing} \\
    \frac{140}{\sqrt{KD\times{}N}} & \textrm{, if ED95 \& DP are missing} \\
    \frac{140}{\sqrt{K\_NORM\times{}N}} & \textrm{, if ED95, DP \& KD are missing} \\
    \frac{140}{\sqrt{K\_NORM\times{}B}} & \textrm{, if ED95, DP, KD \& N are missing} \\
    \frac{140}{\sqrt{1.7\times{}B}} & \textrm{, if ED95, DP, KD \& K\_NORM are missing,} \\
    & \textrm{using the lowest KD value ${\sim}1.7$ in GPMDB4.6b,}
    \end{array}\right.\end{equation*}}
    finally $w$=0 if this $\alpha$95 completely overlaps with another smaller
    $\alpha$95 whose paleopole is exactly derived from the same place and same
    rock.
    \end{minipage} \\ \hline
4 & Age uncertainty Position to bin (more overlap, greater weight): \\
  & \begin{minipage}{5.5in}wha, window high age; wla, window low age \\
    \vbox{\begin{equation*}w=\left\{\begin{array}{ll}
    \frac{wha-LOMAGAGE}{age\_range} & \textrm{, if LOMAGAGE$>$wla \& HIMAGAGE$>$wha} \\
    \frac{HIMAGAGE-wla}{age\_range} & \textrm{, if LOMAGAGE$<$wla \& HIMAGAGE$<$wha} \\
    \frac{wha-wla}{age\_range} & \textrm{, if LOMAGAGE$<$wla \& HIMAGAGE$>$wha} \\
    1 & \textrm{, if LOMAGAGE$>$wla \& HIMAGAGE$<$wha}
    \end{array}\right.\end{equation*}}
    \end{minipage} \\ \hline
5 & Combining 3 and 4: average of the two weights from 3 and 4 \\
\end{longtable}
%
\paragraph{Weighting by sample and site number (Wt 1).} Paleopoles derived
from more individually oriented samples (observations; N) collected from more
sampling levels/sites (B) are more likely to average out secular variation and
accurately sample the GAD field~\citep{T19,B02,v90}, and are given a weighting
closer to 1. Unfortunately, in the GPMDB, a paleopole's B or N is not always
given. As shown in (Table~\ref{tab-weit} no. 1), in such cases the calculated
weights were modified to give lower weights overall whilst still accounting for
partial information, such as N being reported but not B.

For number of sampling sites B$>$1 and number of samples N$\leq$1, there are 8
such paleopoles for 120\textendash0 Ma North America, India 4, and Australia 1.
For B$\leq$1 and N$>$1, there are 20 such paleopoles for 120\textendash0 Ma
North America, India 26, and Australia 22. For B$\leq$1 and N$\leq$1, there are
only 23 such paleopoles for the whole GPMDB 4.6b, including 18 with neither B
value nor N value given; while for 120\textendash0 Ma there is no such paleopole
found in North America, India and Australia.

\paragraph{Weighting by age uncertainty (Wt 2)} Above a maximum age range
that represents a well-constrained age, defined as half of each geological
period in the Phanerozoic Eon (e.g., Quaternary, Neogene; here the geological
period that the middle point of the paleopole's age range falls within is
assigned)~\citep{v90,T19} or 15 Myr (the halves of the Paleogene, Cretaceous,
and Jurassic periods are all at least 20 Myr, which is large for these
relatively young geologic periods), whichever is smaller, paleopoles are given
an increasingly small weight as the age uncertainty (the high magnetic age $-$
the low magnetic age) increases (No. 2 in Table~\ref{tab-weit}).

\paragraph{Weighting by spatial uncertainty (Wt 3).} Paleopoles with a
smaller $\alpha$95 confidence ellipse are given a higher weighting than those
with a larger $\alpha$95, using a Gaussian/Normal distribution centered on 0
with standard deviation of 10. However, a paleopole's $\alpha$95 is not always
given in the database. If $\alpha$95 is not given, DP (the semi-axis of the
confidence ellipse along the great circle path from site to pole) is assigned as
$\alpha$95. If DP is not given either, $\alpha$95 was further approximated by
$\frac{140}{\sqrt{KD\times{}N}}$, where KD is Fisher precision parameter for
mean direction if this parameter is given, or Fisher precision parameter for
Normal directions (K\_NORM) if only K\_NORM is given when KD is missing. If N is
not given, B is used as N. If even K\_NORM is also missing, the lowest KD value
${\sim}1.7$ in GPMDB 4.6b is used as KD\@. It is also worthwhile to mention that
if specimens, where two paleopoles are derived, are exactly from the same place
and same rock (by checking if ROCKNAME, ROCKTYPE and sampling site are the
same), and one $\alpha$95 is completely inside the other $\alpha$95, a zero is
assigned as the weight of the data with the larger $\alpha$95. In fact, in the
above described procedure A95 (circle of 95\% confidence about mean pole) is a
better alternative instead of $\alpha$95, because A95 is directly reflecting the
spatial uncertainty of the paleopoles. However, most paleopoles' A95s are not
given in GPMDB 4.6b, so $\alpha$95 is used instead since $\alpha$95 is also
indirectly reflecting the quality of the paleopole.

\paragraph{Weighting by degree of overlap between moving window and pole age
uncertainty (Wt 4).} If a large fraction of the age range for an individual
paleopole falls within the current window, it is given a higher weighting than a
pole where the overlap is smaller, because it is more likely to be close to the
true pole position in the window interval. In other words, if window intersects
with part of age range, weight= (intersecting part) / (age range width).

\paragraph{Weighting by both spatial and temporal uncertainty (Wt 5).} This
weighting method is a combination of Wt 3 (but here the standard deviation of
the weighting function is 15 though) and Wt 4. It takes the average of sums of
the weights generated by Wt 3 and 4.

\subsubsection{Scaling of Weights}

The weights obtained from the six different weighting functions
(Table~\ref{tab-weit}) are then integrated into Fisher mean function~\citep{F53}
to calculate a weighted Fisher mean. First, weight values are used to scale
Cartesian x, y and z components of each individual paleopole's geographic
coordinate. Then these individual coordinates are combined through Fisher
resultant vector $R$ function~\citep[see][chap.~11]{T19}. Therefore the mean
pole location, its spatial uncertainty A95, and precision parameter are all
weighted along with $R$.

Traditionally, weights are directly integrated by being multiplied with the
variable we would like to do weighting to. For example, here weights can be
directly multiplied with the Cartesian x, y and z components of each paleopole.
However, this direct multiplying causes the decreasing of $R$, which further
sensitively and extremely increases $\alpha$95 (Fig.~\ref{fig-alpha95}),
especially because $R$ is always less than $N$ and $N$ is usually not that high
(more than ${\sim}50$ is rather rare, around 10 averagely). The consequence
would be that all the $\alpha$95 ellipses of mean poles are extremely large in
size and difficult to be spatially differentiated.

\begin{figure*}
  \centering
  \begin{subfigure}{.49\textwidth}
    \includegraphics[width=\textwidth]{../../paper/tex/GeophysJInt/figures/alpha95_000082.png}
    \caption{perspective view a}
  \end{subfigure}
  \begin{subfigure}{.49\textwidth}
    \includegraphics[width=\textwidth]{../../paper/tex/GeophysJInt/figures/alpha95_000094.png}
    \caption{perspective view b}
  \end{subfigure}
  \caption[Visualization of equation that relates $\alpha$95, and R and
N]{Visualization of Equation 11.9 of Essentials of Paleomagnetism: Fifth Web
Edition, illustrating the relationship between the radius of the circle of 95\%
confidence ($p$=0.05) about the mean, $\alpha$95, resultant vector $R$ and
number of directions (or paleopoles) $N$. Note that R$<$N.}\label{fig-alpha95}
\end{figure*}

Therefore, here weights are scaled before being multiplied with the Cartesian x,
y and z components by
%
\begin{equation*}
  ScaledW_i=\frac{N \times W_i}{\sum\limits_{i=1}^{N} W_i},
\end{equation*}
%
where $N$ is number of paleopoles for making a mean pole. So the scaled weight
$ScaledW_i$ could be greater than 1 because it is actually scaled through being
divided by the mean of the weights. This scaling does not only keep the effect
of weighting but also avoid dramatically changing $R$ and indirectly and
extremely changing $\alpha$95.

\subsection{Reference Paths}\label{sec:refpath}

A prediction of the expected APWP for any plate can be generated using a plate
kinematic model (e.g.\ the last ${\sim}180$\textendash200 Myr of plate motions
reconstructed from spreading ridges in ocean basins) that is tied in to an
absolute reference frame. Here, we use the rotations of~\citet{O05}, which
describe motion of Nubian Plate relative to the Indo-Atlantic hotspots back to
120 Ma. North America is linked to this absolute frame of reference across the
Mid-Atlantic ridge, using North America-Nubia rotations from chron C1n
(0\textendash0.78 Ma) from~\citet{D10}, to chron C2An (2.7 Ma)
from~\citet{Sh12}, to C5n.1ny (9.74 Ma) from~\citet{M99}, to C5n.2o (10.949 Ma)
from~\citet{G13}, to C6ny (19.05 Ma) from~\citet{M99}, to C6no (20.131 Ma)
from~\citet{G13}, to C34ny (83.5 Ma) from~\citet{M99}, from C34ny to
${\sim}118.1$ Ma from~\citet{S12}, and to closure at C34no (120.6 Ma)
from~\citet{G13}. India is linked via the East African Rift Valley
(Somalia-Nubia rotations from chron C1n (0\textendash0.78 Ma) from~\citet{D17},
to C2A.2no (3.22 Ma) from~\citet{H05}, and to closure at C7.2m (25.01 Ma) and
chron C34 (85 Ma) from~\citet{R16}, and finally extended to 120 Ma because there
was no known relative motion between Somalia and Nubia from 120 Ma to 85 Ma
according to the rotations from~\citet{M16}). Australia is linked via the East
African Rift Valley, then the SW Indian Ridge (E Antarctica-Somalia rotations
from chron C1n (0\textendash0.78 Ma) from~\citet{D17}, to C2A.2no (3.22 Ma)
from~\citet{H05}, to C5n.2no (10.95 Ma) from~\citet{L02}, to C13ny (33.06 Ma)
from~\citet{P08}, to C29no (64.75 Ma) from~\citet{C10}, to C34y (83 Ma)
from~\citet{R16}, to 96 Ma from~\citet{M01}, and to closure at chron M0 (120.6
Ma) from~\citet{M08}), and SE Indian Ridge (Australia-East Antarctica rotations
from chron C1n (0\textendash0.78 Ma) from~\citet{D17}, to C6no (20.13 Ma)
from~\citet{C04}, to C8o (26 Ma) from~\citet{G18}, to C17n.3no (38.11 Ma)
from~\citet{C04}, to C34ny (83.5 Ma) from~\citet{Wh13}, to the Quiet Zone
Boundary (96 Ma) from~\citet{W07}, to full closure at 136 Ma from~\citet{Wh13}).
The geomagnetic polarity timescales of~\citet{C95} for Late Cretaceous and
Cenozoic, and of~\citet{Gr94} for Early to Late Cretaceous time are used to
convert from chron boundaries to absolute ages. A long table listing these
rotation parameters with covariance uncertainties (Table~\ref{tab:rot}) is
included in Appendix~\ref{appen4chp3}, and the calculated rotations for the
North American, Indian and Australian reference APWPs in the hotspot reference
(Figs.~\ref{fig-mhsPred}-\ref{fig-mhsPred801}) are also listed in
Table~\ref{tab:refAPWP}.

\begin{figure}[!ht]
  \centering
  \includegraphics[width=1.01\textwidth]{../../paper/tex/GeophysJInt/figures/fhs_m.pdf}
  \caption[120\textendash0 Ma MHM vs FHM predicted APWP of North America]{MHM
predicted 120\textendash0 Ma APWP (solid line) for North America through the
North America\textendash{}Nubia\textendash{}Mantle plate circuit. The FHM
predicted path (dashed line with shaded uncertainties) is also shown for
comparison. The age step is 5 Myr. Compared with the 10 Ma paleomagnetic mean
pole calculated by the AMP method (dark triangle), the coeval mean pole derived
from the APP method is closer to both FHM and MHM predicted 10 Ma poles, which
indicates more data diluting the effect of outliers. See also the paleopoles
that the two mean poles are composed of in
Fig.~\ref{fig-nac-maplat}.}\label{fig-mhsPred}
\end{figure}

Where possible, poles which were published uncertainty estimates were used.
Where no uncertainty estimates were available, values of the covariance matrix
were set to an arbitrarily small value (1E–15). Where this occurs, the spatial
uncertainties for the reference APWP are likely underestimated. However, but
because the uncertainties for the Nubia-hotspot rotations are substantially
larger than for rotations derived from fitting magnetic isochrons, the effect is
small.

To reconstruct a reference APWP at the required time steps for comparison with
the paleomagnetic APWPs, rotations and their associated uncertainties were
interpolated between constraining finite rotation poles according to the method
of~\citet{D08}, assuming constant rates.

Neither the hotspot reference frame nor the paleomagnetic reference frame are
truly fixed with respect to the solid Earth. In the former case, hotspots are
not truly stationary in the mantle~\citep{S98}; in the latter, true polar wander
(TPW) may also lead to differential movements of the solid earth with respect to
the spin axis~\citep{E03}. In reality, it is difficult to untangle these
effects. Whilst there is little clear evidence for significant TPW in the past
${\sim}120$ Myr~\citep{C00,R04}, modeling suggests that the effects of hotspot
drift can start to become significant over 80\textendash100 Myr
timescales~\citep{O05}. Because paleomagnetic APWPs have large associated
spatial uncertainties, a synthetic APWP calculated using a fixed-hotspot
reference frame is unlikely to deviate significantly from the `true' APWP, and
most comparison experiments use a fixed-hotspot model (FHM) reference path for
North America (Fig.~\ref{fig-mhsPred}), India (Fig.~\ref{fig-mhsPred501}) and
Australia (Fig.~\ref{fig-mhsPred801}). However, the full set of comparisons for
the 28 picking methods and 6 weighting methods was also run for reference paths
generated using the moving-hotspot model (MHM) rotations of~\citet{O05}, which
incorporate motions of the Indo-Atlantic hotspots relative to the mantle derived
from mantle convection modeling.

\begin{figure}[!ht]
  \centering
  \includegraphics[width=1.01\textwidth]{../../paper/tex/GeophysJInt/figures/fhs501_m.pdf}
  \caption[120\textendash0 Ma MHM vs FHM predicted APWP of India]{MHM predicted
120\textendash0 Ma APWP (solid line) for India through the
India\textendash{}Somalia\textendash{}Nubia\textendash{}Mantle plate circuit.
Its age step is 5 Myr. The dashed line is the FHM predicted path shown for
comparison. The inset shows paths for fast moving India and also much slower
moving North America shown in Fig.~\ref{fig-mhsPred}.}\label{fig-mhsPred501}
\end{figure}

When comparing the synthetic APW paths for the three plates (inset,
Fig.~\ref{fig-mhsPred801}), there are clear differences. The predicted mean
north pole for North America at 120 Myr is still at ${\sim}75^{\circ}$N
(Fig.~\ref{fig-mhsPred}), indicating rather slow drift with respect to the spin
axis; this is due to a large component of the North American plate's absolute
motion in the past 120 Myr being to the east. In contrast, the rapid northward
motion of the Indian plate in the same period, particularly prior to its
collision with Asia at ${\sim}50$\textendash55 Ma~\citep{N10}, is reflected by
the 120 Ma predicted mean north pole being located at ${\sim}20^{\circ}$N
(Fig.~\ref{fig-mhsPred501}). Australia represents an intermediate case, with
north westerly plate motion from ${\sim}120$\textendash60 Myr changing to more
rapid northward motion from ${\sim}60$\textendash55 Ma to the present~\citep{W07}.
When comparing the FHM and MHM tracks, significant differences in the oldest
parts (before ${\sim}80$ Ma) are apparent for India and North America.

\begin{figure}[!ht]
  \centering
  \includegraphics[width=1.01\textwidth]{../../paper/tex/GeophysJInt/figures/mhs801.pdf}
  \caption[120\textendash0 Ma MHM vs FHM predicted APWP of Australia]{MHM
predicted 120\textendash0 Ma APWP (solid line) for Australia through the
Australia\textendash{}East
Antarctica\textendash{}Somalia\textendash{}Nubia\textendash{}Mantle plate
circuit. Its age step is 5 Myr. The dashed line is the FHM predicted path shown
for comparison. The inset shows paths for fast moving India shown in
Fig.~\ref{fig-mhsPred501}, much slower moving North America shown in
Fig.~\ref{fig-mhsPred}, and also relatively intermediate moving
Australia.}\label{fig-mhsPred801}
\end{figure}

These differences in the reference path due to different plate kinematics is
another variable that may affect the performance of the different weighting
algorithm for different plates, in addition to the distribution and type of the
contributing mean poles used to generate the paleomagnetic APWPs.

\subsection{Comparison Algorithm}

Comparisons between APWPs generated using different picking and weighting
algorithms and the synthetic reference APWPs were performed using the composite
path difference ($\mathcal{CPD}$) algorithm described in
Chapter~\ref{chap:Metho}, with equal weighting given to the spatial, length and
angular differences (i.e., W$_s$ = W$_l$ = W$_a$ = $\frac{1}{3}$). A lower
$\mathcal{CPD}$ score generally indicates a `better' fit, although a lower score
can also potentially result from comparison to a more poorly constrained path
with large uncertainties, which are less likely to be significantly different.
An additional `Fit Quality' (FQ) metric can help to distinguish such cases, by
assigning the two paths being compared a letter score based on the average size
of their uncertainty ellipses (Chapter~\ref{sec:FQ}). Here, the first letter
refers to the generated APWP, and the second letter refers to the reference
path. In this study, the reference path FQ score is fixed for each plate;
because the spatial uncertainties for paths generated from plate motion models
are small compared to those typical for paleomagnetic data, this reference path
FQ is always rated `A'.

This does not only help find the most similar paleomagnetic APWP (from the best
algorithm) to the reference APWP, but also help further test and demonstrate the
validity of the similarity measuring tool in practise.

\section{Results}

\subsection{Baseline Results: 10/5 Myr Window/Step, Fixed-hotspot
Reference}\label{sec:base}

Fig.~\ref{fig-dif} shows the $\mathcal{CPD}$ scores for the APWPs generated with
all 28 picking methods (AMP and APP with one of 14 separate filters applied) and
one of 6 weighted mean calculations then applied, compared to the FHM reference
path (squares and dashed line in Figs.~\ref{fig-mhsPred}-\ref{fig-mhsPred801})
for North America (Fig.~\ref{fig-na-dif}), India (Fig.~\ref{fig-in-dif}) and
Australia (Fig.~\ref{fig-au-dif}). The 27 lowest and highest of the 168
scores for each plate (values greater than 1 standard deviation from the mean)
are marked in green and red, respectively. Different combinations of windowing
method, filtering and weighting clearly affect the difference score, with
$\mathcal{CPD}$ values ranging from 0.0023 to 0.5137. The fits for paths with
low difference scores are clearly much better than for those with high ones
(Fig.~\ref{fig-difbw}). From Fig.~\ref{fig-dif}, it is clear that:

\begin{figure}
  \vspace*{-1.1cm}
  \centering
  \begin{subfigure}{.94\textwidth}
    \includegraphics[width=\textwidth]{../../paper/tex/GeophysJInt/figures/101_120_0.pdf}
    \caption{North America: minimum 0.00573 (25(0)), maximum
    0.08601 (16(2)), mean 0.032403, median 0.032395}\label{fig-na-dif} % subcaption
  \end{subfigure}
  \vspace{.1em} % here you can insert horizontal or vertical space
  \begin{subfigure}{.94\textwidth}
    \includegraphics[width=\textwidth]{../../paper/tex/GeophysJInt/figures/501_120_0.pdf}
    \caption{India: minimum 0.023 (19(0)), maximum 0.5137 (8(3)),
    mean 0.1182, median 0.0835}\label{fig-in-dif} % subcaption
  \end{subfigure}
  \vspace{.1em}
  \begin{subfigure}{.94\textwidth}
    \includegraphics[width=\textwidth]{../../paper/tex/GeophysJInt/figures/801_120_0.pdf}
    \caption{Australia: minimum 0.00227 (11(0)), maximum
    0.3934 (26(3)), mean 0.08373, median 0.05}\label{fig-au-dif} % subcaption
  \end{subfigure}
  \caption[$\mathcal{CPD}$ of each plate's paleomagnetic APWPs vs its FHM
predicted APWP]{Equal-weight composite path difference ($\mathcal{CPD}$) values
between each continent's paleomagnetic APWPs and its predicted APWP from FHM and
related plate circuits. The paths are in 10 Myr bin and 5 Myr step. The
difference values less than one-standard-deviation interval of the whole 168
values (lower 15.866 per cent) are colored in green, more than
one-standard-deviation interval (upper 15.866 per cent) colored in red. Exactly
the same columns are connected. The percentages of removed paleopoles are
derived relative to Pk 1, corrected relative to each corresponding picking
method (Pk 8,9, 12,13; 1 paleopole removed and 1 corrected by Pk 20,21 for
India). Fit quality (FQ) for each score is color coded.}\label{fig-dif} % caption for whole figure
\end{figure}

\begin{figure}
  \centering
  \begin{subfigure}{.42\textwidth} % width of left subfigure
    \includegraphics[width=\textwidth]{../../paper/tex/GeophysJInt/figures/ay18_101comb_10_5_25_0.pdf}
    \caption{North America: minimum 0.00573 (25(0)), FQ
    B-A}\label{fig-nac-105250}
  \end{subfigure}
  \begin{subfigure}{.42\textwidth} % width of right subfigure
    \includegraphics[width=\textwidth]{../../paper/tex/GeophysJInt/figures/ay18_101comb_10_5_16_2.pdf}
    \caption{North America: maximum 0.08601 (16(2)), FQ
    B-A}\label{fig-nac-105162}
  \end{subfigure}
  \vspace{.1em}
  \begin{subfigure}{.42\textwidth}
    \includegraphics[width=\textwidth]{../../paper/tex/GeophysJInt/figures/ay18_501comb_10_5_19_0.pdf}
    \caption{India: minimum 0.023 (19(0)), FQ C-A}\label{fig-ind-105190}
  \end{subfigure}
  \begin{subfigure}{.42\textwidth}
    \includegraphics[width=\textwidth]{../../paper/tex/GeophysJInt/figures/ay18_501comb_10_5_8_3.pdf}
    \caption{India: maximum 0.5137 (8(3)), FQ C-A}\label{fig-ind-10583}
  \end{subfigure}
  \vspace{.1em}
  \begin{subfigure}{.42\textwidth}
    \includegraphics[width=\textwidth]{../../paper/tex/GeophysJInt/figures/ay18_801comb_10_5_11_0.pdf}
    \caption{Australia: minimum 0.00227 (11(0)), FQ C-A}\label{fig-au-105110}
  \end{subfigure}
  \begin{subfigure}{.42\textwidth}
    \includegraphics[width=\textwidth]{../../paper/tex/GeophysJInt/figures/ay18_801comb_10_5_26_3.pdf}
    \caption{Australia: maximum 0.3934 (26(3)), FQ C-A}\label{fig-au-105263}
  \end{subfigure}
  \caption[Best and worst $\mathcal{CPD}$s (10/5 Myr window/step)]{Path
comparisons with best and worst difference values shown in Fig.~\ref{fig-dif}.
The parenthetical remarks are Pk No and Wt No.}\label{fig-difbw}
\end{figure}

\begin{enumerate}
  \item There is much more variation in scores along the horizontal axes than
	the vertical axes (see self-explanatory topography of bands in
	Fig.~\ref{fig-dif}), suggesting that the choice of windowing and filtering
	method (Table~\ref{tab-pick}) has a much greater impact than weighting
	(Table~\ref{tab-weit}).
  \item Many of the highest scores (worst fits) are associated with
	even-numbered picking and filtering methods, i.e., those which use the AMP
	windowing algorithm. Even so, Pk 4 and 6 are among the best methods for
	India.
  \item The magnitude and range of $\mathcal{CPD}$ scores for each of the three
	plates is different, with the North American plate having the lowest
	magnitudes and range (Fig.~\ref{fig-na-dif}), and the Indian plate having
	the highest (Fig.~\ref{fig-in-dif}). The scores for the Australian plate are
	generally closer to the equivalent scores for the North American plate than
	to the scores for the Indian plate (Fig.~\ref{fig-d-dif}).
  \item Although there is some overlap (e.g., Pk 19, 21 [best], and
	2, 16 [worst] for all the three plates or for both India and Australia; 1,
	11, 13, 19, 21, 25 [best], and 2, 14, 16, 22, 26 [worst] for both North
	America and Australia; 5, 7, 19, 21 [best], 2, 16, 18 [worst] for both North
	America and India), the best- and worst-performing picking/filtering and
	weighting algorithms are not exactly the same for each plate.
  \item Relating to FQ ratings: North America's lower $\mathcal{CPD}$ scores are
	also associated with consistently good FQ B-A grades; India's higher
	$\mathcal{CPD}$ scores are also associated with lower FQ (mostly C-A,
	including the no-filter methods, so there are some improvements in FQ with
	some filters), and Australia is a mix of B-A (including the no-filter
	methods) and C-A (which represent a reduction in FQ with some filters). In
	general, lower $\mathcal{CPD}$ scores also appear to be associated with
	better FQ (e.g. B-A rather than C-A).
\end{enumerate}

\subsection{Effects of Windowing Method}

Dividing $\mathcal{CPD}$ scores according to whether the AMP or APP windowing
method was used (Fig.~\ref{fig-difAMPvsAPP}) confirms that whilst the lowest
$\mathcal{CPD}$ scores for paths generated by the AMP windowing algorithm are
close to the lowest scores generated using the APP method, the highest scores
are much higher (Fig.~\ref{fig-boxAMPvsAPP}). The mean of the $\mathcal{CPD}$
scores for APWPs generated using AMP is greater than the maximum APP-derived
score for the Indian and Australian plates
(Figs.~\ref{fig-in-difAMPvsAPP},~\ref{fig-au-difAMPvsAPP},~\ref{fig-in-boxAMPvsAPP},~\ref{fig-au-boxAMPvsAPP}),
and more than 1 standard deviation greater than the APP mean for North American
APWPs (Figs.~\ref{fig-na-difAMPvsAPP},~\ref{fig-na-boxAMPvsAPP}).

\begin{figure}
  \centering
  \begin{subfigure}{1.01\textwidth}
    \includegraphics[width=\textwidth]{../../paper/tex/GeophysJInt/figures/101_120_0APPvsAMP.pdf}
    \caption{AMP\@: minimum 0.0265 (4(1)), maximum 0.086 (16(2)), mean
    0.0451, median 0.0457; APP\@: minimum 0.0057 (25(0)), maximum
    0.05835 (17(5)), mean 0.0197, median 0.01494}\label{fig-na-difAMPvsAPP}
  \end{subfigure}
  \vspace{.1em}
  \begin{subfigure}{1.01\textwidth}
    \includegraphics[width=\textwidth]{../../paper/tex/GeophysJInt/figures/501_120_0APPvsAMP.pdf}
    \caption{AMP\@: minimum 0.047 (4(3)), maximum 0.5137 (8(3)), mean
    0.1688, median 0.1892; APP\@: minimum 0.023 (19(0)), maximum 0.1359
    (3(4)), mean 0.0676, median 0.0602}\label{fig-in-difAMPvsAPP}
  \end{subfigure}
  \vspace{.1em}
  \begin{subfigure}{1.01\textwidth}
    \includegraphics[width=\textwidth]{../../paper/tex/GeophysJInt/figures/801_120_0APPvsAMP.pdf}
    \caption{AMP\@: minimum 0.0286 (3(0)), maximum 0.3934 (26(3)), mean
    0.12675, median 0.0938; APP\@: minimum 0.00227 (11(0)), maximum 0.11445
    (3(2)), mean 0.0407, median 0.0256}\label{fig-au-difAMPvsAPP}
  \end{subfigure}
  \caption[$\mathcal{CPD}$ of each plate's paleomagnetic APWPs vs its FHM
predicted APWP (AMP vs APP)]{Separated results from AMP and APP in
Fig.~\ref{fig-dif}. For each grid block (left: AMP, right: APP), the difference
values less than one-standard-deviation interval of the whole 84 values are
labeled in green, more than one-standard-deviation interval labeled in
red.}\label{fig-difAMPvsAPP}
\end{figure}

For each of the 84 possible combinations of filter method and weighting, the
AMP-derived score is typically 3\textendash5 times higher than the equivalent
APP-derived score (Fig.~\ref{fig-boxAMPvsAPP}, insets). APP-generated paths
yield a lower $\mathcal{CPD}$ score than the equivalent AMP-generated path for
82 (97.6\%) of the North America scores, 72 (85.7\%) of the India scores, and 84
(100\%) of the Australia scores. For the North American and Indian plates,
filtering that prefers igneous poles (Pk 4,5 and 6,7) or removes
poles with large various uncertainties (Pk 22,23 and 26,27) is most likely
to produce AMP scores close to (less than 1.5 times) or less than the APP scores
(Figs.~\ref{fig-na-difAMPvsAPP},~\ref{fig-in-difAMPvsAPP}). In the former case,
the scores are comparable and relatively low; in the latter case, they are
comparable but relatively high. For the Australian plate, only correcting for
sedimentary inclination shallowing (Pk 8,9) produces comparable but moderate
scores (Fig.~\ref{fig-au-difAMPvsAPP}).

\begin{figure}
  \centering
  \begin{subfigure}{.9\textwidth}
    \includegraphics[width=\textwidth]{../../paper/tex/GeophysJInt/figures/101_10_5FHM.pdf}
    \caption{}\label{fig-na-boxAMPvsAPP}
  \end{subfigure}
  \vspace{.1em}
  \begin{subfigure}{.9\textwidth}
    \includegraphics[width=\textwidth]{../../paper/tex/GeophysJInt/figures/501_10_5FHM.pdf}
    \caption{}\label{fig-in-boxAMPvsAPP}
  \end{subfigure}
  \vspace{.1em}
  \begin{subfigure}{.9\textwidth}
    \includegraphics[width=\textwidth]{../../paper/tex/GeophysJInt/figures/801_10_5FHM.pdf}
    \caption{}\label{fig-au-boxAMPvsAPP}
  \end{subfigure}
  \caption[]{Box-and-whisker and cross (inset) plots of
Fig.~\ref{fig-difAMPvsAPP}. The $\mathcal{CPD}$s from same filter and weighting
method (red and blue dots plotted with box-and-whisker) are connected; some
special cases where $\mathcal{CPD}$ from AMP lower than from APP are highlighted
using darker connecting lines. Dot symbols are semi-transparent so a darker
color indicates a greater number of data at a given
$\mathcal{CPD}$.}\label{fig-boxAMPvsAPP}
\end{figure}

On the North American plate, the APP-derived $\mathcal{CPD}$ score is most
likely to be significantly better (defined as more than 5 times higher) when
Wt 0 (no weighting) and 1 (by sample and site number) have been
applied (Fig.~\ref{fig-na-difAMPvsAPP}). This pattern is also seen for the
Australian plate, but removal of poles published after 1983 (Pk 16,17) also
results in significantly better performance of the APP method
(Fig.~\ref{fig-au-difAMPvsAPP}). For the Indian plate, the largest difference
occurs when poles with sedimentary inclination shallowing (Pk 8,9) are
corrected, or suspected overprints or local rotations are removed (Pk
18,19, Fig.~\ref{fig-in-difAMPvsAPP}).

\subsection{Effects of Filtering and Weighting}

When $\mathcal{CPD}$ scores are separated by windowing method
(Fig.~\ref{fig-difAMPvsAPP}), the effects of particular filtering and weighting
methods become easier to discern. In general, different filters (rows) produce
larger variations in scores than different weighting methods (columns). With
the exception of AMP-derived paths for India, the $\mathcal{CPD}$ score for
paths with no filtering (Pk 0,1) and no weighting (Wt
0) is lower than the mean scores for that plate and windowing method. Therefore
filtering and weighting at best slightly improves, and at worst significantly
degrades, the APWP fit to the reference path.

\subsubsection{Filter Aggression}

When considering the effects of filtering, it is important to consider how many
paleopoles within the data set have been removed (related to Pk 2\textendash7,
10,11, and 14\textendash27) or corrected (Pk 8,9 and 12,13): if there is very
little alteration of the data set, little change from no filtering (Pk 0,1)
would be expected. In terms of the numbers of paleopoles affected, the most
consequential filters are:
%
\begin{enumerate}
  \item removal of sedimentary paleopoles (Pk 4,5 and 6,7), which removes
		${\sim}40$\textendash50\% of the datasets on all 3 plates, with the
		highest proportion being removed on the Indian plate. Although Pk 4,5
		are more strict, it does not remove many more paleopoles than Pk 6,7.
		For North America and India, the numbers of filtered paleopoles for Pk
		4,5 and 6,7 are actually the same.
  \item correction of sedimentary paleopoles for inclination flattening (filter
		in Pk 8,9), which affects 38\textendash48\% of the dataset, with the
		highest affected proportion on the Indian plate.
  \item removal of paleopoles with large temporal and spatial uncertainty (Pk
		2,3, 22,23), particularly for the Australian plate, where the SS05
		filtering criteria removes ${\sim}70$\% of the paleopoles. Filter in Pk
		26,27 combines Pk 22,23 and 24,25, but no or very few (in the only case
		of Australia only 1 additional paleopole) are actually removed.
  \item filtering based on publication date (Pk 14,15 and 16,17), with the
		ratio of pre/post 1983 poles varying from about 50/50 on the North
		American plate to about 70/30 on the Australian plate.
\end{enumerate}

Conversely, filtering or correction for redbeds (Pk 10,11 and 12,13), local
rotations and overprints (Pk 18,19 and 20,21%[only 1 paleopole influenced by local rotation removed, and one corrected, for only India; ${\sim}2.7$\%, Fig.~\ref{fig-in-dif}]
), or superseded data (Pk 24,25) affected 4\% or less of
the paleopoles on any plate.

Note that there is an important trade-off that precision in a mean pole is
determined by the number of contributing paleopoles (A95 is proportional to
1/$\sqrt{n}$), so an overly aggressive filter might improve accuracy at the
expense of precision. This is particularly an issue where data density is low
(see Section~\ref{sec:datden}).

\subsubsection{Filter Performance}

Focussing on the filtering and weighting methods with aggressive filtering,
some commonalities in the best- and worst-performing methods can be observed,
although there are usually exceptions for particular plates and/or windowing
methods:
%
\begin{enumerate}
  \item For all 3 plates, higher $\mathcal{CPD}$ scores are commonly associated
		with filtering based on $\alpha$95 and age range (Pk 2,3, 22,23, 26,27),
		with the exception of AMP-derived paths for India, where Pk 22 and 26
		produce some of the lowest scores, and also improve FQ\@.
  \item For North America and India, low scores are commonly associated with
		removal of non-igneous poles (Pk 4,5 and 6,7), particularly for
		AMP-derived paths. On the Australian plate, these filters are less
		effective, and also reduce FQ\@.
  \item For North America and Australia, correction for inclination flattening
		generates $\mathcal{CPD}$ scores very similar to scores with no
		filtering for AMP-derived paths (Pk 8 versus Pk 0; for Australia FQ
		reduced), and increases scores for APP-derived paths (Pk 9 versus Pk 1).
		In contrast, for India generally there is a small decrease in
		$\mathcal{CPD}$ scores compared to no filtering for both AMP- and
		APP-derived paths.
  \item Many of the highest difference scores for North America and India occur
		when paleopoles published after 1983 are removed (Pk 16,17), whilst
		removing paleopoles published before 1983 (Pk 14,15) generates
		$\mathcal{CPD}$ scores comparable to scores with no filtering
		(relatively much lower). In contrast, for Australia Pk 14,15
		produces some of the highest $\mathcal{CPD}$ scores, and Pk 16,17
		have little effect on the $\mathcal{CPD}$ score, although FQ is
		commonly reduced. It is noteworthy that the number of older studies
		(65/68; Fig.~\ref{fig-au-dif}) is almost 2.5 times of the number of
		newer studies (27/29) for the Australian plate.
\end{enumerate}

Whilst it is generally true that methods with low-aggression filters do not
generate scores that differ much from the no-filtering scores, there are some
exceptions:
%
\begin{enumerate}
  \item For North America and Australia, removing superseded paleopoles (Pk
		24,25) produces lower $\mathcal{CPD}$ scores.
  \item Removing paleopoles suspected to be affected by overprints or local
		rotations (Pk 18,19) consistently produces lower
		$\mathcal{CPD}$ scores for Indian APP-derived paths.
\end{enumerate}

\begin{figure*}
  \centering
  \begin{subfigure}{1.01\textwidth}
    \includegraphics[width=\textwidth]{../../paper/tex/GeophysJInt/figures/f501d101.pdf}
    \caption{Differences between Fig.~\ref{fig-in-dif} (India) and
    Fig.~\ref{fig-na-dif} (North America)}\label{fig-i-n-dif}
  \end{subfigure}
  \vspace{.1em}
  \begin{subfigure}{1.01\textwidth}
    \includegraphics[width=\textwidth]{../../paper/tex/GeophysJInt/figures/f801d101.pdf}
    \caption{Differences between Fig.~\ref{fig-au-dif} (Australia) and
    Fig.~\ref{fig-na-dif} (North America)}\label{fig-a-n-dif}
  \end{subfigure}
  \vspace{.1em}
  \begin{subfigure}{1.01\textwidth}
    \includegraphics[width=\textwidth]{../../paper/tex/GeophysJInt/figures/f501d801.pdf}
    \caption{Differences between Fig.~\ref{fig-in-dif} (India) and
    Fig.~\ref{fig-au-dif} (Australia)}\label{fig-i-a-dif}
  \end{subfigure}
  \caption[Differences of $\mathcal{CPD}$ of each plate's paleomagnetic APWPs vs
its FHM predicted APWP]{Differences between grids in Fig.~\ref{fig-dif}. The
absolute difference values less than 1.96-standard-deviation interval of the
whole 168 values are labeled in green, more than 1.96-standard-deviation
interval labeled in red.}\label{fig-d-dif}
\end{figure*}

\subsubsection{Weighting Performance}

Compared to the variations resulting from different windowing methods and
filters, Figs.~\ref{fig-difAMPvsAPP},~\ref{fig-nads},~\ref{fig-nadl}
and~\ref{fig-nada} indicate that the effect of weighting the data prior to
calculating a Fisher mean is generally small or diluted. Where an effect can be
seen, it is negative, generating larger $\mathcal{CPD}$ scores.

\begin{figure}[!ht]
	\centering
	\begin{subfigure}{.495\textwidth}
		\includegraphics[width=\textwidth]{../../paper/tex/GeophysJInt/figures/101npoles105w0.pdf}
		\caption{For Wt 0}\label{fig-na-dsw0}
	\end{subfigure}
	\vspace{.1em}
	\begin{subfigure}{.495\textwidth}
		\includegraphics[width=\textwidth]{../../paper/tex/GeophysJInt/figures/101npoles105w1.pdf}
		\caption{For Wt 1}\label{fig-na-dsw1}
	\end{subfigure}
	\vspace{.1em}
	\begin{subfigure}{.495\textwidth}
		\includegraphics[width=\textwidth]{../../paper/tex/GeophysJInt/figures/101npoles105w2.pdf}
		\caption{For Wt 2}\label{fig-na-dsw2}
	\end{subfigure}
	\vspace{.1em}
	\begin{subfigure}{.495\textwidth}
		\includegraphics[width=\textwidth]{../../paper/tex/GeophysJInt/figures/101npoles105w3.pdf}
		\caption{For Wt 3}\label{fig-na-dsw3}
	\end{subfigure}
	\vspace{.1em}
	\begin{subfigure}{.495\textwidth}
		\includegraphics[width=\textwidth]{../../paper/tex/GeophysJInt/figures/101npoles105w4.pdf}
		\caption{For Wt 4}\label{fig-na-dsw4}
	\end{subfigure}
	\vspace{.1em}
	\begin{subfigure}{.495\textwidth}
		\includegraphics[width=\textwidth]{../../paper/tex/GeophysJInt/figures/101npoles105w5.pdf}
		\caption{For Wt 5}\label{fig-na-dsw5}
	\end{subfigure}
	\caption[$d_s$ of each pair of poles for North American 10/5 Myr APWPs]{Tested
spatial difference ($d_s$) values (color shaded) between North American
paleomagnetic APWPs and its predicted APWP from the FHM and related plate
circuits. The paths are in 10 Myr bin and 5 Myr step. The number labels on the
grids (including grid heights) are the numbers of site mean poles that are
contributing to make each mean path pole.}\label{fig-nads}
\end{figure}

\begin{figure}[!ht]
	\centering
	\begin{subfigure}{.495\textwidth}
		\includegraphics[width=\textwidth]{../../paper/tex/GeophysJInt/figures/101nsegs105w0.pdf}
		\caption{For Wt 0}\label{fig-na-dlw0}
	\end{subfigure}
	\vspace{.1em}
	\begin{subfigure}{.495\textwidth}
		\includegraphics[width=\textwidth]{../../paper/tex/GeophysJInt/figures/101nsegs105w1.pdf}
		\caption{For Wt 1}\label{fig-na-dlw1}
	\end{subfigure}
	\vspace{.1em}
	\begin{subfigure}{.495\textwidth}
		\includegraphics[width=\textwidth]{../../paper/tex/GeophysJInt/figures/101nsegs105w2.pdf}
		\caption{For Wt 2}\label{fig-na-dlw2}
	\end{subfigure}
	\vspace{.1em}
	\begin{subfigure}{.495\textwidth}
		\includegraphics[width=\textwidth]{../../paper/tex/GeophysJInt/figures/101nsegs105w3.pdf}
		\caption{For Wt 3}\label{fig-na-dlw3}
	\end{subfigure}
	\vspace{.1em}
	\begin{subfigure}{.495\textwidth}
		\includegraphics[width=\textwidth]{../../paper/tex/GeophysJInt/figures/101nsegs105w4.pdf}
		\caption{For Wt 4}\label{fig-na-dlw4}
	\end{subfigure}
	\vspace{.1em}
	\begin{subfigure}{.495\textwidth}
		\includegraphics[width=\textwidth]{../../paper/tex/GeophysJInt/figures/101nsegs105w5.pdf}
		\caption{For Wt 5}\label{fig-na-dlw5}
	\end{subfigure}
	\caption[$d_l$ of each pair of segments for North American 10/5 Myr
APWPs]{Tested length difference ($d_l$) values (color shaded) between North
American paleomagnetic APWPs and its predicted APWP from FHM and related plate
circuits. The paths are in 10 Myr bin and 5 Myr step. The labeled numbers on the
grids are the averaged numbers of site mean poles that are contributing to each
segment's two mean path poles.}\label{fig-nadl}
\end{figure}

\begin{figure}[!ht]
	\centering
	\begin{subfigure}{.495\textwidth}
		\includegraphics[width=\textwidth]{../../paper/tex/GeophysJInt/figures/101nocs105w0.pdf}
		\caption{For Wt 0}\label{fig-na-daw0}
	\end{subfigure}
	\vspace{.1em}
	\begin{subfigure}{.495\textwidth}
		\includegraphics[width=\textwidth]{../../paper/tex/GeophysJInt/figures/101nocs105w1.pdf}
		\caption{For Wt 1}\label{fig-na-daw1}
	\end{subfigure}
	\vspace{.1em}
	\begin{subfigure}{.495\textwidth}
		\includegraphics[width=\textwidth]{../../paper/tex/GeophysJInt/figures/101nocs105w2.pdf}
		\caption{For Wt 2}\label{fig-na-daw2}
	\end{subfigure}
	\vspace{.1em}
	\begin{subfigure}{.495\textwidth}
		\includegraphics[width=\textwidth]{../../paper/tex/GeophysJInt/figures/101nocs105w3.pdf}
		\caption{For Wt 3}\label{fig-na-daw3}
	\end{subfigure}
	\vspace{.1em}
	\begin{subfigure}{.495\textwidth}
		\includegraphics[width=\textwidth]{../../paper/tex/GeophysJInt/figures/101nocs105w4.pdf}
		\caption{For Wt 4}\label{fig-na-daw4}
	\end{subfigure}
	\vspace{.1em}
	\begin{subfigure}{.495\textwidth}
		\includegraphics[width=\textwidth]{../../paper/tex/GeophysJInt/figures/101nocs105w5.pdf}
		\caption{For Wt 5}\label{fig-na-daw5}
	\end{subfigure}
	\caption[$d_a$ of each pair of segment-oreintation-changes for North American
10/5 Myr APWPs]{Tested angular difference ($d_a$) values (color shaded) between
North American paleomagnetic APWPs and its predicted APWP from FHM and related
plate circuits. The paths are in 10 Myr bin and 5 Myr step. The labeled numbers
on the grids are the averaged numbers of site mean poles that are contributing
to each segment-orientation-change's three mean path poles.}\label{fig-nada}
\end{figure}
%
\begin{enumerate}
  \item For APP-derived paths from the North American and Australian plates,
		Wt 0 (no weighting) and 1 (weighting by sample and site
		number) usually produce slightly better $\mathcal{CPD}$ scores than
		other weighting methods.
  \item Wt 3 (weighting by spatial uncertainty) seems most likely
		to generate much higher $\mathcal{CPD}$ scores, particularly for
		AMP-derived paths, and particular for the Indian plate.
  \item Wt 0, 1 and 5 generally produce lower similarity scores than Wt 2, 3 and
		4.
\end{enumerate}

\subsection{Effects of Window Size}

Fig.~\ref{fig-dif2010} shows the $\mathcal{CPD}$ scores for the APWPs generated
with the 28 picking and 6 weighting methods, compared to the FHM reference
paths, with the picking time window width increased from 10 to 20 Myr, and the
window step increased from 5 to 10 Myr.

\begin{figure*}
	\centering
	\begin{subfigure}{.96\textwidth}
		\includegraphics[width=\textwidth]{../../paper/tex/GeophysJInt/figures/101_20_10_120_0.pdf}
		\caption{North America: minimum 0.00991 (15(0)),
		maximum 0.0654 (14(2)), mean 0.0339, median 0.0296371}\label{fig-na-dif2010}
	\end{subfigure}
	\vspace{.1em}
	\begin{subfigure}{.96\textwidth}
		\includegraphics[width=\textwidth]{../../paper/tex/GeophysJInt/figures/501_20_10_120_0.pdf}
		\caption{India: minimum 0.0142822 (6(3)), maximum 0.154278 (16(3)),
		mean 0.07384, median 0.072088}\label{fig-in-dif2010}
	\end{subfigure}
	\vspace{.1em}
	\begin{subfigure}{.96\textwidth}
		\includegraphics[width=\textwidth]{../../paper/tex/GeophysJInt/figures/801_20_10_120_0.pdf}
		\caption{Australia: minimum 0 (11(0,1),19(3)), maximum
		0.324438 (22(3)), mean 0.0682, median 0.036432}\label{fig-au-dif2010}
	\end{subfigure}
	\caption[$\mathcal{CPD}$ of each plate's paleomagnetic APWPs vs its FHM
predicted APWP (20/10 Myr window/step)]{As Fig.~\ref{fig-dif}, here the paths
are generated in 20 Myr bin and 10 Myr step. The difference values less than
one-standard-deviation interval of the whole 168 values are colored in green,
more than one-standard-deviation interval colored in red. Compare the numbers of
picked paleopoles with those in Fig.~\ref{fig-dif}.}\label{fig-dif2010}
\end{figure*}

\begin{figure*}
	\centering
	\begin{subfigure}{.43\textwidth}
		\includegraphics[width=\textwidth]{../../paper/tex/GeophysJInt/figures/ay18_101comb_20_10_15_0.pdf}
		\caption{North America: minimum 0.00991 (15(0)), FQ B-A}
	\end{subfigure}
	\begin{subfigure}{.43\textwidth}
		\includegraphics[width=\textwidth]{../../paper/tex/GeophysJInt/figures/ay18_101comb_20_10_14_2.pdf}
		\caption{North America: maximum 0.0654 (14(2)), FQ B-A}
	\end{subfigure}
	\vspace{.1em}
	\begin{subfigure}{.43\textwidth}
		\includegraphics[width=\textwidth]{../../paper/tex/GeophysJInt/figures/ay18_501comb_20_10_6_3.pdf}
		\caption{India: minimum 0.0143 (6(3)), FQ C-A}
	\end{subfigure}
	\begin{subfigure}{.43\textwidth}
		\includegraphics[width=\textwidth]{../../paper/tex/GeophysJInt/figures/ay18_501comb_20_10_16_3.pdf}
		\caption{India: maximum 0.1543 (16(3)), FQ C-A}
	\end{subfigure}
	\vspace{.1em}
	\begin{subfigure}{.43\textwidth}
		\includegraphics[width=\textwidth]{../../paper/tex/GeophysJInt/figures/ay18_801comb_20_10_11_0.pdf}
		\caption{Australia: minimum 0 (11(0)), FQ B-A}\label{fig-au-2010110}
	\end{subfigure}
	\begin{subfigure}{.43\textwidth}
		\includegraphics[width=\textwidth]{../../paper/tex/GeophysJInt/figures/ay18_801comb_20_10_22_3.pdf}
		\caption{Australia: maximum 0.3244 (22(3)), FQ C-A}\label{fig-au-2010223}
	\end{subfigure}
	\caption[Best and worst $\mathcal{CPD}$s (20/10 Myr window/step)]{Path
comparisons with best and worst difference values shown in
Fig.~\ref{fig-dif2010}. The parenthetical remarks are Pk No and Wt
No.}\label{fig-dif2010bw}
\end{figure*}

\subsubsection{Overall Change}

The overall effect of increased window and step size varies between plates
(Fig.~\ref{fig-f105d2010}). 118/168 (${\sim}70$\%) of equivalent $\mathcal{CPD}$
scores for the North American plate increase, indicating reduced similarity with
respect to the reference path (Fig.~\ref{fig-101f105d2010}). The increased
occurrence of A-A rather than B-A FQ ratings (Fig.~\ref{fig-na-dif2010} versus
Fig.\ref{fig-na-dif}) indicate that in some cases this may be due to a better
constrained path becoming more distinguishable. Decreased scores are confined to
particular, mostly AMP-derived picking methods: Pk 2, 22 and 26 (filtering based
on ${\alpha}95$ and age uncertainties), 4 and 6 (removal of sedimentary
paleopoles), and 16,17 (removal of post–1983 results). It is clearly due to the
corresponding increase of N in each increased window for these AMP-derived
quality filters.

For the Indian and Australian plates, there is a trend towards decreased
$\mathcal{CPD}$ scores, with 153/168 (${\sim}91$\%) and 104/168 (${\sim}62$\%)
of equivalent $\mathcal{CPD}$ scores being lower than for the 10/5 Myr
window/step, respectively (Fig.~\ref{fig-501f105d2010},~\ref{fig-801f105d2010}).
Where the score increases, the actual difference is close to 0, indicating very
little change, with the exception of Pk 14 (AMP, removal of pre–1983 results) on
the Australian plate. The largest score decreases are associated with Wt 3 (AMP,
${\alpha}95$ and age filtering) on the Indian plate, and Pk 15 (APP, removal of
pre–1983 results) on the Australian plate. Most FQ ratings remain unchanged
(Fig.~\ref{fig-in-dif2010} versus Fig.~\ref{fig-in-dif},
Fig.~\ref{fig-au-dif2010} versus Fig.~\ref{fig-au-dif}): most obviously, ratings
for Pk 5 and 7 (APP, removal of sedimentary poles) on the Indian plate and Pk 14
on the Australian plate change from B-A to C-A.

\begin{figure*}
	\centering
	\begin{subfigure}{1\textwidth}
		\includegraphics[width=\textwidth]{../../paper/tex/GeophysJInt/figures/101f105d2010.pdf}
		\caption{North America: Generally Bin/Step 10/5 Myr is better
		($\frac{59}{84}\approx70.24\%$)}\label{fig-101f105d2010}
	\end{subfigure}
	\vspace{.1em}
	\begin{subfigure}{1\textwidth}
		\includegraphics[width=\textwidth]{../../paper/tex/GeophysJInt/figures/501f105d2010.pdf}
		\caption{India: Generally Bin/Step 20/10 Myr is better
		($\frac{153}{168}\approx91.07\%$)}\label{fig-501f105d2010}
	\end{subfigure}
	\vspace{.1em}
	\begin{subfigure}{1\textwidth}
		\includegraphics[width=\textwidth]{../../paper/tex/GeophysJInt/figures/801f105d2010.pdf}
		\caption{Australia: Generally Bin/Step 20/10 Myr is better
		($\frac{13}{21}\approx61.9\%$)}\label{fig-801f105d2010}
	\end{subfigure}
	\caption[]{Differences between grids in Fig.~\ref{fig-dif} (10 Myr bin, 5
Myr step) and Fig.~\ref{fig-dif2010} (20 Myr bin, 10 Myr step). The absolute
difference values less than 1.96-standard-deviation interval of the whole 168
values are labeled in green, more than 1.96-standard-deviation interval labeled
in red.}\label{fig-f105d2010}
\end{figure*}

\subsubsection{Relative Performance of Methods}

As for the baseline results (Section~\ref{sec:base}, Fig.~\ref{fig-dif}), the
effects of windowing and filtering, particularly the windowing method (APP
versus AMP), are much more apparent than the effects of weighting. APP-derived
scores still outperform AMP-derived ones, but the overall degree of difference
is reduced for the North American and Indian plates, with percentage of
APP-derived $\mathcal{CPD}$ scores more than 3 times the equivalent AMP-derived
score reducing to ${\sim}26$\% and ${\sim}4$\% (from ${\sim}44$\% and
${\sim}40$\%), respectively. For the Australian plate, this percentage increases
from ${\sim}58$\% to ${\sim}70$\%.

This is largely the result of larger changes in the AMP-derived $\mathcal{CPD}$
scores compared to changes in the APP-derived ones with the wider time window.
The scores for APP-derived methods Pk 1 (no filtering), other non-aggressive
filters (Pk 13, 19, 21, 25), and Pk 9 (correction for inclination flattening),
are particularly stable (Fig.~\ref{fig-f105d2010}) for all three plates.
Filtering according to the SS05 criteria (Pk 23 and 27) also yields comparable
scores for both time windows on the North American and Indian plates.

As for the 10/5 window, the lowest AMP/APP differences are for Pk 4,5 and 6,7
(igneous poles preferred \textendash{} low scores) and Pk 2,3 and 22,23
(filtering for spatial and temporal uncertainty \textendash{} high scores) for
the North American and Indian plates; and 8,9 (correction for inclination
shallowing \textendash{} moderate scores) for the Australian plates. Likewise,
the highest AMP/APP difference is still associated with Pk 0,1 (no filtering) on
the North American and Australian plates, and Pk 16,17 on the Australian plate.

Overall, the relative performance of the different filtering methods when using
the 20/10 window/step remains similar to that observed for the 10/5 window/step
(Figs.~\ref{fig-dif},~\ref{fig-dif2010}). Some of the lowest scores are still
produced by Pk 1 (APP, no filtering). Of the aggressive filters, removing
sedimentary paleopoles (Pk 4,5 and 6,7) still perform well, selecting only
spatially and temporally well-constrained paleopoles (Pk 2,3 and 22,23) still
perform relatively poorly, and correction for inclination flattening (Pk 8,9)
has little effect.

In addition, using only paleopoles published before 1983 (Pk 16,17) still
produces relatively high $\mathcal{CPD}$ scores for the North American and
Indian plates, although the scores for the AMP-derived Pk 16 are substantially
reduced with the wider time window (Figs.~\ref{fig-101f105d2010},~\ref{fig-501f105d2010}).
Likewise, for Australia, using only paleopoles published after 1983 (Pk 14,15)
still produces higher scores than Pk 16,17: scores for the wider time window are
even higher for the AMP-derived Pk 14, but substantially lower for the
APP-derived Pk 15 (Fig.~\ref{fig-801f105d2010}).

\begin{landscape}
\begin{table}[]
\centering
\caption{Consistency check on comparisons of picking methods' performance between 20/10 and 10/5 Myr window/step. Notes: E means expected; UE means unexpected.}\label{tab-2010vs105F}
\resizebox{.9\width}{!}{%
\begin{tabular}{@{}cccllcllcccccl@{}}
\toprule
\multicolumn{2}{c}{Comparisons} & \multicolumn{3}{c}{Consistency of best} & \multicolumn{3}{c}{Consistency of worst} & \multicolumn{6}{c}{If CPD values for 20/10 are lower (Y/N)} \\ \midrule
10/5 & 20/10 & Y/N & \multicolumn{1}{c}{Special case} & Same Pk & Y/N & \multicolumn{1}{c}{Special case} & \multicolumn{1}{c}{Same Pk} & Mean & Median & Max & Min & All & If N, UE case \\ \midrule
\multicolumn{14}{c}{FHM} \\ \midrule
\multirow{4}{*}{\begin{tabular}[c]{@{}c@{}}Fig.\\\ref{fig-na-dif}\end{tabular}} & \multirow{4}{*}{\begin{tabular}[c]{@{}c@{}}Fig.\\\ref{fig-na-dif2010}\end{tabular}} & \multirow{4}{*}{Y} & \multirow{4}{*}{\begin{tabular}[c]{@{}l@{}}3 more best:\\ Pk 4, 6, 9\\ only for\\ 20/10 (E)\end{tabular}} & \multirow{4}{*}{\begin{tabular}[c]{@{}l@{}}1, 5, 7,\\ 11, 13,\\ 15, 19,\\ 21, 25\end{tabular}} & \multirow{4}{*}{N} & \multirow{4}{*}{\begin{tabular}[c]{@{}l@{}}Pk 2, 5, 7, 17, 22,\\ 26 for 10/5 (E);\\ 0, 8, 10, 12, 20,\\ 24 for 20/10 (UE)\end{tabular}} & \multirow{4}{*}{\begin{tabular}[c]{@{}l@{}}14,\\ 16,\\ 18\end{tabular}} & \multirow{4}{*}{N} & \multirow{4}{*}{Y} & \multirow{4}{*}{Y} & \multirow{4}{*}{N} & \multirow{4}{*}{N} & \multirow{4}{*}{\begin{tabular}[c]{@{}l@{}}Positive\\ values\\ in\\ Fig.~\ref{fig-101f105d2010}\end{tabular}} \\
 &  &  &  &  &  &  &  &  &  &  &  &  &  \\
 &  &  &  &  &  &  &  &  &  &  &  &  &  \\
 &  &  &  &  &  &  &  &  &  &  &  &  &  \\ \midrule
\multirow{4}{*}{\begin{tabular}[c]{@{}c@{}}Fig.\\\ref{fig-in-dif}\end{tabular}} & \multirow{4}{*}{\begin{tabular}[c]{@{}c@{}}Fig.\\\ref{fig-in-dif2010}\end{tabular}} & \multirow{4}{*}{Y} & \multirow{4}{*}{\begin{tabular}[c]{@{}l@{}}4 more best:\\ Pk 9, 21, 22,\\ 26 only for\\ 10/5 (UE)\end{tabular}} & \multirow{4}{*}{\begin{tabular}[c]{@{}l@{}}4\textendash7,\\ 19\end{tabular}} & \multirow{4}{*}{N} & \multirow{4}{*}{\begin{tabular}[c]{@{}l@{}}Pk 8 for\\ 10/5 (E);\\ 23, 27 for\\ 20/10 (UE)\end{tabular}} & \multirow{4}{*}{\begin{tabular}[c]{@{}l@{}}0, 2,\\ 10, 12,\\ 16, 18,\\ 20, 24\end{tabular}} & \multirow{4}{*}{Y} & \multirow{4}{*}{Y} & \multirow{4}{*}{Y} & \multirow{4}{*}{Y} & \multirow{4}{*}{N} & \multirow{4}{*}{\begin{tabular}[c]{@{}l@{}}Positive\\ values\\ in\\ Fig.~\ref{fig-501f105d2010}\end{tabular}} \\
 &  &  &  &  &  &  &  &  &  &  &  &  &  \\
 &  &  &  &  &  &  &  &  &  &  &  &  &  \\
 &  &  &  &  &  &  &  &  &  &  &  &  &  \\ \midrule
\multirow{3}{*}{\begin{tabular}[c]{@{}c@{}}Fig.\\\ref{fig-au-dif}\end{tabular}} & \multirow{3}{*}{\begin{tabular}[c]{@{}c@{}}Fig.\\\ref{fig-au-dif2010}\end{tabular}} & \multirow{3}{*}{Y} & \multirow{3}{*}{\begin{tabular}[c]{@{}l@{}}2 more best:\\ Pk 23, 27 only\\ for 20/10 (E)\end{tabular}} & \multirow{3}{*}{\begin{tabular}[c]{@{}l@{}}1, 11, 13,\\ 17, 19,\\ 21, 25\end{tabular}} & \multirow{3}{*}{Y} & \multirow{3}{*}{\begin{tabular}[c]{@{}l@{}}1 more worst:\\ Pk 4 only for\\ 10/5 (E)\end{tabular}} & \multirow{3}{*}{\begin{tabular}[c]{@{}l@{}}2, 14,\\ 16, 22,\\ 26\end{tabular}} & \multirow{3}{*}{Y} & \multirow{3}{*}{Y} & \multirow{3}{*}{Y} & \multirow{3}{*}{Y} & \multirow{3}{*}{N} & \multirow{3}{*}{\begin{tabular}[c]{@{}l@{}}Positive\\ values in\\ Fig.~\ref{fig-801f105d2010}\end{tabular}} \\
 &  &  &  &  &  &  &  &  &  &  &  &  &  \\
 &  &  &  &  &  &  &  &  &  &  &  &  &  \\ \midrule
\multicolumn{14}{c}{MHM} \\ \midrule
\multirow{5}{*}{\begin{tabular}[c]{@{}c@{}}Fig.\\\ref{fig-na-difm}\end{tabular}} & \multirow{5}{*}{\begin{tabular}[c]{@{}c@{}}Fig.\\\ref{fig-na-dif2010m}\end{tabular}} & \multirow{5}{*}{N} & \multirow{5}{*}{\begin{tabular}[c]{@{}l@{}}Pk 1, 9, 11,\\ 13, 19, 21,\\ 25 for 10/5\\ (UE); 22, 26\\ for 20/10 (E)\end{tabular}} & \multirow{5}{*}{\begin{tabular}[c]{@{}l@{}}5,\\ 7,\\ 15\end{tabular}} & \multirow{5}{*}{N} & \multirow{5}{*}{\begin{tabular}[c]{@{}l@{}}Pk 16,17\\ only for 10/5\\ (E); 8 only\\ for 20/10\\ (UE)\end{tabular}} & \multirow{5}{*}{\begin{tabular}[c]{@{}l@{}}0, 10\\ 12, 14,\\ 18,\\ 20,\\ 24\end{tabular}} & \multirow{5}{*}{N} & \multirow{5}{*}{Y} & \multirow{5}{*}{N} & \multirow{5}{*}{N} & \multirow{5}{*}{N} & \multirow{5}{*}{\begin{tabular}[c]{@{}l@{}}(0,8,10,11,12,14,18,20,24,25)(0\textendash5)\\ (1,13,19,21)(0,1,5)  3(0,3,5)\\ (5,7)(0\textendash2,4,5)  15(0\textendash4)\\ (9,23,27)(0,1,3,5)\\ account for 60.71\%\end{tabular}} \\
 &  &  &  &  &  &  &  &  &  &  &  &  &  \\
 &  &  &  &  &  &  &  &  &  &  &  &  &  \\
 &  &  &  &  &  &  &  &  &  &  &  &  &  \\
 &  &  &  &  &  &  &  &  &  &  &  &  &  \\ \midrule
\multirow{5}{*}{\begin{tabular}[c]{@{}c@{}}Fig.\\\ref{fig-in-difm}\end{tabular}} & \multirow{5}{*}{\begin{tabular}[c]{@{}c@{}}Fig.\\\ref{fig-in-dif2010m}\end{tabular}} & \multirow{5}{*}{N} & \multirow{5}{*}{\begin{tabular}[c]{@{}l@{}}Pk 19, 21\\ only for 10/5\\ (UE); 4, 6\\ only for\\ 20/10 (E)\end{tabular}} & \multirow{5}{*}{\begin{tabular}[c]{@{}l@{}}5,\\ 7,\\ 22,\\ 26\end{tabular}} & \multirow{5}{*}{Y} & \multirow{5}{*}{\begin{tabular}[c]{@{}l@{}}3 more\\ worst: Pk\\ 8, 14, 20\\ for 10/5\\ (E)\end{tabular}} & \multirow{5}{*}{\begin{tabular}[c]{@{}l@{}}0, 2,\\ 10, 12,\\ 16,\\ 18,\\ 24\end{tabular}} & \multirow{5}{*}{Y} & \multirow{5}{*}{Y} & \multirow{5}{*}{Y} & \multirow{5}{*}{Y} & \multirow{5}{*}{N} & \multirow{5}{*}{\begin{tabular}[c]{@{}l@{}}(1,25)(0,3,5)  15(3)\\ (19,23,27)(0\textendash5)\\ 21(0,1,3\textendash5)\\ (22,26)(4);\\ account for 19.05\%\end{tabular}} \\
 &  &  &  &  &  &  &  &  &  &  &  &  &  \\
 &  &  &  &  &  &  &  &  &  &  &  &  &  \\
 &  &  &  &  &  &  &  &  &  &  &  &  &  \\
 &  &  &  &  &  &  &  &  &  &  &  &  &  \\ \midrule
\multirow{5}{*}{\begin{tabular}[c]{@{}c@{}}Fig.\\\ref{fig-au-difm}\end{tabular}} & \multirow{5}{*}{\begin{tabular}[c]{@{}c@{}}Fig.\\\ref{fig-au-dif2010m}\end{tabular}} & \multirow{5}{*}{Y} & \multirow{5}{*}{\begin{tabular}[c]{@{}l@{}}3 more best:\\ Pk 15, 23,\\ 27 only\\ for 20/10\\ (E)\end{tabular}} & \multirow{5}{*}{\begin{tabular}[c]{@{}l@{}}1, 11,\\ 13, 17,\\ 19,\\ 21,\\ 25\end{tabular}} & \multirow{5}{*}{Y} & \multirow{5}{*}{\textit{None}} & \multirow{5}{*}{\begin{tabular}[c]{@{}l@{}}2,\\ 14,\\ 16,\\ 22,\\ 26\end{tabular}} & \multirow{5}{*}{Y} & \multirow{5}{*}{Y} & \multirow{5}{*}{Y} & \multirow{5}{*}{N} & \multirow{5}{*}{N} & \multirow{5}{*}{\begin{tabular}[c]{@{}l@{}}(0,20)(5)  (5,7)(3)\\ (1,11,13,16,18,19,21,25)(0,1,3,5)\\ (8,17)(0\textendash3,5)  9(0,1)  10(1,5)\\ 12(2)  14(2,4,5)  (22,26)(0\textendash2,4,5)\\ 24(1,2,5); account for 39.88\%\end{tabular}} \\
 &  &  &  &  &  &  &  &  &  &  &  &  &  \\
 &  &  &  &  &  &  &  &  &  &  &  &  &  \\
 &  &  &  &  &  &  &  &  &  &  &  &  &  \\
 &  &  &  &  &  &  &  &  &  &  &  &  &  \\ \bottomrule
\end{tabular}%
}
\end{table}
\end{landscape}

\subsubsection{Other Window Sizes and Steps}

\paragraph{What to expect is}
the difference values for larger window/step size should be generally lower than
those for smaller window/step size, which further could result in more best
methods and less worst methods.

\begin{landscape}
\renewcommand*{\arraystretch}{.9}  % arraystretch sets the vertical (row) spacing
\begin{table}[]
\centering
\caption{Equal-weight 120\textendash0 Ma $\mathcal{CPD}s$ for the three representative continents' paleomagnetic APWPs compared with their FHM predicted APWPs. The best are in dark green and underlined, second best in green and third in light green.}\label{tab-pk0vs1bs}
\resizebox{.78\width}{!}{%
\begin{tabular}{@{}lllllllllllll@{}}
\toprule
\multicolumn{1}{c}{\multirow{2}{*}{\begin{tabular}[c]{@{}c@{}}Window/step\\ size (Myr)\end{tabular}}} & \multicolumn{6}{c}{Pk 0} & \multicolumn{6}{c}{Pk 1} \\ \cmidrule(l){2-13} 
\multicolumn{1}{c}{} & \multicolumn{1}{c}{Wt 0} & \multicolumn{1}{c}{Wt 1} & \multicolumn{1}{c}{Wt 2} & \multicolumn{1}{c}{Wt 3} & \multicolumn{1}{c}{Wt 4} & \multicolumn{1}{c}{Wt 5} & \multicolumn{1}{c}{Wt 0} & \multicolumn{1}{c}{Wt 1} & \multicolumn{1}{c}{Wt 2} & \multicolumn{1}{c}{Wt 3} & \multicolumn{1}{c}{Wt 4} & \multicolumn{1}{c}{Wt 5} \\ \midrule
\multicolumn{13}{c}{North America} \\ \midrule
2/1 & 0.2864 & 0.2925 & 0.28685 & 0.280944 & 0.287 & 0.29324 & {\color[HTML]{34FF34} \textbf{0.005759}} & 0.0091586 & {\color[HTML]{32CB00} \textbf{0.01528}} & {\color[HTML]{009901} {\ul\textbf{0.0090683}}} & 0.03601 & {\color[HTML]{32CB00} \textbf{0.009034}} \\
4/2 & 0.10584 & 0.09228 & 0.10938 & 0.11301 & 0.1131 & 0.11263 & {\color[HTML]{009901} {\ul\textbf{0.00525}}} & 0.007665 & 0.01701 & {\color[HTML]{32CB00} \textbf{0.009612}} & 0.03504 & {\color[HTML]{009901} {\ul\textbf{0.007863}}} \\
6/3 & 0.06486 & 0.064078 & 0.077426 & 0.073621 & 0.082867 & 0.075522 & {\color[HTML]{32CB00} \textbf{0.005418}} & {\color[HTML]{34FF34} \textbf{0.007522}} & 0.017936 & {\color[HTML]{34FF34} \textbf{0.0097317}} & {\color[HTML]{34FF34} \textbf{0.01983}} & 0.0095782 \\
8/4 & 0.06934 & 0.07168 & 0.082474 & 0.0707846 & 0.11473 & 0.071686 & 0.006 & {\color[HTML]{32CB00} \textbf{0.006217}} & {\color[HTML]{34FF34} \textbf{0.01556}} & 0.01242 & 0.02163 & 0.011483 \\
10/5 & 0.04412 & 0.04486 & 0.04739 & 0.04212 & 0.04753 & 0.04445 & 0.00591 & {\color[HTML]{009901} {\ul\textbf{0.00616}}} & {\color[HTML]{009901} {\ul\textbf{0.01262}}} & 0.01208 & {\color[HTML]{009901} {\ul\textbf{0.01529}}} & {\color[HTML]{34FF34} \textbf{0.00941}} \\
12/6 & {\color[HTML]{32CB00} \textbf{0.03271}} & {\color[HTML]{32CB00} \textbf{0.02987}} & {\color[HTML]{009901} {\ul\textbf{0.03024}}} & {\color[HTML]{009901} {\ul\textbf{0.0276143}}} & {\color[HTML]{009901} {\ul\textbf{0.03012}}} & {\color[HTML]{009901} {\ul\textbf{0.02988}}} & 0.008736 & 0.00902 & 0.01715 & 0.015665 & {\color[HTML]{32CB00} \textbf{0.01642}} & 0.01453 \\
14/7(119\textendash0) & 0.0367 & 0.03695 & 0.03498 & 0.0347 & {\color[HTML]{34FF34} \textbf{0.0319}} & 0.0352 & 0.0115 & 0.01198 & 0.023922 & 0.01477 & 0.02205 & 0.012985 \\
16/8 & 0.052386 & 0.04923 & 0.04964 & 0.04653 & 0.047712 & 0.048523 & 0.01138 & 0.0117 & 0.02177 & 0.015409 & 0.023068 & 0.01616 \\
20/10 & 0.04972 & 0.05356 & 0.0562 & 0.05352 & 0.0542 & 0.04902 & 0.014 & 0.01728 & 0.02896 & 0.0181 & 0.02702 & 0.0185 \\
24/12 & {\color[HTML]{009901} {\ul\textbf{0.031642}}} & {\color[HTML]{34FF34} \textbf{0.0342}} & {\color[HTML]{34FF34} \textbf{0.03273}} & {\color[HTML]{34FF34} \textbf{0.03469}} & {\color[HTML]{32CB00} \textbf{0.0314765}} & {\color[HTML]{32CB00} \textbf{0.031253}} & 0.0164 & 0.016737 & 0.02521 & 0.01907 & 0.02397 & 0.01924 \\
30/15 & {\color[HTML]{34FF34} \textbf{0.0345}} & {\color[HTML]{009901} {\ul\textbf{0.0298}}} & {\color[HTML]{32CB00} \textbf{0.0317}} & {\color[HTML]{32CB00} \textbf{0.0307}} & 0.03402 & {\color[HTML]{34FF34} \textbf{0.0341}} & 0.0206 & 0.0211 & 0.0313 & 0.0272 & 0.0293 & 0.0277 \\ \midrule
\multicolumn{13}{c}{India} \\ \midrule
2/1 & 0.25838 & 0.258757 & 0.258701 & 0.275184 & 0.265442 & 0.258692 & 0.05446 & 0.05639 & 0.06456 & 0.06075 & 0.0677596 & 0.0583844 \\
4/2 & 0.19345 & 0.20276 & 0.19344 & 0.24597 & 0.203264 & 0.19423 & 0.05588 & 0.056691 & 0.066803 & 0.062283 & 0.067216 & 0.060121 \\
6/3 & 0.17396 & 0.17476 & 0.173975 & 0.229325 & 0.18514 & 0.175174 & 0.057887 & 0.05882 & 0.0664754 & 0.0627534 & 0.06725 & 0.06083 \\
8/4 & 0.15309 & 0.14966 & 0.15321 & 0.17412 & 0.172154 & 0.15124 & 0.05761 & 0.05848 & 0.06671 & 0.06139 & 0.0669 & 0.05941 \\
10/5 & 0.1839 & 0.1845 & 0.1909 & 0.2586 & 0.1972 & 0.1852 & 0.0545 & 0.0554 & 0.0662 & 0.0589 & 0.066 & 0.06 \\
12/6 & 0.108924 & 0.1035 & {\color[HTML]{34FF34} \textbf{0.11205}} & 0.11831 & 0.121497 & {\color[HTML]{34FF34} \textbf{0.10587}} & 0.059897 & 0.06068 & 0.06993 & 0.064499 & 0.06951 & 0.062945 \\
14/7(119\textendash0) & 0.112537 & 0.112885 & 0.126554 & 0.139516 & 0.132359 & 0.114195 & {\color[HTML]{32CB00} \textbf{0.04942}} & {\color[HTML]{34FF34} \textbf{0.0502588}} & 0.0579931 & 0.060018 & 0.0654112 & 0.0582519 \\
16/8 & {\color[HTML]{34FF34} \textbf{0.104461}} & 0.104463 & 0.121002 & 0.110942 & 0.119599 & 0.118336 & {\color[HTML]{34FF34} \textbf{0.051735}} & 0.052813 & {\color[HTML]{34FF34} \textbf{0.055188}} & {\color[HTML]{34FF34} \textbf{0.056389}} & 0.0574883 & 0.055042 \\
20/10 & 0.1052 & {\color[HTML]{34FF34} \textbf{0.1015}} & 0.1198 & {\color[HTML]{34FF34} \textbf{0.1096}} & {\color[HTML]{34FF34} \textbf{0.1174}} & 0.1166 & {\color[HTML]{009901} {\ul\textbf{0.0492}}} & {\color[HTML]{32CB00} \textbf{0.0501}} & 0.0585 & {\color[HTML]{009901} {\ul\textbf{0.053}}} & {\color[HTML]{009901} {\ul\textbf{0.0536}}} & {\color[HTML]{009901} {\ul\textbf{0.052}}} \\
24/12 & {\color[HTML]{009901} {\ul\textbf{0.053143}}} & {\color[HTML]{009901} {\ul\textbf{0.05356}}} & {\color[HTML]{009901} {\ul\textbf{0.056986}}} & {\color[HTML]{009901} {\ul\textbf{0.05747}}} & {\color[HTML]{009901} {\ul\textbf{0.05558}}} & {\color[HTML]{009901} {\ul\textbf{0.0553047}}} & 0.051926 & {\color[HTML]{009901} {\ul\textbf{0.045995}}} & {\color[HTML]{009901} {\ul\textbf{0.04868}}} & {\color[HTML]{32CB00} \textbf{0.0557455}} & {\color[HTML]{34FF34} \textbf{0.05698}} & {\color[HTML]{34FF34} \textbf{0.05456}} \\
30/15 & {\color[HTML]{32CB00} \textbf{0.05617}} & {\color[HTML]{32CB00} \textbf{0.0754578}} & {\color[HTML]{32CB00} \textbf{0.0775947}} & {\color[HTML]{32CB00} \textbf{0.0575459}} & {\color[HTML]{32CB00} \textbf{0.0565421}} & {\color[HTML]{32CB00} \textbf{0.056635}} & 0.0523614 & 0.0519862 & {\color[HTML]{32CB00} \textbf{0.054158}} & 0.0563985 & {\color[HTML]{32CB00} \textbf{0.0555998}} & {\color[HTML]{32CB00} \textbf{0.0543501}} \\ \midrule
\multicolumn{13}{c}{Australia} \\ \midrule
2/1 & 0.471554 & 0.529417 & 0.52834 & 0.563622 & 0.628921 & 0.55071 & 0.00498465 & 0.0047458 & 0.0247455 & 0.0072776 & 0.035936 & {\color[HTML]{34FF34} \textbf{0.0039155}} \\
4/2 & 0.26822 & 0.29862 & 0.25881 & 0.276464 & 0.33741 & 0.268965 & 0.004543 & 0.004578 & 0.023895 & 0.005417 & 0.032511 & {\color[HTML]{009901} {\ul\textbf{0.0037674}}} \\
6/3 & 0.090985 & 0.0947676 & 0.09448 & 0.08977 & 0.10083 & 0.091376 & {\color[HTML]{32CB00} \textbf{0.0040955}} & 0.004847 & 0.02199 & {\color[HTML]{34FF34} \textbf{0.0046074}} & 0.023489 & 0.0064495 \\
8/4 & 0.0870445 & 0.097201 & 0.057284 & 0.086078 & 0.06245 & 0.04463 & 0.004265 & {\color[HTML]{34FF34} \textbf{0.004191}} & 0.026156 & 0.0067075 & 0.02533 & 0.00757145 \\
10/5 & 0.0315 & 0.039 & {\color[HTML]{34FF34} \textbf{0.0509}} & 0.0305 & 0.0601 & 0.031 & 0.0045 & 0.0048 & 0.0199 & 0.0058 & 0.0288 & 0.0089 \\
12/6 & 0.027348 & {\color[HTML]{34FF34} \textbf{0.026942}} & 0.057592 & 0.026687 & 0.055426 & {\color[HTML]{009901} {\ul\textbf{0.0270495}}} & 0.004341 & 0.0049061 & 0.020529 & 0.007813 & 0.020667 & 0.0054794 \\
14/7(119\textendash0) & 0.02725 & 0.033687 & {\color[HTML]{32CB00} \textbf{0.046242}} & {\color[HTML]{34FF34} \textbf{0.0252737}} & {\color[HTML]{34FF34} \textbf{0.04651}} & {\color[HTML]{32CB00} \textbf{0.027222}} & {\color[HTML]{009901} {\ul\textbf{0.0017226}}} & {\color[HTML]{009901} {\ul\textbf{0.00354029}}} & {\color[HTML]{34FF34} \textbf{0.0119085}} & {\color[HTML]{009901} {\ul\textbf{0.00157378}}} & 0.0121774 & 0.00765843 \\
16/8 & 0.028261 & 0.037993 & 0.0536856 & 0.0278993 & 0.05263 & {\color[HTML]{34FF34} \textbf{0.0282745}} & 0.0050338 & {\color[HTML]{32CB00} \textbf{0.0037223}} & 0.018136 & 0.0047548 & 0.014537 & 0.01061 \\
20/10 & {\color[HTML]{34FF34} \textbf{0.0257}} & 0.0378 & 0.0616 & 0.0254 & 0.0526 & 0.0373 & 0.00544 & 0.012 & 0.023 & 0.0162 & {\color[HTML]{34FF34} \textbf{0.0119}} & 0.0137 \\
24/12 & {\color[HTML]{32CB00} \textbf{0.0219624}} & {\color[HTML]{32CB00} \textbf{0.0214394}} & {\color[HTML]{009901} {\ul\textbf{0.043667}}} & {\color[HTML]{32CB00} \textbf{0.0216832}} & {\color[HTML]{32CB00} \textbf{0.0372685}} & 0.0345906 & 0.0058473 & 0.0063788 & {\color[HTML]{32CB00} \textbf{0.0059937}} & 0.0125858 & {\color[HTML]{009901} {\ul\textbf{0.002742}}} & 0.0051643 \\
30/15 & {\color[HTML]{009901} {\ul\textbf{0.014614}}} & {\color[HTML]{009901} {\ul\textbf{0.0183819}}} & 0.0786527 & {\color[HTML]{009901} {\ul\textbf{0.014254}}} & {\color[HTML]{009901} {\ul\textbf{0.031029}}} & 0.0293416 & {\color[HTML]{34FF34} \textbf{0.00412276}} & 0.00444694 & {\color[HTML]{009901} {\ul\textbf{0.00448627}}} & {\color[HTML]{32CB00} \textbf{0.00397289}} & {\color[HTML]{32CB00} \textbf{0.0054747}} & {\color[HTML]{32CB00} \textbf{0.00387337}} \\ \bottomrule
\end{tabular}%
}
\end{table}
\end{landscape}

\begin{figure}[!ht]
\centering
\includegraphics[width=1\textwidth]{../../paper/tex/GeophysJInt/figures/WinStpVsCPD.pdf}
\caption[Sliding window and step sizes vs $\mathcal{CPD}$]{Plot of the
equal-weight $\mathcal{CPD}$ scores collected in Table~\ref{tab-pk0vs1bs}. Note
that here the step size is always half of the sliding window size and the
reference path is the FHM derived.}\label{fig-WinStpVsCPD}
\end{figure}

\paragraph{The results}
are summarised in Table~\ref{tab-2010vs105F}, Table~\ref{tab-pk0vs1bs} and
Fig.~\ref{fig-WinStpVsCPD}. Based on these results, the effect of doubling the
size of the time window and step is negligible. $\mathcal{CPD}$ scores for APWPs
generated with no filtering (Pk 0,1) over a wider range of time windows and
steps, from 2/1 to 30/15 are shown in Fig.~\ref{fig-WinStpVsCPD} and
Table~\ref{tab-pk0vs1bs}. Scores for APP-derived paths (Pk 1) remain stable over
the whole tested range. AMP-derived paths (Pk 0) have higher scores for windows
narrower than 10 Myr, but are much more stable for windows wider than
10\textendash12 Myr, whilst remaining generally higher than the equivalent APP
scores.


\subsection{Moving versus Fixed Hotspot Reference}

Fig.~\ref{fig-difm} and Fig.~\ref{fig-dif2010m} show the $\mathcal{CPD}$ scores
for the APWPs generated with all 28 picking and 6 weighting methods, compared to
the MHM reference paths (stars and solid line in Figs.~\ref{fig-mhsPred}-\ref{fig-mhsPred801})
with picking time windows and steps of 10/5 Myr and 20/10 Myr.

\subsubsection{Overall Change}

Although the reference path is changed to MHM, mean, median and range values of
$\mathcal{CPD}$ scores are almost unchanged and comparable (Fig.~\ref{fig-dif}
versus Fig.~\ref{fig-difm}, and Fig.~\ref{fig-dif2010} versus
Fig.~\ref{fig-dif2010m}).

\begin{figure*}
	\centering
	\begin{subfigure}{1.01\textwidth}
		\includegraphics[width=\textwidth]{../../paper/tex/GeophysJInt/figures/101_120_0m.pdf}
		\caption{North America: minimum 0.00268588 (5(1)),
		maximum 0.0892467 (16(3)), mean 0.03674, median 0.03177075}\label{fig-na-difm}
	\end{subfigure}
	\vspace{.1em}
	\begin{subfigure}{1.01\textwidth}
		\includegraphics[width=\textwidth]{../../paper/tex/GeophysJInt/figures/501_120_0m.pdf}
		\caption{India: minimum 0.0232517 (19(1)), maximum 0.51876 (8(3)),
		mean 0.11364, median 0.076556}\label{fig-in-difm}
	\end{subfigure}
	\vspace{.1em}
	\begin{subfigure}{1.01\textwidth}
		\includegraphics[width=\textwidth]{../../paper/tex/GeophysJInt/figures/801_120_0m.pdf}
		\caption{Australia: minimum 0.00122074 (17(5)), maximum
		0.367952 (26(3)), mean 0.077, median 0.03893}\label{fig-au-difm}
	\end{subfigure}
	\caption[$\mathcal{CPD}$ of each plate's paleomagnetic APWPs vs
its MHM predicted APWP]{As Fig.~\ref{fig-dif}, here the reference path is
predicted from MHM\@. See the numbers of the picked paleopoles for methods in
Fig.~\ref{fig-dif}.}\label{fig-difm}
\end{figure*}

\begin{figure*}
	\centering
	\begin{subfigure}{.42\textwidth}
		\includegraphics[width=\textwidth]{../../paper/tex/GeophysJInt/figures/ay18_101comb_10_5_5_1m.pdf}
		\caption{North America: minimum 0.002686 (5(1)), FQ C-A}
	\end{subfigure}
	\begin{subfigure}{.42\textwidth}
		\includegraphics[width=\textwidth]{../../paper/tex/GeophysJInt/figures/ay18_101comb_10_5_16_3m.pdf}
		\caption{North America: maximum 0.08925 (16(3)), FQ B-A}
	\end{subfigure}
	\vspace{.1em}
	\begin{subfigure}{.42\textwidth}
		\includegraphics[width=\textwidth]{../../paper/tex/GeophysJInt/figures/ay18_501comb_10_5_19_1m.pdf}
		\caption{India: minimum 0.0233 (19(1)), FQ C-A}
	\end{subfigure}
	\begin{subfigure}{.42\textwidth}
		\includegraphics[width=\textwidth]{../../paper/tex/GeophysJInt/figures/ay18_501comb_10_5_8_3m.pdf}
		\caption{India: maximum 0.5188 (8(3)), FQ C-A}
	\end{subfigure}
	\vspace{.1em}
	\begin{subfigure}{.42\textwidth}
		\includegraphics[width=\textwidth]{../../paper/tex/GeophysJInt/figures/ay18_801comb_10_5_17_5m.pdf}
		\caption{Australia: minimum 0.00122 (17(5)), FQ B-A}
	\end{subfigure}
	\begin{subfigure}{.42\textwidth}
		\includegraphics[width=\textwidth]{../../paper/tex/GeophysJInt/figures/ay18_801comb_10_5_26_3m.pdf}
		\caption{Australia: maximum 0.368 (26(3)), FQ C-A}
	\end{subfigure}
	\caption[Best and worst $\mathcal{CPD}$s (10/5 Myr window/step; MHM)]{Path
comparisons with best and worst difference values shown in
Fig.~\ref{fig-difm}. The parenthetical remarks are Pk No with Wt
No.}\label{fig-difbwm}
\end{figure*}

\begin{figure*}
	\centering
	\begin{subfigure}{1.01\textwidth}
		\includegraphics[width=\textwidth]{../../paper/tex/GeophysJInt/figures/101_20_10_120_0m.pdf}
		\caption{North America: minimum 0.00565784 (5(1)),
		maximum 0.104618 (14(2)), mean 0.043777, median 0.02679955}\label{fig-na-dif2010m}
	\end{subfigure}
	\vspace{.1em}
	\begin{subfigure}{1.01\textwidth}
		\includegraphics[width=\textwidth]{../../paper/tex/GeophysJInt/figures/501_20_10_120_0m.pdf}
		\caption{India: minimum 0.0177 (6(3)), maximum 0.15937 (16(1)), mean
		0.0745, median 0.061346}\label{fig-in-dif2010m}
	\end{subfigure}
	\vspace{.1em}
	\begin{subfigure}{1.01\textwidth}
		\includegraphics[width=\textwidth]{../../paper/tex/GeophysJInt/figures/801_20_10_120_0m.pdf}
		\caption{Australia: minimum 0.00282 (23(4)), maximum
		0.31998 (22(3)), mean 0.062766, median 0.02803}\label{fig-au-dif2010m}
	\end{subfigure}
	\caption[$\mathcal{CPD}$ of each plate's paleomagnetic APWPs vs its MHM
predicted APWP (20/10 Myr window/step)]{As Fig.~\ref{fig-dif2010}, here the
reference path is predicted from MHM\@. See the numbers of picked paleopoles in
Fig.~\ref{fig-dif}.}\label{fig-dif2010m}
\end{figure*}

\begin{figure*}
	\centering
	\begin{subfigure}{.43\textwidth}
		\includegraphics[width=\textwidth]{../../paper/tex/GeophysJInt/figures/ay18_101comb_20_10_5_1m.pdf}
		\caption{North America: minimum 0.00566 (5(1)), FQ B-A}
	\end{subfigure}
	\begin{subfigure}{.43\textwidth}
		\includegraphics[width=\textwidth]{../../paper/tex/GeophysJInt/figures/ay18_101comb_20_10_14_2m.pdf}
		\caption{North America: maximum 0.1046 (14(2)), FQ B-A}
	\end{subfigure}
	\vspace{.1em}
	\begin{subfigure}{.43\textwidth}
		\includegraphics[width=\textwidth]{../../paper/tex/GeophysJInt/figures/ay18_501comb_20_10_6_3m.pdf}
		\caption{India: minimum 0.0177 (6(3)), FQ C-A}
	\end{subfigure}
	\begin{subfigure}{.43\textwidth}
		\includegraphics[width=\textwidth]{../../paper/tex/GeophysJInt/figures/ay18_501comb_20_10_16_1m.pdf}
		\caption{India: maximum 0.1594 (16(1)), FQ C-A}
	\end{subfigure}
	\vspace{.1em}
	\begin{subfigure}{.43\textwidth}
		\includegraphics[width=\textwidth]{../../paper/tex/GeophysJInt/figures/ay18_801comb_20_10_23_4m.pdf}
		\caption{Australia: minimum 0.0028 (23(4)), FQ C-A}
	\end{subfigure}
	\begin{subfigure}{.43\textwidth}
		\includegraphics[width=\textwidth]{../../paper/tex/GeophysJInt/figures/ay18_801comb_20_10_22_3m.pdf}
		\caption{Australia: maximum 0.32 (22(3)), FQ C-A}
	\end{subfigure}
	\caption[Best and worst $\mathcal{CPD}$s (20/10 Myr window/step;
MHM)]{Path comparisons with best and worst difference values shown in
Fig.~\ref{fig-dif2010m}. The parenthetical remarks are Pk No and Wt No.}\label{fig-dif2010bwm}
\end{figure*}

For 10/5 Myr window/step, the absolute differences (Fig.~\ref{fig-dmf}) are all
lower than 0.066, and actually most are less than 0.01. This indicates that for
comparing with paleomagnetic APWPs choosing fixed or moving hotspot model for
generating a reference path is not quite different. Therefore selecting fixed or
moving model for having a reference path is not a priority. However, based on
the signs of the differences between the scores from FHM and MHM
(Fig.~\ref{fig-dmf}), for North America (Fig.~\ref{fig-mf101}), FHM derived path
is a slighly better reference in general, while for both India
(Fig.~\ref{fig-mf501}) and Australia (Fig.~\ref{fig-mf801}), generally MHM
derived path is a slightly better choice.

\begin{figure*}
	\centering
	\begin{subfigure}{1.01\textwidth}
		\includegraphics[width=\textwidth]{../../paper/tex/GeophysJInt/figures/mf101.pdf}
		\caption{Fig.~\ref{fig-na-difm}-Fig.~\ref{fig-na-dif}; Percentage for
		positive values: ${\sim}66.67$\%}\label{fig-mf101}
	\end{subfigure}
	\vspace{.1em}
	\begin{subfigure}{1.01\textwidth}
		\includegraphics[width=\textwidth]{../../paper/tex/GeophysJInt/figures/mf501.pdf}
		\caption{Fig.~\ref{fig-in-difm}-Fig.~\ref{fig-in-dif}; Percentage for
		positive values: ${\sim}34.52$\%}\label{fig-mf501}
	\end{subfigure}
	\vspace{.1em}
	\begin{subfigure}{1.01\textwidth}
		\includegraphics[width=\textwidth]{../../paper/tex/GeophysJInt/figures/mf801.pdf}
		\caption{Fig.~\ref{fig-au-difm}-Fig.~\ref{fig-au-dif}; Percentage for
		positive values: ${\sim}19.62$\%}\label{fig-mf801}
	\end{subfigure}
	\caption[Differences between results from FHM and MHM]{Differences between
results from two different reference paths, FHM (Fig.~\ref{fig-dif}) and MHM
(Fig.~\ref{fig-difm}) derived. The absolute difference values less than
1.96-standard-deviation interval of the whole 168 values are labeled in green,
more than 1.96-standard-deviation interval labeled in red.}\label{fig-dmf}
\end{figure*}

For North America, large changes seem favored by Wt 3, and small changes seem
favored by Wt 5 or Pk 15 (Fig.~\ref{fig-dmf}). Pk 3 brings minor changes to both
North America and India. Pk 22 and 26 show large changes and Pk 17, 19, 21 and
25 show minor changes for both India and Australia.

\subsubsection{Relative Performance of Methods}

\begin{enumerate}
  \item When 10/5 Myr bin/step is applied, Pk 19 is still among the best,
	and Pk 16 still one of the worst, for all 3 plates. Even when 20/10 Myr
	bin/step is applied, 19 is still relatively a ``good'' method, and 16 a
	``bad'' one.
  \item APP still outperforms AMP\@. For 10/5 Myr bin/step, the percentage of
	factor of greater than 3 (about 19.05\%, 38.1\% and 48.81\% for North
	America, India and Australia respectively) is less than FHM (about 40.5\%,
	39.3\% and 50\%). For 20/10 Myr bin/step, the percentage of factor of
	greater than 3 (about 3.6\% and 46.43\% for India and Australia) is still
	less than FHM (about 3.8\% and 71.4\%) whilst for North America the
	percentage (${\sim}44.05$\%) is more than FHM (28.6\%).
  \item Both scores and AMP/APP difference are still low for Pk 4,5, 6,7.
  \item Both scores and AMP/APP difference are still high for Pk 16,17
	(Pk 17 gives low scores for Australia).
  \item Still scores are high but AMP/APP difference are low for Pk 2,3.
  \item Still scores are low but AMP/APP difference are high for Pk 0,1 on
	Australia.
  \item Comparable AMP/APP (20/10 versus 10/5, or FHM versus MHM) still appears
        for Pk 8,9 on Australian plate.
  \item Pk 1 is still a good performer.
  \item Pk 4,5, 6,7 are still `good' aggressive filters.
  \item Pk 2,3 (at least 2) are still `bad' aggressive filters.
  \item For 10/5 Myr bin/step, Pk 22 is still better than 23 for India. For
	20/10 Myr bin/step, Pk 22 is still better than 23 for both North
	America and India.
  \item Pk 16,17 are still `bad' for North America and India.
\end{enumerate}

\begin{landscape}
%\setlength\tabcolsep{3pt}
\begin{table}[]
\centering
\caption{Performance statistics of all the picking and weighting methods for 10/5 and 20/10 Myr window/step path comparisons.}\label{tab-bw}
\resizebox{.92\width}{!}{%
\begin{tabular}{@{}clllllcccccccc@{}}
\toprule
\multirow{3}{*}{\begin{tabular}[c]{@{}c@{}}Hotspot\\ model\\ for ref path\end{tabular}} & \multicolumn{1}{c}{\multirow{3}{*}{\begin{tabular}[c]{@{}c@{}}CPD\\ grid\end{tabular}}} & \multicolumn{2}{c}{\multirow{2}{*}{Best}} & \multicolumn{2}{c}{\multirow{2}{*}{Worst}} & \multirow{3}{*}{\begin{tabular}[c]{@{}c@{}}\% of\\ APP better\\ than AMP\end{tabular}} & \multicolumn{6}{c}{\multirow{2}{*}{\begin{tabular}[c]{@{}c@{}}Count occurrences of each\\ Wt being best for 28 Pks\end{tabular}}} & \multirow{3}{*}{\begin{tabular}[c]{@{}c@{}}Pk14,15 (New\\ Studies) better than\\ Pk16,17(Old)\end{tabular}} \\
 & \multicolumn{1}{c}{} & \multicolumn{2}{c}{} & \multicolumn{2}{c}{} &  & \multicolumn{6}{c}{} &  \\ \cmidrule(lr){3-6} \cmidrule(lr){8-13}
 & \multicolumn{1}{c}{} & \multicolumn{1}{c}{Pk} & \multicolumn{1}{c}{Wt} & \multicolumn{1}{c}{Pk} & \multicolumn{1}{c}{Wt} &  & 0 & 1 & 2 & 3 & 4 & 5 &  \\ \midrule
\multirow{12}{*}{FHM} & \multirow{2}{*}{\begin{tabular}[c]{@{}l@{}}Fig.\\\ref{fig-na-dif}\end{tabular}} & \multirow{2}{*}{\begin{tabular}[c]{@{}l@{}}1, 5, 7, 11, 13\\ 15, \textbf{19}, \textbf{21}, 25\end{tabular}} & \multirow{2}{*}{\begin{tabular}[c]{@{}l@{}}0, 1,\\ 3, 5\end{tabular}} & \multirow{2}{*}{\begin{tabular}[c]{@{}l@{}}2, 5, 7, 14, \textbf{16},\\ 17, 18, 22, 26\end{tabular}} & \multirow{24}{*}{0\textendash5} & \multirow{2}{*}{97.6} & \multirow{2}{*}{\textbf{9}} & \multirow{2}{*}{4} & \multirow{2}{*}{4} & \multirow{2}{*}{7} & \multirow{2}{*}{0} & \multirow{2}{*}{5} & \multirow{4}{*}{Y} \\
 &  &  &  &  &  &  &  &  &  &  &  &  &  \\ \cmidrule(lr){2-5} \cmidrule(lr){7-13}
 & \multirow{2}{*}{\begin{tabular}[c]{@{}l@{}}Fig.\\\ref{fig-in-dif}\end{tabular}} & \multirow{2}{*}{\begin{tabular}[c]{@{}l@{}}4\textendash7, 9, \textbf{19}\\ \textbf{21}, 22, 26\end{tabular}} & \multirow{2}{*}{0\textendash5} & \multirow{2}{*}{\begin{tabular}[c]{@{}l@{}}0, 2, 8, 10, 12,\\ \textbf{16}, 18, 20, 24\end{tabular}} &  & \multirow{2}{*}{85.7} & \multirow{2}{*}{\textbf{15}} & \multirow{2}{*}{2} & \multirow{2}{*}{2} & \multirow{2}{*}{6} & \multirow{4}{*}{2} & \multirow{2}{*}{1} &  \\
 &  &  &  &  &  &  &  &  &  &  &  &  &  \\ \cmidrule(lr){2-5} \cmidrule(lr){7-11} \cmidrule(l){13-14} 
 & \multirow{2}{*}{\begin{tabular}[c]{@{}l@{}}Fig.\\\ref{fig-au-dif}\end{tabular}} & \multirow{2}{*}{\begin{tabular}[c]{@{}l@{}}1, 11, 13, 17,\\ \textbf{19}, \textbf{21}, 25\end{tabular}} & \multirow{4}{*}{\begin{tabular}[c]{@{}l@{}}0,\\ 1,\\ 3,\\ 5\end{tabular}} & \multirow{2}{*}{\begin{tabular}[c]{@{}l@{}}2, 4, 14,\\ \textbf{16}, 22, 26\end{tabular}} &  & \multirow{2}{*}{100} & \multirow{2}{*}{\textbf{10}} & \multirow{2}{*}{5} & \multirow{2}{*}{1} & \multirow{2}{*}{7} &  & \multirow{2}{*}{3} & \multirow{2}{*}{N} \\
 &  &  &  &  &  &  &  &  &  &  &  &  &  \\ \cmidrule(lr){2-3} \cmidrule(lr){5-5} \cmidrule(l){7-14} 
 & \multirow{2}{*}{\begin{tabular}[c]{@{}l@{}}Fig.\\\ref{fig-na-dif2010}\end{tabular}} & \multirow{2}{*}{\begin{tabular}[c]{@{}l@{}}1, 4\textendash7, 9, 11, 13\\ 15, \textbf{19}, \textbf{21}, 25\end{tabular}} &  & \multirow{2}{*}{\begin{tabular}[c]{@{}l@{}}0, 8, 10, 12, 14,\\ \textbf{16}, 18, 20, 24\end{tabular}} &  & \multirow{2}{*}{72.6} & \multirow{2}{*}{\textbf{14}} & \multirow{2}{*}{2} & \multirow{6}{*}{0} & \multirow{2}{*}{3} & \multirow{2}{*}{3} & \multirow{2}{*}{6} & \multirow{2}{*}{N, Y} \\
 &  &  &  &  &  &  &  &  &  &  &  &  &  \\ \cmidrule(lr){2-5} \cmidrule(lr){7-9} \cmidrule(l){11-14} 
 & \multirow{2}{*}{\begin{tabular}[c]{@{}l@{}}Fig.\\\ref{fig-in-dif2010}\end{tabular}} & \multirow{2}{*}{\begin{tabular}[c]{@{}l@{}}4\textendash7,\\ \textbf{19}\end{tabular}} & \multirow{2}{*}{0\textendash5} & \multirow{2}{*}{\begin{tabular}[c]{@{}l@{}}0, 2, 10, 12, \textbf{16},\\ 18, 20, 23, 24, 27\end{tabular}} &  & \multirow{2}{*}{69} & \multirow{2}{*}{\textbf{11}} & \multirow{4}{*}{6} &  & \multirow{2}{*}{9} & \multirow{4}{*}{1} & \multirow{2}{*}{1} & \multirow{2}{*}{Y, 4y2n} \\
 &  &  &  &  &  &  &  &  &  &  &  &  &  \\ \cmidrule(lr){2-5} \cmidrule(lr){7-8} \cmidrule(lr){11-11} \cmidrule(l){13-14} 
 & \multirow{2}{*}{\begin{tabular}[c]{@{}l@{}}Fig.\\\ref{fig-au-dif2010}\end{tabular}} & \multirow{2}{*}{\begin{tabular}[c]{@{}l@{}}1, 11, 13, 17, \textbf{19},\\ \textbf{21}, 23, 25, 27\end{tabular}} & \multirow{2}{*}{\begin{tabular}[c]{@{}l@{}}0, 1,\\ 3\textendash5\end{tabular}} & \multirow{2}{*}{\begin{tabular}[c]{@{}l@{}}2, 14, \textbf{16},\\ 22, 26\end{tabular}} &  & \multirow{2}{*}{100} & \multirow{2}{*}{\textit{9}} &  &  & \multirow{2}{*}{\textbf{10}} &  & \multirow{4}{*}{2} & \multirow{2}{*}{N} \\
 &  &  &  &  &  &  &  &  &  &  &  &  &  \\ \cmidrule(r){1-5} \cmidrule(lr){7-12} \cmidrule(l){14-14} 
\multirow{12}{*}{MHM} & \multirow{2}{*}{\begin{tabular}[c]{@{}l@{}}Fig.\\\ref{fig-na-difm}\end{tabular}} & \multirow{2}{*}{\begin{tabular}[c]{@{}l@{}}1, 5, 7, 9, 11, 13,\\ 15, \textbf{19}, \textbf{21}, 25\end{tabular}} & \multirow{4}{*}{0\textendash5} & \multirow{2}{*}{\begin{tabular}[c]{@{}l@{}}0, 10, 12, 14,\\ \textbf{16}\textendash18, 20, 24\end{tabular}} &  & \multirow{2}{*}{97.6} & \multirow{2}{*}{\textbf{10}} & \multirow{2}{*}{9} & \multirow{2}{*}{3} & \multirow{2}{*}{4} & \multirow{2}{*}{0} &  & \multirow{4}{*}{Y} \\
 &  &  &  &  &  &  &  &  &  &  &  &  &  \\ \cmidrule(lr){2-3} \cmidrule(lr){5-5} \cmidrule(lr){7-13}
 & \multirow{2}{*}{\begin{tabular}[c]{@{}l@{}}Fig.\\\ref{fig-in-difm}\end{tabular}} & \multirow{2}{*}{\begin{tabular}[c]{@{}l@{}}5, 7, \textbf{19},\\ \textbf{21}, 22, 26\end{tabular}} &  & \multirow{2}{*}{\begin{tabular}[c]{@{}l@{}}0, 2, 8, 10, 12,\\ 14, \textbf{16}, 18, 20, 24\end{tabular}} &  & \multirow{2}{*}{85.7} & \multirow{2}{*}{\textbf{12}} & \multirow{2}{*}{2} & \multirow{2}{*}{0} & \multirow{2}{*}{7} & \multirow{4}{*}{4} & \multirow{2}{*}{3} &  \\
 &  &  &  &  &  &  &  &  &  &  &  &  &  \\ \cmidrule(lr){2-5} \cmidrule(lr){7-11} \cmidrule(l){13-14} 
 & \multirow{2}{*}{\begin{tabular}[c]{@{}l@{}}Fig.\\\ref{fig-au-difm}\end{tabular}} & \multirow{2}{*}{\begin{tabular}[c]{@{}l@{}}1, 11, 13, 17,\\ \textbf{19}, \textbf{21}, 25\end{tabular}} & \multirow{2}{*}{\begin{tabular}[c]{@{}l@{}}0\textendash3,\\ 5\end{tabular}} & \multirow{2}{*}{\begin{tabular}[c]{@{}l@{}}2, 14, \textbf{16},\\ 22, 26\end{tabular}} &  & \multirow{2}{*}{98.8} & \multirow{2}{*}{\textit{6}} & \multirow{2}{*}{4} & \multirow{2}{*}{2} & \multirow{2}{*}{\textbf{10}} &  & \multirow{2}{*}{2} & \multirow{2}{*}{N} \\
 &  &  &  &  &  &  &  &  &  &  &  &  &  \\ \cmidrule(lr){2-5} \cmidrule(l){7-14} 
 & \multirow{2}{*}{\begin{tabular}[c]{@{}l@{}}Fig.\\\ref{fig-na-dif2010m}\end{tabular}} & \multirow{2}{*}{\begin{tabular}[c]{@{}l@{}}5, 7, 15,\\ 22, 26\end{tabular}} & \multirow{6}{*}{0\textendash5} & \multirow{2}{*}{\begin{tabular}[c]{@{}l@{}}0, 8, 10, 12,\\ 14, 18, 20, 24\end{tabular}} &  & \multirow{2}{*}{76.2} & \multirow{2}{*}{\textbf{10}} & \multirow{2}{*}{8} & \multirow{2}{*}{1} & \multirow{2}{*}{4} & \multirow{2}{*}{1} & \multirow{2}{*}{4} & \multirow{2}{*}{1y5n, Y} \\
 &  &  &  &  &  &  &  &  &  &  &  &  &  \\ \cmidrule(lr){2-3} \cmidrule(lr){5-5} \cmidrule(l){7-14} 
 & \multirow{2}{*}{\begin{tabular}[c]{@{}l@{}}Fig.\\\ref{fig-in-dif2010m}\end{tabular}} & \multirow{2}{*}{\begin{tabular}[c]{@{}l@{}}4\textendash7,\\ 22, 26\end{tabular}} &  & \multirow{2}{*}{\begin{tabular}[c]{@{}l@{}}0, 2, 10, 12,\\ \textbf{16}, 18, 24\end{tabular}} &  & \multirow{2}{*}{70.2} & \multirow{2}{*}{\textit{6}} & \multirow{2}{*}{\textbf{7}} & \multirow{4}{*}{0} & \multirow{2}{*}{6} & \multirow{2}{*}{3} & \multirow{2}{*}{6} & \multirow{2}{*}{Y, 3y3n} \\
 &  &  &  &  &  &  &  &  &  &  &  &  &  \\ \cmidrule(lr){2-3} \cmidrule(lr){5-5} \cmidrule(lr){7-9} \cmidrule(l){11-14} 
 & \multirow{2}{*}{\begin{tabular}[c]{@{}l@{}}Fig.\\\ref{fig-au-dif2010m}\end{tabular}} & \multirow{2}{*}{\begin{tabular}[c]{@{}l@{}}1, 11, 13, 15, 17,\\ \textbf{19}, \textbf{21}, 23, 25, 27\end{tabular}} &  & \multirow{2}{*}{\begin{tabular}[c]{@{}l@{}}2, 14, \textbf{16},\\ 22, 26\end{tabular}} &  & \multirow{2}{*}{100} & \multirow{2}{*}{\textbf{12}} & \multirow{2}{*}{1} &  & \multirow{2}{*}{10} & \multirow{2}{*}{4} & \multirow{2}{*}{1} & \multirow{2}{*}{N} \\
 &  &  &  &  &  &  &  &  &  &  &  &  &  \\ \bottomrule
\end{tabular}%
}
\end{table}
\end{landscape}

\section{Discussion}

\subsection{Question: Why APP Methods Generally Produce Better Similarities than
AMP Methods?}

\subsubsection{Perspective of Conceptual Difference between AMP and APP}

Paleomagnetic (Mean) A95 represents precision (how well constrained calculated
poles are), and (mean) coeval poles' GCD represents accuracy (how close
calculated poles are to the reference path; Fig.~\ref{fig-A95GCD105F} and
Fig.~\ref{fig-A95SGCD105F}). Compared with AMP, APP usually improves both and
generates paths with higher accuracy and also higher precision (generally
increasing number of contributing paleopoles).

\begin{figure}[!ht]
\captionsetup[subfigure]{singlelinecheck=off,justification=raggedright,aboveskip=-6pt,belowskip=-6pt}
\centering
  \begin{subfigure}[htbp]{.49\textwidth}
	\caption{}\includegraphics[width=\textwidth]{../../paper/tex/GeophysJInt/figures/a95gcd_10_5F.pdf}\label{fig-A95GCD105F}
  \end{subfigure}
  \begin{subfigure}[htbp]{.49\textwidth}
	\caption{}\includegraphics[width=\textwidth]{../../paper/tex/GeophysJInt/figures/a95Sgcd_10_5F.pdf}\label{fig-A95SGCD105F}
  \end{subfigure}
  \begin{subfigure}[htbp]{.49\textwidth}
	\caption{}\includegraphics[width=\textwidth]{../../paper/tex/GeophysJInt/figures/a95mSad10_5F.pdf}\label{fig-A95mSad105F}
  \end{subfigure}
  \begin{subfigure}[htbp]{.49\textwidth}
	\caption{}\includegraphics[width=\textwidth]{../../paper/tex/GeophysJInt/figures/a95mSld10_5F.pdf}\label{fig-A95mSld105F}
  \end{subfigure}
  \caption[APP spatially better than AMP (arrow)]{Paleomagnetic APWP's mean A95 versus
(a) ``mean GCD'', (b) ``mean significant GCD'', (c) ``mean significant
orientation difference'', and (d) ``mean significant length difference'' between
paleomagnetic APWP and its corresponding FHM-and-plate-circuit predicted APWP\@.
Arrowtails are the results from AMP, while arrowheads are from APP\@. Black
color filled arrowheads are the small number of special cases of AMP derived
equal-weight $\mathcal{CPD}$s better than APP (see details in Fig.~\ref{fig-dif}
and Table~\ref{tab-bw}).}\label{fig-A95mG105F}
\end{figure}

\begin{figure}
\captionsetup[subfigure]{singlelinecheck=off,justification=raggedright,aboveskip=-6pt,belowskip=-6pt}
\centering
  \begin{subfigure}[htbp]{.49\textwidth}
	\caption{}\includegraphics[width=\textwidth]{../../paper/tex/GeophysJInt/figures/a95gcdd.pdf}\label{fig-A95GCD105Fd}
  \end{subfigure}
  \begin{subfigure}[htbp]{.49\textwidth}
	\caption{}\includegraphics[width=\textwidth]{../../paper/tex/GeophysJInt/figures/a95Sgcd_10_5Fd.pdf}\label{fig-A95SGCD105Fd}
  \end{subfigure}
  \begin{subfigure}[htbp]{.49\textwidth}
	\caption{}\includegraphics[width=\textwidth]{../../paper/tex/GeophysJInt/figures/a95mSad10_5Fd.pdf}\label{fig-A95mSad105Fd}
  \end{subfigure}
  \begin{subfigure}[htbp]{.49\textwidth}
	\caption{}\includegraphics[width=\textwidth]{../../paper/tex/GeophysJInt/figures/a95mSld10_5Fd.pdf}\label{fig-A95mSld105Fd}
  \end{subfigure}
  \caption[APP spatially better than AMP (dot)]{Differences of APP and AMP coordinates
shown in Fig.~\ref{fig-A95mG105F}. Crosses locates the small minority cases of
AMP derived equal-weight $\mathcal{CPD}$s better than APP (see details in
Fig.~\ref{fig-dif} and Table~\ref{tab-bw}).}\label{fig-A95mG105Fd}
\end{figure}

The fact that APP increases the number of paleopoles (N) in each sliding window
would potentially average out some ``bad'' (i.e.\ inaccurate) poles and improves
the fit between the paleomagnetic APWPs and the model-predicted APWPs. The
general effects that APP brings include the decreases in paleomagnetic A95s,
or/and distances between compared coeval poles of paleomagnetic APWP and
reference APWP (Fig.~\ref{fig-A95mG105F} and Fig.~\ref{fig-A95mG105Fd}).
However, if the added paleopoles were all or mostly ``bad'', the improvement of
fit would not occur. So the improvement of fit is not only because of the
increase in N, but also because the majority of the additional poles are
``good''. AMP only regards the time uncertainty of each pole as one mid-point.
Then this mid-point is treated as the most likely age of that mean pole. This is
actually incorrect. The age uncertainty of paleopole is not obtained from a
probability density function derived from an observed frequency distribution. As
defined, the time uncertainty's lower (older) limit is a stratigraphic age, and
its upper (younger) limit could be also a stratigraphic age or be constrained by
a tectonic event using the field tests (e.g.\ fold/tilt test and conglomerate
test). So the true age of the pole could be any one that is not older than the
lower limit and also not younger than the upper limit. In other words, the
mid-point could be the true age of the pole, but it is not known as the most
likely age of that pole. If the mid-point is the most likely age of a pole, AMP
should generate a path that is closer to the reference. However, mostly APP
generates better similarities (See the high proportions of APP better than AMP
in Table~\ref{tab-bw}). Most reasonably, the mid-point should be regarded as
one possibility of all uniformly (not necessarily normally bell shaped, or U
shaped, or left or right skewed) distributed ages between the two time limits.

So APP remains the effect of a paleopole borne on the mean poles during all the
period of its age uncertainty, and use the increased number of paleopoles (N)
to average out the negative effect of those ``bad'' poles, including the
paleopoles that should not be included at that age for mean pole.

\subsubsection{Perspective of Stability Comparison between AMP and APP}

Fitting curves by moving averaging change with different time window lengths and
time increment lengths (i.e.\ steps) (e.g., the similarity of the pair in
Fig.~\ref{fig-au-2010110} is improved a bit compared to
Fig.~\ref{fig-au-105110}). A balance needs to be made between having windows
that are too wide and steps that are too long which will smooth the data so much
we miss actual details in the APWP (e.g.\ those 20 Myr window 10 Myr step
paleomagnetic paths in Fig.~\ref{fig-dif2010bw} and even 30/15 Myr window and
step; Table~\ref{tab-pk0vs1bs} and Fig.~\ref{fig-WinStpVsCPD}) and windows that
are too narrow and steps that are too short which introduces noise by having too
few poles in each window (e.g.\ 2 Myr window 1 Myr step;
Table~\ref{tab-pk0vs1bs} and Fig.~\ref{fig-WinStpVsCPD}). There is a dependence
here on data density: higher density allows smaller windows/steps (this is one
of the things we want to test with selective data removal in the future). A
variety of ways of binning the data (here 30\textendash2 Myr window size and
half of the size as step) are being tested to see which one produces the better
and more appropriately smoothed fit.

Note that there are 135, 75 and 99 paleopoles that compose of 120\textendash0 Ma
APWPs of North America, India and Australia respectively. Does the reason of
10/5 Myr generally better than 20/10 could be the relatively larger number of
paleopoles for North America? Since theoretically for each sliding window, the
more ``bad'' paleopoles it contains, the worse similarity we should obtain. In
the contrary, the less paleopoles the window contains, the weaker the effect of
averaging out ``bad'' poles' influence would be. So is there a threshold number
of paleopoles for making an paleomagnetic APWP\@? For example, for making a
120\textendash0 Ma APWP, do the results indicate the best number of paleopoles
we need should be some value between 99 and 135? Or is there a threshold number
of average paleopoles per window for making a reliable paleomagnetic APWP\@?
Here a test is implemented as
follows. With the results from the 10/5 and 20/10 bin/step together, 2/1, 4/2,
6/3, 8/4, 12/6, 14/7, 16/8, 20/10, 24/12 and 30/15 Myr bin/step are also used to
generate paleomagnetic APWPs for North America, India and Australia to see which
bin/step size would make paleomagnetic APWP closest to reference path. Will the
similarities they generate be generally worse than those the 10/5 Myr bin/step
generates? Or will they be better first and then worse than those the 10/5 Myr
bin/step generates when the bin/step sizes increase up to 20/10 Myr? For the
best results (Table~\ref{tab-pk0vs1bs}), as expected, AMP needs wider sliding
window and step to get closer to the reference path while APP does not
(Fig.~\ref{fig-WinStpVsCPD}). Even the best sizes of sliding window and step are
assigned for AMP, the results from APP are still much better than those from
AMP\@. Picking methods (directly related to N) are still the key influence
factor of choosing a better sliding window size and step size of moving
averaging, although weighting methods are also important.

\subsubsection{Summary}

\paragraph{If AMP has to be used,} better results can be obtained through
using large sizes of sliding window and step, commonly more than 24/12 Myr. In
addition, we should be cautious when Wt 3 is used with AMP\@.

\paragraph{APP is still recommended,} not only because the temporal uncertainty
is incorporated into the algorithm but also the results from APP are not as that
sensitive as AMP to the changes of sliding window and step sizes. In fact, for
APP the results from different window and step sizes are much more stable than
those from AMP (Fig.~\ref{fig-WinStpVsCPD}). This means we actually do not need
to worry about what sizes should be chosen for the sliding window and step when
we use APP method.

\subsection{Question: Why AMP Methods Sometimes Unexceptionally Produce Better
Similarities than APP Methods?}

Because of small number of paleopoles (not necessarily ``bad'') involved in
each sliding window, the produced mean poles by AMP should be relatively far
from its contemporary model-predicted pole. In other words, AMP intends to give
fairly small change in accuracy. This also could potentially bring more
distinguishable $d_s$ for AMP\@ if the corresponding A95 is not large enough.
For example, for Fig.~\ref{fig-na-dif}, there are only two special (of 84 APP
versus AMP comparisons) cases Pk (Wt) 4(3), 6(3) better than 5(3), 7(3)
respectively. Compared with the Pk (Wt) 4(3) APWP, although most of the mean
paleopoles are closer to the FHM predicted APWP and also the number of the
significant pole pairs is two less for the APP derived path (i.e. 5(3)), the A95s
are smaller and most importantly there are one more significant $d_a$
orientation-change pair and one more significant $d_l$ segment pair
(Table~\ref{tab-w3p4vs5}). If we observe carefully, it is because of the much
smaller 15 Ma A95 for 5(3). The similar phenomenon occurs to the case of 6(3)
versus 7(3), a relatively much smaller paleomagnetic A95 causes more distinguishable
$d_a$ and $d_l$ for the APP results, and they offset the improvement of spatial
similarity $d_s$ APP brings.

For Pk (Wt) 2(0) versus 3(0) for 20/10 Myr window/step North America, all their $d_a$ and
$d_l$ are indistinguishable. Compared with the results from AMP, although the
coeval pole GCDs are generally unchanged or decreased or even increased (but not
too much) for APP, this spatial improvement is not able to offset the negative
effects of also generally unchanged or decreased or even increased (but not too
much) paleomagnetic A95s, which potentially brings more statistically
distinguishable coeval poles (e.g.\ the 20 Ma and 110 Ma poles for Pk 3 and
Wt 0; Table~\ref{tab-w5p4vs5}). This further causes greater
distinguishable mean $d_s$ from the APP methods. The similar phenomenon occurs
to Fig.~\ref{fig-na-dif2010} Pk 2 versus 3 with Wt 2, 3 and 5, Pk
4 versus 5 with Wt 1, 3 and 5, and Pk 6 versus 7 with Wt 1, 3 and 5, and
so on.

In addition, compared with AMP, APP potentially could generate more mean poles,
because sometimes for some sliding window there is no paleopole involved at all
for AMP\@ while there are paleopoles involved for APP\@. For APP, the
mean poles at all ages should be composed of more paleopoles than it is for
AMP, which should generally decrease both coeval pole distance and paleomagnetic
A95. However, sometimes a rare case (e.g.\ the 0 Ma comparison shown in
Table~\ref{tab-501w0p22vs23}) happens. It is sometimes that an additional
very ``bad'' paleopole gets included by APP and this increases both coeval pole
distance and paleomagnetic A95 even though N increases. Such cases include
Fig.~\ref{fig-in-dif} Pk 22 versus 23 (actually exactly the same as Pk 26
versus 27) with all the six types of weightings.

So generally as we discussed in the last section APP decreases the distances
between paleomagnetic APWPs and the hotspot and ocean-floor spreading model
predicted APWP, and also the uncertainties of paleomagnetic APWPs. However, as
we described in this section, special cases like decreased A95 potentially
intends to make coeval poles differentiated if the coeval poles' distance is
not decreased effectively or even increased, or very ``bad'' paleopoles got
involved in some sliding windows, occurs. In summary, when the negative effect
from these types of rare cases is beyond the positive effect the generally
improved mean poles contribute, the composite difference score would increase.
However, this phenomenon seldom occur (Table~\ref{tab-bw}).

\begin{table}[!ht]
\centering
\caption{One example of the Type 1 rare cases where AMP gives better similarity
result than APP does from North America (10/5 Myr window/step). Only
statistically significant values are listed here.}\label{tab-w3p4vs5}
\resizebox{\textwidth}{!}{%
\begin{tabular}{@{}llllllllll@{}}
\toprule
\multicolumn{2}{c}{\multirow{2}{*}{\begin{tabular}[c]{@{}c@{}}FHM\\ predicted\end{tabular}}} & \multicolumn{4}{c}{Pk4(Wt3)} & \multicolumn{4}{c}{Pk5(Wt3)} \\ \cmidrule(l){3-10} 
\multicolumn{2}{c}{} & \multicolumn{2}{c}{ds} & \multicolumn{2}{c}{dl} & \multicolumn{2}{c}{ds} & \multicolumn{2}{c}{da} \\ \midrule
Age (Ma) & DM/DP () & Pmag A95 () & Dist () & Age (Ma) & Diff () & Pmag A95 () & Dist () & Age (Ma) & Diff () \\ \midrule
10 & 1.44607/0.793714 & 14.876819 & \textbf{9.937} & 105\textendash110 & 5.91855 & \multicolumn{2}{l}{} & \textit{\textbf{10\textendash15\textendash20}} & \textbf{126.59} \\ \cmidrule(lr){5-6}
15 & 1.2875/0.816514 & \multicolumn{2}{l}{\multirow{2}{*}{}} & \multicolumn{2}{l}{\multirow{11}{*}{}} & 2.0857 & \textbf{11.805} & \multicolumn{2}{l}{} \\ \cmidrule(l){9-10} 
25 & 2.48031/1.10915 & \multicolumn{2}{l}{} & \multicolumn{2}{l}{} & 6.3358 & \textbf{6.873} & \multicolumn{2}{c}{dl} \\ \cmidrule(l){9-10} 
55 & 3.58782/2.14032 & 4.6347 & \textbf{5.372} & \multicolumn{2}{l}{} & \multicolumn{2}{l}{} & Age (Ma) & Diff () \\ \cmidrule(l){9-10} 
60 & 4.85938/3.17602 & \multicolumn{2}{l}{\multirow{2}{*}{}} & \multicolumn{2}{l}{} & 6.5922 & 6.215 & \textit{\textbf{10\textendash15}} & \textbf{13.52} \\
65 & 3.68984/2.30014 & \multicolumn{2}{l}{} & \multicolumn{2}{l}{} & 8.6632 & \textbf{7.6} & \textit{\textbf{15\textendash20}} & \textbf{14.68} \\ \cmidrule(l){9-10} 
75 & 2.6435/1.54052 & 9.0812 & \textbf{8.836} & \multicolumn{2}{l}{} & \multicolumn{2}{l}{\multirow{4}{*}{}} & \multicolumn{2}{l}{\multirow{6}{*}{}} \\
100 & 2.8983/2.68346 & 8.892 & \textbf{8.455} & \multicolumn{2}{l}{} & \multicolumn{2}{l}{} & \multicolumn{2}{l}{} \\
105 & 2.32328/1.74639 & 5.3 & \textbf{5.03} & \multicolumn{2}{l}{} & \multicolumn{2}{l}{} & \multicolumn{2}{l}{} \\
110 & 4.13015/2.25964 & 3.8 & \textbf{9.8064} & \multicolumn{2}{l}{} & \multicolumn{2}{l}{} & \multicolumn{2}{l}{} \\
115 & 4.63512/2.58006 & 19.6676 & \textbf{9.3345} & \multicolumn{2}{l}{} & 8.5 & \textbf{11.704} & \multicolumn{2}{l}{} \\
120 & 7.34408/4.06043 & 3.515 & \textbf{17.35} & \multicolumn{2}{l}{} & 7.728 & \textbf{15.258} & \multicolumn{2}{l}{} \\ \cmidrule(r){1-4} \cmidrule(lr){7-8}
\end{tabular}%
}
\end{table}

\begin{table}[!ht]
\centering
\caption{One example of the Type 2 rare cases where AMP gives better similarity
result than APP does from North America (20/10 Myr window/step). Only
statistically significant values are listed here.}\label{tab-w5p4vs5}
\resizebox{\textwidth}{!}{%
\begin{tabular}{@{}llllll@{}}
\toprule
\multicolumn{1}{c}{\multirow{3}{*}{\begin{tabular}[c]{@{}c@{}}Age\\ (Ma)\end{tabular}}} & \multicolumn{1}{c}{\multirow{2}{*}{\begin{tabular}[c]{@{}c@{}}FHM\\ predicted\end{tabular}}} & \multicolumn{4}{c}{ds} \\ \cmidrule(l){3-6} 
\multicolumn{1}{c}{} & \multicolumn{1}{c}{} & \multicolumn{2}{c}{Pk2(Wt0)} & \multicolumn{2}{c}{Pk3(Wt0)} \\ \cmidrule(l){2-6} 
\multicolumn{1}{c}{} & DM/DP (\degree) & Pmag A95 (\degree) & Dist (\degree) & Pmag A95 (\degree) & Dist (\degree) \\ \midrule
0 & 0 & 3.97 & \textbf{5.714} & 3.97 & \textbf{5.714} \\
10 & 1.44607/0.793714 & 3.879 & \textbf{6.034} & 3.879 & \textbf{6.034} \\
20 & 1.58039/1.10047 & \multicolumn{2}{l}{} & 6.771 & \textbf{6.934} \\
50 & 3.57782/1.61328 & 3.8644 & \textbf{7.304} & 4.03 & \textbf{8.6} \\
60 & 4.85938/3.17602 & 5.716 & \textbf{8.457} & 5.55 & \textbf{7.367} \\
100 & 2.8983/2.68346 & 10.769 & \textbf{7.308} & 10.769 & \textbf{7.308} \\
110 & 4.13015/2.25964 & \multicolumn{2}{l}{} & 3.29 & \textbf{8.311} \\
120 & 7.34408/4.06043 & 3.38 & \textbf{16.41} & 3.083 & \textbf{16.728} \\ \bottomrule
\end{tabular}%
}
\end{table}

\begin{table}[!ht]
\centering
\caption{One example of the Type 3 rare cases where AMP gives better similarity
result than APP does from India (10/5 Myr window/step). Only statistically
significant values are listed here. Note that for the bold-number ages, there is
no mean poles at all for the ``Pk 22 (Wt 2)'' case.}\label{tab-501w0p22vs23}
\resizebox{\textwidth}{!}{%
\begin{tabular}{@{}llllllllll@{}}
\toprule
\multicolumn{2}{c}{\multirow{2}{*}{\begin{tabular}[c]{@{}c@{}}FHM\\ predicted\end{tabular}}} & \multicolumn{3}{c}{Pk22(Wt 2)} & \multicolumn{5}{c}{Pk23(Wt2)} \\ \cmidrule(l){3-10} 
\multicolumn{2}{c}{} & \multicolumn{3}{c}{ds} & \multicolumn{3}{c}{ds} & \multicolumn{2}{c}{dl} \\ \midrule
\multicolumn{1}{c}{Age (Ma)} & DM/DP (\degree) & Pmag DM/DP (\degree) & Dist (\degree) & N & Pmag DM/DP (\degree) & Dist (\degree) & N & Age (Ma) & Diff (\degree) \\ \midrule
0 & 0 & \textbf{6.28} & \textbf{12.72} & \textit{\textbf{2}} & \textbf{23.54} & \textbf{18.14} & \textit{\textbf{3}} & \textbf{80\textendash85} & 6.286 \\
10 & 1.12124/0.673225 & 5.4/3.1 & 29.9 & 1 & 5.4/3.1 & 29.9 & 1 & \textbf{110\textendash115} & 16.684 \\ \cmidrule(l){9-10} 
\textbf{15} & 1.1347/0.8127 & \multicolumn{3}{l}{\multirow{5}{*}{}} & 5.4/3.1 & 28.28 & 1 & \multicolumn{2}{l}{\multirow{13}{*}{}} \\
\textbf{60} & 4.79687/3.07133 & \multicolumn{3}{l}{} & 8.817 & 8.28 & 20 & \multicolumn{2}{l}{} \\
\textbf{70} & 4.26508/2.48783 & \multicolumn{3}{l}{} & 3.26 & 4.464 & 20 & \multicolumn{2}{l}{} \\
\textbf{75} & 2.6777/1.57975 & \multicolumn{3}{l}{} & 5 & 4.477 & 1 & \multicolumn{2}{l}{} \\
\textbf{80} & 4.20828/2.50294 & \multicolumn{3}{l}{} & 5 & 3.358 & 1 & \multicolumn{2}{l}{} \\
85 & 2.50744/1.24746 & 5 & 7.632 & 1 & 5 & 7.632 & 1 & \multicolumn{2}{l}{} \\
\textbf{90} & 3.88998/1.43423 & \multicolumn{3}{l}{\multirow{5}{*}{}} & 5 & 10.884 & 1 & \multicolumn{2}{l}{} \\
\textbf{95} & 2.23389/1.6247 & \multicolumn{3}{l}{} & 5 & 11.099 & 1 & \multicolumn{2}{l}{} \\
\textbf{100} & 2.8062/2.59819 & \multicolumn{3}{l}{} & 5 & 11.4155 & 1 & \multicolumn{2}{l}{} \\
\textbf{105} & 2.32328/1.74639 & \multicolumn{3}{l}{} & 5 & 14.908 & 1 & \multicolumn{2}{l}{} \\
\textbf{110} & 4.55519/2.49218 & \multicolumn{3}{l}{} & 6.8/4.9 & 13.962 & 1 & \multicolumn{2}{l}{} \\
115 & 4.63512/2.58006 & 10.73 & 10.508 & 5 & 10.73 & 10.508 & 5 & \multicolumn{2}{l}{} \\
120 & 6.02639/3.3319 & 10.73 & 10.508 & 5 & 10.73 & 10.508 & 5 & \multicolumn{2}{l}{} \\ \cmidrule(r){1-8}
\end{tabular}%
}
\end{table}

Other Type 1 (e.g. Table~\ref{tab-w3p4vs5}) cases: Fig.~\ref{fig-na-dif2010}
Pk (Wt) 2(1) versus 3(1). Fig.~\ref{fig-na-difm} 4(3) versus 5(3), 6(3) versus 7(3).

Other Type 2 (e.g. Table~\ref{tab-w5p4vs5}) cases: Fig.~\ref{fig-na-dif2010}
Pk (Wt) 2(0, 2, 3, 5) versus 3(0, 2, 3, 5), 4(1, 3, 5) versus 5(1, 3, 5), 6(1,
3, 5) versus 7(1, 3, 5). Fig.~\ref{fig-in-dif2010} 4(0\textendash5) versus 5(0\textendash5),
6(0\textendash5) versus 7(0\textendash5), 14(2, 3) versus 15(2, 3), 22(0, 2, 3) versus 23(0, 2, 3),
26(0, 2, 3) versus 27(0, 2, 3). Fig.~\ref{fig-au-dif2010} 4(2) versus 5(2).
Fig.~\ref{fig-au-difm} 8(5) versus 9(5). Fig.~\ref{fig-na-dif2010m} 2(0\textendash5)
versus 3(0\textendash5), 4(2) versus 5(2), 6(2) versus 7(2), 22(0\textendash5) versus
23(0\textendash5), 26(0\textendash5) versus 27(0\textendash5).
Fig.~\ref{fig-in-dif2010m} 4(0\textendash5) versus 5(0\textendash5), 6(0\textendash5)
versus 7(0\textendash5), 14(3) versus 15(3).

Combined Type 1 and 2 cases: Fig.~\ref{fig-na-dif2010} Pk (Wt)
22(0\textendash5) versus 23(0\textendash5), 26(0\textendash5) versus 27(0\textendash5).
Fig.~\ref{fig-in-dif2010} Pk (Wt) 22(1, 4, 5) versus 23(1, 4, 5), 26(1, 4, 5) versus
27(1, 4, 5). Fig.~\ref{fig-in-dif2010m} 22(0\textendash5) versus 23(0\textendash5),
26(0\textendash5) versus 27(0\textendash5).

Other Type 3 (e.g. Table~\ref{tab-501w0p22vs23}) cases: Fig.~\ref{fig-na-difm}
4(2\textendash5) versus 5(2\textendash5). Fig.~\ref{fig-in-difm} 22(0\textendash5)
versus 23(0\textendash5), 26(0\textendash5) versus 27(0\textendash5).


\subsection{Question: Why Weighting Is Not Affecting?}

Generally, weighting does not affect the similarities dramatically, because the
six results from the six weighting methods are mostly very close to each other
(Fig.~\ref{fig-dif}, Fig.~\ref{fig-dif2010}, Fig.~\ref{fig-difm},
Fig.~\ref{fig-dif2010m}). This closeness is also generally observed in the form
of clusters in Fig.~\ref{fig-w} and Fig.~\ref{fig-wu}. In addition, from the
general statistics of performance of the six weighting methods shown in
Table~\ref{tab-bw}, Wt 0 mostly performs the best or at least the
second best, which means no weighting works better in general.

When the above-mentioned question, about why APP generally produces better fits
than AMP, is tackled, we already find that both accuracy (how closely do the
pairs/segments/angles match) and precision (how large are the uncertainties on
the pairs/segments/angles, i.e.\ how difficult do they distinguish) can be the
factors that finally determine the difference score. Although another factor,
resolution (how many pairs/segments/angles are actually being compared) can
also influence the difference score (e.g. Table~\ref{tab-pk0vs1bs}), here this
factor is not relevant to comparisons between different weightings for a
certain picking method, because the numbers of picked paleopoles are the same
for the six weighting methods.

Therefore, at the very basic level, for example, a lower score is the result of
one of, or combination of:

1. A reduction in the difference scores of significantly different
pairs/segments, straightforwardly interpreted as a better fit (improved
accuracy).

2. Previously significantly different pairs/segments becoming insignificant.
This can occur either because the fit is better (improved accuracy), or because
of an increase in the uncertainty of the mean poles (they become less
distinguishable \textemdash{} decreased precision).

Fig.~\ref{fig-wp} and Fig.~\ref{fig-wpu} show that the proportions of results
from Wt 1\textendash5 to result from 0 are generally in the second
quadrant of the coordinate plane, where proportional change in difference score
is positive whereas proportional change in paleomagnetic FQ score is negative.
This means the five weightings (Wt 1\textendash5) do make effects, not obviously
in accuracy but mainly in improving precision, which could potentially expose
more pairs of distinguishable poles/segments/angle-changes. Or even accuracy is
improved in a small amount, improved precision intends to cancel out the effects
from improved accuracy.

There are a few special cases that one or two of the six weighting methods gives
a result with a dramatically worsened difference score, e.g.\ for weighting
method Wt 3 (Fig.~\ref{fig-dif}). From the labeled dots of Wt 3 in
Fig.~\ref{fig-w}, Fig.~\ref{fig-wu}, Fig.~\ref{fig-wp} and Fig.~\ref{fig-wpu},
we can get a general impression that Wt 3 is indeed improving precision
but not accuracy (at least not enough to offset the effects from improved
precision) so that this precision improvement is also the culprit that worsens
the final score. Highly improved precision without corresponding improved
accuracy would potentially bring more significant differences in shape metrics.
Wt 3 is exactly performing in this form.

\begin{figure}
\centering
\includegraphics[width=1\textwidth]{../../paper/tex/GeophysJInt/figures/w.pdf}
\caption[Paleomagnetic APWP's FQ score vs significant difference score
(i.e. $\mathcal{CPD}$)]{10/5 Myr bin/step paleomagnetic APWP's FQ score (different
from FQ, see the definitions of FQ and FQ score in Chapter~\ref{chap:Metho})
versus $\mathcal{CPD}$ significant score (reference path: FHM predicted) for the
28 different picking methods. See the $\mathcal{CPD}$ scores and Ppath-Rpath FQ
in Fig.~\ref{fig-dif}. Only those results dramatically worsened by Wt 3
are labeled.}\label{fig-w}
\end{figure}

\begin{figure}
\centering
\includegraphics[width=1\textwidth]{../../paper/tex/GeophysJInt/figures/wu.pdf}
\caption[Paleomagnetic APWP's FQ score vs raw difference score (i.e., no
statistical testing)]{10/5 Myr bin/step paleomagnetic APWP's FQ score (different
from FQ, see the definitions of FQ and FQ score in Chapter~\ref{chap:Metho})
versus raw difference score (reference path: FHM predicted) for the 28 different
picking methods. See FQ versus significant difference score in Fig.~\ref{fig-w}.
Only those results dramatically worsened by Wt 3 are labeled.}\label{fig-wu}
\end{figure}

\begin{figure}
\centering
\includegraphics[width=1\textwidth]{../../paper/tex/GeophysJInt/figures/wp.pdf}
\caption[Proportional changes of Wt 1\textendash5 to 0: Paleomagnetic
APWP's FQ score vs $\mathcal{CPD}$]{Proportion of Wt
1\textendash5 to 0: Proportional change in 10/5 Myr bin/step paleomagnetic
APWP's FQ score (different from FQ, see the definitions of FQ and FQ score in
Chapter~\ref{chap:Metho}) versus proportional change in $\mathcal{CPD}$ significant score
(reference path: FHM predicted) for the 28 different picking methods. See the
$\mathcal{CPD}$ scores and Ppath-Rpath FQ in Fig.~\ref{fig-dif}.
Only those results dramatically worsened by Wt 3 are
labeled.}\label{fig-wp}
\end{figure}

\begin{figure}
\centering
\includegraphics[width=1\textwidth]{../../paper/tex/GeophysJInt/figures/wpu.pdf}
\caption[Proportional changes of Wt 1\textendash5 to 0: Paleomagnetic
APWP's FQ score vs raw difference score]{Proportion of Wt
1\textendash5 to 0: Proportional change in 10/5 Myr bin/step paleomagnetic
APWP's FQ score (different from FQ, see the definitions of FQ and FQ score in
Chapter~\ref{chap:Metho}) versus proportional change in raw difference score (reference path:
FHM predicted) for the 28 different picking methods. See FQ versus significant
difference score in Fig.~\ref{fig-wp}. Only those results dramatically worsened
by Wt 3 are labeled.}\label{fig-wpu}
\end{figure}

In addition, for Wt 3, a small size of $\alpha$95 (high precision) could be
caused by those sampled directions not covering enough long period (thought to
be at least about $10^4$ years) to ``average out'' secular variation for giving
a paleopole. That is to say, the smallest $\alpha$95s could get the greatest
weights that they should not deserve.

Generally, weighting is affecting because different weighting functions give
obviously different results. However, interestingly weighting does not improve
fit and generally no weighting (Wt 0) is giving the best fit, although in most
cases weighting does improve precision. Wt 2 or 4 is not recommended, because
they never have generated the best similarities (Table~\ref{tab-bw}), compared
with other weighting methods. There is no general pattern about which weighting
(of Wt 1\textendash5) is better or worse. So weighting, for making a
paleomagnetic APWP, is not absolutely necessary. However, there are some
patterns about which weighting is better or worse for some specific continent or
some specific picking methods. For example, Wt 3 works generally fine with
Australian data (Table~\ref{tab-bw}). However, Wt 3 is not recommended for North
America and India.

\subsection{Question: Why Best and Worst Methods Sometimes Are Not Consistent?}

For all the three continents, North America, India and Australia, Pk
19 (APP with local rotation or secondary print excluded) and 21 (APP
with local rotation and secondary print corrected) are consistently the best or
at least the relatively better (for example, in Fig.~\ref{fig-in-dif2010},
Pk 21 results are not colored in green, but ranging from
0.0483\textendash0.0561 that is still much less than the mean, 0.074 and also
the median, 0.072); whereas Pk 16 (AMP with publications before
1983) and 18 (AMP with local rotation or secondary print excluded) are
consistently the worst or at least the relatively worse.

Nevertheless, for each single continent, they have their own consistently best
and worst picking methods that are not the best or worst for other continents.
For example, Pk 15 (APP with publications after 1983) works well with North
America. We know that 70 North American paleopoles (about 53\%) have contributed
to Pk 15 for 120\textendash0 Ma. However, only 28 Indian paleopoles (about 38\%)
and 29 Australian paleopoles (about 30\%) have contributed to Pk 15. In
contrast, the Pk 17 (APP with publications before 1983) works well with only
Australia. This also tells that number of paleopoles is a key factor that
affects the fit score. The fact that Pk 22 (AMP with SS05 criteria) doesn't work
well with only Australia also reflects the importance of this factor.

In summary, the reason why some best and worst methods are not consistent is
generally that different continents have their unique data sets and get their
own paleomagnetic studies in varying degrees.

\subsection{Question: Are There Particular Parts of the Path That Are More
Variable? Do Different Methods Affect Different Parts of the Path Differently?}

The results may highlight the trade-off between more data diluting the effect of
outliers, and fewer but `better' data being more easily affected by a bad point
that gets through the filters (Fig.~\ref{fig-nads}, Fig.~\ref{fig-nadl} and
Fig.~\ref{fig-nada}).



\section{Final Conclusions}

From the perspective of the general similarities between those paleomagnetic
APWPs and the hot spot model and ocean-floor spreading model predicted APWPs,
GAD hypothesis is proved valid for at least the last 120 Myr.

\subsection{Universal Rules of Ways of Processing Paleomagnetic Data}
%
\begin{description}
  \item Although effects of filters (all the picking methods where number of
	paleopoles shrinks compared to Pk 0 and Pk 1; see Fig.~\ref{fig-dif}) have a
	marginal change in reducing N (precision potentially going down), some
	filters do improve the similarity score, for example, Pk 25 (APP without
	superseded data) is always giving better scores than Pk 1 for all the
	three continents. However, Pk 0 and Pk 1 (no filtering and corrections
	applied) still generally perform well.
  \item APP (adding data to a time window with overlapping age selection
	criterion) is better than AMP for making paleomagnetic APWPs, for both kinds
	of situations when there are lots of data (APP even better, e.g.\ for North
	America and Australia) and not much data (APP still a better option, e.g.\
	for India; Table~\ref{tab-bw}). APP with most filters/corrections (Pk 3, 5,
	7, 9, 11, \ldots, 27) are generally giving worse scores than APP without any
	filter/correction (Pk 1).
  \item In most cases the APP methods produce better similarities than the AMP
	methods (Table~\ref{tab-bw}).
  \item Actually weighting is not improving the fit but improving precision
	generally. For quite many of the methods, no weighting is the best performer
	(Table~\ref{tab-bw}). For example, score is likely worse for the combined
	methods of weight Pk 3 and AMP\@.
  \item APP itself helps incorporate temporal uncertainty into the algorithm.
	With the bootstraps test helping incorporate spatial uncertainty into the
	algorithm together, both spatial and temporal uncertainties are successfully
	considered in APP methods.
\end{description}

\subsection{Conditional Rules of Ways of Processing Paleomagnetic Data}
%
\begin{description}
  \item Pk 16 (AMP with data from old studies) and 18 (AMP
	without data affected by rotations and secondary print) are not recommended
	for generating a paleomagnetic APWP\@.
\end{description}

\subsection{Summary}

According to the results we have from the three continents, North America, India
and Australia, using the similarity measuring tool developed in Chapter~\ref{chap:Metho}, it is
recommended that APP should be used to select the input paleopoles. According to
all the paleomagnetic data we have from the three continents, the results from
any size for sliding window and step are interestingly and extremely close to
each other when the APP method is used, compared with the results from the AMP
method (Table~\ref{tab-pk0vs1bs} and Fig.~\ref{fig-WinStpVsCPD}). So any size
for binning and stepping is ok when APP is used. Then filtering is actually not
necessary. However, some filtering methods (e.g.\ Pk 5, 7 (igneous-derived), 11,
13 (nonredbeds or corrected redbeds derived), 19, 21
(non-local-rotation/reprinted or corrected-rotation derived) and 25
(non-superseded data derived)) are fine too and will not give worst or worse
results than the other filtering methods (i.e.\ the left Picking methods). With
APP used, weighting is actually not necessary either. If a weighting has to be
used, Wt 1 (related to the number of paleomagnetic sampling sites and
observations) is generally better than the other four given weighting methods
(Fig.~\ref{fig-flow}).

If AMP has to be used, relatively wide sliding window and step are needed.
According to our tests, more than 20/10 Myr is recommended. In addition, AMP
works relatively better with igneous-derived data (i.e.\ Pk 4 and 6),
which indicates that if we have fewer data, these data need to be better in
quality (Fig.~\ref{fig-flow}).

\begin{figure}
  \centering
  \includegraphics[width=1.01\textwidth]{../../paper/tex/GeophysJInt/figures/flowchart.pdf}
  \caption[Flowchart of making a reliable paleomagnetic APWP]{Flowchart for
recommended procedure of processing paleomagnetic data.}\label{fig-flow}
\end{figure}

%\chapter{How Much Data Needed to Make a Reliable APWP}\label{chap:DatNeed}
\textit{This chapter mainly describes how the mean poles with their raw VGPs
at random reduced densities can make a reliable APWP\@. Further we will see how
much data (raw VGPs) are needed on earth to make a reliable APWP, and how the
``bad'' poles influence the final result when we have less data. Are we be able
to make a final determination of best number of VGPs in each sliding window in
average for moving-averaging out an APWP? (No, different situations for
different continents.)}
\vfill
\minitoc\newpage

In the past, especially in deep time, the density and quality of paleomagnetic
data are lower, compared with younger geological times. Reducing the data
density can help see if our methodology is still able to reliably give
reasonable results from data aged in deep times.

\section{Reference Path}

The fixed hotspot model and related plate circuit predicted APWPs are used as
references. In fact, as mentioned in the last chapter, choosing FHM or MHM does
not make much difference at all.

\section{Extration Fraction}

The raw VGPs are extracted before being moving-averaged (i.e.\ picking and
weighting; see the last chapter for more details).

80\%, 60\%, 40\% and 20\% mean 20, 40, 60 and 80 percent of raw VGPs are
removed.

\begin{figure}
    \centering
        \includegraphics[width=0.88\textwidth]{fig/Fay18_10_5.pdf}
    \captionsetup{width=.95\textwidth}
    \caption{Random VGP samplings (30 times) for the best and worst results for
	the 10 Myr window and 5 Myr step paleomagnetic APWPs vs FHM \& plate circuit
	predicted APWP\@. The lower and upper bound lines connect the 1st and 3th
	quantiles ($Q_1$ and $Q_3$) of the 30 samples. The bold line connects their
	means. The numbers in small parentheses are actual quantity of VGPs after
	filtered by the corresponding picking methods for the case with no data
	removal. The $Q_1$\textendash$Q_3$ interquartile range from best method is
	also shown (shadowed) in the plot of the worst method for clarity. Green
	dots are the lowest value for each method; dark green dots are the lowest
	for the 20\% case.}\label{Fig:Fay18_10_5bw}
\end{figure}

\begin{figure}
    \centering
        \includegraphics[width=0.88\textwidth]{fig/Fay18_10_5to101bw.pdf}
    \captionsetup{width=.95\textwidth}
    \caption{Comparisons of results from the best and worst methods for North
	America (101), also applied on the other two continents (501 and 801). The
	$Q_1$\textendash$Q_3$ interquartile range from Picking No. 11 is also shown
	(shadowed) in the plot of Picking No. 16 for clarity.}\label{Fig:Fay18_10_5to101bw}
\end{figure}

\begin{figure}
    \centering
        \includegraphics[width=0.88\textwidth]{fig/Fay18_10_5to501bw.pdf}
    \captionsetup{width=.95\textwidth}
	\caption{Comparisons of results from the best and worst methods for India
	(501), also applied on the other two continents (101 and 801). The
	$Q_1$\textendash$Q_3$ interquartile range from Picking No. 19 is also shown
	(shadowed) in the plot of Picking No. 8 for clarity.}\label{Fig:Fay18_10_5to501bw}
\end{figure}

\begin{figure}
    \centering
        \includegraphics[width=.88\textwidth]{fig/Fay18_10_5to801bw.pdf}
    \captionsetup{width=.95\textwidth}
    \caption{Comparisons of results from the best and worst methods for
	Australia (801), also applied on the other two continents (101 and 501). The
	$Q_1$\textendash$Q_3$ interquartile range from Picking No. 17 is also shown
	(shadowed) in the plot of Picking No. 22 for clarity.}\label{Fig:Fay18_10_5to801bw}
\end{figure}

We can see that the best picking and weighting methods are statistically always
better than the worst ones even if only 20 percent of thei 120\textendash0 Ma
paleopoles are used to compose the APWPs (Fig.~\ref{Fig:Fay18_10_5bw}) for the
three continents, North America, India and Australia.

For the worst methods applied onto Indian and Australian data, the equal-weight
$\mathcal{CPD}$ surprisingly decreases when the percentage of extracted data
decreases (Fig.~\ref{Fig:Fay18_10_5bw}). This is because after the data density
is reduced the left data are not always enough to cover all the time range of
120\textendash0 Ma but only part, or even though the 120 and 0 Ma mean poles
(two ends) exist, the number of intermediate mean poles between 120 Ma and 0 Ma
is much less than the APWP from data without reducing density.

\subsection{Number of Samples}
Here because the thousands of times of testing for each percentage of data
removal and each picking and weighting method is rather expensive, 30 samples
(a rule of thumb; e.g.~\cite{H19} says ``greater than 25 or 30'') are obtained.
In fact, the 25th percentiles ($Q_1$), 75th percentiles ($Q_3$) and the means of
30, 60, 100, 200 and 1000 samples are not quite different
(Fig.~\ref{Fig:Fay18_10_5_0_0}), although 200 seems a better and relatively
cheaper option.

\begin{figure}
    \centering
        \includegraphics[width=0.88\textwidth]{fig/Fay18_10_5_0_0.pdf}
    \captionsetup{width=.95\textwidth}
    \caption{Testing differences of results from different numbers of samples.
	See Fig.~\ref{Fig:Fay18_10_5bw} for more details.}\label{Fig:Fay18_10_5_0_0}
\end{figure}

Describe the function fill-df-a95nan in ma.py:
Situation 1: Note that because of reducing of pole quantity in sliding windows,
in some window there could be only one VGP which does not have given ED95, DM
and DP, then we use $140/\sqrt{kN}$~\cite{T91,T19} to calculate a ED95 for this
VGP as its A95.

\subsection{Extreme Value Study}

It is easy for us to think that less paleomagnetic data means poorer similarity
between paleomagnetic APWPs and the reference path. However, it is noticeable
that even though the data density is tremendously reduced (e.g.\ by 80\%), it is
still possible to have a better similarity for paleomagnetic APWPs and the
reference, even better than for the paleomagnetic APWP with all original
datasets (e.g.\ the green dots in Fig.~\ref{Fig:Fay18_10_5bw}). For example,
for the case (a) (Fig.~\ref{Fig:Fay18_10_5bw}), even though 60\% of the
paleopoles are removed, we still can get a better similarity using the
paleomagnetic APWP composed of the left 40\% of the paleopoles than the
original. Although this 40\% data generated paleomagnetic APWP owns the same
number of mean poles as the 100\% data generated paleomagnetic APWP, the
number of paleopoles for each mean pole is actually much less. The main reason
why this 40\% data generated paleomagnetic APWP is more similar to the
reference path is that this APWP's spacial errors are much larger than those of
the 100\% data generated paleomagnetic APWP
(Fig.~\ref{Fig:Fay18_10_5b101l40p_vs_100p}). Even only 20\% of the paleopoles
could also give a better similarity (the dark green dot in
Fig.~\ref{Fig:Fay18_10_5bw} (a)). The reason why this 20\% data generated
paleomagnetic APWP is more similar to the reference path is the same as for the
40\% data generated path. It's the same for the lowest difference given by the
60\% Australian paleomagnetic data (the green dot in Fig.~\ref{Fig:Fay18_10_5bw}
(e)).

\begin{figure*}[tbp]
  \captionsetup[subfigure]{labelformat=empty,aboveskip=-6pt,belowskip=-6pt}
  \centering
  \begin{subfigure}[htbp]{.49\textwidth}
    \captionsetup{skip=0pt}  % local setting for this subfigure
    \centering
    \includegraphics[width=1.01\linewidth]{/home/g/Desktop/git/paper/tex/GeophysJInt/figures/ay18_101comb_10_5_11_0.pdf}
	\caption{100\% data; Fig.~\ref{Fig:Fay18_10_5bw}
	(a)}\label{Fig:Fay18_10_5b101}
  \end{subfigure}
  \begin{subfigure}[htbp]{.49\textwidth}
    \captionsetup{skip=0pt}
    \centering
    \includegraphics[width=1.01\linewidth]{fig/ay18_101comb_10_5_11_0_40perc_Best.pdf}
    \caption{The lowest for 40\% data;
	Fig.~\ref{Fig:Fay18_10_5bw} (a)}\label{Fig:Fay18_10_5b101l40p}
  \end{subfigure}
  \caption[Less data, better similarity?]{Comparing the 100\% North American
  120\textendash0 Ma paleomagnetic data derived result with the best of the only
  40\% data (giving even better similarity) derived results (the green dot in
  Fig.~\ref{Fig:Fay18_10_5bw} (a)).}\phantomsection\label{Fig:Fay18_10_5b101l40p_vs_100p}
\end{figure*}

For the case (b) (Fig.~\ref{Fig:Fay18_10_5bw}), the reason why the only 20\%
data could give better result than the 100\% data does is that not only the
20\% data gives less mean poles, but also the 20\% data gives larger spacial
errors (Fig.~\ref{Fig:Fay18_10_5w101l20p_vs_100p}).

\begin{figure*}[tbp]
  \captionsetup[subfigure]{labelformat=empty,aboveskip=-6pt,belowskip=-6pt}
  \centering
  \begin{subfigure}[htbp]{.49\textwidth}
    \captionsetup{skip=0pt}
    \centering
    \includegraphics[width=1.01\linewidth]{/home/g/Desktop/git/paper/tex/GeophysJInt/figures/ay18_101comb_10_5_16_3.pdf}
	\caption{100\% data; Fig.~\ref{Fig:Fay18_10_5bw}
	(b)}\label{Fig:Fay18_10_5w101}
  \end{subfigure}
  \begin{subfigure}[htbp]{.49\textwidth}
    \captionsetup{skip=0pt}
    \centering
    \includegraphics[width=1.01\linewidth]{fig/ay18_101comb_10_5_16_3_20perc_Best.pdf}  %Fay18_101comb_10_5_16_3/_2/026
    \caption{The lowest for 20\% data;
	Fig.~\ref{Fig:Fay18_10_5bw} (b)}\label{Fig:Fay18_10_5w101l20p}
  \end{subfigure}
  \caption[Less data, better similarity?]{Comparing the 100\% North American
  120\textendash0 Ma paleomagnetic data derived result with the best of the only
  20\% data (giving even better similarity) derived results (the green dot in
  Fig.~\ref{Fig:Fay18_10_5bw} (b)).}\phantomsection\label{Fig:Fay18_10_5w101l20p_vs_100p}
\end{figure*}

For the case (c) (Fig.~\ref{Fig:Fay18_10_5bw}), the reason why the 80\% data is
able to give better result than the 100\% data does is that the 10 Ma mean pole
of the 80\% data derived path is a bit closer to reference, because both the 10
Ma pole pair in Fig.~\ref{Fig:Fay18_10_5b501l80p_vs_100p} are distinguishable.
Although the 80\% data derived paleomagnetic APWP
(Fig.~\ref{Fig:Fay18_10_5b501l80p}) generally owns relatively larger spacial
errors, the related pole pairs are distinguishable for both path pairs in
Fig.~\ref{Fig:Fay18_10_5b501l80p_vs_100p}.

\begin{figure*}[tbp]
  \captionsetup[subfigure]{labelformat=empty,aboveskip=-6pt,belowskip=-6pt}
  \centering
  \begin{subfigure}[htbp]{.49\textwidth}
    \captionsetup{skip=0pt}
    \centering
    \includegraphics[width=1.01\linewidth]{/home/g/Desktop/git/paper/tex/GeophysJInt/figures/ay18_501comb_10_5_19_0.pdf}
	\caption{100\% data; Fig.~\ref{Fig:Fay18_10_5bw}
	(c)}\label{Fig:Fay18_10_5b501}
  \end{subfigure}
  \begin{subfigure}[htbp]{.49\textwidth}
    \captionsetup{skip=0pt}
    \centering
    \includegraphics[width=1.01\linewidth]{fig/ay18_501comb_10_5_19_0_80perc_Best.pdf}  %Fay18_501comb_10_5_19_0/_8/005
    \caption{The lowest for 80\% data;
	Fig.~\ref{Fig:Fay18_10_5bw} (c)}\label{Fig:Fay18_10_5b501l80p}
  \end{subfigure}
  \caption[Less data, better similarity?]{Comparing the 100\% Indian
  120\textendash0 Ma paleomagnetic data derived result with the best of the only
  80\% data (giving even better similarity) derived results (the green dot in
  Fig.~\ref{Fig:Fay18_10_5bw} (c)).}\phantomsection\label{Fig:Fay18_10_5b501l80p_vs_100p}
\end{figure*}

Still for the case (c) (Fig.~\ref{Fig:Fay18_10_5bw}), of the 30 samples for the
20\% data, none is able to give better result compared with the 100\% data, but
the lowest difference value we can get from these 30 samples (20\% in
Fig.~\ref{Fig:Fay18_10_5bw} (c)) indicates that 20\% data is still able to give
good (not dramatically different from that 100\% data gives) similarity
(Fig.~\ref{Fig:Fay18_10_5b501l20p}).

\begin{figure*}[tbp]
  \captionsetup[subfigure]{labelformat=empty,aboveskip=-6pt,belowskip=-6pt}
  \centering
  \begin{subfigure}[htbp]{.49\textwidth}
    \captionsetup{skip=0pt}
    \centering
    \includegraphics[width=1.01\linewidth]{/home/g/Desktop/git/paper/tex/GeophysJInt/figures/ay18_501comb_10_5_19_0.pdf}
	\caption{100\% data; Fig.~\ref{Fig:Fay18_10_5bw}
	(c)}\label{Fig:Fay18_10_5b501_}
  \end{subfigure}
  \begin{subfigure}[htbp]{.49\textwidth}
    \captionsetup{skip=0pt}
    \centering
    \includegraphics[width=1.01\linewidth]{fig/ay18_501comb_10_5_19_0_20perc_Best.pdf}  %Fay18_501comb_10_5_19_0/_2/028
    \caption{The lowest for 20\% data;
	Fig.~\ref{Fig:Fay18_10_5bw} (c)}\label{Fig:Fay18_10_5b501l20p}
  \end{subfigure}
  \caption[Less data, better similarity?]{Comparing the 100\% Indian
  120\textendash0 Ma paleomagnetic data derived result with the best of the only
  20\% data (giving even better similarity) derived results (the dark green dot
  in Fig.~\ref{Fig:Fay18_10_5bw}
  (c)).}\phantomsection\label{Fig:Fay18_10_5b501l20p_vs_100p}
\end{figure*}

For the case (d) (Fig.~\ref{Fig:Fay18_10_5bw}), the reason why the only 40\%
data could give better result than the 100\% data does is that for the green
dot (Fig.~\ref{Fig:Fay18_10_5bw} (d)) not only the 40\% data gives less mean
poles (but two ends 120 Ma and 0 Ma still exist), but also the 40\% data does
not contain some ``bad'' paleopoles that are far away from the reference path
(Fig.~\ref{Fig:Fay18_10_5w501l40p}). It's the same for the lowest difference
given by the 20\% data samples (the dark green dot in
Fig.~\ref{Fig:Fay18_10_5bw} (d)).

\begin{figure*}[tbp]
  \captionsetup[subfigure]{labelformat=empty,aboveskip=-6pt,belowskip=-6pt}
  \centering
  \begin{subfigure}[htbp]{.49\textwidth}
    \captionsetup{skip=0pt}
    \centering
    \includegraphics[width=1.01\linewidth]{/home/g/Desktop/git/paper/tex/GeophysJInt/figures/ay18_501comb_10_5_8_3.pdf}
	\caption{100\% data; Fig.~\ref{Fig:Fay18_10_5bw}
	(d)}\label{Fig:Fay18_10_5w501}
  \end{subfigure}
  \begin{subfigure}[htbp]{.49\textwidth}
    \captionsetup{skip=0pt}
    \centering
    \includegraphics[width=1.01\linewidth]{fig/ay18_501comb_10_5_8_3_40perc_Best.pdf}  %Fay18_501comb_10_5_19_0/_2/028
    \caption{The lowest for 40\% data;
	Fig.~\ref{Fig:Fay18_10_5bw} (d)}\label{Fig:Fay18_10_5w501l40p}
  \end{subfigure}
  \caption[Less data, better similarity?]{Comparing the 100\% Indian
  120\textendash0 Ma paleomagnetic data derived result with the best of the only
  40\% data (giving even better similarity) derived results (the green dot in
  Fig.~\ref{Fig:Fay18_10_5bw} (d)).}\phantomsection\label{Fig:Fay18_10_5w501l40p_vs_100p}
\end{figure*}

For the case (e) (Fig.~\ref{Fig:Fay18_10_5bw}), although the 20\% data derived
paleomagnetic paths are not closer to the reference path than the 100\% data
derived one, the closest one (the dark green dot in Fig.~\ref{Fig:Fay18_10_5bw}
(e)) still performs well (Fig.~\ref{Fig:Fay18_10_5b801l20p}). This is mainly
because the number of mean poles becomes less when there are only 20\% of the
paleopoles, especially for 120 Ma and 115 Ma, where the mean poles are far from
the reference path for the 100\% data (Fig.~\ref{Fig:Fay18_10_5b801}), mean
poles are missing for the dark green dot case (Fig.~\ref{Fig:Fay18_10_5bw} (e)).
The same situation happens to the green dot case (Fig.~\ref{Fig:Fay18_10_5bw}
(f)).

\begin{figure*}[tbp]
  \captionsetup[subfigure]{labelformat=empty,aboveskip=-6pt,belowskip=-6pt}
  \centering
  \begin{subfigure}[htbp]{.49\textwidth}
    \captionsetup{skip=0pt}
    \centering
    \includegraphics[width=1.01\linewidth]{/home/g/Desktop/git/paper/tex/GeophysJInt/figures/ay18_801comb_10_5_17_0.pdf}
	\caption{100\% data; Fig.~\ref{Fig:Fay18_10_5bw}
	(e)}\label{Fig:Fay18_10_5b801}
  \end{subfigure}
  \begin{subfigure}[htbp]{.49\textwidth}
    \captionsetup{skip=0pt}
    \centering
    \includegraphics[width=1.01\linewidth]{fig/ay18_801comb_10_5_17_0_20perc_Best.pdf} %Fay18_801comb_10_5_17_0/_2/007/ay18_801comb
    \caption{The lowest for 20\% data;
	Fig.~\ref{Fig:Fay18_10_5bw} (e)}\label{Fig:Fay18_10_5b801l20p}
  \end{subfigure}
  \caption[Less data, better similarity?]{Comparing the 100\% Australian
  120\textendash0 Ma paleomagnetic data derived result with the best of the only
  20\% data derived results (the dark green dot in Fig.~\ref{Fig:Fay18_10_5bw}
  (e)).}\phantomsection\label{Fig:Fay18_10_5b801l20p_vs_100p}
\end{figure*}

For the case (f) (Fig.~\ref{Fig:Fay18_10_5bw}), most of the 20\% data derived
paleomagnetic paths are closer to the reference path than the 100\% data derived
one, especially for the dark green dot case in Fig.~\ref{Fig:Fay18_10_5bw} (f)
(Fig.~\ref{Fig:Fay18_10_5w801l20p}). This is mainly because the number of mean
poles becomes much less when there are only 20\% of the paleopoles, especially
two end mean poles missing.

\begin{figure*}[tbp]
  \captionsetup[subfigure]{labelformat=empty,aboveskip=-6pt,belowskip=-6pt}
  \centering
  \begin{subfigure}[htbp]{.49\textwidth}
    \captionsetup{skip=0pt}
    \centering
    \includegraphics[width=1.01\linewidth]{/home/g/Desktop/git/paper/tex/GeophysJInt/figures/ay18_801comb_10_5_22_3.pdf}
	\caption{100\% data; Fig.~\ref{Fig:Fay18_10_5bw}
	(f)}\label{Fig:Fay18_10_5w801}
  \end{subfigure}
  \begin{subfigure}[htbp]{.49\textwidth}
    \captionsetup{skip=0pt}
    \centering
    \includegraphics[width=1.01\linewidth]{fig/ay18_801comb_10_5_22_3_20perc_Best.pdf} %Fay18_801comb_10_5_22_3/_2/018/ay18_801comb
    \caption{The lowest for 20\% data;
	Fig.~\ref{Fig:Fay18_10_5bw} (f)}\label{Fig:Fay18_10_5w801l20p}
  \end{subfigure}
  \caption[Less data, better similarity?]{Comparing the 100\% Australian
  120\textendash0 Ma paleomagnetic data derived result with the best of the only
  20\% data derived results (the dark green dot in Fig.~\ref{Fig:Fay18_10_5bw}
  (f)).}\phantomsection\label{Fig:Fay18_10_5w801l20p_vs_100p}
\end{figure*}

\section{Are the rules we obtained in the last chapter are still true for less
data?}

First, that if APP is still better, and weighting is still not affecting for
less dense paleomagnetic data, is needed to be tested.

\begin{figure}
    \centering
        \includegraphics[width=1\textwidth]{fig/Fay18_10_5_101appamp.pdf}
    \captionsetup{width=1\textwidth}
    \caption{Comparisons of results from Picking No. 1 (APP) and Picking No. 0
	(AMP) with all the listed weighting methods for the three continents. The
	$Q_1$\textendash$Q_3$ interquartile range from Picking No. 1 is also shown
	(shadowed) in the plot of Picking No. 0 for clarity.}\label{Fig:Fay18_10_5_101appamp}
\end{figure}

From the perspective of checking if the $Q_1$\textendash$Q_3$ intervals overlap,
APP is indeed still better than AMP and weighting is indeed still not affecting
for more than 15 VGPs making a 10/5 Myr bin/step APWP (ideally composed of 25
mean poles for 120\textendash0 Ma). For the 20\% Indian data case, which
contains not more than 15 VGPs making a 120\textendash0 Ma APWP, there is
overlapping between APP's Mean\textendash$Q_3$ interval and AMP's
$Q_1$\textendash{}Mean interval for weighted cases (i.e.\ for Weighting No.
1\textendash5). Even so, APP's means are still lower than AMP's for this
no-more-than-15-VGPs case (Fig.~\ref{Fig:Fay18_10_5_101appamp}). So here is the
question: is 15 the threshold?

\chapter{Conclusions}\label{chap:Concl}
\textit{2\textendash3 pages of summary of results, significance and future
directions/work.}
\vfill
\minitoc\newpage

\section{Summary of Results and Significance}

\section{Recommendations for Future Research}

Although this thesis proposed methodologies that starts right from the
beginning, compiling data, processing data \ldots step by step to analyze results,
and finally found and validated an approach to make a reliable paleomagnetic
APWP, there are still many further studies that need to be done.
%
\begin{enumerate}
  \item As mentioned in the objectives of Chapter~\ref{chap:Intro}, we have
  investigated the limits of paleomagnetic data on reconstructing individual
  plate motions like of North America, India and Australia in
  Chapter~\ref{chap:Reliab}. We still need to further investigate the limits of
  paleomagnetic data on reconstructing supercontinents, and even global tectonic
  parameters like average rate of plate motion, number of plates and so on.
  \item 
\end{enumerate}


%next line adds the Bibliography to the contents page
\addcontentsline{toc}{chapter}{Bibliography}
%uncomment next line to change bibliography name to references
%\renewcommand{\bibname}{References}
\bibliography{../../doc_template_config/tex/refs}        %use a bibtex bibliography file refs.bib
\bibliographystyle{plainnat}  %use the plain bibliography style if we don't use natbib

%now enable appendix numbering format and include any appendices
\begin{appendices}
\appendix
\chapter{Methods for Manipulating Data on A Globe in Chapter 2}\label{appen4chp2}

\section{Test if a coeval pole pair is distinguishable with the Bootstrap}
Whether a pair of coeval mean poles are statistically distinguishable from each
other is investigated, as it can be determined by checking if the confidence
intervals of their bootstrap means (based on two poles' uncertainty attributes)
overlap~\citep{T91}. The 95\% confidence bounds of the Cartesian coordinates of
the bootstrap means are determined and compared. If the poles are
distinguishable, the confidence bounds along at least one coordinate axis do not
overlap. Otherwise, if the confidence bounds along all the three coordinate axes
overlap, the poles are indistinguishable~\citep{T91}. The actual test used is
dependent on the number of paleopoles ($\mathbf{N}$) used to calculate the mean
pole in the APWP$\colon$

\begin{itemize}
\item when $\mathbf{N}>25$ a simple bootstrap~\citep{T91} generates a
pseudo-mean pole from $\mathbf{N}$ directions drawn randomly from the original
set of paleopoles. 1000 such simple Bootstraps are implemented here.
\item when $1<\mathbf{N}\leq25$ a parametric bootstrap~\citep{T91} generates a
pseudo-mean from $\mathbf{N}$ directions drawn from a Fisher distribution with
the same $\mathbf{K}$ and $\mathbf{N}$ as the mean pole. 1000 such parametric
Bootstraps are implemented here.
\item when $\mathbf{N}=1$, a pseudo-mean is drawn from a bivariate normal
distribution, defined by the properties of the associated A95 uncertainty
circle or dm/dp ellipse (see the following Section~\ref{sec:biv}). Here 1000
samples are drawn from such a normal distribution.
\item if $\mathbf{N}$ is not given, because for example sometimes the pole could
be an interpolated result, a negligible A95 like 0.1\degree\ or 0\degree\ is
assigned and the same sampling way as used for the $\mathbf{N}=1$ case is
applied here. This is for the situation when only one of the coeval poles is
interpolated, and one would like to keep this pair of poles. Note that if the
coeval poles are both interpolated, we suggest directly removing this pair of
poles.
\end{itemize}

\paragraph{Special cases} Sometimes, like in the cases in Fig. 3 and Fig. 10, we
have complete access to the parameters of the mean poles, e.g.\ $\mathbf{N}$ and
precision parameter $\mathbf{K}$, and also the paleopoles. However this is not
necessarily true. If, for instance, we only have access to the path with only
its mean poles and spatial uncertainties, we can keep the way of doing bootstrap
sampling consistent for all the mean poles, and just draw bootstrapped means
from a bivariate normal distribution based on each spatial uncertainty's
geometry. This is implemented through arbitrarily setting $\mathbf{N}$=1. The
consistency of bootstrap sampling makes it independent of the state of knowledge
of the underlying dataset and even the underlying method used to calculate the
uncertainty. This means the method can be more generalisable beyond APWPs,
because the metrics and the significance testing procedure are more broadly
applicable to comparison of other trajectories with associated spatial
uncertainties, such as hurricane tracks and bird migration routes.

The final results for each coeval pole pair of all the seven APWP pairs (Fig. 3),
are given in the sub-folder ``0.result\_tables'', which is contained in the main
``data'' folder. The results for length and angular differences are listed
starting from the rows for the second and third poles respectively, simply
because one pole can not compose a APWP segment and at least three poles could
constitute an APWP orientation change.

\section{Bivariate sampling}\label{sec:biv}
For some specific poles of the APWP, e.g., only one paleomagnetic pole makes up
that ``mean'', i.e., $N=1$, or even there is no paleomagnetic pole in that
specific bin (i.e., $N=0$) but an interpolated pole that might be given by
authors at that specific age, the bivariate normal distribution is used to
generate random samples based on its uncertainty ellipse's semi axes and the
major axis' azimuth, then we use the cumulative distributions of Cartesian
coordinates of those random samples to see if the confidence intervals overlap.

However, here the scenario is not a two dimensional (2D) domain, but rather a
spherical surface. Directly simulating random points for an ellipse on a
sphere is a complicated problem~\citep{K82}. An analogue approach is proposed
here as follows. First, random points of a 2D bivariate normal distribution are
generated with NumPy's random sampling routine
``multivariate\_normal''~\citep{W11}. The lengths of the uncertainty ellipse's
semi-major and semi-minor axes are used as about 1.96 standard deviations of the
bivariate normal distribution. The center of the ellipse is located at the
intersection of the equator (0\degree\ latitude) and the prime meridian
(0\degree\ longitude) with its major axis lying equator-ward (blue point cloud in
Fig.~\ref{fig:ellip_sim}). Then according to the actual pole coordinates (red
star in Fig.~\ref{fig:ellip_sim}), an Euler rotation~\citep{G99} (black star and
blue angle arc) can be calculated along the great circle (progressing from blue
to red) from the location (0\degree, 0\degree) to the actual pole location. After
those random points (blue points) are rotated using the same Euler rotation to
the new locations (red point cloud in Fig.~\ref{fig:ellip_sim}), this elliptical
cloud (red point cloud) then is adjusted to its actual azimuth (i.e., the
major-axis azimuth of the pole's uncertainty ellipse; the red dashed line
rotated to the yellow dashed line using the red star as the Euler pole shown in
Fig.~\ref{fig:ellip_sim}).

Note that directly using NumPy's ``random.multivariate\_normal'' or
``random.normal'' routine (2D calculations) and spherical trigonometry to draw
random points for an elliptical uncertainty distorts the point cloud out of a
bivariate normal distribution, especially at high-latitude areas~\cite[see the
examples given by their Figure 7]{P18} and makes the simulation inaccurate.
This analogue approach avoids producing declination and inclination
vectors beforehand and directly generates random pole vectors, which saves the
transformation from declination and inclination to pole and further helps keep
us away from the distortion.

\begin{figure}[tbp]
\includegraphics[width=1.0\linewidth]{../../paper/tex/ComputGeosci/figures/rd.pdf}
\caption[How N$<=$1 uncertainty ellipse is simulated]{Example of modeling random
points for an ellipse uncertainty on the Earth's surface. Sample points (blue)
from a bivariate normal distribution centered at the intersection of the equator
and the prime meridian are rotated to their new locations (red points) together
with the uncertainty ellipse center (i.e.\ the 0\degree\ longitude 0\degree\ latitude
point prior to the rotation) exactly rotated to its actual pole coordinate (red
star), then adjusted to the true orientation (yellow dashed
line).}\label{fig:ellip_sim}\end{figure}

\section{Synchronization}
This algorithm is developed for comparing time-synchronized APWPs. In other
words, the compared APWPs should have the same timestamps. If the number of
their timestamps are different, the unpaired $pole(s)$ would be removed to make
the timestamps the same before the comparison. APWPs with a pole interpolated
for pairing an unpaired pole can be processed by our tool, as we noted earlier,
but it is not recommended for a valid analysis. For example, for paleomagnetic
APWPs, sometimes there are no paleopoles for a given time window ($N=0$);
sometimes a mean pole is an interpolated result.

\subsection{Equally treated random weights}
Assigning equally likely (not necessarily equal in value) random values to
$W_s,W_a,W_l$ is also tested. Three uniformly distributed random numbers with a
given sum 1 are generated for, for example, 10 000 times here, and then are
substituted into the $\mathcal{CPD}$ formula for deriving the seven APWP
pairs' $D_{full},D_{0-100Ma}$ etc.\ to check the possibility that ``one pair is
superior to the other pair'' (Fig.~\ref{fig:rall}).

\begin{figure}[tbp]
\centering
\includegraphics[width=.9\linewidth]{../../paper/tex/ComputGeosci/figures/rAll.png}
\caption[Comparisons of Pairs a-g with random weights involved]{Differences of
$\mathcal{CPD}s$ between \emph{Pair}~\textbf{a}, \emph{Pair}~\textbf{b},
\emph{Pair}~\textbf{c}, \emph{Pair}~\textbf{d}, \emph{Pair}~\textbf{e},
\emph{Pair}~\textbf{f} and \emph{Pair}~\textbf{g}, when 10 000 sets of three
uniformly random weights (with
their sum 1) are applied. If the difference $D$ is positive, the subtrahend pair
ranks higher in similarity, and if it is negative, the minuend pair ranks
higher. The $y$ axis in each upper plot is for the percentage $P$ that the
subtrahend pair owns higher similarity.}\label{fig:rall}
\end{figure}

The full-path results (Fig.~\ref{fig:rall}) again re-verify Order (4) and
the results shown in Fig. 8. Although the possibility that
\emph{Pair}~\textbf{d} is more similar than \emph{Pair}~\textbf{e} is not
significant (around 50\%), the possibility that
\emph{Pairs}~\textbf{f},\textbf{g} are more similar than
\emph{Pairs}~\textbf{c},\textbf{d},\textbf{e} is significant (more than 95\%),

All the sub-path results (Fig.~\ref{fig:rall}) are explicable using the results
shown in Fig. 8. For example, for 0\textendash100 Ma, both
\emph{Pair}~\textbf{a} and \emph{Pair}~\textbf{b} are assigned values of zero
for all the three metrics $d_s^{0-100Ma}$, $d_l^{0-100Ma}$ and $d_a^{0-100Ma}$
(Figures 8b, 8d and 8f), which means they are always undifferentiated.

\chapter{Methods for Constraining Paleopoles For Tectonic Plates in Chapter 3}\label{appen4chp3}

A polygon can be drawn around a set of paleomagnetic data, whose sampling sites
we believe belong to a specific plate or rigid block. Each such plate tectonic
polygon is assigned with an unique identification code, which is called Plate
ID\@. Then the {\em Spatial Join\/} technique~\citep{J07} helps join attributes,
especially Plate ID, from the polygon to the paleomagnetic data based on the
spatial relationship allowing data within this polygon to be extracted from the
whole raw large dataset without splitting a subset just for a specific plate.
That allows us to quickly select subsets of the database based on geographic
constraints just as easily as for age. Of course, the boundary of this polygon
must be reasonably along a tectonic boundary. Regions like those close to the
plate boundaries are usually tectonically active (e.g.\ local rotations), so we
should also be careful when we deal with the paleopoles derived from this type
of locations.

\section{120\textendash0 Ma North America}

The data-constraining polygons are from the recently published plate
model~\citep{Y18} (Fig.~\ref{fig_NAfinal}). North American craton (Plate ID 101)
polygon in the plate model~\citep{Y18}, including its children 108
(Avalon/Acadia block) and 109 (Piedmont block) polygons for 120\textendash0 Ma,
is used to select the sampling sites of the paleopoles for North America.
According to the plate model rotation data~\citep{Y18}, 108 is fixed to 101
during the geologic period from Cretaceous to the present day. 109 is also fixed
to 101 since ${\sim}300$ Ma~\citep{C14}. Then in order to be compared with the
FHM (120\textendash0 Ma)~\citep{M93,M99}, the paleopoles with age ranging
120\textendash0 Ma are further selected through constraining the lower magnetic
age ``LOMAGAGE $<=$ 135'' (here it is not 120 but 135, because for the lower
resolution case when the window length is 30 Myr, the Age Position Picking
method will include those data with their lower magnetic age between 120 Ma and
135 Ma). In addition, the RESULTNO=6007 dataset should also be included
according to a published plate kinematic model~\citep{Mc06}
with a relatively higher resolution of polygons and
rotations, although the dataset is in the PlateID=178 polygon. In the end, 193
datasets in total are extracted (both white circles and red
triangle-inside-circles in Fig.~\ref{fig_NAfinal}).

Also based on this model of southwestern North America since 36 Ma~\citep{Mc06},
part of the paleopoles constrained by the four small western terranes whose
Plate IDs are also 101 (white circles in Fig.~\ref{fig_NAfinal}) in fact had gone through regional
rotations and here are removed. However, the poles with age younger than 10 Ma
located within the largest western 101 terrane (on the south of the smallest
western 101 terrane; corresponding to the RANGE\_ID=74 polygon in the
model~\citep{Mc06}) should be included. So finally 135 of the 193 datasets remain
(Fig.~\ref{fig_NAfinal}). Spatially North American paleomagnetic data are mainly
from the western and eastern margins of the plate.

\begin{figure}
\includegraphics[scale=.7]{../../paper/tex/GeophysJInt/figures/Appendix/101.pdf}
  \caption[Final filtered datasets for analysis on 120\textendash0 Ma North
America]{The final filtered datasets (red triangle-inside-circles) for later
analysis on 120\textendash0 Ma North America. Those poles that had been
influenced by local tectonic rotations are shown as white
circles.}\label{fig_NAfinal}
\end{figure}

\section{120\textendash0 Ma India}

Plate ID 501 polygons in the recently published Plate Model~\citep{Y18} also
include the two small polygons of the northern ``Lesser Himalayan passive
margin of Greater Indian Basin'' and ``Tethyan Himalayan microcontinent of
Greater India'' (Fig.~\ref{fig_INfinal}). The polygons are used to select the
sampling sites of the paleopoles for India (Fig.~\ref{fig_INfinal}).

\begin{figure}
\includegraphics[scale=.7]{../../paper/tex/GeophysJInt/figures/Appendix/501.pdf}
\caption[Final filtered datasets for analysis on 120\textendash0 Ma India]{The
final filtered datasets (red triangle-inside-circles) for later
analysis on 120\textendash0 Ma India. Those poles that had been influenced by
local tectonic rotations are shown as white circles. The rifts, faults and
detachments (red lines) around India are used to filter out those data that
are influenced by local tectonic rotations.}\label{fig_INfinal}
\end{figure}

Based on the model of the tectonic interactions between India, Arabia and Asia
since the Jurassic~\citep{G15} (Fig.~\ref{fig_INfinal}), part of the paleopoles
constrained by the north two small terranes whose Plate IDs are also 501 in fact
had gone through regional rotations and here are removed. So finally 75 datasets
are left (Fig.~\ref{fig_INfinal}). Spatially Indian paleomagnetic data are more
evenly distributed on the India plate than North American and Australian poles.

\section{120\textendash0 Ma Australia}

Plate ID 801 polygon in the recently published Plate Model~\citep{Y18}, including
its children 675 (Sumba block) and 684 (Timor block) polygons for
120\textendash0 Ma (Fig.~\ref{fig_AUfinal}), is used to select the sampling
sites of the paleopoles for Australia. According to the plate model rotation
data~\citep{Y18}, 675 and 684 are fixed to 801 during the geologic period from
${\sim}145$ Ma to the present.

On the southeast of the main Australia plate (the blue polygon in
Fig.~\ref{fig_AUfinal}), there is a triangle-shaped small polygon 850
(Tasmania block) which is fixed to 801 since ${\sim}100$ Ma according to
the~\citet{Y18} rotation data. With that attribute, 850 contributes more data
younger than ${\sim}100$ Ma for the later analysis. Ultimately the final 99 extracted
datasets is shown in Fig.~\ref{fig_AUfinal}.

\begin{figure}
\includegraphics[scale=.7]{../../paper/tex/GeophysJInt/figures/Appendix/801.pdf}
\caption[Final filtered datasets for analysis on 120\textendash0 Ma
Australia]{The final filtered datasets (red triangle-inside-circles) for later
analysis on 120\textendash0 Ma Australia. Those poles that had been influenced
by local tectonic rotations are shown as white circles. The Plate ID 850 helps
increase the amount of qualified datasets for 100\textendash0
Ma.}\label{fig_AUfinal}
\end{figure}

\begin{landscape}
\scriptsize
\setlength\LTleft{-7mm}
  \begin{longtable}{@{}lllllllllllllp{3.5cm}@{}}
\caption{Rotation parameters for making 3 reference paths}\label{tab:rot}\\
\toprule
Moving-Fixed & Chron & Age/Ma & Latitude & Longitude & Rotation & kappa & a & b & c & d & e & f & Source \\* \midrule
\endfirsthead
%
\multicolumn{14}{c}%
{{\bfseries Table \thetable\ continued from previous page}} \\
\toprule
Moving-Fixed & Chron & Age/Ma & Latitude & Longitude & Rotation & kappa & a & b & c & d & e & f & Source \\* \midrule
\endhead
%
\bottomrule
\endfoot
%
\endlastfoot
%
NAm-Nubia & 1no & 0.78 & 79.2 & 40.2 & 0.18 & 1 & 7.41E-9 & -5.77E-9 & 4.29E-9 & 5.93E-9 & -3.35E-9 & 5.15E-9 & Demets et al. 2010 \\
NAm-Nubia(NWA) & 2An & 2.7 & 78.8 & 38.3 & 0.65 & 1 & 1E-15 & 1E-15 & 1E-15 & 1E-15 & 1E-15 & 1E-15 & Cande et al. 1995 Shephard et al. EPSL2012 Gurnis et al. CG2012 \\
NAm-Nubia(NWA) & 5n.1ny & 9.74 & 80.98 & 22.82 & 2.478 & 2.46 & 4.72E-5 & -4.05E-5 & 2.79E-5 & 4.35E-5 & -2.9E-5 & 2.06E-5 & Muller et al. 1999 \\
NAm-Nubia(NWA) & 5n.2o & 10.949 & 81 & 22.9 & 2.85 & 1 & 1E-15 & 1E-15 & 1E-15 & 1E-15 & 1E-15 & 1E-15 & Gaina et al. 2013 \\
NAm-Nubia(NWA) & 6ny & 19.05 & 80.89 & 23.28 & 5.244 & 1.96 & 6.13E-5 & -4.97E-5 & 3.56E-5 & 5.2E-5 & -3.59E-5 & 2.68E-5 & Muller et al. 1999 \\
NAm-Nubia(NWA) & 6no & 20.131 & 80.6 & 24.4 & 5.54 & 1 & 1E-15 & 1E-15 & 1E-15 & 1E-15 & 1E-15 & 1E-15 & Gaina et al. 2013 \\
NAm-Nubia(NWA) & 8n.1ny & 25.82 & 79.34 & 28.56 & 7.042 & 2.53 & 1.71E-4 & -1.71E-4 & 1.24E-4 & 1.96E-4 & -1.41E-4 & 1.04E-4 & Muller et al. 1999 \\
NAm-Nubia(NWA) & 13ny & 33.058 & 75.99 & 5.98 & 9.767 & 1.19 & 8.32E-5 & -8.47E-5 & 5.97E-5 & 1.03E-4 & -7.14E-5 & 5.21E-5 & Muller et al. 1999 Gaina et al. 2013 \\
NAm-Nubia(NWA) & 16ny & 35.343 & 75.3 & 3.1 & 10.68 & 1 & 1E-15 & 1E-15 & 1E-15 & 1E-15 & 1E-15 & 1E-15 & Gaina et al. 2013 \\
NAm-Nubia(NWA) & 18n1ny? & 38.43 & 74.54 & 0.19 & 11.918 & 1.65 & 1.98E-4 & -2.08E-4 & 1.43E-4 & 2.41E-4 & -1.64E-4 & 1.16E-4 & Muller et al. 1999 \\
NAm-Nubia(NWA) & 18n.2no & 40.13 & 74.5 & -1.2 & 12.61 & 1 & 1E-15 & 1E-15 & 1E-15 & 1E-15 & 1E-15 & 1E-15 & Gaina et al. 2013 \\
NAm-Nubia(NWA) & 20ny & 42.5 & 74.38 & -2.8 & 13.56 & 1 & 1E-15 & 1E-15 & 1E-15 & 1E-15 & 1E-15 & 1E-15 & Muller et al. 1993 Shephard et al. 2012 \\
NAm-Nubia(NWA) & 20no & 43.789 & 74.3 & -3.6 & 14.09 & 1 & 1E-15 & 1E-15 & 1E-15 & 1E-15 & 1E-15 & 1E-15 & Gaina et al. 2013 \\
NAm-Nubia(NWA) & 21ny & 46.26 & 74.23 & -5.01 & 15.106 & 1.19 & 1.8E-4 & -2.03E-4 & 1.36E-4 & 2.68E-4 & -1.8E-4 & 1.26E-4 & Muller et al. 1999 \\
NAm-Nubia(NWA) & 21no & 47.906 & 74.9 & -4.6 & 15.61 & 1 & 1E-15 & 1E-15 & 1E-15 & 1E-15 & 1E-15 & 1E-15 & Gaina et al. 2013 \\
NAm-Nubia(NWA) & 22ny & 49.037 & 75.29 & -4.26 & 15.95 & 1 & 1E-15 & 1E-15 & 1E-15 & 1E-15 & 1E-15 & 1E-15 & Muller et al. 1993 Shephard et al. 2012 \\
NAm-Nubia(NWA) & 22no & 49.714 & 75.9 & -3.5 & 16.15 & 1 & 1E-15 & 1E-15 & 1E-15 & 1E-15 & 1E-15 & 1E-15 & Gaina et al. 2013 \\
NAm-Nubia(NWA) & 24n.1ny & 52.364 & 77.34 & -1.61 & 16.963 & 3.08 & 2.65E-4 & -3.13E-4 & 2.07E-4 & 4.05E-4 & -2.67E-4 & 1.8E-4 & Muller et al. 1999 Gaina et al. 2013 \\
NAm-Nubia(NWA) & 25ny & 55.904 & 80.64 & 6.57 & 17.895 & 1.26 & 1.19E-4 & -1.36E-4 & 8.64E-5 & 1.87E-4 & -1.18E-4 & 7.87E-5 & Muller et al. 1999 Gaina et al. 2013 \\
NAm-Nubia(NWA) & 30ny & 65.58 & 82.74 & 2.93 & 20.84 & 1.07 & 9.12E-5 & -1.07E-4 & 6.64E-5 & 1.58E-4 & -9.83E-5 & 6.45E-5 & Muller et al. 1999 \\
NAm-Nubia(NWA) & 31ny & 67.735 & 82.3 & -1.8 & 21.53 & 1 & 1E-15 & 1E-15 & 1E-15 & 1E-15 & 1E-15 & 1E-15 & Gaina et al. 2013 \\
NAm-Nubia(NWA) & 32n.2ny & 71.59 & 81.35 & -8.32 & 22.753 & 1.33 & 1.23E-4 & -1.62E-4 & 9.7E-5 & 2.42E-4 & -1.46E-4 & 9.1E-5 & Muller et al. 1999 \\
NAm-Nubia(NWA) & 33ny & 73.619 & 81.11 & -10.64 & 23.74 & 1 & 1E-15 & 1E-15 & 1E-15 & 1E-15 & 1E-15 & 1E-15 & Muller et al. 1993Shephard et al. 2012 \\
NAm-Nubia(NWA) & 33no & 79.075 & 78.64 & -18.16 & 26.981 & 1.85 & 4.09E-5 & -4.02E-5 & 2.34E-5 & 5.73E-5 & -3.48E-5 & 2.38E-5 & Muller et al. 1999 \\
NAm-Nubia(NWA) & 34ny & 83.5 & 76.81 & -20.59 & 29.506 & 1.82 & 6.2E-5 & -4.83E-5 & 2.47E-5 & 6.96E-5 & -4.29E-5 & 3.07E-5 & Muller et al. 1999\_2008 Gaina et al. 2013 \\
NAm-Nubia(NWA) & $\mathord{?}$ & 89.9 & 74.33 & -22.65 & 33.86 & 1 & 1E-15 & 1E-15 & 1E-15 & 1E-15 & 1E-15 & 1E-15 & Muller\_Roest1992 Muller et al. 2008 Shephard et al. 2012 \\
NAm-Nubia(NWA) & $\mathord{?}$ & 94.1 & 72.0 & -24.39 & 36.49 & 1 & 1E-15 & 1E-15 & 1E-15 & 1E-15 & 1E-15 & 1E-15 & Muller\_Roest1992 Muller et al. 2008 Shephard et al. 2012 \\
NAm-Nubia(NWA) & $\mathord{?}$ & 100 & 69.42 & -23.52 & 40.46 & 1 & 1E-15 & 1E-15 & 1E-15 & 1E-15 & 1E-15 & 1E-15 & Muller\_Roest1992 Muller et al. 2008 Shephard et al. 2012 \\
NAm-Nubia(NWA) & $\mathord{?}$ & 106.9 & 68.08 & -22.66 & 45.36 & 1 & 1E-15 & 1E-15 & 1E-15 & 1E-15 & 1E-15 & 1E-15 & Muller\_Roest1992 Muller et al. 2008 Shephard et al. 2012 \\
NAm-Nubia(NWA) & $\mathord{?}$ & 118.1 & 66.21 & -21.0 & 53.19 & 1 & 1E-15 & 1E-15 & 1E-15 & 1E-15 & 1E-15 & 1E-15 & Muller\_Roest1992 Muller et al. 2008 Shephard et al. 2012 \\
NAm-Nubia(NWA) & 34no & 120.6 & 65.9 & -20.5 & 54.56 & 1 & 1E-15 & 1E-15 & 1E-15 & 1E-15 & 1E-15 & 1E-15 & Gaina et al. 2013 \\
Nubia-Mantle & interpol & 0 & 57.728 & -51.2596 & 0 & 1 & 1E-15 & 1E-15 & 1E-15 & 1E-15 & 1E-15 & 1E-15 & O'Neill et al. 2005fixedHotspots \\
Nubia-Mantle & interpol & 10 & 57.728 & -51.2596 & -1.3645 & 3.3043 & 9.495E-5 & -6.072E-6 & 1.085E-5 & 2.8857E-5 & -2.436E-5 & 6.987E-5 & O'Neill et al. 2005fixedHotspots \\
Nubia-Mantle & interpol & 20 & 40.412 & -23.4948 & -3.611 & 2.545 & 1.57E-4 & 4.682E-6 & 2.293E-5 & 8.242E-5 & -6.609E-5 & 1.49E-4 & O'Neill et al. 2005fixedHotspots \\
Nubia-Mantle & interpol & 30 & 49.42 & -61.8981 & -5.795 & 3.172 & 8.149E-5 & 2.568E-5 & -6.617E-6 & 8.819E-5 & -5.922E-5 & 8.762E-5 & O'Neill et al. 2005fixedHotspots \\
Nubia-Mantle & interpol & 40 & 44.555 & -55.735 & -7.4188 & 0.7337 & 7.166E-4 & -5.577E-5 & -2.8646E-4 & 1.5166E-4 & -5.034E-5 & 2.725E-4 & O'Neill et al. 2005fixedHotspots \\
Nubia-Mantle & interpol & 50 & 8.836 & -34.845 & -12.431 & 0.826 & 6.97E-4 & 2.808E-5 & -3.607E-4 & 1.587E-4 & -1.04E-4 & 3.44E-4 & O'Neill et al. 2005fixedHotspots \\
Nubia-Mantle & interpol & 60 & 14.244 & -40.32 & -13.725 & 2.081 & 1.496E-3 & 3.451E-4 & -5.396E-4 & 7.096E-4 & -2.723E-4 & 5.502E-4 & O'Neill et al. 2005fixedHotspots \\
Nubia-Mantle & interpol & 70 & 26.518 & -47.4563 & -14.604 & 0.65 & 8.564E-4 & 1.195E-4 & -3.269E-4 & 2.863E-4 & -1.276E-4 & 4.452E-4 & O'Neill et al. 2005fixedHotspots \\
Nubia-Mantle & interpol & 80 & 26.85 & -46.9249 & -17.705 & 0.473 & 7.972E-4 & 1.314E-4 & -3.436E-4 & 2.874E-4 & -1.396E-4 & 3.485E-4 & O'Neill et al. 2005fixedHotspots \\
Nubia-Mantle & interpol & 90 & 7.2356 & -33.7327 & -22.745 & 0.692 & 6.805E-4 & 2.6E-5 & -1.336E-4 & 1.834E-4 & -2.302E-4 & 6.163E-4 & O'Neill et al. 2005fixedHotspots \\
Nubia-Mantle & interpol & 100 & 10.826 & -46.482 & -26.59 & 1.308 & 3.705E-4 & 2.897E-6 & 5.225E-5 & 4.458E-4 & 3.313E-5 & 3.037E-4 & O'Neill et al. 2005fixedHotspots \\
Nubia-Mantle & interpol & 110 & 5.063 & -34.8933 & -31.787 & 0.475 & 6.694E-4 & 1.7826E-4 & -1.243E-5 & 1.875E-4 & 6.704E-5 & 4.335E-4 & O'Neill et al. 2005fixedHotspots \\
Nubia-Mantle & interpol & 120 & 22.488 & -37.2224 & -30.632 & 66.181 & 1.436E-3 & 4.318E-5 & 1.78E-3 & 1E-3 & 1.134E-3 & 4.44E-3 & O'Neill et al. 2005fixedHotspots \\
Nubia-Mantle & None & 0 & 46.19 & -87.86 & 0 & 1 & 1E-15 & 1E-15 & 1E-15 & 1E-15 & 1E-15 & 1E-15 & O'Neill et al. 2005moving \\
Nubia-Mantle & None & 10 & 46.19 & -87.86 & -1.92 & 0.92 & 9.94E-5 & -3.8E-6 & -1.18E-6 & 2.84E-5 & -2.3E-5 & 8.68E-5 & O'Neill et al. 2005moving \\
Nubia-Mantle & None & 30 & 43.54 & -69.67 & -6.05 & 3.31 & 9.89E-5 & 2.14E-5 & -3.56E-5 & 8.19E-5 & -6.45E-5 & 1.18E-4 & O'Neill et al. 2005moving \\
Nubia-Mantle & None & 40 & 44.56 & -54.31 & -8.08 & 0.6 & 6.39E-4 & -9.58E-5 & -3.09E-4 & 1.54E-4 & -2.27E-5 & 3.49E-4 & O'Neill et al. 2005moving \\
Nubia-Mantle & None & 50 & 36.97 & -58.9 & -10.26 & 1.24 & 7.6E-4 & -1.29E-4 & -4.44E-4 & 1.65E-4 & -1.35E-5 & 4.53E-4 & O'Neill et al. 2005moving \\
Nubia-Mantle & None & 60 & 23.73 & -42.14 & -12.53 & 2.31 & 1.61E-3 & 8.69E-5 & -6.22E-4 & 6.07E-4 & -2.63E-4 & 7.51E-4 & O'Neill et al. 2005moving \\
Nubia-Mantle & None & 90 & 14.6 & -33.26 & -16.24 & 2.47 & 5.47E-4 & -9.38E-5 & 5.78E-6 & 1.18E-4 & -7.96E-5 & 5.36E-4 & O'Neill et al. 2005moving \\
Nubia-Mantle & None & 100 & 14.4 & -29.63 & -20.08 & 3.25 & 2.41E-4 & -1.79E-5 & 1.48E-5 & 3.16E-4 & -1.69E-5 & 1.92E-4 & O'Neill et al. 2005moving \\
Nubia-Mantle & None & 120 & 17.03 & -27 & -29.72 & 9.06 & 7.81E-4 & 3.73E-4 & 4.38E-4 & 6.44E-4 & 4.17E-4 & 5.87E-4 & O'Neill et al. 2005moving \\
Ind-Somalia & 1no & 0.781 & 19.57 & 27.97 & -.347 & 1 & 1.99E-6 & 3.41E-6 & 1.86E-9 & 6.82E-6 & 6.01E-7 & 4.67E-7 & Bull et al. 2010 \\
Ind-Somalia & 2ny & 1.778 & 21.59 & 30.83 & -.755 & 1 & 4.26E-7 & 6.51E-7 & -4.44E-8 & 1.32E-6 & 1.74E-7 & 2.11E-7 & Bull et al. 2010 \\
Ind-Somalia & 2An.1ny & 2.581 & 22.48 & 30.6 & -1.074 & 1 & 2.94E-6 & 4.33E-6 & -6.73E-7 & 7.77E-6 & -7.56E-9 & 9.55E-7 & Bull et al. 2010 \\
Ind-Somalia & 2An.3no & 3.596 & 18.7 & 34.62 & -1.642 & 1 & 4.51E-6 & 7.3E-6 & -5.4E-7 & 1.42E-5 & 7.06E-7 & 1.35E-6 & Bull et al. 2010 \\
Ind-Somalia & 3n.1ny & 4.187 & 22.11 & 28.45 & -1.74 & 1 & 4.31E-6 & 7.11E-6 & -2.86E-7 & 1.34E-5 & 5.49E-7 & 8.8E-7 & Bull et al. 2010 \\
Ind-Somalia & 3n.4no & 5.235 & 22.05 & 31.64 & -2.181 & 1 & 2.18E-6 & 3.53E-6 & -3.32E-7 & 6.89E-6 & 1.84E-7 & 6.19E-7 & Bull et al. 2010 \\
Ind-Somalia & 3An.1ny & 5.89 & 22.89 & 28.11 & -2.333 & 1 & 6.98E-6 & 1.15E-5 & -1.71E-6 & 2.29E-5 & -3.39E-7 & 2.37E-6 & Bull et al. 2010 \\
Ind-Somalia & 3An.2no & 6.57 & 21.32 & 30.92 & -2.748 & 1 & 2.41E-6 & 4.68E-6 & -5.41E-10 & 9.68E-6 & 4.53E-7 & 4.53E-7 & Bull et al. 2010 \\
Ind-Somalia & 4n.1ny & 7.43 & 22.61 & 30.57 & -2.982 & 1 & 5.75E-6 & 1.17E-5 & -2.81E-7 & 2.68E-5 & 1.7E-6 & 1.92E-6 & Bull et al. 2010 \\
Ind-Somalia & 4n.2no & 8.07 & 22.01 & 30.79 & -3.281 & 1 & 4.62E-6 & 8.01E-6 & -1.16E-6 & 1.58E-5 & -6.77E-7 & 1.4E-6 & Bull et al. 2010 \\
Ind-Somalia & 4Any & 8.67 & 22.48 & 30.77 & -3.489 & 1 & 4.46E-6 & 6.32E-6 & -1.54E-6 & 1.36E-5 & 1.13E-6 & 3.14E-6 & Bull et al. 2010 \\
Ind-Somalia & 4Ano & 9.03 & 22.59 & 30.78 & -3.685 & 1 & 1.45E-6 & 1.82E-6 & -6.92E-7 & 3.84E-6 & 3.34E-7 & 1.33E-6 & Bull et al. 2010 \\
Ind-Somalia & 5n.1ny & 9.74 & 23.59 & 30.49 & -3.89 & 1 & 3.08E-6 & 4.97E-6 & -2.38E-7 & 1.01E-5 & 1.36E-6 & 1.63E-6 & Bull et al. 2010 \\
Ind-Somalia & 5n.2no & 10.95 & 23.62 & 29.3 & -4.311 & 1 & 4.43E-6 & 7.25E-6 & -1.58E-7 & 1.58E-5 & 2.01E-6 & 1.56E-6 & Bull et al. 2010 \\
Ind-Somalia & 5An.2no & 12.4 & 23.8 & 29.26 & -4.879 & 1 & 3.39E-6 & 5.42E-6 & -5.85E-7 & 1.04E-5 & 5.02E-7 & 1.44E-6 & Bull et al. 2010 \\
Ind-Somalia & 5ADno & 14.607 & 24.6 & 29.14 & -5.76 & 1 & 2.83E-6 & 6.13E-6 & 4.51E-7 & 1.4E-5 & 1.29E-6 & 3.29E-7 & Bull et al. 2010 \\
Ind-Somalia & 5Cn.1ny & 16.01 & 24.8 & 29.3 & -6.399 & 1 & 1.27E-6 & 2.26E-6 & -1.7E-7 & 5.7E-6 & 4.81E-7 & 5.08E-7 & Bull et al. 2010 \\
Ind-Somalia & 5Dny & 17.235 & 24.83 & 30.29 & -7.149 & 1 & 1.76E-6 & 2.98E-6 & -2.06E-7 & 6.53E-6 & 6.03E-7 & 7.15E-7 & Bull et al. 2010 \\
Ind-Somalia & 5Eny & 18.28 & 24.77 & 30.27 & -7.637 & 1 & 2.47E-6 & 5.44E-6 & 3.58E-7 & 1.32E-5 & 1.44E-6 & 5.17E-7 & Bull et al. 2010 \\
Ind-Somalia & 6no & 20.13 & 25.41 & 30.6 & -8.469 & 1 & 3E-5 & 3.6E-5 & -1.63E-5 & 6.11E-5 & -7.2E-6 & 1.81E-5 & Bull et al. 2010 \\
Ind-Somalia & 13no & 33.55 & -19.35 & -137.7 & 16.05 & 1 & 3.85E-5 & 5.31E-5 & -2.14E-5 & 1.05E-4 & -1.84E-5 & 2.19E-5 & Calculated \\
Ind-Somalia & 18n.2no & 40.13 & -19.48 & -135.85 & 19.31 & 1 & 4.06E-5 & 6.05E-5 & -2.33E-5 & 1.31E-4 & -2.52E-5 & 2.37E-5 & Calculated \\
Ind-Somalia & 20ny & 42.54 & -19.94 & -136.28 & 20.23 & 1 & 3.74E-5 & 4.99E-5 & -2.03E-5 & 9.48E-5 & -1.51E-5 & 2.08E-5 & Calculated \\
Ind-Somalia & 20no & 43.79 & -20.44 & -136.77 & 20.63 & 1 & 3.73E-5 & 4.98E-5 & -2.03E-5 & 9.5E-5 & -1.52E-5 & 2.1E-5 & Calculated \\
Ind-Somalia & 21ny & 46.26 & -19.6 & -137.27 & 21.82 & 1 & 3.73E-5 & 4.94E-5 & -2.02E-5 & 9.33E-5 & -1.46E-5 & 2.08E-5 & Calculated \\
Ind-Somalia & 21no & 47.91 & -19.49 & -138.01 & 22.9 & 1 & 3.7E-5 & 4.87E-5 & -1.99E-5 & 9.13E-5 & -1.41E-5 & 2.05E-5 & Calculated \\
Ind-Somalia & 22no & 49.71 & -19.63 & -140.64 & 24.1 & 1 & 3.72E-5 & 4.96E-5 & -2.01E-5 & 9.42E-5 & -1.48E-5 & 2.07E-5 & Calculated \\
Ind-Somalia & 23n.2no & 51.74 & -17.83 & -142.26 & 26.82 & 1 & 3.8E-5 & 5.26E-5 & -2.09E-5 & 1.06E-4 & -1.75E-5 & 2.15E-5 & Calculated \\
Ind-Somalia & 24n.3no & 53.35 & -17.69 & -144.8 & 28.76 & 1 & 3.86E-5 & 5.46E-5 & -2.13E-5 & 1.14E-4 & -1.9E-5 & 2.17E-5 & Calculated \\
Ind-Somalia & 25ny & 55.9 & -17.68 & -147.99 & 31.52 & 1 & 4.13E-5 & 6.48E-5 & -2.29E-5 & 1.52E-4 & -2.53E-5 & 2.29E-5 & Calculated \\
Ind-Somalia & 26ny & 57.55 & -16.81 & -148.61 & 33.86 & 1 & 3.91E-5 & 5.72E-5 & -2.14E-5 & 1.25E-4 & -2E-5 & 2.16E-5 & Calculated \\
Ind-Somalia & 27ny & 60.92 & -15.03 & -149.25 & 38.15 & 1 & 3.97E-5 & 5.93E-5 & -2.15E-5 & 1.34E-4 & -2.02E-5 & 2.15E-5 & Calculated \\
Ind-Somalia & 28ny & 62.5 & -14.48 & -149.55 & 40.23 & 1 & 4.93E-5 & 9.58E-5 & -2.61E-5 & 2.74E-4 & -3.79E-5 & 2.38E-5 & Calculated \\
Ind-Somalia & 29no & 64.75 & -15.39 & -152.91 & 42.42 & 1 & 3.92E-5 & 5.79E-5 & -2.08E-5 & 1.32E-4 & -1.79E-5 & 2.1E-5 & Calculated \\
Ind-Somalia & 34ny & 83 & 21.571 & 21.698 & -52.702 & 1 & 1E-15 & 1E-15 & 1E-15 & 1E-15 & 1E-15 & 1E-15 & Rowan and Rowley 2016 \\
Ind-Nubia(Madagas) & synthetic & 83 & 22.16 & 19.22 & -52.74 & 1 & 1E-15 & 1E-15 & 1E-15 & 1E-15 & 1E-15 & 1E-15 & Gibbons et al. 2013 Muller et al. 2017 \\
Ind-Nubia(Madagas) & synthetic & 100 & -23.19 & -152.79 & 55.76 & 1 & 1E-15 & 1E-15 & 1E-15 & 1E-15 & 1E-15 & 1E-15 & Gibbons et al. 2013 Muller et al. 2017 \\
Ind-Nubia(Madagas) & synthetic & 106 & -23.19 & -152.79 & 55.76 & 1 & 1E-15 & 1E-15 & 1E-15 & 1E-15 & 1E-15 & 1E-15 & Gibbons et al. 2013 Muller et al. 2017 \\
Ind-Nubia(Madagas) & synthetic & 120.6 & -24.58 & -153.92 & 54.51 & 1 & 1E-15 & 1E-15 & 1E-15 & 1E-15 & 1E-15 & 1E-15 & Gibbons et al. 2013 Muller et al. 2017 \\
Somalia-Nubia & 1no & 0.78 & -33.83 & 34.4 & 0.04602 & 1 & 3.737E-9 & 3.426E-9 & -8.39E-10 & 3.957E-9 & -8.03E-10 & 6.66E-10 & Demets et al. 2017 replace 2010 inverted DoF from 2010 \\
Somalia-Nubia & 2A.2no & 3.22 & -44.7 & 2.8 & 0.27048 & 1 & 4.29E-9 & 1.26E-9 & -5.01E-9 & 1.62E-9 & -1.44E-9 & 7.19E-9 & Horner-Johnson et al. 2005 \\
Somalia-Nubia & 5n.2no & 10.95 & -27.4 & 43.28 & 0.4 & 1 & 1E-15 & 1E-15 & 1E-15 & 1E-15 & 1E-15 & 1E-15 & Rowan and Rowley 2016 \\
Somalia-Nubia & 6no & 20.13 & -27.4 & 43.28 & 0.8 & 1 & 1E-15 & 1E-15 & 1E-15 & 1E-15 & 1E-15 & 1E-15 & Rowan and Rowley 2016 \\
Somalia-Nubia & C7.2m & 25.01 & -27.4 & 43.28 & 1 & 1 & 1E-15 & 1E-15 & 1E-15 & 1E-15 & 1E-15 & 1E-15 & Rowan and Rowley 2016 \\
Somalia-Nubia & C34 & 85 & -27.4 & 43.28 & 1 & 1 & 1E-15 & 1E-15 & 1E-15 & 1E-15 & 1E-15 & 1E-15 & Rowan and Rowley 2016 \\
Somalia-Nubia & $>$85NoRelMot & 120 & -27.4 & 43.28 & 1 & 1 & 1E-15 & 1E-15 &
1E-15 & 1E-15 & 1E-15 & 1E-15 & Inferred From Global\_EarthByte\_230-0Ma\_GK07\_AREPS.rot \\
Australia-EAnt & 1no & 0.78 & 11.3 & 41.8 & -.49374 & 1 & 2.605E-8 & 1.846E-8 & 9.725E-9 & 2.184E-8 & 1.785E-9 & 1.031E-8 & Demets et al. 2010errInverted \\
Australia-EAnt & 2An.1ny & 2.58 & -11.164 & -139.7 & 1.655 & 5.33 & 2.81E-7 & -3.35E-7 & 2.6E-7 & 5.15E-7 & -4.87E-7 & 9.06E-7 & Cande and Stock2004 \\
Australia-EAnt & 3An.1ny & 5.89 & -11.591 & -139.23 & 3.83 & 2.12 & 2.94E-7 & -3.9E-7 & 3.19E-7 & 6.27E-7 & -5.83E-7 & 8.97E-7 & Cande and Stock2004 Krijgsman et al. 1999 Age From Meckel2005 \\
Australia-EAnt & 5n.2no & 10.95 & -11.896 & -142.058 & 6.79 & 1.02 & 1.36E-7 & -1.71E-7 & 5.56E-8 & 3.05E-7 & -1.89E-7 & 4.47E-7 & Cande and Stock2004 \\
Australia-EAnt & 6no & 20.13 & -13.393 & -145.63 & 12.051 & 1.06 & 1.61E-7 & -1.85E-7 & 2.56E-8 & 3E-7 & -1.59E-7 & 4.22E-7 & Cande and Stock2004 \\
Australia-EAnt & 8o & 26 & -13.79 & -146.43 & 15.92 & 0.95 & 1.68E-7 & -2.05E-7 & 1.05E-7 & 3.57E-7 & -2.72E-7 & 5.04E-7 & Granot and Dyment2018 \\
Australia-EAnt & 8n.2no & 26.55 & -13.805 & -146.444 & 15.919 & 1.01 & 1.95E-7 & -2.24E-7 & 5.79E-8 & 3.71E-7 & -2.39E-7 & 5.9E-7 & Cande and Stock2004 \\
Australia-EAnt & 10n.2no & 28.75 & -13.58 & -146.016 & 17.319 & 1.17 & 2.32E-7 & -2.56E-7 & 8.48E-8 & 3.94E-7 & -2.77E-7 & 7.51E-7 & Cande and Stock2004 \\
Australia-EAnt & 12no & 30.94 & -13.396 & -145.623 & 18.89 & 0.75 & 2.6E-7 & -3.14E-7 & 5.53E-8 & 5.18E-7 & -3.1E-7 & 9.82E-7 & Cande and Stock2004 \\
Australia-EAnt & 13no & 33.55 & -13.451 & -145.623 & 20.495 & 1.07 & 2.44E-7 & -3.26E-7 & 1.49E-7 & 5.56E-7 & -3.43E-7 & 6.93E-7 & Cande and Stock2004 \\
Australia-EAnt & 17n.3no & 38.11 & -14.65 & -146.525 & 22.882 & 0.49 & 6.88E-7 & -8.39E-7 & -2.51E-7 & 1.34E-6 & -1.37E-7 & 1.31E-6 & Cande and Stock2004 \\
Australia-EAnt & 20no & 43.79 & 14.92 & 32.5 & -24.51 & 0.74 & 6.3E-7 & -1E-7 & 1.68E-6 & 1.87E-6 & -3.1E-6 & 6.76E-6 & Whittaker et al. 2007\_2013 \\
Australia-EAnt & 21ny & 46.26 & 13.42 & 33.83 & -24.62 & 5.32 & 1.328E-5 & -2.024E-5 & 8.2E-7 & 3.203E-5 & -2.28E-6 & 5.76E-6 & Whittaker et al. 2013\_replace2007 \\
Australia-EAnt & 24n.3no & 53.35 & 10.48 & 35.17 & -25.24 & 0.11 & 3.38E-7 & -2.17E-7 & 4.29E-7 & 4.84E-7 & -1.139E-6 & 4.281E-6 & Whittaker et al. 2013\_replace2007 \\
Australia-EAnt & 27ny & 60.92 & 9.22 & 35.42 & -25.43 & 0.17 & 3.09E-7 & -1.93E-7 & 3.6E-7 & 5.07E-7 & -1.206E-6 & 4.474E-6 & Whittaker et al. 2013\_replace2007 \\
Australia-EAnt & 31no & 68.74 & 7.89 & 35.75 & -25.61 & 0.19 & 3.11E-7 & -1.85E-7 & 3.23E-7 & 4.95E-7 & -1.158E-6 & 4.303E-6 & Whittaker et al. 2013\_replace2007 \\
Australia-EAnt & 32n.1ny & 71.07 & 6.67 & 36.14 & -25.75 & 0.14 & 2.5E-7 & -1.82E-7 & 3.25E-7 & 4.82E-7 & -1.087E-6 & 3.991E-6 & Whittaker et al. 2013\_replace2007 \\
Australia-EAnt & 33no & 79.08 & 4.53 & 36.64 & -26.13 & 0.13 & 2.52E-7 & -1.79E-7 & 2.97E-7 & 4.75E-7 & -1.031E-6 & 3.788E-6 & Whittaker et al. 2013\_replace2007 \\
Australia-EAnt & 34ny & 83.5 & 1.02 & 37.28 & -26.62 & 0.13 & 5.37E-7 & -6E-9 & -6.56E-7 & 5.73E-7 & -8.19E-7 & 2.356E-6 & Williams et al. 2011 Whittaker et al. 2013replace2007 \\
Australia-EAnt & QZB & 96 & -12.69 & 46.58 & -29.06 & 1.03 & 3.53E-6 & -4.06E-6 & 8.19E-6 & 6.09E-6 & -1.35E-5 & 3.571E-5 & Whittaker et al. 2007 \\
Australia-EAnt & Full-fit & 136. & -3.91 & 37.9 & -30.86 & 1.04 & 2.85E-6 & -2.69E-6 & 3.77E-6 & 4.01E-6 & -6.93E-6 & 1.38E-5 & Williams et al. 2011 Whittaker et al. 2013 \\
EAnt-Somalia & 1no & 0.78 & -20.58 & 115.29 & -.10062 & 1 & 3.661E-9 & 3.38E-9 & -8.209E-10 & 4.124E-9 & -1.068E-9 & 2.415E-9 & Demets et al. 2017inverted\_replace2010inverted \\
EAnt-Somalia & 2A.2no & 3.22 & 7.9 & -44.1 & 0.42826 & 1 & 7.9E-10 & 8.8E-10 & -8.2E-10 & 1.49E-9 & -1.22E-9 & 1.15E-9 & Horner-Johnson et al. 2005 \\
EAnt-Somalia & 5n.2no & 10.95 & 14.6 & -49.1 & 1.53 & 1 & 2.21E-7 & 2.36E-7 & -9.2E-8 & 3.04E-7 & -1.67E-7 & 2.45E-7 & LeMaux et al. 2002 \\
EAnt-Somalia & 6no & 20.131 & 10.8 & -46 & 2.7 & 1 & 1.028E-7 & 8.83E-8 & -2.34E-8 & 1.759E-7 & -1.347E-7 & 2.237E-7 & Patriat et al. 2008 \\
EAnt-Somalia & 8n.2no & 26.554 & 14.3 & -46.9 & 3.91 & 1 & 1.484E-7 & 1.489E-7 & -9.66E-8 & 2.58E-7 & -2.458E-7 & 3.702E-7 & Patriat et al. 2008 \\
EAnt-Somalia & 13ny & 33.06 & 16.2 & -44.7 & 5.66 & 1 & 8.56E-7 & 6.73E-7 & -1.66E-7 & 6.9E-7 & -3.74E-7 & 5.82E-7 & Patriat et al. 2008 \\
EAnt-Somalia & 13no & 33.55 & 12.69 & -44.61 & 5.67 & 0.59 & 5.71E-7 & 5.12E-7 & 5.63E-7 & -2.35E-7 & -2.53E-7 & 3.99E-7 & Cande et al. 2010basic \\
EAnt-Somalia & 18n.2no & 40.13 & 13.8 & -43.75 & 7.05 & 0.63 & 6.6E-7 & 5.69E-7 & 5.56E-7 & -1E-9 & -7.9E-8 & 6.8E-7 & Cande et al. 2010basic \\
EAnt-Somalia & 20ny & 42.54 & 11.84 & -42.32 & 7.53 & 0.93 & 2.36E-7 & 1.91E-7 & 2.31E-7 & -2.28E-7 & -2.63E-7 & 5.98E-7 & Cande et al. 2010all \\
EAnt-Somalia & 20no & 43.79 & 11.84 & -42.16 & 7.87 & 0.78 & 1.64E-7 & 1.23E-7 & 1.79E-7 & -1.95E-7 & -2.42E-7 & 4.97E-7 & Cande et al. 2010all \\
EAnt-Somalia & 21ny & 46.26 & 11.29 & -41.54 & 8.49 & 0.55 & 1.89E-7 & 1.6E-7 & 2.24E-7 & -2.76E-7 & -3.25E-7 & 6.86E-7 & Cande et al. 2010all \\
EAnt-Somalia & 21no & 47.91 & 9.82 & -40.7 & 8.83 & 1.86 & 2.26E-7 & 1.98E-7 & 2.67E-7 & -3.69E-7 & -4E-7 & 8.38E-7 & Cande et al. 2010all \\
EAnt-Somalia & 22no & 49.71 & 9.19 & -40.63 & 9.21 & 1.4 & 2.73E-7 & 2.54E-7 & 3.7E-7 & -4.7E-7 & -5.4E-7 & 1.084E-6 & Cande et al. 2010all \\
EAnt-Somalia & 23n.2no & 51.74 & 9.31 & -41.53 & 9.61 & 0.74 & 2.78E-7 & 2.77E-7 & 4.22E-7 & -4.41E-7 & -5.38E-7 & 9.45E-7 & Cande et al. 2010withSWIR \\
EAnt-Somalia & 24n.3no & 53.35 & 10.16 & -43.3 & 9.96 & 0.72 & 4.74E-7 & 4.34E-7 & 5.65E-7 & -7.41E-7 & -7.62E-7 & 1.398E-6 & Cande et al. 2010withSWIR \\
EAnt-Somalia & 25ny & 55.9 & 9.86 & -45.24 & 10.49 & 0.52 & 1.149E-6 & 1.375E-6 & 1.932E-6 & -2.016E-6 & -2.601E-6 & 3.954E-6 & Cande et al. 2010withSWIR \\
EAnt-Somalia & 26ny & 57.55 & 10.64 & -47.47 & 10.78 & 1.35 & 1.267E-6 & 1.409E-6 & 1.747E-6 & -1.885E-6 & -2.253E-6 & 3.33E-6 & Cande et al. 2010withSWIR \\
EAnt-Somalia & 27ny & 60.92 & 7.1 & -45.8 & 11.08 & 1.15 & 1.623E-6 & 1.751E-6 & 2.193E-6 & -2.459E-6 & -2.843E-6 & 4.288E-6 & Cande et al. 2010withSWIR \\
EAnt-Somalia & 28ny & 62.5 & 4.75 & -44.79 & 11.39 & 0.66 & 1.359E-6 & 1.549E-6 & 2.054E-6 & -2.354E-6 & -2.838E-6 & 4.486E-6 & Cande et al. 2010withSWIR \\
EAnt-Somalia & 29no & 64.75 & 4.79 & -45.56 & 11.7 & 0.53 & 9.16E-7 & 9.94E-7 & 1.27E-6 & -1.651E-6 & -1.913E-6 & 3.288E-6 & Cande et al. 2010withSWIR \\
EAnt-Somalia & 34y & 83 & -1.8 & -38.7 & 17.9 & 1 & 1E-15 & 1E-15 & 1E-15 & 1E-15 & 1E-15 & 1E-15 & Rowan and Rowley 2016 \\
EAnt-Somalia & EAn-Afr & 96 & -3.06 & -33.49 & 26.8 & 1 & 1E-15 & 1E-15 & 1E-15 & 1E-15 & 1E-15 & 1E-15 & Marks and Tikku 2001EPSL \\
EAnt-Somalia & M0 & 120.6 & 10.36 & 153.67 & -41.56 & 1 & 1E-15 & 1E-15 & 1E-15 & 1E-15 & 1E-15 & 1E-15 & Muller et al. 2008G3 \\* \bottomrule
\end{longtable}
\end{landscape}

\end{appendices}

\end{document}
