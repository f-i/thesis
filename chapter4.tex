\chapter{How Much Data Needed to Make a Reliable APWP}\label{chap:DatNeed}
\textit{This chapter mainly describes how the mean poles with their raw VGPs
at random reduced densities can make a reliable APWP\@. Further we will see how
much data (raw VGPs) are needed on earth to make a reliable APWP, and how the
``bad'' poles influence the final result when we have less data. Are those
rules we obtained in the last chapter are still true for less data (test if APP
is still better, and weighting is still not affecting)? As we all know, in the
past, especially in deep time, the data density and quality are lower, compared
with younger geological times. Reducing the data density can help see if our
methodology is still able to reliably give reasonable results from data aged in
deep times. Are we be able to make a final determination of best number of VGPs
in each sliding window in average for moving-averaging out an APWP? (No,
different situations for different continents.)}
\vfill
\minitoc\newpage

\section{Reference Path}

The fixed hotspot model and related plate circuit predicted APWPs are used as
references. In fact, as mentioned in the last chapter, choosing FHM or MHM does
not make much difference at all.

\section{Extration Fraction}

The raw VGPs are extracted before being moving-averaged (i.e.\ picking and weighting;
see the last chapter).

\subsection{80\%}
20 percent of raw VGPs are removed.

\subsection{60\%}
Now it is changed to 40 percent removal.

\subsection{40\%}
Now it is changed to 60 percent removal.

\subsection{20\%}
Now it is changed to 80 percent removal.

\begin{figure}
    \centering
        \includegraphics[width=0.88\textwidth]{fig/Fay18_10_5.pdf}
    \captionsetup{width=.95\textwidth}
    \caption{Random VGP samplings (30 times) for the best and worst results for
	the 10 Myr window and 5 Myr step paleomagnetic APWPs vs FHM \& plate circuit
	predicted APWP\@. The lower and upper bound lines connect the 1st and 3th
	quantiles of the 30 samples. The bold line connects their means. The
	numbers in small parentheses are actual quantity of VGPs after filtered by
	the corresponding picking methods for the case with no data
	removal.}\label{Fig:Fay18_10_5bw}
\end{figure}

\begin{figure}
    \centering
        \includegraphics[width=0.88\textwidth]{fig/Fay18_10_5to101bw.pdf}
    \captionsetup{width=.95\textwidth}
    \caption{Comparisons of results from the best and worst methods for North
	America, applied on all the three continents.}\label{Fig:Fay18_10_5to101bw}
\end{figure}

\begin{figure}
    \centering
        \includegraphics[width=0.88\textwidth]{fig/Fay18_10_5to501bw.pdf}
    \captionsetup{width=.95\textwidth}
    \caption{Comparisons of results from the best and worst methods for India,
	applied on all the three continents.}\label{Fig:Fay18_10_5to501bw}
\end{figure}

\begin{figure}
    \centering
        \includegraphics[width=0.88\textwidth]{fig/Fay18_10_5to801bw.pdf}
    \captionsetup{width=.95\textwidth}
    \caption{Comparisons of results from the best and worst methods for
	Australia, applied on all the three continents.}\label{Fig:Fay18_10_5to801bw}
\end{figure}

We can see that the best picking and weighting methods are statistically always
better than the worst ones no matter how much data we have to compose the APWPs
(Fig.~\ref{Fig:Fay18_10_5bw}) for the three continents.

\subsection{Number of Samples}
Here because the thousands of times of testing for each percentage of data
removal and each picking and weighting method is rather expensive, 30 samples
(a rule of thumb; e.g.~\cite{H19} says ``greater than 25 or 30'') are obtained.
In fact, the 25th percentiles ($Q_1$), 75th percentiles ($Q_3$) and the means of
30, 60, 100, 200 and 1000 samples are not quite different
(Fig.~\ref{Fig:Fay18_10_5_0_0}), although 200 seems a better and relatively
cheaper option.

\begin{figure}
    \centering
        \includegraphics[width=0.88\textwidth]{fig/Fay18_10_5_0_0.pdf}
    \captionsetup{width=.95\textwidth}
    \caption{Testing differences of results from different numbers of samples.
	See Fig.~\ref{Fig:Fay18_10_5bw} for more details.}\label{Fig:Fay18_10_5_0_0}
\end{figure}

Describe the function fill-df-a95nan in ma.py:
Situation 1: Note that because of reducing of pole quantity in sliding windows,
in some window there could be only one VGP which does not have given ED95, DM
and DP, then we use $140/\sqrt{kN}$~\cite{T91,T19} to calculate a ED95 for this
VGP as its A95.
